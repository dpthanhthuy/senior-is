%%%%%%%%%%%%%%%%%%%%%%%%%%%%%%%%%%%%%%%%%%%%%%%%%%%%%%%
%
%                                                       Example IS Template
%
% \documentclass{woosterthesis} must be at the beginning of every IS. Options are the same as
% for the report class with some additional options, abstractonly, blacklinks, code, kaukecopyright, palatino, picins,
% maple, index, verbatim, dropcaps, euler, gauss, alltt,  woolshort, colophon, woosterchicago, and
% achemso. The kaukecopyright option will put the arch symbol with the word mark on the
% copyright page. The woosterthesis class is based on the report class. One thing to note is that
% the ``%'' symbol comments out all characters that follow it on the line.
%
%%%%%%%%%%%%%%%%%%%%%%%%%%%%%%%%%%%%%%%%%%%%%%%%%%%%%%%

%%%%%%%%%%%%%%%%%%%%%%%%%%%%%%%%%%%%%%%%%%%%%%%%%%%%%%%
% use this declaration for a draft  version of your IS
\documentclass[10pt,palatino,code,picins,kaukecopyright,openright,woolshort,dropcaps,verbatim,index,euler,woosterchicago]{woosterthesis}
%\documentclass[10pt,code,picins,kaukecopyright,openright,woolshort,dropcaps,verbatim,euler,index,colophon,blacklinks,twoside]{woosterthesis}
% note that you can specify the woosterchicago option to use Chicago citation style and achemso to use the American Chemical Society citation format
%
%%%%%%%%%%%%%%%%%%%%%%%%%%%%%%%%%%%%%%%%%%%%%%%%%%%%%%%
%
% use this declaration for the print version of your IS
%\documentclass[12pt,code,palatino,picins,blacklinks,kaukecopyright,openright,twoside]{woosterthesis} % probably what most students would use
%
%%%%%%%%%%%%%%%%%%%%%%%%%%%%%%%%%%%%%%%%%%%%%%%%%%%%%%%
%
% use this declaration for the PDF version of your IS
%\documentclass[12pt,code,palatino,picins,kaukecopyright,openright,twoside]{woosterthesis}
%
%%%%%%%%%%%%%%%%%%%%%%%%%%%%%%%%%%%%%%%%%%%%%%%%%%%%%%%

%%%%%%%%%%%%%%%%%%%%%%%%%%%%%%%%%%%%%%%%%%%%%%%%%%%%%%%
%
%                                                       Load Packages
%
%   To load packages in addition to the ones that are loaded by default, please place your
%   usepackage commands in the packages.tex file in the styles folder.
%
%%%%%%%%%%%%%%%%%%%%%%%%%%%%%%%%%%%%%%%%%%%%%%%%%%%%%%%

%%%%%%%%%%%%%%%%%%%%%%%%%%%%%%%%%%%%%%%%%%%%%%%%%%%%%%%%%%%%%%%%%%%%%%%%%%%%%%%%%%%%%%%%%%%%%%
%
%                                                       Packages
%
% Do not add any other packages without consulting with Dr. Breitenbucher as they may break the functionality of the class.
%
%%%%%%%%%%%%%%%%%%%%%%%%%%%%%%%%%%%%%%%%%%%%%%%%%%%%%%%%%%%%%%%%%%%%%%%%%%%%%%%%%%%%%%%%%%%%%%

\ifxetex%
	\defaultfontfeatures{Mapping=tex-text}%
		\setmainfont[Numbers=OldStyle,BoldFont={* Semibold}]{Adobe Garamond Pro}% select the body font other choices would be Baskerville, Optima Regular, Didot, Georgia, Cochin
                      \setmathrm{Adobe Garamond Pro}
                      \setmathfont[Digits,Latin]{Adobe Garamond Pro}
		\setsansfont[Scale=.87,Fractions=On,Numbers=Lining]{Myriad Pro}% select the sans serif font other choices would be Skia, Arial, Helvetica, Helvetica Neue
%		\setmonofont[Scale=.88,Fractions=On]{Prestige Elite Std Bold}% set the mono font other choices would be Courier, Monaco, American Typewriter
	           \setmonofont[Scale=.9]{Courier Std}%
%	    \setromanfont[Fractions=On,Numbers=OldStyle, BoldFont={Warnock Pro Semibold}]{Warnock Pro}%
%	    \setsansfont[Scale=.95,Fractions=On,Numbers=Lining]{Myriad Pro}%
%	    \setmonofont[Scale=.91,Fractions=On]{Courier Std Medium}%
%	    \setmonofont[Scale=.88,Fractions=On]{American Typewriter}%
%		\setmonofont[Scale=.94,Fractions=On]{Prestige Elite Std Bold}
%    		\setromanfont[Fractions=On,Numbers=OldStyle]{Minion Pro}
 %    	\setsansfont[Scale=.9,Fractions=On,Numbers=Lining]{Myriad Pro}
%     	\setmonofont[Scale=.93,Fractions=On]{Courier Std Medium}
%     	\setromanfont[Fractions=On,Numbers=OldStyle]{Minion Pro}
%     	\setsansfont[Scale=.85,Fractions=On,Numbers=Lining]{News Gothic Std}
%    		\setmonofont[Scale=.93,Fractions=On]{Prestige Elite Std}
%		\setromanfont[Fractions=On,Numbers=OldStyle]{Minion Pro}
%		\setsansfont[Scale=.9,Fractions=On,Numbers=Lining]{Bell Gothic Std Bold}
%		\setmonofont[Scale=.95,Fractions=On]{Prestige Elite Std Bold}
\fi

\usepackage[utf8]{inputenc}
\usepackage[english,vietnamese]{babel}
\usepackage[T5]{fontenc}
\usepackage{epigraph}
\usepackage[ragged,bottom,hang]{footmisc}
\usepackage{titlesec}
\titlespacing*{\section}{0pt}{.3\baselineskip}{.3\baselineskip}
\titlespacing*{\subsection}{0pt}{.2\baselineskip}{.2\baselineskip}
\titlespacing*{\subsubsection}{0pt}{.1\baselineskip}{.1\baselineskip}
\usepackage{wrapfig}
\interfootnotelinepenalty=10000
\raggedbottom
\setlength{\footnotemargin }{.8em}


%%%%%%%%%%%%%%%%%%%%%%%%%%%%%%%%%%%%%%%%%%%%%%%%%%%%%%%
%
%                                                       Load Personal commands
%                                                                    
%  There will be certain commands that you use frequently in the thesis. You can give these
%  commands new names which are easier for you to remember. You can also combine several
%  commands into a new command of your own. See The LaTeX Companion or Guide to LaTeX
%  for examples on defining your own commands. These are commands that I defined to cut
%  down on typing. You can enter your commands in the personal.tex file in the styles folder.
%
%%%%%%%%%%%%%%%%%%%%%%%%%%%%%%%%%%%%%%%%%%%%%%%%%%%%%%%

%%%%%%%%%%%%%%%%%%%%%%%%%%%%%%%%%%%%%%%%%%%%%%%%%%%%%%%%%%%%%%%%%%%%%%%%%%%%%%%%%%%%%%%%%%%%%%
%
%                                                       Personal Commands
%                                                                    
% There will be certain commands that you use frequently in the thesis. You can give these
% commands new names which are easier for you to remember. You can also combine several
% commands into a new command of your own. See The LaTeX Companion or Guide to LaTeX for
% examples on defining your own commands. These are commands that I defined to cut down on typing.
%
%%%%%%%%%%%%%%%%%%%%%%%%%%%%%%%%%%%%%%%%%%%%%%%%%%%%%%%%%%%%%%%%%%%%%%%%%%%%%%%%%%%%%%%%%%%%%%

\newcommand{\fl}{\ell}
\newcommand{\lt}{\LaTeX\ }
\newcommand{\msw}{Word\texttrademark\ }
\newcommand{\xt}{\ifthenelse{\boolean{xetex}}{\XeTeX\ }{XeTeX} }
\newcommand{\en}{\selectlanguage{english}}
\newcommand{\vi}{\selectlanguage{vietnamese}}
\newcommand{\tl}{\hspace{-2.3pt}$\bullet$ \hspace{5pt}}
%\newcommand{\Cl}{\ensuremath{\textup{C}_\fl}}
%\newcommand{\bCl}{C$_{\ell}$}
%\newcommand{\Al}{\ensuremath{\textup{A}_\fl}}
%\newcommand{\msum}{{(m_1+\cdots+m_\ell)}}
%\newcommand{\Nsum}{{(N_1+\cdots+N_\ell)}}
%\newcommand{\ysum}{{(y_1+\cdots+y_\ell)}}
%\newcommand{\Nsub}{{N_1+\cdots+N_\ell}}
%\newcommand{\ysub}{{y_1+\cdots+y_\ell}}
%\newcommand{\xsub}{{x_1+\cdots+x_\ell}}
%\newcommand{\ysqsum}{{y_1^2+\cdots +y_{\fl}^2}}
%\newcommand{\msqsum}{{m_1^2+\cdots +m_{\fl}^2}}
%\newcommand{\ratio}{\left(\frac{\beta}{\alpha}\right)}
%\newcommand{\LT}{\ensuremath{\LaTeX{}}}

%%%%%%%%%%%%%%%%%%%%%%%%%%%%%%%%%%%%%%%%%%%%%%%%%%%%%%%%%%%%%%%%%%%%%%%%%%%%%%%%%%%%%%%%%%%%%%
% These commands have one argument and are entered as \commandname{argument}.
%%%%%%%%%%%%%%%%%%%%%%%%%%%%%%%%%%%%%%%%%%%%%%%%%%%%%%%%%%%%%%%%%%%%%%%%%%%%%%%%%%%%%%%%%%%%%%

%\newcommand{\bd}[1]{\textbf{#1}}
\newcommand{\mbd}[1]{{\mathbf{#1}}}
%\newcommand{\abs}[1]{\vert{#1}\vert}
\newcommand{\bvec}[1]{{\mbd{#1}}}
%\newcommand{\lvec}[1]{\abs{\bvec{#1}}}
%\newcommand{\nesmallprod}[1]{\prod_{\substack{#1=1\\
%#1\neq p}}^{\fl}}
%\newcommand{\esec}[1]{e_{2}({#1}_1,\ldots ,{#1}_\fl)}
%\newcommand{\smallprod}[1]{\prod_{#1=1}^{\fl}}
%\newcommand{\incsum}[1]{{#1}_2+2{#1}_3+\cdots +(\fl -1){#1}_\fl}
%\newcommand{\binomsum}[1]{\binom{{#1}_1}{2}+\cdots +\binom{{#1}_\fl}{2}}
%\newcommand{\imultsum}[1]{\multsum{{#1}_k\ge 0}{k=1,\ldots ,\fl}}
%\newcommand{\diagsum}[1]{\sum _{\substack{{#1}_k\ge 0\\
%k=1, \ldots ,\fl\\
%\lvec{#1}=m}}}
%\newcommand{\Mb}[1][\fl]{\ensuremath{\textup{\bd{M}}_b^{(#1)}}}
%\newcommand{\HLV}[1]{\ensuremath{\textup{\bd{H}}_{#1}}}
%\newcommand{\Rq}[1][p]{\ensuremath{\textup{R}_q^{(#1)}}}
\newcommand{\degree}[1]{\ensuremath{#1^{\circ}}}
\newcommand{\ip}[1]{\texttt{#1}\index{packages!#1}}
\newcommand{\ic}[1]{\texttt{$\backslash$#1}\index{commands!#1}}
\newcommand{\ie}[1]{#1\index{#1}}

%%%%%%%%%%%%%%%%%%%%%%%%%%%%%%%%%%%%%%%%%%%%%%%%%%%%%%%%%%%%%%%%%%%%%%%%%%%%%%%%%%%%%%%%%%%%%%
% These commands have 2 or more arguments some with default values for the first argument. You
% can learn a lot about constructing complicated equations by studying the commands in this %section.
%%%%%%%%%%%%%%%%%%%%%%%%%%%%%%%%%%%%%%%%%%%%%%%%%%%%%%%%%%%%%%%%%%%%%%%%%%%%%%%%%%%%%%%%%%%%%%

%\newcommand{\qbinom}[2]{\ensuremath{\left[{#1}\atop{#2}\right]_q}}
%\newcommand{\sqprod}[2]{\prod_{#1,#2=1}^{\fl}}
%\newcommand{\triprod}[2]{\prod_{1\le #1<#2\le \fl}}
%\newcommand{\nesqprod}[2]{\prod_{\substack{#1,#2=1\\
%#1,#2\neq p}}^{\fl}}
%\newcommand{\netriprod}[2]{\prod_{\substack{1\le #1<#2\le \fl\\
%#1,#2\neq p}}}
\newcommand{\qrfac}[3][\ ]{\left({#2}\right)_{#3}^{#1}}
%\newcommand{\multsum}[2]{\sum_{\substack{{#1}\\
%\\
%{#2}}}}
%\newcommand{\fmultsum}[2][N]{\multsum{0\le {{#2}_k}\le {{#1}_k}}{k=1,\ldots ,\fl}}
%\newcommand{\pq}[2]{\ _{#1}\varphi_{#2}}
%\newcommand{\mess}[2][y_k]{\frac{\qrfac{\alpha x_k}{#2}\qrfac{qx_k\beta^{-1}}{#2}}{\qrfac{\beta x_k}{#1}
%\qrfac{qx_k\alpha^{-1}}{#1}}}
%\newcommand{\MG}[7][\fl]{\ensuremath{\left[\textup{MG}\right]_{#2}^{(#1)}{#3}q;{#4};{#5}^{#6}{#7}}}

%%%%%%%%%%%%%%%%%%%%%%%%%%%%%%%%%%%%%%%%%%%%%%%%%%%%%%%%%%%%%%%%%%%%%%%%%%%%%%%%%%%%%%%%%%%%%%
% These commands define new environments
%%%%%%%%%%%%%%%%%%%%%%%%%%%%%%%%%%%%%%%%%%%%%%%%%%%%%%%%%%%%%%%%%%%%%%%%%%%%%%%%%%%%%%%%%%%%%%

\newcounter{unnumft}
\setcounter{unnumft}{0}
\newenvironment{unnumft}[2]{\renewcommand{\thefootnote}{}\footnote{#1}\footnote{#2}} {\addtocounter{footnote}{-2}}
\newenvironment{wooexample}{\small
\begin{singlespace}
\begin{example}}{\end{example}
\end{singlespace}}

\graphicspath{{./figures/}}% for setting where to look for figures
\citestyle{wooster}% change the style of citations. Math and CS people should leave this alone.

\usepackage{color}

\DeclareFixedFont{\ttb}{T1}{txtt}{bx}{n}{11} % for bold
\DeclareFixedFont{\ttm}{T1}{txtt}{m}{n}{11} 

\definecolor{deepblue}{rgb}{0,0,0.5}
\definecolor{deepred}{rgb}{0.6,0,0}
\definecolor{deepgreen}{rgb}{0,0.5,0} 
 
\lstset{
language=Python,
otherkeywords={self},             % Add keywords here
keywordstyle=\bfseries\color{deepblue},
emph={MyClass,__init__},          % Custom highlighting
emphstyle=\bfseries\color{deepred},    % Custom highlighting style
stringstyle=\color{deepgreen},
breaklines=true, commentstyle=\textit,
showspaces=false, numbers=left,
stringstyle=\upshape, numberstyle=\tiny, xleftmargin=30pt,
breakautoindent=true, captionpos=b,
basicstyle=\small,
frame=tb
}



%%%%%%%%%%%%%%%%%%%%%%%%%%%%%%%%%%%%%%%%%%%%%%%%%%%%%%%
%
%                                                       Load Theorem formatting information
%
%  If you need to define an new theorem style or want to see what theorem like environments 
%  are available please look at the theorems.tex file in the styles folder.
%
%%%%%%%%%%%%%%%%%%%%%%%%%%%%%%%%%%%%%%%%%%%%%%%%%%%%%%%

%%%%%%%%%%%%%%%%%%%%%%%%%%%%%%%%%%%%%%%%%%%%%%%%%%%%%%%%%%%%%%%%%%%%%%%%%%%%%%%%%%%%%%%%%%%%%%
%
% This is where one would tell \LaTeX{} how to format Theorems, Definitions, etc. and also
% indicate the environment names. You need the amsthm package (loaded in the woosterthesis %class) in order for these commands to work.
%
%%%%%%%%%%%%%%%%%%%%%%%%%%%%%%%%%%%%%%%%%%%%%%%%%%%%%%%%%%%%%%%%%%%%%%%%%%%%%%%%%%%%%%%%%%%%%%

% an example of defining your own theoremstyle
%\newtheoremstyle{break}% name
%  {\topsep}%      Space above
%  {\topsep}%      Space below
%  {\itshape}%         Body font
%  {}%         Indent amount (empty = no indent, \parindent = para indent)
%  {\bfseries}% Thm head font
%  {.}%        Punctuation after thm head
%  {\newline}%     Space after thm head: " " = normal interword space;
%        %       \newline = linebreak
%  {}%         Thm head spec (can be left empty, meaning `normal')
\newtheoremstyle{scthm}{\topsep}{\topsep}{\itshape}{}{\bfseries\scshape}{}{ }{}% small cap font for the heading
\newtheoremstyle{itdefn}{\topsep}{\topsep}{\itshape}{}{\bfseries}{.}{ }{}% italic definitions
\newtheoremstyle{scdefn}{\topsep}{\topsep}{\itshape}{}{}{}{ }{\thmname{\textbf{#1}}\thmnumber{ \textbf{#2}}\thmnote{ \scshape #3:}}% small cap headings and italic text.

\theoremstyle{break}% this theoremstyle will put the text of the theorem on a new line.
\newtheorem{thm}{Theorem}[chapter]%number theorems within chapters 
%\newtheorem{cor}[thm]{Corollary}%by using [thm] we are numbering these environments with the theorems.
\newtheorem{cor}{Corollary}[chapter]%number corollaries within chapters .
%\newtheorem{lem}[thm]{Lemma}
\newtheorem{lem}{Lemma}[chapter]
%\newtheorem{prop}[thm]{Proposition}
\newtheorem{prop}{Proposition}[chapter]

\theoremstyle{scdefn}
%\newtheorem{defn}[thm]{Definition}
\newtheorem{defn}{Definition}[chapter]
\theoremstyle{remark}
%\newtheorem{rem}[thm]{Remark}
\newtheorem{rem}{Remark}[chapter]
\renewcommand{\therem}{}
%\newtheorem{ex}[thm]{Example}
\newtheorem{ex}{Example}[chapter]

\theoremstyle{plain}
%\newtheorem{note}[thm]{Notation}
\newtheorem{note}{Notation}[chapter]
\renewcommand{\thenote}{}
%\newtheorem{nts}[thm]{Note to self}%use to remind yourself of things yet to do
\newtheorem{nts}{Note to self}[chapter]
\renewcommand{\thents}{}
%\newtheorem{terminology}[thm]{Terminology}
\newtheorem{terminology}{Terminology}[chapter]
\renewcommand{\theterminology}{}

\theoremstyle{itdefn}
\newtheorem{bdefn}{Definition}[chapter]
\newsavebox{\fmbox} 
\newenvironment{boxeddefn}[2] 
{\begin{lrbox}{\fmbox}\begin{minipage}{0.9 \linewidth }\begin{singlespace}\begin{bdefn}[{#1}]\label{#2}\vspace{0.2cm}} 
{\end{bdefn}\end{singlespace}\end{minipage}\end{lrbox}\fbox{\usebox{\fmbox}}}

\setcounter{secnumdepth}{5}% controls the numbering of sections
\setcounter{tocdepth}{3}% controls the number of levels in the Contents


%%%%%%%%%%%%%%%%%%%%%%%%%%%%%%%%%%%%%%%%%%%%%%%%%%%%%%%
%
%  This is where one enters the information about the thesis.
%
%%%%%%%%%%%%%%%%%%%%%%%%%%%%%%%%%%%%%%%%%%%%%%%%%%%%%%%


\title{Remembering the City: Mapping the Memory of Ho Chi Minh City using Augmented Reality}
\thesistype{Independent Study Thesis} % you should make this Independent Study Thesis
\author{Thuy Dinh}
%\presentdegrees{Ph.D.} % you should comment this line
\degreetoobtain{Bachelor of Arts}
\presentschool{The College of Wooster}
\academicprogram{Departments of Computer Science and History}
\gradyear{2020}
\advisor{Denise Byrnes (Computer Science)}
\secondadvisor{Margaret Wee Siang Ng}
%\reader{Reader}
\copyrighted   
%\copyrightdate{}                  
\makeindex % comment this line if you do not have an index

%%%%%%%%%%%%%%%%%%%%%%%%%%%%%%%%%%%%%%%%%%%%%%%%%%%%%%%
%
%  This is where the commands for the document begin. All \LaTeX{} documents must have a
%  \begin{document} text .... \end{document} structure.
%
%%%%%%%%%%%%%%%%%%%%%%%%%%%%%%%%%%%%%%%%%%%%%%%%%%%%%%%

\begin{document}
\en
%%%%%%%%%%%%%%%%%%%%%%%%%%%%%%%%%%%%%%%%%%%%%%%%%%%%%%%
%
%  The front matter includes acknowledgments, dedications, vitas, list of tables, list of figures,
%  copyright, abstract, title page, and contents.
%
%%%%%%%%%%%%%%%%%%%%%%%%%%%%%%%%%%%%%%%%%%%%%%%%%%%%%%%

\frontmatter
\maketitle
\ClearShipoutPicture
\clearpage\thispagestyle{empty}\null\clearpage
\disscopyright 

%%%%%%%%%%%%%%%%%%%%%%%%%%%%%%%%%%%%%%%%%%%%%%%%%%%%%%%
%                                                                                       
%                                                       Abstract						
%                                                                                       
%%%%%%%%%%%%%%%%%%%%%%%%%%%%%%%%%%%%%%%%%%%%%%%%%%%%%%%

%\begin{abstract}
%Include a short summary of your thesis, including any pertinent results.  This section is \emph{not} optional for the Mathematics and Computer Science or Physics Department ISs, and the reader should be able to learn the meat of your thesis by reading this (short) section.
%\end{abstract}

%%%%%%%%%%%%%%%%%%%%%%%%%%%%%%%%%%%%%%%%%%%%%%%%%%%%%%%
%                                                                                       
%                                                       Dedications					
%                                                                                       
%%%%%%%%%%%%%%%%%%%%%%%%%%%%%%%%%%%%%%%%%%%%%%%%%%%%%%%

%\dedication{This work is dedicated to the future generations of Wooster students.}


%%%%%%%%%%%%%%%%%%%%%%%%%%%%%%%%%%%%%%%%%%%%%%%%%%%%%%%
%                                                                                       
%                                                       Acknowledgments					
%                                                                                       
%%%%%%%%%%%%%%%%%%%%%%%%%%%%%%%%%%%%%%%%%%%%%%%%%%%%%%%

%\begin{acknowl}  
%I would like to acknowledge Prof. Lowell Boone in the Physics Department for his suggestions and code.
%\end{acknowl}

%%%%%%%%%%%%%%%%%%%%%%%%%%%%%%%%%%%%%%%%%%%%%%%%%%%%%%%
%
%  We now create the contents page and if necessary the list of figures and list of tables.
%
%%%%%%%%%%%%%%%%%%%%%%%%%%%%%%%%%%%%%%%%%%%%%%%%%%%%%%%


\cleardoublepage
\phantomsection
\addcontentsline{toc}{chapter}{Contents}

\tableofcontents
\listoffigures %Use if you have a list of figures.
\listoftables%Use if you have a list of tables.
\lstlistoflistings% Use if you are using the code option

%%%%%%%%%%%%%%%%%%%%%%%%%%%%%%%%%%%%%%%%%%%%%%%%%%%%%%%

%%!TEX root = ../username.tex
\chapter*{Preface}\label{pref}
\addcontentsline{toc}{chapter}{Preface}
\lettrine[lines=2, lhang=0.33, loversize=0.1]{T}he purpose of this document is to provide you with a template for typesetting your IS using \LaTeX\index{LaTeX@\LaTeX}. \lt is very similar to HTML in the sense that it is a markup language. What does this mean? Well, basically it means you need only enter the commands for structuring your IS, i.e., identify chapters, sections, subsections, equations, quotes, etc. You do not need to worry about any of the formatting. The  \texttt{woosterthesis} class takes care of all of the formatting.

Here is how I plan on introducing you to \LaTeX. The Introduction gives some reasons for why one might find \lt superior to MS Word\texttrademark. Chapter \ref{text} will demonstrate how one starts typesetting a document and works with text in \LaTeX. Chapter \ref{graphics} discusses the creation of tables and how one puts figures into a thesis. Chapter \ref{bibind} talks about creating a bibliography/references section and an index. There are three Appendices which discuss typesetting mathematics and computer program code. The Afterword will discuss some of the particulars of how a \lt document gets processed and what packages the \texttt{woosterthesis} class uses and are assumed to be available on your system.

Hopefully, this document will be enough to get you started. If you have questions please refer to \citet{mgbcr04,kd03,ophs03,feu02,fly03}, or \citet{gra96}. % most theses do not have a preface so this should be commented

%%%%%%%%%%%%%%%%%%%%%%%%%%%%%%%%%%%%%%%%%%%%%%%%%%%%%%%
\mainmatter

%%%%%%%%%%%%%%%%%%%%%%%%%%%%%%%%%%%%%%%%%%%%%%%%%%%%%%%
%
%                                                       Thesis Chapters
%
% This is where the main text of the thesis goes. I have written this template assuming that
% each chapter is a separate file. You do not have to do this but it makes things easier to find
% for editing. You can use the sample chapters to help you figure out how to type things into
% your thesis. To include a chapter just use the \include{chaptername} command. Chapters are
% included in the order listed.
%
%%%%%%%%%%%%%%%%%%%%%%%%%%%%%%%%%%%%%%%%%%%%%%%%%%%%%%%

%!TEX root = ../username.tex
\chapter{Introduction}\label{intro}
\vi
\section{Saigon in \textit{Miền Ký Ức}}
\vi In 2019, Saigon Bitexco Financial Tower held the exhibition “Memory Museum: Once There Was Saigon,” showcasing miniature replicas of old Saigon scenes (Figure~\ref{sgxua}).\footnote{“‘Bảo Tàng Ký Ức’ Lần Đầu Tiên Của Người Sài Gòn,” \textit{Dân Trí}, accessed February 26, 2020, \url{https://dantri.com.vn/van-hoa/bao-tang-ky-uc-lan-dau-tien-cua-nguoi-sai-gon-20190425180415510.htm}.} The exhibition was the brainchild of the youth art group SG-Xưa (Old-SG), whose business model profits from the creation and sales of handcrafted models replicating old symbols of Ho Chi Minh City. The commodification of memory in Ho Chi Minh City is not a new phenomenon. In recent years, there has been a growing obsession with antiques and romanticized representations of the past in the forms of cafés, restaurants, bookstores, etc.\footnote{Má Lúm, “Trào Lưu Quán Cà Phê Bao Cấp Khắp Ba Miền,” \textit{VnExpress}, April 10, 2017, \textit{\url{https://dulich.vnexpress.net/photo/am-thuc/trao-luu-quan-ca-phe-bao-cap-khap-ba-mien-3568152.html}}; Tâm An, “Uống cà phê, nhớ những tháng ngày xưa cũ ở Đà Nẵng,” \textit{Dân Trí}, November 3, 2017, \url{https://dantri.com.vn/doi-song/uong-ca-phe-nho-nhung-thang-ngay-xua-cu-o-da-nang-20171103065604954.htm}.} Such is the spread of this trend that writers have taken to call Ho Chi Minh City a place in "miền ký ức." "Miền ký ức" is a term often associated with remembering Saigon's past.\footnote{Lam Phong, “Sài Gòn của một miền ký ức,” \textit{Tuổi Trẻ Online}, January 19, 2014, \url{https://tuoitre.vn/news-590460.htm}} Saigon in "miền ký ức" is a realm (\textit{miền}) of memory (\textit{ký ức}) where life was simpler and societal morals were not corroded. This realm of memory elicits nostalgic recollections of and longing for values and embedded in sites that are now thought of as lost or at risk from the current rate of urbanization.\footnote{Ngô Minh Hùng, “Ký Ức Sài Gòn - Chợ Lớn Xưa,” \textit{Tạp Chí Kiến Trúc}, 2017.} The nostalgia for a former Saigon is often connected to notions of an ideal living environment, whether that is a Western metropolis or a close-knit Asian society. Much as they are romantic and idealistic, these projections often turn out contradictory, misleading, unfounded, and superficial. What they do indicate, nevertheless, is the contested processes that result in the formation of such memories. 
\en

\begin{figure}[!ht]
\begin{center}
\woopic{sgxua}{0.25}
\vspace{-.2 in}
\caption[\vi Sg-Xưa's Exhibit]{\vi A mini model of Bến Thành Market in SG-Xưa's exhibit (Photograph by SG Xưa. Facebook, February 21, 2020.\footnotemark)}\label{sgxua}
\end{center}
\vspace{-.4 in}
\end{figure}

\vi A city is concrete, but memory is an abstract concept.\footnotetext{SG Xưa, "Chợ Bến Thành,"  \textit{ Facebook}, February 21, 2020. \url{https://www.facebook.com/mohinhsaigonxua/photos/pcb.566548354205522/566539340873090/?type=3&theater}} How is the way a city is remembered related to the bricks and mortar used to construct it? In the case of Ho Chi Minh City, the parallels between these two concepts are even more striking, considering its relatively short history.\footnote{This 300-year periodization is only concerned with its recent history as a Vietnamese urban center. The history of this region started much earlier before the arrival of the Vietnamese.} Mythical tales of the city's miraculous development and fighting spirit through the wars have consolidated a notion of Ho Chi Minh City that exists outside of its pre-1600s history of wilderness. From the myths of its origin, it is obvious that these histories must have also been crafted alongside the physical work of building this urban center. In light of the tumultuous events and drastic transformations that transpired in the last three centuries, it is remarkable that these narratives could prove so pervasive in solidifying the position of the city in memory.

Ho Chi Minh City's history is one of foreign expansion and control. The Vietnamese were not natives in this land. Systemic Vietnamese settlement in South Vietnam only started with the waves of conquests into the South by the Vietnamese ruling dynasties since the 16th century. Before Vietnamese presence in the area, Baigaur/Prey Nokor (former names of Ho Chi Minh City) was part of Indianized principalities like Funan, Chenla, the Kingdom of Champa and other proto-Cambodian polities.\footnote{Philippe Peycamm, “Saigon, From the Origins to 1859,” in \textit{Saigon, Ba Thế Kỷ Phát Triển Và Xây Dựng [Three Centuries of Urban Development]}, ed. Quang Ninh Lê and Stéphane Dovert, 4th ed. (Ha Noi: Nhà xuất bản Hồng Đức, 2015), 28–29.} Saigon fell into the hands of French colonizers after their attack in 1859 and remained under French control until their defeat in 1954, except for a brief stint of Japanese occupation in World War Two.\footnote{Hữu Ngọc, Ca Van Thinh, and Ta Xuan Linh, “Old Saigon,” in \textit{From Saigon to Ho Chi Minh City: A Path of 300 Years}, ed. Nguyễn Khắc Viện (Ha Noi: Thế Giới Publishers, 1998), 144–45.} Following the division of Vietnam after the Geneva Conference in 1954, Ngô Đình Diệm established the South Vietnamese government with the support of the United States.\footnote{Nghia M. Vo, \textit{Saigon: A History} (Jefferson, N.C: McFarland, 2011), 126.} Communist forces retook Saigon and gained control of the entire country in April 30, 1975, known as Liberation Day in Vietnamese national discourse or the Fall of Saigon in Western media. The immediate aftermath of the war was known as the \textit{bao cấp} (subsidy) period, during which the government aggressively nationalized private enterprises and properties as well as monopolized the distribution of goods and resources.\footnote{Hy V. Luong, \textit{Postwar Vietnam: Dynamics of a Transforming Society}, Asian Voices (Singapore: Rowman \& Littlefield, 2003), 2–8.} These restrictions had devastating effects on the postwar economy and were finally lifted in 1986 by a series of reforms called Đổi Mới (Renovation).\footnote{Vo, \textit{Saigon}, 198–210.} Today, Ho Chi Minh City (renamed from Saigon in 1976) is the industrial center of the hybrid semi-capitalist economy of Vietnam. It also has the highest population in the country with almost ten million people.\footnote{Thông tấn xã Việt Nam, “Dân Số TPHCM Gần 9 Triệu Người, Đông Nhất Cả Nước,” \textit{Báo Sài Gòn Đầu Tư Tài Chính}, October 12, 2019, \url{https://saigondautu.com.vn/content/NjcxMDQ=.html}.}  

Considering past turbulence, it is unlikely that any simple narrative can uncover the true complexity of its history. The city of Ho Chi Minh painted in the current light of progress and modernity is an elaborately constructed memory.  Despite having all the designs of a modern metropolis, the area known as Ho Chi Minh City today was mostly devoid of human habitation for the larger part of history, an environment inhospitable to all except for mosquitos and crocodiles. The bifurcation between past and present also includes the change in demographics from mostly Cham, Khmer, and Chinese to predominantly Vietnamese. Power transitions from colonial to democratic to socialist created conflicting memories, forcing these ruling regimes all had to grapple with the tensions and contradictions. The rapid transformations of Ho Chi Minh City resulted a multilayered mesh of identities, warped to account for the changes to the city's environmental and social makeup. But these changes were far from smooth. After one century of colonization, three major wars, four regime changes, all initiated by foreigners to this land, it seems inevitable that the city’s history has become a contested ground for defining and legitimizing these sometimes violent alterations. As a result, memory has become such a central concept to the building and governing of the city. With this in mind, how are strands of memory embedded in the physical components of urban construction, by whom, and for what reason? This research seeks to unravel the connections between memory, power, and identity, centered around the theme of environment, including the natural, built, and symbolic. The authorities of Ho Chi Minh City have harnessed their power to shape these environments of the city to create memory, which solidifies their legitimacy and justifies future augmentations to the city.

\section{Memory, Vietnam, and Ho Chi Minh City}
The framework for understanding memory, according to scholars like Maurice Halbwachs and Pierre Nora, is based on the premise that memory is fallible, malleable, and as a result, subject to distortion.\footnote{For more on memory, see Maurice Halbwachs, \textit{The Collective Memory}, 1st ed (New York: Harper \& Row, 1980); Pierre Nora, \textit{Rethinking France: Les Lieux de Mémoire}, trans. Mary Trouille, vol. 1 (Chicago: University of Chicago Press, 2001); Paul Ricœur, \textit{Memory, History, Forgetting} (Chicago: University of Chicago Press, 2004); Alexandre Dessingué and J. M. Winter, eds., \textit{Beyond Memory: Silence and the Aesthetics of Remembrance}, Routledge Approaches to History 13 (New York: Routledge, 2015); Daniel L. Schacter, \textit{The Seven Sins of Memory: How the Mind Forgets and Remembers} (Boston: Houghton Mifflin, 2001); Jeffrey K. Olick, \textit{The Politics of Regret: On Collective Memory and Historical Responsibility} (New York: Routledge, 2007).} Psychologists like Daniel Schacter suggest that every time the human brain recollects an event from the past, it alters the memory to account for the feelings, beliefs and knowledge acquired after the experience.\footnote{Schacter, \textit{The Seven Sins of Memory}.} Because of this impressionable nature, memory changes over time, making the process of influencing recollections, or the process of memory, a historical process with context, agency, and consequences. Memory could then be social and collective.\footnote{Halbwachs, \textit{The Collective Memory}, 50–87.} It can be constructed through the use of \textit{lieux de memoir}, the term Nora uses when referring to the sites of embedded images or narratives designed to evoke a particular reconstruction of the past.\footnote{Nora, \textit{Rethinking France}.} Sites of memory can be physical or immaterial or both. A physical site of memory is meant to trigger a mental response; public sites of memory are tools for constructing national narratives.\footnote{J. M. Winter, \textit{Sites of Memory, Sites of Mourning: The Great War in European Cultural History}, Canto Classics edition, Canto Classics (Cambridge: Cambridge University Press, 2014).} Within this broader framework, there are other micro-processes concerned with specific lenses through which memories are framed, such as through commemoration, nostalgia, or suppression.\footnote{Henry Rousso, \textit{The Vichy Syndrome: History and Memory in France since 1944} (Cambridge, Mass: Harvard University Press, 1991).} 

The existing body of scholarship on remembrance in Vietnam builds upon the same framework for studying memory, which is mostly from a Eurocentric perspective, with the vast majority of works focusing on European or American history. Some of the most important memory scholarship on Vietnam is by Hue-Tam Ho Tai, whose research is concerned with the commemorative mode of reimagining the past in the public sphere through mediums such as memoirs, paintings, tourism, cinematography, or in spaces such as war monuments, cemeteries, shrines, and museums.\footnote{For more on historical memory studies on Vietnam, see Hue-Tam Ho Tai and John Bodnar, \textit{Country of Memory: Remaking the Past in Late Socialist Vietnam} (Berkeley: University of California Press, 2001); Long T. Bui, “The Debts of Memory: Historical Amnesia and Refugee Knowledge in The Reeducation of Cherry Truong,” \textit{Journal of Asian American Studies} 18, no. 1 (February 25, 2015): 73–97; David G. Marr, “History and Memory in Vietnam Today: The Journal ‘Xưa \& Nay,’” \textit{Journal of Southeast Asian Studies} 31, no. 1 (2000): 1–25; Nathalie Huynh Chau Nguyen, \textit{Memory Is Another Country: Women of the Vietnamese Diaspora} (Santa Barbara: Praeger, 2009); Christina Schwenkel, “Recombinant History: Transnational Practices of Memory and Knowledge Production in Contemporary Vietnam,” \textit{Cultural Anthropology} 21, no. 1 (2006): 3–30.} The American War of Resistance, internationally known as the Vietnam War, is a significant moment in Vietnam’s collective and personal memory. New modes of remembrance emerge as a result of the war’s disruptive nature, allowing for reconciliation and rebuilding, whether in celebration or in grief.\footnote{Nathalie Huynh Chau Nguyen, \textit{South Vietnamese Soldiers: Memories of the Vietnam War and After} (Santa Barbara: Praege, 2016); Karen Turner-Gottschang and Thanh Hao Phan, \textit{Even the Women Must Fight: Memories of War from North Vietnam} (New York: Wiley, 1998).} Authors like Viet Thanh Nguyen and Scott Laderman write about narratives that have materialized from the struggle to make sense of the war, mostly from a top-down approach through examining the national historical discourse.\footnote{Viet Thanh Nguyen, \textit{Nothing Ever Dies: Vietnam and the Memory of War} (Cambridge: Harvard University Press, 2016); Scott Laderman, \textit{Tours of Vietnam: War, Travel Guides, and Memory}, American Encounters/Global Interactions (Durham: Duke University Press, 2009).} Research on remembrance at the community level is limited to Vietnamese in the diaspora due to state suppression, but recently, especially after Renovation (1986), alternative modes to commemoration have become more common as Vietnamese learn to navigate various layers of bureaucracy and censorship to inject revolutionary beliefs into the collective, state-sanctioned public memory.\footnote{Rivka Syd Eisner, “Performing Prospective Memory,” \textit{Cultural Studies} 25, no. 6 (November 2011): 892–916.}

On the topic of Ho Chi Minh City specifically, one sees a curious case with Vietnamese-language texts where remembrance is vibrant, but its growth is at the expense of critical memory works. The most popular genre for historical writings about Ho Chi Minh City is \textit{tản văn}. These are expressive and descriptive vignettes on a particular subject and are usually without a central argument.\footnote{Hoài Nam, “Tản Văn, Từ Một Cái Nhìn Lướt,” \textit{Báo Công an Nhân Dân Điện Tử}, January 30, 2015, \url{http://antgct.cand.com.vn/Nhan-dam/Tan-van-tu-mot-cai-nhin-luot-340089/}.} The most accomplished writer of this genre on South Vietnam is Sơn Nam. His writings are compilations of accounts about the city’s history centered around themes such as nature, bureaucrats, and land reclamation.\footnote{Sơn Nam, \textit{Đất Gia Định - Bến Nghé Xưa \& Người Sài Gòn} (Ho Chi Minh: Nhà xuất bản Trẻ, 2016).} The growing fascination with Ho Chi Minh City's past has also encouraged the rise in popularity of works such as \textit{Sài Gòn Những Biểu Tượng (Saigon’s Icons)} and \textit{Vọng Sài Gòn (Saigon Echoes [of memory])}.\footnote{Du Tử Lê et al., \textit{Sài Gòn Những Biểu Tượng} (Ho Chi Minh: Nhà Xuất Bản Văn Hóa - Văn Nghệ, 2018); Trác Thúy Miêu, \textit{Vọng Sài Gòn} (Ha Noi: Nhà Xuất Bản Hội Nhà Văn, 2019).} These books have similar themes that evoke nostalgia and pride in the city, which borders on exceptionalism.\footnote{Vo, Saigon, 54; Nguyễn Khắc Viện and Hữu Ngọc, eds., \textit{From Saigon to Ho Chi Minh City: A Path of 300 Years} (Ha Noi: Thế Giới Publishers, 1998), 11–15.} There are very few instances where these memories of Ho Chi Minh City are questioned, and often they are done in conjunction with criticism of French colonialism.\footnote{Đoàn Khắc Tình, “Cái Lý Của Nghệ Thuật Kiến Trúc Thuộc Địa,” \textit{Tạp Chí Kiến Trúc}, August 11, 2014, \url{https://www.tapchikientruc.com.vn/chuyen-muc/ly-luan-phe-binh-kien-truc/cai-ly-cua-nghe-thuat-kien-truc-thuoc-dia.html}.} Critical memory studies that challenge the agency of the narratives around Ho Chi Minh City's creation and development outside the French colonization period are almost non-existent, in spite of the burgeoning business of memorializing some idealized, romanticized, and depoliticized versions of the past.\footnote{For more on Ho Chi Minh City, see Nguyễn Viết Ngoạn, \textit{Di Sản Sài Gòn [Saigon Heritage]} (Ha Noi: Nhà xuất bản Thời đại, 2014); Nguyễn Thanh Lợi, \textit{Sài Gòn Đất và Người} (Ho Chi Minh: Nhà xuất bản Tổng hợp Thành phố Hồ Chí Minh, 2015); Huỳnh Ngọc Trảng, \textit{Sài Gòn - Gia Định xưa: tư liệu \& hình ảnh} (Ho Chi Minh: Nhà xuất bản Thành phố Hồ Chí Minh, 1997).}

The interest in memory in an urban setting often stems from the fear of heritage loss in a fast-paced environment.\footnote{For more on urban memory, see Barbara E. Thornbury and Evelyn Schulz, eds., \textit{Tokyo: Memory, Imagination, and the City} (Lanham, Maryland: Lexington Books, 2018); Steve Nash and Austin Wiliams, “The Historic City: False Urban Memory Syndrome,” in \textit{The Lure of the City: From Slums to Suburbs}, ed. Austin Williams and Alastair Donald (London: Pluto Press, 2011), 98–116; Teresa Stoppani, \textit{Unorthodox Ways to Think the City: Representations, Constructions, Dynamics} (New York: Routledge, 2019).} The past is a myth, and historic sites are “little more than the carcasses of former functions,” as Mark Crinson aptly describes.\footnote{Mark Crinson, ed., \textit{Urban Memory: History and Amnesia in the Modern City} (London: Routledge, 2005), xi.} Jordan Sand’s analysis on Tokyo vernacular reveals that such concern is especially true in places where there is little to preserve. The building stock of Tokyo was repeatedly destroyed and rebuilt, and in the place of this disappearing heritage, urban planners and residents have started to recreate spaces and objects to commemorate and preserve the past.\footnote{Jordan Sand, \textit{Tokyo Vernacular: Common Spaces, Local Histories, Found Objects} (Berkeley: University of California Press, 2013), 1–5.} The materials of urban memory can be personal, in the form of literature, artwork, or public, like architecture and city plans. These tools of remembrance are political, and as Christine Boyer argues, engage with concepts such as “the public sphere” or “collective memory” in an attempt to master and dominate urban spaces and the experiences they accommodate.\footnote{M. Christine Boyer, \textit{The City of Collective Memory: Its Historical Imagery and Architectural Entertainments} (Cambridge, Mass: MIT Press, 1994), 3.} The memory phenomenon in Ho Chi Minh City is manifested in voracious consumption of historic imagery and ideals engendered in the materials of urban memory. Studying their connection to the city’s history is integral to understanding citizens’ experiences and responses to efforts by the state and other players in the urban memory environment.

This project considers the question of memory and city building from an ideological angle, using physical sites of memory including maps and buildings. Both of these sources are connected to the manipulation of space to produce meaning. To examine how cartography and architecture have been employed by different governments or enterprises to create memory, this study investigates their use of symbology, design, structure, function, and technology to convey the builders’ intentions for the city and how it should be remembered. In addition to critiquing this process of memory formation through the spatial mediums of maps and buildings, this research employs a digital spatial medium to visualize this process, in what could be described as an attempt at critical augmented reality.

\section{An Augmented Urban History}
Augmented reality (AR) describes a technology that allows for computer-generated (or virtual) contents to be superimposed or projected on the environment around us using special interfaces such as digital glasses, phone screens, or projectors. AR relies on computer vision techniques to generate a map of our surroundings, called the physical (or real) environment, and align virtual contents to this map.\footnote{H. Durrant-Whyte and T. Bailey, “Simultaneous Localization and Mapping: Part I,” \textit{IEEE Robotics Automation Magazine} 13, no. 2 (June 2006): 99–110.} It has applications in numerous fields, including healthcare, education, entertainment, and history. Augmented reality has been used in the research phase for textual and image analysis as well as for sharing findings through AR-enabled public history projects.\footnote{Kevin B. Kee and Timothy Compeau, eds., \textit{Seeing the Past with Computers: Experiments with Augmented Reality and Computer Vision for History}, Digital Humanities (Ann Arbor: University of Michigan Press, 2019).} Technologies like augmented reality and virtual reality are increasingly commonplace at sites of public memory such as museums and historic monuments. It has grown to become a site of memory of its own, with the same characteristics as augmented reality's traditional counterpart, concerning issues with agency, audience, and agenda.

The desire to control the surrounding environment and to “augment” it with interactive and maneuverable data is what motivates the various iterations of augmented reality in history. In 1968, Ivan Sutherland introduced the first augmented reality system, but the term “augmented reality” did not officially come about until 1992.\footnote{Clemens Arth et al., “The History of Mobile Augmented Reality,” \textit{ArXiv}, no. 1505.01319 (November 10, 2015): 3, \url{http://arxiv.org/abs/1505.01319}.} As the field of computer vision develops, scientists like Paul Milgram and Fumio Kishino have come up with different ways to describe the emerging technologies, such as the Reality-Virtuality Continuum (explained in Chapter 2).\footnote{Arth et al., 5.} Since then, AR contents have gone from simple wireframe drawings to complicated three-dimensional models and clunky hardware is gradually replaced by more portable and efficient processors and displays.  Alongside hardware, AR software also benefits from the advances in detection and tracking algorithms. In 1991, Durrant-Whyte et al. made groundbreaking progress with the simultaneous localization and mapping problem, which is used extensively in augmented and virtual reality.\footnote{Durrant-Whyte and Bailey, “Simultaneous Localization and Mapping.”} The increased portability and compactness of mobile devices have allowed for AR to be deployed more effectively and with advanced features empowered by motion measuring components in the devices’ hardware. These developments make AR accessible to many disciplines and industries without the technical overhead.

The parallels regarding space usage between the sites of memory of interest in this project (maps and buildings) and augmented reality can provide an illuminating case study on how effective these new computer vision techniques could help historians better understand processes of history and convey their understanding to the general public, especially the communities whose lives these histories directly affect. In this example of Ho Chi Minh City, the introduction of an augmented reality component is an attempt to present a top-down perspective (government and ideology) to a grassroots audience. The critical part of this project is delivered through a close examination of the back-end implementation of augmented reality as well as a critique of the ethics and implications of this technology, particularly in relation to memory and power.

Through close reading of maps and architectural structures of Ho Chi Minh City, this Independent Study argues that the ruling regimes of the city as well as private enterprises and local communities have fashioned these sites of memory into devices for mass-mediating the façade and function of the city’s spaces. The mobilization of cartography and monumental architecture to create memory is a form of control made permanent by accumulation of wealth and power. These spaces are the manifestation of ideologies such as modernity and urbanism, conceptualized on plans and blueprints and realized with bricks and mortar. The augmented reality exhibit accompanying this project provides an opportunity for users to deconstruct these narratives by giving them a vantage point to view these historical processes, enabled by computer vision algorithms and techniques. Such an attempt also serves to demonstrate how perspectives are created with the assistance of power (AR technology) and that science is also a form of power capable of manufacturing memory. Cartography and architecture are two examples of how science has participated in both the physical and mental construction of Ho Chi Minh City.

This independent study is made up of two main sections: the first part discusses the technical aspect of augmented reality and the process of designing an AR historical exhibit on the Vuforia platform, while the second part includes historical analysis of maps and sites from four major periods in Vietnam’s history. Although this organization creates a dichotomy between technology and history, each chapter considers both aspects within the context of each other. Chapter 2 offers a high-level overview of the technical underpinnings of augmented reality as well as an analysis of feature detection algorithms, while considering the implication of these methods from a humanities perspective. Chapter 3 moves on to the workflow of designing with Vuforia and the elements of digital storytelling for historical narratives. The contents of the exhibit are drawn from the analyses in chapters 4 and 5, which discuss cartography and architecture respectively and their relationship with memory. Each chapter concludes with a description of the augmented reality experience associated with each site of memory. Exhibit items include an augmented map and several 3D models of the buildings considered in chapter 5. All software AR components are included in a mobile app available for both Android and Apple mobile devices.

\en

%!TEX root = ../username.tex
\chapter[Technicalities of AR and Applications]{The Technicalities of Augmented Reality and Applications in Digital Humanities}\label{text}
\vi
In 2007, an Italian research group from the Polytechnic University of Marche collaborated with Huế College of Sciences in cataloguing "heritage cities" and architecture to produce a 3D model of the city of Huế.\footnote{Fausto Pugnaloni, Giovanni Issini, and Nam Dang Minh, “3D City Model of the Ancient Hue, Vietnam; Reconstruction of the City Environment for the Cultural Heritage Identity Conservation,” in \textit{Virtual Systems and Multimedia}, ed. Theodor G. Wyeld, Sarah Kenderdine, and Michael Docherty, Lecture Notes in Computer Science (Berlin, Heidelberg: Springer, 2008), 13–23.} The project's goals emphasized the need for Asian countries to codify and analyze their heritage and illustrated how technologies such as virtual reality can assist in this process. In this case study, the former imperial city of Huế was chosen as a site for experimenting with spatial computing in the reconstruction of the urban environment and surveying the transformations of its historic monuments. One of the motivations for this collaboration was the fear of a loss of identity and cultural roots, especially among younger generations, amidst economic growth and social transformation. The underlying premise of heritage preservation using emerging technologies is that the production of a kind of virtual replication can preserve the history of these sites of heritage. From a historical perspective, these virtual reproductions are what Pierre Nora and other historians of memory call "sites of memory."\footnote{Pierre Nora, \textit{Rethinking France: Les Lieux de Mémoire}, trans. Mary Trouille, vol. 1 (Chicago: University of Chicago Press, 2001).} The prevalence of spatial computing mediums such as virtual reality and augmented reality in fields such as museum studies and heritage preservation requires an understanding of how these technologies work and their historical implications when applied to historical projects. This section provides a brief overview augmented reality, its algorithmic and software requirements, as well as discusses the emergence of digital humanities and relevant considerations.

\section{Spatial Computing}
Augmented reality is an application of spatial computing. Spatial computing is the digital technology that blends computer-generated contents with the real world, allowing users to interact with digital data in their physical space in real time.\footnote{Magic Leap, “What Is Spatial Computing?,” Magic Leap Creator, March 29, 2019, \url{https://creator.magicleap.com/learn/guides/design-spatial-computing}.} What defines spatial computing is the type of interactions that it affords: natural gestures in three dimensions with the physical environment as the interface.\footnote{In this I.S., I will use the terms "physical environment/world/space" and "real environment/world/space" interchangeably. They refer to the actual lived-in environment that we occupy.}  Spatial hardware and software can calculate the device's relative position in the physical world and create meaningful interactions using their understanding of the surrounding space. Spatial computing is also often known as the integration of different realities. In 1994, Paul Milgram and Fumio Kishino defined the reality-virtuality continuum as a spectrum with the physical environment on one extreme and the  virtual environment on the other (Figure \ref{realityvirtual}).\footnote{Jon Peddie, \textit{Augmented Reality: Where We Will All Live} (Cham, Switzerland: Springer, 2017), 1–28.} In the middle of Figure~\ref{realityvirtual}, augmented reality indicates the use of digital contents to add to (augment) the physical world, while augmented virtuality is using real-world elements to augment the virtual space. Other definitions continue to build on this understanding about the extent to which technology augments and ev\en modifies our view of the world. The emerging technologies that use spatial computing to create realities are commonly known as \textbf{Extended Reality} (XR), where X is a variable for the different kinds of realities along the Reality-Virtuality Continuum. This term includes mediums such as \textbf{Augmented Reality}, \textbf{Mixed Reality} and \textbf{Virtual Reality}.\footnote{Alexandria Hexton, “The Revolution of Spatial Computing: Emerging Design Frontiers in VR/AR” (October 4, 2019), \url{http://signage.showprg.com/ghc19/9d29dc65-3774-41bb-9d33-6c2d1d76a575-96117-Alexandria-Heston.pdf}.}

\begin{figure}[!ht]
\rightline{
\begin{minipage}{\textwidth}
\begin{center}
\woopic{realityvirtual}{1.0}
\vspace{-.2 in}
\caption[Milgram and Kishino's Reality-Virtuality Continuum]{Milgram and Kishino's Reality-Virtuality Continuum. (Illustration by Paul Milgram et al. In \textit{Augmented Reality: Where We Will All Live}. By Jon Peddie.)}\label{realityvirtual}
\end{center}
\end{minipage}
}
\vspace{-.1 in}
\end{figure}

\begin{enumerate}
	\item Augmented Reality (AR)
	\newline
	Augmented Reality is a technology that uses spatial computing to overlay computer-generated contents onto the physical environment. Virtual data is projected on top of the real world using mediums such as phone screens, heads-up displays, and wearable devices. An example of an augmented robot in front of a real couch is shown in Figure~\ref{AR}.
	\item Mixed Reality (MR)
	\newline
	Mixed Reality is an extended version of Augmented Reality. The main difference lies in the virtual contents' ability to interact with real-world elements following the laws and principles of physics. Virtual objects in MR respect the presence of other objects in the physical environment. In the example in Figure~\ref{MR}, the robot's position is behind the couch. Therefore, part of its body is hidden. MR contents also react to real objects; for instance, when the robot encounters the couch, it walks around it.
	\item Virtual Reality (VR)
	\newline
	Virtual Reality is a technology that creates an immersive experience by blocking the real world and presenting an alternative reality that simulates a real environment. The key in VR is convincing users that the virtual environment is real and suspending disbelief through two channels, sound and touch. In Figure~\ref{VR}, the same robot is placed in a computer-generated 3D space. Virtual Reality is at the virtuality end of the Reality-Virtuality Continuum.
\end{enumerate}

\begin{figure}[!ht]\centering
\subfigure[AR][AR]
{\woopic{AR}{.25}\label{AR}}
\qquad
\subfigure[MR][MR]
{\woopic{MR}{.25}\label{MR}}
\qquad
\subfigure[VR][VR]
{\woopic{VR}{.25}\label{VR}}
\caption{Three XR mediums (Illustration by Alexandria Heston. Grace Hopper Celebration 2019, Orlando, October 4, 2019.)}\label{fig3}
\end{figure}
\vspace{-.1 in}

\section{AR Taxonomy}
As an emerging technology, the taxonomy of augmented reality is constantly evolving to describe the ever-changing technological landscape. The standards for classification in AR are shifting with the introduction of new devices, methods of augmentation, and contents. As it is currently impossible to use a one-size-fits-all taxonomy for augmented reality, this section introduces some of the most common ways to categorize AR, by types of displays, types of devices, and types of tracking technology.

\subsection{Types of Displays}
When discussing the requirements and characteristics of an AR visual system, Tobias Höllerer and Dieter Schmalstieg suggest that ideally an AR display is capable of creating 3D augmentations that occupy physical spaces.\footnote{Dieter Schmalstieg and Tobias Höllerer, “Displays,” in \textit{Augmented Reality: Principles and Practice} (Boston: Addison-Wesley, 2016).} To convince the human eye, an AR system has to conform to principles of human vision under the limitations of visual display technology. Humans have a field of vision ranging from $200^o$ to $220^o$, but the area with the highest clarity (fovea) is only from $1^o$ to $2^o$.\footnote{Schmalstieg and Höllerer.} Humans can make up for this by moving their eyes and heads, so the actual fovea can cover up to $50^o$. An AR display needs to ensure that its \textbf{field of view} and \textbf{resolution} accommodate human field of vision and range of fovea. Humans' ability to adjust to different lighting conditions through pupil dilation also means that AR lighting must be able to simulate all levels of hues and contrast and/or have agnostic contents, or contents that are viewable in different lighting conditions.\footnote{Bushra Mahmood, “A Quick Guide to Designing for Augmented Reality on Mobile (Part 3),” Medium, February 3, 2019, \url{https://medium.com/@goatsandbacon/a-quick-guide-to-designing-for-augmented-reality-on-mobile-part-3-2380f253467a}.} Another important consideration in designing for the human eye is monocular and binocular depth cues. The use of one eye (monocular field of vision) provides information such as size, height, occlusions (objects hidden behind others), shadows, and linear perspective (illusion of depth which makes further objects appear smaller). Perception of depth increases with the use of two eyes (binocular field of vision). The positional disparity between the image perceived by each eye is processed by the brain to create a sense of depth. This binocular depth cue is especially important for designing contents for AR eyeware.

\begin{figure}[!ht]
\rightline{
\begin{minipage}{\textwidth}
\begin{center}
\woopic{hololens}{0.7}
\vspace{-.2 in}
\caption[Microsoft HoloLens 2]{Microsoft HoloLens 2 (Photograph by Microsoft.\footnotemark)}\label{hololens}
\end{center}
\end{minipage}
}
\vspace{-.1 in}
\end{figure}

\begin{enumerate}
	\item See-through Displays

One of the challenges of augmented reality is how to combine the real and virtual environments into a seamless and believable world.\footnotetext{“HoloLens 2—Pricing and Options,” Microsoft, accessed December 5, 2019, \url{https://img-prod-cms-rt-microsoft-com.akamaized.net/cms/api/am/imageFileData/RE2PHcl?ver=a890&q=100&m=6&h=291&w=517&b=\%23FFFFFFFF&l=f&o=t}.} The most intrinsic way to achieve this is to use a lens overlaid with virtual contents to view the environment. This method of augmentation is called a \textbf{see-through display}.\footnote{Schmalstieg and Höllerer, “Displays.”} Consider camera filters in digital photography. These transparent pieces of colored glass correct white balance and filter out unwanted color while preserving the overall image. The idea of a see-through display is also to ensure the integrity of the real environment while enhancing it with other contents. Two types of see-through displays are \textbf{optical see-through display} and \textbf{video see-through display}. Optical see-through displays describe the use of an optical element for the transmission and reflection of real and virtual imagery respectively.\footnote{Schmalstieg and Höllerer.} Video see-through displays use a camera to capture images of the environment and add a computer-generated component on top. The final digitally enhanced imagery is rendered to a viewing screen.

\begin{figure}[!ht]
\rightline{
\begin{minipage}{\textwidth}
\begin{center}
\woopic{optical}{2.0}
\vspace{-.2 in}
\caption[Optical see-through display]{How real-world imagery is transmitted to the eye using an optical see-through display. (Illustration by Schmalstieg and Höllerer. In \textit{Augmented Reality: Principles and Practice}.)}\label{optical}
\end{center}
\end{minipage}
}
\vspace{-.1 in}
\end{figure}

Figure \ref{optical} describes the mechanism of an optical see-through display. The optical combiners transmit real-world images to the human eye while simultaneously reflecting virtual imagery. Both external light and the computer-generated image travel to the eye through the same optical device, creating the illusion of an enhanced reality. Optical combiners use several technologies to achieve this effect. The most common technique in the current wearable AR market is waveguide based, used by the Microsoft Hololens (Figure \ref{hololens}) and Magic Leap One.\footnote{“HoloLens 2—Overview, Features, and Specs,” Microsoft, accessed December 5, 2019, \url{https://www.microsoft.com/en-us/hololens/hardware}; Magic Leap, “Magic Leap One Creator Edition,” Magic Leap, accessed December 5, 2019, \url{https://www.magicleap.com/magic-leap-one}.} Waveguide displays combine virtual images and transport both external and virtual light through a tube so that the reflection out of the other end of the tube is completely preserved. Waveguide technology is popular for near-eye optical see-through displays because it allows imaging optics and display to be moved from the eye’s field of vision to the temples or the forehead, creating a wider field of view.\footnote{Lauren Bedal, “Designing for the Human Body in XR,” Virtual Reality Pop, November 16, 2017, \url{https://virtualrealitypop.com/designing-for-the-human-body-in-xr-e9ac88931e45}.}

\begin{figure}[!ht]
\rightline{
\begin{minipage}{\textwidth}
\begin{center}
\woopic{video}{2.0}
\vspace{-.2 in}
\caption[Video see-through display]{How real-world imagery is transmitted to the eye using a video see-through display. (Illustration by Schmalstieg and Höllerer. In \textit{Augmented Reality: Principles and Practice})}\label{video}
\end{center}
\end{minipage}
}
\vspace{-.1 in}
\end{figure}

Handheld AR devices and some other head-mounted displays are video see-through. Video see-through displays block the real world from the users’ view; the real environment is captured by a video camera and presented to the user on a screen. In Figure~\ref{video}, the digital combiner combines the video signal of the external world and the video signal from the computer graphics system to produce augmented images that are displayed using a monitor.\footnote{Alan B. Craig, “Augmented Reality Hardware,” in \textit{Understanding Augmented Reality: Concepts and Applications} (Waltham, MA: Morgan Kaufmann, 2013).}
	
For both types of see-through displays, the generated virtual content can be monoscopic (single-eye content) or stereoscopic (creating illusion of depth using the same image with a slight angular difference for each eye). Mobile AR is usually monoscopic, as the content is viewed on a 2D screen. AR with head-mounted displays can be stereoscopic, where the display for each eye shows a different angle of the same scene to create a sense of depth.

\item{Spatial Augmented Reality}

\begin{figure}[!ht]
\rightline{
\begin{minipage}{\textwidth}
\begin{center}
\woopic{lookingglass}{.8}
\vspace{-.2 in}
\caption[The Looking Glass]{A volumetric video recording shown on the Looking Glass (Video sequence by Looking Glass Factory.\footnotemark)}\label{lookingglass}
\end{center}
\end{minipage}
}
\vspace{-.1 in}
\end{figure}

Spatial AR, also known as projection-based AR, makes use of projection to display the digitally created content.\footnotetext{Looking Class Factory, “Introducing The Looking Glass: A New, Interactive Holographic Display,” \textit{Looking Glass Factory Blog} (blog), July 24, 2018, \url{https://blog.lookingglassfactory.com/content/images/2019/09/1_NYk5gfgwuA5ZJb3X1BB72g.gif}}. The AR system projects a special kind of light onto a projection surface, which could be real-world objects; the combination of virtual and real information, in this case, takes place in the physical world.\footnote{Schmalstieg and Höllerer, “Displays.”} Holograms are an example of a spatial AR display. The Looking Glass in Figure~\ref{lookingglass} is an interactive light-field volumetric display showing a sequence of moves by a Tai Chi master. The Looking Glass optics use 45 unique views of the 3D content to create a superstereoscopic, full-color scene.\footnote{Looking Class Factory} Head-up displays for cars (Figure~\ref{hud}) are another successful application of spatial AR.\footnote{“HUDWAY Drive,” HUDWAY, accessed December 5, 2019, \url{https://hudway.co/drive}.} The system projects vital information about the car onto the inside of the windshield, allowing the driver to view this information without requiring them to take their eyes off the road.

\begin{figure}[!ht]
\rightline{
\begin{minipage}{\textwidth}
\begin{center}
\woopic{hud}{.25}
\vspace{-.2 in}
\caption[HUDWAY Drive]{A map projected by the head-up display HUDWAY Drive through a car's windshield. (Photograph by HUDWAY.\footnotemark)}\label{hud}
\end{center}
\end{minipage}
}
\vspace{-.1 in}
\end{figure}

\item Non-visual AR Displays\footnotetext{“HUDWAY Drive,” HUDWAY, accessed December 5, 2019, \url{https://hudway.co/assets/images/tild3435-3265-4135-b866-383933393636__untitled-8.jpg}}\newline
AR displays do not have to be visual; displays that target other sensory modalities such as smell, sound, taste to enhance an experience with virtual stimuli are also considered augmented reality.\footnote{Schmalstieg and Höllerer, “Displays.”} Since humans perceive the environment using multiple senses, AR products should also be multimodal to cater to these senses. Currently, audio AR is the most common display, but tangible, tactile, and haptic AR have also gained traction among researchers recently as more products seek to integrate these modes into the conventional visual display.
\end{enumerate}
\clearpage

\subsection{Types of Devices}

\begin{figure}[!ht]
\rightline{
\begin{minipage}{\textwidth}
\begin{center}
\woopic{eyedistance}{1.1}
\vspace{-.2 in}
\caption[Eye Distance AR Classification]{Classification of AR displays by distance from eye to display. (Illustration by Schmalstieg and Höllerer. In \textit{Augmented Reality: Principles and Practice})}\label{eyedistance}
\end{center}
\end{minipage}
}
\vspace{-.1 in}
\end{figure}

Augmented reality can also be classified by the type of devices. The previous section has briefly mentioned some of these devices: mobile handheld devices, head-mounted displays and heads-up displays. These devices can be broadly categorized into two classes: wearable and non-wearable devices. \textbf{Wearable devices} are usually glasses or headsets (head-mounted displays), but can also include helmets and contact lenses. \textbf{Non-wearable devices} for AR on the market today feature mobile devices, stationary devices (desktops, televisions, etc.) and projected displays (holographic displays, heads-up displays). Another way to categorize AR devices is by order of distance from the eye. Figure~\ref{eyedistance} exemplifies the ranking of the above-mentioned displays in terms of the space they occupy: head space, body space, or world space.\footnote{Schmalstieg and Höllerer.}

\subsection{Types of Tracking Technology}\label{trackingtech}
In addition to classifying AR by types of hardware (like displays and devices), augmented reality can also be differentiated by software implementation. In terms of tracking technologies, there are two main types of augmented reality: marker-based AR and markerless AR. A key requirement of AR systems is real-time computer vision to perform instantaneous tracking and registration (alignment of objects in the device coordinate system). \textbf{Marker-based AR} relies on the use of predefined signs or images that are easily detectable with image processing, pattern recognition, and other computer vision techniques.\footnote{Sanni Siltanen, \textit{Theory and Applications of Marker-Based Augmented Reality} (Espoo, Finland: VTT, 2012), 39.} In marker-based tracking, the pose and scale of real-world objects are calculated relative to the position and orientation of the markers. An alternative to using artificial markers is markerless AR, which incorporates natural feature detection and other hybrid tracking methods (combining multiple techniques). Location-based AR is another branch of \textbf{markerless AR}. Markerless trackers remove the hindrance of markers and allow the AR system to make use of objects in the scene as tracking anchors. Natural feature detection is the most common method in markerless tracking. There is some overlap between marker-based AR and markerless AR when it comes to image detection. Markerless detection techniques that employ natural feature identification algorithms share some similarities with marker-based tracking algorithms. The following sections include an in-depth discussion of natural feature tracking in image detection.

\section{AR Pipeline}
A simple augmented reality system typically requires three components: a camera, a computational unit and a display. The pipeline (in Figure~\ref{ar_pipeline}) follows a process of capturing, tracking and rendering.\footnote{Siltanen, 19–20.} An AR system builds an understanding of the environment by capturing images and using mapping algorithms to generate topological maps of the physical space. The system tracks the device’s relative position in the environment by determining the six degrees of freedom (6DOF) position of the camera, also known as the pose (see Figure~\ref{6dof}). Using these positional and environmental data, developers can create virtual content that overlays and/or interacts with the physical space. These computer-generated elements are rendered to a display, which could be a see-through device or a computer screen. The following section only discusses briefly capturing and rendering, focusing instead on tracking concepts and techniques.

\begin{figure}[!ht]
\rightline{
\begin{minipage}{\textwidth}
\begin{center}
\woopic{ar_pipeline.png}{.7}
\vspace{-.2 in}
\caption[Simple AR pipeline]{Simple AR pipeline. (Adapted from Siltanen, \textit{Theory and Applications of Marker-Based Augmented Reality}, 20.)}\label{ar_pipeline}
\end{center}
\end{minipage}
}
\end{figure}

\begin{figure}[!ht]
\rightline{
\begin{minipage}{\textwidth}
\begin{center}
\woopic{6dof}{.6}
\vspace{-.2 in}
\caption[6DOF]{Six degrees of freedom refers to the freedom of movement by an object in 3D spaces (translation and rotation about the x, y, and z axes). (Illustration by Ola Podgórska. Pinterest.\footnotemark)}\label{6dof}
\end{center}
\end{minipage}
}
\vspace{-.1 in}
\end{figure}

\subsection{Tracking \& Pose Determination: Simultaneous Localization and Mapping}
The first two steps in the AR pipeline, capturing and tracking, involve solving the problem of \textbf{simultaneous localization and mapping}.\footnotetext{Ola Podgórska, “Six Degrees of Freedom,” Pinterest. \url{https://i.pinimg.com/236x/41/08/0b/41080b26d9f8cdcc69b830d719721534--degrees-of-freedom-octopus.jpg}.} When a user turns on an AR device, the system has no prior (a priori) knowledge of the environment surrounding the device or its position in this environment. The AR system has to construct its internal version of the environment while at the same time estimate the device's pose in this internal map. This problem is known as simultaneous localization and mapping (SLAM). SLAM is the computational problem of building the map of the environment and simultaneously computing the device’s pose in this map.\footnote{H. Durrant-Whyte and T. Bailey, “Simultaneous Localization and Mapping: Part I,” \textit{IEEE Robotics Automation Magazine} 13, no. 2 (June 2006): 99–110.} SLAM is solved using methods of mapping, sensing and modeling. Most AR software requires some sort of initial configuration. Initial scanning of the environment enables the system to map the physical space and deduce its position before any rendering happens. The actual implementation of tracking and pose determination in SLAM is described below.

Tracking is one of the most important stages in the Augmented Reality pipeline. \textbf{Tracking} refers to the dynamic capturing and measuring of the physical environment by AR systems using tracking devices and sensors.\footnote{Dieter Schmalstieg and Tobias Höllerer, “Tracking,” in \textit{Augmented Reality: Principles and Practice} (Boston: Addison-Wesley, 2016).} Tracking enables reconstruction of the real world, specifically by determining the pose of tracked objects. As discussed in Section~\ref{trackingtech}, there are two main sets of tracking algorithms: marker-based and markerless. Marker-based tracking algorithms use the pose of the marker (relative to the viewer) to deduce the position and orientation of other objects in the environment. Some markerless methods also rely on the detection of natural markers to perform immediate registration of the scene, but other techniques can also be used to register the environment's layout.

The technology that enables pose determination in AR systems is \textbf{visual inertial odometry}. In computer vision and robotics, visual odometry is the process of determining the pose of the device by analyzing its camera images. Visual inertial odometry is a visual odometry system that applies camera image analysis and sensor fusion of inertial measurement units (IMU) to track acceleration and rotation. IMU sensors include the accelerometer (measures movements along the three axes) or the gyroscope (measures rotation about the three axes). These sensors enable the device to estimate the 6DOF pose of the camera. The measurement is taken using a mechanism called a proof mass. Each sensor has a proof mass which changes from a neutral position under external influences such as acceleration or rotation  This change is used to quantify the movement or the rotation made by the camera.\footnote{Steve Aukstakalnis, “Sensors for Tracking Position, Orientation, And Motion,” in \textit{Practical Augmented Reality: A Guide to the Technologies, Applications, and Human Factors for AR and VR} (Boston: Addison-Wesley Professional, 2016).} This information is used in conjunction with computer vision analysis of images captured by the device’s camera to estimate its relative pose. Platforms like ARCore align the pose of the virtual camera and the device's camera to render virtual elements to their correct positions. The following section includes an in-depth discussion of marker detection algorithms and pose calculation based on these techniques.

\section{AR Tracking System}
\subsection{Marker Detection}
Every visual AR system requires at least one camera. Images captured by the camera(s) undergo a process of marker/feature detection to extract the pose of the camera. The process of image formation (representing a 3D scene on a 2D image) requires switching from the world coordinate system to the image coordinate system, or in other words converting from a physical space to a flat image space. Computer graphics use the pinhole camera model to perform this translation. A pinhole camera is a box that has a hole on one side that only takes in a single ray of light, which is captured by a film placed on the other side of the box. The pinhole camera model is the ideal model for quickly calculating the 2D projection of a 3D object. However, digital cameras cannot fully simulate the pinhole model because of constraints with focal length, depth of field and field of view. Therefore, image formation by conventional cameras requires additional considerations.

The fluid nature of augmented reality demands AR systems to be able to seamlessly switch between world space and image space, which means rapid mapping of world coordinates to image coordinates and vice versa. This requirement is true for both marker-based and markerless systems. In marker-based tracking, markers are objects or images that allow the AR system to quickly deduce the positions of the camera and other objects in the scene relative to the markers. Marker detection provides the image coordinates of the marker, which are then mapped back to their world coordinates (actual position in the physical world). By acquiring the marker’s position in both image and world coordinates, the system can then calculate the camera pose, which is applied to other objects in the image to find their positions in the real world. The matrix transformation $M$ that maps the world coordinates of a point to its image coordinates is called the \textbf{perspective projection} and is defined in Equation~\ref{perspectiveprojection}.\footnote{Dieter Schmalstieg and Tobias Höllerer, “Computer Vision for Augmented Reality,” in \textit{Augmented Reality: Principles and Practice} (Boston: Addison-Wesley, 2016)}
\begin{equation}\label{perspectiveprojection}
M = K [R | t]
\end{equation}

Matrix $M$ is the result of the multiplication of the camera calibration matrix $K$ and the concatenation of rotation matrix $R$ and translation vector $t$. Matrix $K$ contains information about the camera's focal length and other offsets that affect its image formation. Rotation matrix $R$ and translation vector $t$ describe the orientation and position of the camera respectively. The image coordinates $x$ of an arbitrary point with world coordinates $X$ is calculated by Equation~\ref{mapworldimage}\footnote{Siltanen, \textit{Theory and Applications of Marker-Based Augmented Reality}, 49.}
\begin{equation}\label{mapworldimage}\
x = MX
\end{equation}
From Equation~\ref{perspectiveprojection}, it is clear that the position of the projected image of a point depends on both the internal geometric properties of a camera and its external positioning and orientation. Assuming that matrix $K$ is constant for each camera, the perspective projection $M$ needs to be updated every single frame to account for any external changes to the camera's pose. With Equation~\ref{mapworldimage}, $M$ can easily be calculated when world coordinates $X$ and image coordinates $x$ are known. In marker-based systems, finding image coordinates requires the use of marker detection.

\subsubsection{Artificial Marker Detection}

Markers are essential for tracking because there is not enough information to deduce scale and pose at the beginning. Many AR systems depend on the use of markers to perform SLAM. \textbf{Artificial markers} are the most efficient and inexpensive solution to pose estimation. A marker is a unique image that is easily detectable using computer vision and image processing techniques. Markers are often black and white because detection methods recognize differences in brightness better than differences in color, and black and white images provide the best contrast in brightness.\footnote{Siltanen, 39.} After the marker is detected, only four points are necessary for calculating the pose of the camera relative to the marker. The four-point rule is the reason why most markers, including QR codes, are in the shape of a square.

The AR marker system pipeline (Figure~\ref{markerbased}) has four main stages:
\begin{enumerate}
	\item Capturing the image
	\item Tracking
	\begin{enumerate}
		\item Detecting the marker
		\item Calculating the pose of the camera
	\end{enumerate}
	\item Rendering
\end{enumerate}

\begin{figure}[!ht]
\rightline{
\begin{minipage}{\textwidth}
\begin{center}
\woopic{markerbased.png}{0.7}
\vspace{-.1 in}
\caption[Marker-based system pipeline]{Marker-based system pipeline. (Illustration adapted from Schmalstieg and Höllerer, “Tracking.”)}\label{markerbased}
\end{center}
\end{minipage}
}
\end{figure}

After the system has successfully acquired an image from the camera feed, the processing function applies a threshold operation and performs edge detection and quadrilateral fitting on the resulting binary image.\footnote{Siltanen, \textit{Theory and Applications of Marker-Based Augmented Reality}, 41–43.} Thresholded images only have two colors, black and white, separating the background from the objects. Each object is a closed contour. Scanline examination can be performed to detect the quadrilateral marker's edges; however to optimize performance, quadrilateral fitting is performed to check if an object is the marker. Figure \ref{quadrilateralfitting} demonstrates the process of searching for furthest points from edges and diagonals to detect the square. The fitting algorithm first picks an arbitrary point $a$ on the contour, then traces the entire contour to find a point with the greatest distance to $a$ (labeled $p_1$). $p_1$ becomes the first corner of the quadrilateral. Next, the centroid $m$ of the contour is calculated. It is clear that two of the remaining three corners must be on opposite sides of the diagonal through $p_1$ and $m$ and are the furthest points from the diagonal from each side. These two corners are denoted as $p_2$ and $p_3$. Finally, the half plane formed by the line through $p_2$ and $p_3$ that does not contain $p_1$ has to contain $p_4$, which also turns out to be the furthest point from the $p_2$ and $p_3$ diagonal. This method of finding the furthest point from each edge is repeated to ensure that no new corners exist.

\begin{figure}[!ht]
\rightline{
\begin{minipage}{\textwidth}
\begin{center}
\woopic{quadrilateralfitting}{1.3}
\vspace{-.2 in}
\caption{Quadrilateral fitting. (Illustration by Schmalstieg and Höllerer. In \textit{Augmented Reality: Principles and Practice}.)}\label{quadrilateralfitting}
\end{center}
\end{minipage}
}
\vspace{-.1 in}
\end{figure}

After finding the image coordinates of the four corner points, the system can now calculate linear transformation matrix $M$. Assume the world coordinates of these points are $(0, 0, 0)$, $(1, 0, 0)$, $(1, 1, 0)$, and $(0, 1, 0)$. Finding the matrix is to find the relationship between these coordinates and the image coordinates gained from marker detection. To solve the relationship, AR systems commonly use \textbf{direct linear transformation} (Equation~\ref{dlt}), a method used to solve the linear transformation $A$ from known vectors $x_k$ and $y_k$.\footnote{Siltanen, 52; Schmalstieg and Höllerer, “Computer Vision for Augmented Reality.”}

\begin{equation}\label{dlt}
x_k \propto Ay_k \text{ for } k = 1,...,N
\end{equation}
$\propto$ denotes equality for an unknown scalar multiplication. To solve for $M$, replace $x_k$ and $y_k$ with the known image and world coordinates of the marker's corners. Recall from Equation~\ref{perspectiveprojection} that $M$ is the matrix product of the calibration matrix $K$ and the pose matrix $[R|t]$. Since $K$ is constant, the pose matrix, which provides the rotation and translation of the camera, can be recovered from $M$. Since point correspondence can be imperfect, further iterative error minimization is performed to refine the pose.\footnote{Schmalstieg and Höllerer, “Computer Vision for Augmented Reality.”}

\subsubsection{Natural Feature Detection}
Using a black and white square marker is efficient but its obstructive behavior is not always desirable. In terms of enhancing immersion, AR systems can be more intuitive if pose estimation can be performed on natural markers, markers that are part of the physical environment. One approach is to use \textbf{natural feature detection}. Natural feature tracking relies on sparse matching, the problem of finding the correspondences between a number of interest points in 2D images and their real 3D locations. In tracking by natural feature detection, the camera pose is recalculated for every frame, meaning that in every single frame interest points are repeatedly detected and matched.\footnote{Schmalstieg and Höllerer.} This behavior means that occlusions (accidentally covering the camera, extreme changes in lighting) in previous frames do not affect tracking performance in later frames. Another advantage of interest point tracking is compactness since tracking models only have to match a select number of points and require no memory to store information from prior frames.

Figure~\ref{naturalfeature} breaks down the steps in the tracking phase for markerless AR systems using natural feature detection.

\begin{figure}[!ht]
\rightline{
\begin{minipage}{\textwidth}
\begin{center}
\woopic{natural_feature_pipeline}{1.0}
\vspace{-.4 in}
\caption[Natural feature tracking pipeline]{Natural feature tracking pipeline. (Illustration adapted from Schmalstieg and Höllerer, \textit{Augmented Reality: Principles and Practice}.)}\label{naturalfeature}
\end{center}
\end{minipage}
\vspace{-.1 in}
}
\end{figure}
First the system detects interest points in the image from the camera feed and creates a descriptor to represent each detected point. These descriptors are matched with predefined descriptors to determine if the points they describe are part of the target images. Lastly, the camera pose is estimated using the Perspective-n-Point algorithm.

An interest point (also known as feature point or key point) is a clearly defined area in an image that is visually distinct.\footnote{Schmalstieg and Höllerer; Siltanen, \textit{Theory and Applications of Marker-Based Augmented Reality}, 94.} Interest points are well textured and exhibit considerable changes in intensity compared to their surroundings. Interest point selection algorithms must ensure constant performance across different lighting conditions and from various perspectives, or in other words, the algorithms must be able to select the same interest points every time regardless of external variables. Different methods can be used to select different kinds of features. Special interest points include edges, corners, blobs, and patches.\footnote{Siltanen, \textit{Theory and Applications of Marker-Based Augmented Reality}, 96.} The following discussion considers two corner detection algorithms: Harris detector and Features from Accelerated Segment Test (FAST).

\begin{figure}[!ht]
\rightline{
\begin{minipage}{\textwidth}
\begin{center}
\woopic{harris_corner}{0.35}
\vspace{-.2 in}
\caption[Harris corner detector]{Changes in intensity in Harris corner detector. (Illustration by Robert Collins. In "Lecture 06: Harris Corner Detector," 2007.\footnotemark)}\label{harris}
\end{center}
\end{minipage}
}
\vspace{-.1 in}
\end{figure}
Harris detector recognizes corners by analyzing changes in intensity.\footnotetext{Robert Collins, “Lecture 06: Harris Corner Detector” (University Park, Pennsylvania, 2007), \url{http://www.cse.psu.edu/~rtc12/CSE486/lecture06.pdf}.} Figure~\ref{harris} shows how autocorrelation can be used to detect interest points. This algorithm detects two types of interest points: corners and edges. Consider the pixel marked by the pink window. Let $I(a, b)$ be the function for measuring the intensity of pixel $(x,y)$. In the first case, because the pixel's intensity does not change when shifting the pink window in any direction, the bounded region is therefore flat and does not contain an interest point. In the second case, when the window contains an edge, moving the window along the edge's direction does not yield different intensities. In the last case, however, moving the window in any direction would produce considerable changes in intensity. The sum of squared differences $E(u, v)$ of an image patch $(x, y)$ in window $W$ when shifting $u$ units in the $x$-axis and $v$ units in the $y$-axis is described by Equation~\ref{intensitychange}:\footnote{Collins.}
\begin{equation}\label{intensitychange}
E(u,v) = \sum\limits_{(x, y) \in W} [I(x + u, y + v) - I(x, y)]^2
\end{equation}
By calculating the sum of squared differences, the algorithm compares two patches of the image and assigns a dissimilarity score; the higher the score, the more dissimilar the two patches. Threshold values are specified for edges, corners, and flat regions. The values of $E(u,v)$ are large for corner points and approach $0$ for flat regions.

Since the introduction of the first feature detection algorithms like the Harris detector, many other methods have emerged with increased robustness and computational efficiency. Performance is especially important for technologies such as AR, which require real-time video processing. The Features from Accelerated Segment Test (FAST) corner detector is among some of the more computationally efficient algorithms often used in processing real-time camera feed. FAST detects corners by performing a segment test. The test determines if a specific pixel is an interest point by examining the pixels surrounding it. A pixel is deemed a corner if there are more than $n$ (commonly $n = 12$) contiguous pixels surrounding it that are brighter or darker than the pixel considered. In Figure~\ref{fast}, a pixel is chosen for the segment test, labeled $p$. A Bresenham circle of 16 pixels is drawn around $p$ using the midpoint circle algorithm. The intensity values of the pixels on the circle are tested against $p$'s intensity. First $p$ is compared to pixels 1, 5, 9, and 13.  If at least three of these four intensity values are all greater or all less than $p$'s value, the remaining pixels will be examined. If there is a strip of 12 continuous pixels that satisfy the criteria (all brighter or all darker than $p$), $p$ is a corner pixel.\footnote{Deepanshu Tyagi, “Introduction to FAST (Features from Accelerated Segment Test),” Medium, January 7, 2020, \url{https://medium.com/analytics-vidhya/introduction-to-fast-features-from-accelerated-segment-test-4ed33dde6d65}.}

\begin{figure}[!ht]
\rightline{
\begin{minipage}{\textwidth}
\begin{center}
\woopic{fast}{.85}
\vspace{-.2 in}
\caption[FAST's segment test]{FAST's segment test. (Illustration by Tyagi.\footnotemark)}\label{fast}
\end{center}
\end{minipage}
}
\end{figure}

An implementation of the segment test is described by the pseudocode in Listing~\ref{detect}.\footnotetext{Tyagi, \url{https://miro.medium.com/max/443/0*iB26EPO33F4LSig3.jpg}.} The function \texttt{detect} takes an image \texttt{image} and a threshold value \texttt{threshold} as its parameters. The threshold value determines how much its intensity needs to vary in order for it to be considered brighter or darker than another pixel. For example, if the intensity value of pixel $p$ is $I_p$, then for pixel $p'$ to be brighter than $p$, $I_{p'}$ has to be greater than $I_p + threshold$. The data type of \texttt{image} is not an actual representation of the image, but a two-dimensional matrix containing the intensity values of each pixel in the image. For example, for an image of size $2000 \times 1000$ pixels, the argument passed to parameter \texttt{image} is a $2000 \times 1000$ matrix. In the pseudocode, \texttt{image} is represented by a list, where \texttt{image[x][y]} accesses the intensity of the pixel at row $x$ and column $y$.

\vspace{-.2 in}
\begin{singlespace}
\begin{lstlisting}[mathescape, caption= Find corners in image, label=detect]
def detect(image: list, threshold: float) -> list:
    corners = []
    rows = len(image)
    cols = len(image[0])
    for x in range(4, rows - 4):
        for y in range(4, cols - 4):
            i_max = image[x][y] + threshold
            i_min = image[x][y] - threshold
            circle = []
            # Fill the list circle with the intensity values of the pixels
            # chosen by the Bresenham circle algorithm
            feature = is_corner(circle, i_max, i_min)
            if feature:
                corners.append((x, y))
    return corners
\end{lstlisting}
\end{singlespace}

First, the algorithm starts at pixel $(4,4)$ and loops through every single pixel in the image. Pixels on the first and last four rows and columns are excluded because a Bresenham circle cannot be drawn for these border pixels. For each pixel, the maximum and minimum intensity thresholds are calculated. Then a circle is drawn around the current pixel using the Bresenham's circle drawing algorithm (refer to Figure~\ref{fast} for an example of the circle). Lines 10 and 11 in Listing~\ref{detect} assume that the Bresenham circle has been drawn and the intensity values of the chosen pixels are appended to the list \texttt{circle}. Then the algorithm calls the \texttt{is\_corner} function (Listing~\ref{iscorner}) to check if the pixel is a corner or not. If the pixel is a corner,  its indices $(x, y)$ will appended as a tuple to the list \texttt{corners}. This list stores the indices of all the corners, or interest points, in the image and is returned at the end.
\newpage

\vspace{-.2 in}
\begin{singlespace}
\begin{lstlisting}[mathescape, caption= Check for corner at pixel with given Bresenham circle, label=iscorner]
def is_corner(circle: list, i_max: float, i_min: float) -> bool:
    brighter = 0 # counter of brighter pixels
    darker = 0 # counter of darker pixels
    for i in range(4):
        if circle[i * 4] > i_max:
            brighter += 1
        if circle[i * 4] < i_min:
            darker -= 1
    results = False
    counter = 0
    first_strip = 0
    is_first = True
    if brighter >= 3:
        for i in range(16):    
            # Replace with circle[i] < min for testing darker intensities
            if circle[i] > i_max: 
                counter += 1
                if counter == 12:
                    return True
            else:
                if is_first:
                    is_first = False
                    first_strip = counter
                counter = 0
        if counter + first_strip >= 12:
            return True
    if darker >= 3:
        '''
        Repeat the previous loop. 
        Replace the condition with circle[i] < min.
        '''
    return False
\end{lstlisting}
\end{singlespace}

The segment test described in \texttt{detect} calls \texttt{is\_corner} (pseudocode in Listing~\ref{iscorner}) to determine whether a pixel is a corner or not. \texttt{is\_corner} takes the list of the intensity values of 16 pixels in the Bresenham circle as its parameters.  The maximum and minimum threshold values \texttt{i\_max} and \texttt{i\_min} are also passed to the function. Recall from Figure~\ref{fast} that initially only pixels with indices 1, 5, 9, and 13 are examined. The counters \texttt{brighter} and \texttt{darker} keep track of how many of these four pixels are brighter or darker than the thresholds. If there are more than three that all hold the same criterion (all brighter or darker) then the function will check if there are at least 12 continuous pixels on the circle that also hold this criterion. This process is done by keeping a counter and incrementing it every time the pixel passes the check. The counter is reset to 0 if the test fails or returns \texttt{True} if the counter reaches 12. Because a circle has no beginning or end, there might be a contingency of pixels that pass the test at the start of the loop and others that pass the test at the end of the loop. These also count as one continuous strip. For instance, pixels 1 through 5 and 10 through 16 are all brighter. However, between 5 and 10, the counter might have been reset. To account for the continuity of the first pixels examined and the last ones, some extra variables are used to store this information. Boolean variable \texttt{is\_first} is \texttt{True} only when the loop iterates through the first continuous strip found. \texttt{first\_strip} records the number of continuous pixels that pass the test at the beginning, which is added to the counter at the end of the loop to get the length of this entire strip.

There are some limitations to this method. The order in which the pixels are examined affects the performance of the algorithm. A machine learning approach seeks to optimize this drawback using decision trees. First, the 16 pixels around the pixel in question are divided into three subsets: darker, similar, or brighter, represented by $P_d$, $P_s$, and $P_b$ respectively. Let $x$ be a pixel from one of the 16 in the Bresenham circle and $p$ the center pixel. We have
\begin{equation}
x \in \begin{cases}
	P_d, & \text{if $I_x \leq I_p - threshold$,} \\
	P_s, & \text{if $I_p - threshold < I_x < I_p + threshold$,} \\
	P_b, & \text{if $I_x \geq I_p + threshold$,}
	\end{cases}
\end{equation}
where $I_x$ is the intensity of $x$ and $I_p$ is the intensity of $p$. To create the decision tree, the classifier algorithm is recursively applied to each subset to find the pixel $x$ that provides the most information about whether the center pixel $p$ is a corner. This decision tree determines the sequence of pixels to examine in order to achieve faster detection.\footnote{Alexander Mordvintsev and Abid K., “FAST Algorithm for Corner Detection,” OpenCV-Python Tutorials, 2013, \url{https://opencv-python-tutroals.readthedocs.io/en/latest/py_tutorials/py_feature2d/py_fast/py_fast.html}.}
%to do: clarify if intensity values are assigned from 0-100

Another constraint of FAST is that it can end up detecting multiple corners for the same edges. Better variations of FAST use the non-maximum suppression technique to remove unwanted noise around corners. To achieve this, for each detected corner, a score $V$ is calculated by finding the sum of absolute difference between the intensity of the corner pixel and the other 16 pixels in the Bresenham circle.\footnote{Mordvintsev and K.} When two adjacent corners are found, whichever has the higher $V$ is chosen as the corner.

\section{AR Development Platforms}
Development platforms for AR have been expanding with the evolution of detection algorithms in computer vision and computer graphics. Currently, the main AR software development platforms are ARCore, ARKit, and Vuforia. All of these platforms provide support for the Unity game engine.

\subsection{ARCore} 
ARCore is a platform by Google for AR mobile development. ARCore provides three main functionalities: motion tracking, environmental understanding, and light estimation.\footnote{“Fundamental Concepts | ARCore,” Google Developers, accessed October 17, 2019, \url{https://developers.google.com/ar/discover/concepts}.} Tracking is also done using the same principles as SLAM, by identifying point clusters and using inertial sensors to estimate the camera’s pose. A point cluster is a set of oriented interest points. Each interest point returns a directional vector that can be used for designing interactions between virtual and physical content.\footnote{\textit{Unite Berlin 2018 - Getting Started with Handheld AR} (Berlin, 2018), \url{https://www.youtube.com/watch?v=MqA0XhfKIE0}.} The construction of point clusters can be used to define planar surfaces and deduce their angles.\footnote{“Fundamental Concepts | ARCore.”} Figure~\ref{pointcloud} demonstrates a net of point cluster found from a table surface; virtual Andy Androids are placed on this surface. ARCore provides multiplayer experiences by using Cloud Anchors, which sync selected anchors to Google Cloud, making them accessible to other devices. This feature allows multiple users to view the same AR scene simultaneously. ARCore also supports instant preview of real-time modifications on apps that are currently running on Android device(s).

\begin{figure}[!ht]
\rightline{
\begin{minipage}{\textwidth}
\begin{center}
\woopic{pointcloud.png}{0.3}
\vspace{-.2 in}
\caption[ARCore point cluster]{The point cluster is used in this ARCore app to track a table surface. (Illustration by Google Developers.\footnotemark)}\label{pointcloud}
\end{center}
\end{minipage}
}
\vspace{-.1 in}
\end{figure}

\subsection{ARKit}
ARKit is Apple's AR development platform for iOS devices.\footnotetext{“Explore the HelloAR Sample App Code | ARCore,” Google Developers, accessed March 9, 2020, \url{https://developers.google.com/ar/images/hello-ar.png}.} ARKit 3 introduces people and object occlusion, enabled by motion capture and facial tracking. A child framework, RealityKit, allows for better simulation and rendering with additional functionality that includes audio AR, animation, real-time response to user input, and cross-device multiplayer experiences.\footnote{“Augmented Reality,” Apple Developer, 2020, \url{https://developer.apple.com/augmented-reality/}.} ARKit comes with a companion app (Reality Composer) with a drag-and-drop interface for AR prototyping and development.

\subsection{Vuforia}
Vuforia is a software development kit for AR mobile devices. Acquired by PTC from Qualcomm in 2015, Vuforia is one of the most widely used platforms for AR development.\footnote{“PTC Acquires Vuforia,” PTC, November 3, 2015, \url{https://www.ptc.com/en/about/vuforia}.} Vuforia provides cross-platform support for Android, iOS, and Universal Windows Platform, as well as the Unity game engine. Apps developed with Vuforia can be deployed on various phones, tablets, and other eyewear devices such as the HoloLens and Vuzix M300/M400. Vuforia's main feature is its ability to track an array of targets, including models, images, objects, and a hybrid of these targets. The latest release includes support for tracking horizontal planes.\footnote{“Vuforia Engine Features,” Vuforia Developer Library, accessed March 9, 2020, \url{https://library.vuforia.com/content/vuforia-library/en/features/overview.html.}}

\subsection{Unity}
Unity is a real-time development platform by Unity Technologies. As a game engine, Unity provides support for a range of platforms, including major native platforms such as Android, iOS, Windows, as well as all the aforementioned AR platforms. The Unity environment includes both drag-and-drop functionality and a scripting API in C\#. Unity can be used to create 2D and 3D games in addition to VR/AR and other simulation experiences.\footnote{Unity Technologies, “Unity - Manual: XR,” Unity Documentation, accessed March 9, 2020, \url{https://docs.unity3d.com/Manual/XR.html}.} With its evolving graphics features and usability, Unity has grown to become the most popular game engine and platform for AR and VR content.

AR Foundation is the multi-platform support package for AR by Unity. AR Foundation provides a common abstract API that combines the core functionality of ARCore and ARKit. The package provides a collection of Unity scripts that enable high-level functionalities such as surface detection, point clouds, reference points, light estimation, and world tracking.\footnote{Unity Technologies, “About AR Foundation,” Unity Documentation, accessed March 9, 2020, \url{https://docs.unity3d.com/Packages/com.unity.xr.arfoundation\@3.0/manual/index.html}.}

AR hardware and software continue to evolve with the introduction of new mediums, devices, development platforms, and algorithms. The level of engagement and immersion improve with each iteration, while development and design processes are streamlined. The growth of AR is demonstrated by its expansive applications in gaming, business, education, and medicine.

\section{Augmented Reality \& Digital Humanities}\label{dighuman}
In recent years, computer vision technologies such as virtual reality and augmented reality have found more uses in everyday life. Developments in fundamental AR components, such as feature detection, and supporting development frameworks, such as Vuforia, help improve the technology's accessibility and potential in various fields. New advances are letting humans see in ways that were previously unthinkable. X-Ray machines can see where the unaided eye cannot, through multiple layers, even into the human body. Facial recognition algorithms have gained popularity both as a significant technological accomplishment and a security invasion. Ethical applications of these powerful and sometimes invasive technologies require careful considerations over their implications. For augmented reality, who is allowed to see with AR and who is being seen through become questions of great consequence for designers and users of this technology.

On the other hand, technological progress implies tremendous potential but also possible drawbacks. For history and other humanities and social sciences, these advances provide more methods for research and new perspectives, with the focus on data and more thorough analysis techniques. In fact, a new discipline has emerged to study the intersection between technology and these other fields, called digital humanities. Digital humanities is concerned with the use of computational tools to produce new methods for finding patterns, visualization, and analysis.\footnote{Michael J. Kramer, “What Does Digital Humanities Bring to the Table?,” \textit{Michael J. Kramer} (blog), accessed March 9, 2020, \url{http://www.michaeljkramer.net/what-does-digital-humanities-bring-to-the-table/}.} However, as far as the limitations of technology are concerned, digital humanities is also self-critical as a discipline; alongside utilizing technological affordances, scholars also challenge the pitfalls that come with culturally constructed assumptions behind designs. Technology reinforces privileges that discriminate against certain groups of users, and technology sometimes invades the privacy of non-users as well.\footnote{James E. Dobson, “Protocols, Methods, and Workflows: Digital Ways of Reading,” in \textit{Critical Digital Humanities: The Search for a Methodology} (Urbana: University of Illinois Press, 2019), 6–7.}

In history, AR and VR are most commonly found in public history institutions, especially with museums, archives, and libraries. The obsession with seeing is now extended to the past through historical photographs, texts, and now virtual reconstructions enabled by AR/VR platforms. But the possibilities are much broader. Techniques such as optical character recognition allow for the processing of hundreds of thousands of sources simultaneously, multiplying the scope of research.\footnote{Devon Elliott and William J. Turkel, “Faster than the Eye: Using Computer Vision to Explore Sources in the History of Stage Magic,” in \textit{Seeing the Past with Computers: Experiments with Augmented Reality and Computer Vision for History}, ed. Kevin Kee and Timothy Compeau, Digital Humanities (University of Michigan Press, 2019), 83–94; Ian Milligan, “Learning to See the Past at Scale: Exploring Web Archives through Hundreds of Thousands of Images,” in \textit{Seeing the Past with Computers}, ed. Kee and Compeau, 116–36.} By manipulating spaces, AR brings history outside texts and into the world, transcending physical barriers by employing them as platforms for virtual media. Location-based immersive experiences such as Museum of the Hidden City by Walking Cinema make history accessible in the space where it happened. The app presents a multimodal narrative for a walking tour around the Fillmore neighborhood of San Francisco, where the forces of gentrification uprooted marginalized populations through affordable housing.\footnote{“Museum of the Hidden City,” Museum of the Hidden City, accessed March 9, 2020, \url{http://www.seehidden.city}.} Figure~\ref{fillmore} is taken directly from the app and shows an illustration of historical Fillmore superimposed on an image of today's Fillmore from the camera feed. Multimodality introduces new ways of presenting different perspectives, and in the Fillmore example, these perspectives include oral histories, historic images, and physical primary sources (in the form of the buildings themselves). Museum of the Hidden City and England Originals are  examples of digital storytelling; the former uses location-based AR while the latter employs plane detection. England Originals (Figure~\ref{england_originals}) is a relatively successful example of the tabletop model. Users can place the virtual map of English cities on horizontal planes and click on individual buildings for further context.

\begin{figure}[!ht]
\rightline{
\begin{minipage}{\textwidth}
\begin{center}
\woopic{fillmore.png}{0.3}
\vspace{-.2 in}
\caption[Museum of the Hidden City]{A screenshot of the immersive AR experience exploring the history of housing in Fillmore, San Francisco. (Screen capture from \textit{Museum of the Hidden City Trailer. By Walking Cinema, 2019.\footnotemark}).}\label{fillmore}
\end{center}
\end{minipage}
}
\end{figure}

\begin{figure}[!ht]
\rightline{
\begin{minipage}{\textwidth}
\begin{center}
\woopic{englandoriginals}{0.15}
\vspace{-.2 in}
\caption[England Originals]{A table-top AR application for mobile phones about the history of England. (Screen capture of the England Originals app by England's Historic Cities. Screen capture by Thuy Dinh.)}\label{england_originals}
\end{center}
\end{minipage}
}
\end{figure}

This project is concerned with digital storytelling using augmented reality on a smaller scale\footnotetext{Walking Cinema, \textit{Museum of the Hidden City Trailer}, 2019, \url{https://vimeo.com/358878249}}; it presents an analysis of Ho Chi Minh City's history and its memory. Instead of using the location-based method, the app is designed for table tops and vertical planes. It employs feature detection and tracking to provide a localized spatial experience on the Vuforia and Unity platforms. This project takes into considerations questions about the implications of using these computational tools. As the analysis delves into politics and power, it is important that any product from this project also makes transparent the power and politics underlying such technology.

%!TEX root = ../username.tex
\chapter{Developing with Vuforia}\label{vuforia}
Like most AR platforms, Vuforia's functionality revolves around the concepts of marker detection and tracking. The main difference between Vuforia and other prominent AR development platforms is its focus on customized markers, or targets, as the Vuforia documentations call them. Users are allowed to register their own targets on an online database and then download the target database in a package that can be imported into Unity or deployed in Visual Studio, Xcode or other integrated development environments (IDE). This chapter provides an overview of the functionality of the Vuforia framework and introduces the use of Vuforia with Unity. Section~\ref{workflow} discusses the complete workflow of developing an AR mobile application using Vuforia and Unity.

\section{Features}
As previously noted, the prominent feature of Vuforia is its ability to track customized targets. Figure~\ref{targets} lists the targets that can be tracked in a Vuforia-enabled app. In Vuforia terminology, all of these targets are considered \texttt{Trackable}s. Recall the definition of interest points in the previous discussion on feature detection. A \texttt{Trackable} is a set of interest points that can be recognized and tracked at run time. Since the targets are customized, these interest points are preregistered offline and stored in a database. Interest points detected at run time are compared against this database to identify the targets. Once a target is matched, its pose is retrieved and the orientation and position data are used to align virtual contents to the physical targets. These targets are divided into three main categories: images, objects, and environments. Each category is further divided into several smaller subgroups:

\begin{figure}[!ht]
\rightline{
\begin{minipage}{\textwidth}
\begin{center}
\woopic{targets}{0.3}
\vspace{-.2 in}
\caption{Vuforia targets. Source: Vuforia developer portal}\label{targets}
\end{center}
\end{minipage}
}
\end{figure}

\begin{enumerate}
\item Image targets

An image target is a flat image that Vuforia can detect and track. The engine uses natural feature detection to select the interest points from the camera feed and match them to a known database of targets. Vuforia is capable of tracking a detected image as long as it is partially visible to the camera. Usually, virtual contents are attached to the image target (see Figure~\ref{teapot} for an example of a virtual teapot attached to an image target). Its most common use cases are for augmenting product packaging and posters. (Vuforia)

\begin{figure}[!ht]
\rightline{
\begin{minipage}{\textwidth}
\begin{center}
\woopic{teapot}{0.5}
\vspace{-.2 in}
\caption[Image target]{A detected image target with a virtual teapot attached. Source: Vuforia developer portal}\label{teapot}
\end{center}
\end{minipage}
}
\end{figure}

\begin{figure}[!ht]
\rightline{
\begin{minipage}{\textwidth}
\begin{center}
\woopic{cuboid}{0.5}
\vspace{-.2 in}
\caption[Multi target]{A box can serve as a multi target. Source: Vuforia developer portal}\label{cuboid}
\end{center}
\end{minipage}
}
\end{figure}

\item Multi targets

A multi target is a set of image targets with a predefined geometric arrangement. An example of a multi target is the box in Figure~\ref{cuboid}. The box is comprised of six images (one for each face), each of which is a separate image target. The pose of these image targets are defined relative to one another. When detecting a multi target, the identified interest points must not only match the image targets, but their pose must match the predetermined geometric arrangement.

\item Cylinder targets

A cylinder target is an extension of an image target, wrapped around a cylindrical or conical object. The object's dimensions (diameter and size length) must be defined. The system can also track the images on the top and bottom faces of the cylindrical target if defined.

\item Object recognition

The object recognition feature of Vuforia allows users to create trackable targets from physical objects. The Vuforia Object Scanner creates a digital representation of a real-life 3D object by extrapolating the features and geometry of the object from the scan. Unlike image targets, object targets do not have to be planar. Figure~\ref{object_scanner} shows an example of the result of scanning a toy car. 

\begin{figure}[!ht]
\rightline{
\begin{minipage}{\textwidth}
\begin{center}
\woopic{object_scanner}{0.5}
\vspace{-.2 in}
\caption[Vuforia Object Scanner]{The Vuforia Object Scanner picks up the features and shape of a scanned toy car.}\label{object_scanner}
\end{center}
\end{minipage}
}
\end{figure}

\item VuMark

\begin{figure}[!ht]
\rightline{
\begin{minipage}{\textwidth}
\begin{center}
\woopic{vumark.png}{1.}
\vspace{-.2 in}
\caption{VuMark}\label{vumark}
\end{center}
\end{minipage}
}
\end{figure}

VuMarks (Figure~\ref{vumark}) are Vuforia's version of marker-based tracking. The existence of both image targets and VuMarks can be confusing, but VuMark designs allow for the coexistence of millions of trackable instances as well as the ability to encode a variety of data formats. Because of the uniqueness of each VuMark instance, these markers can also be used as a means of identification among similar products.

\item Model targets

Model targets are targets that are created from a digital 3D model of an object and can be used to track it. The Model Target Generator converts 3D object files into Vuforia Engine databases. Vuforia supports the real-time tracking of the registered models in the databases. In order to initiate the tracking, the Vuforia app must detect the object from a specific angle, prompted by the guide view (see Figure~\ref{guideview} for a screen capture of the guide view).

\begin{figure}[!ht]
\rightline{
\begin{minipage}{\textwidth}
\begin{center}
\woopic{guideview}{0.5}
\vspace{-.2 in}
\caption{Model Target Guide View}\label{guideview}
\end{center}
\end{minipage}
}
\end{figure}

\item Extended tracking

The main problem with tracking targets is that pose calculation desists when the target is no longer visible in the frame. Vuforia offers a solution to this issue in the form of extended tracking. Extended tracking makes use of a Device Tracker feature to enable six-degrees-of-freedom tracking. The Device Tracker provides information about the device's pose at all times relative to the targets and the rest of the world. After the device's position has been established the first time a target is detected, alignment between real-world objects and virtual contents persists even after the camera has moved from the target's position. Extended tracking makes space-aware AR experiences possible, for example games that require lots of space or when tracking a large object.

\item Ground plane

The ground plane detector identifies horizontal surfaces and allows users to place content on top of the plane or elsewhere relative to the detected plane. The Mid Air Stage places virtual content in mid air based on its preset position from the target ground plane. Vuforia ground plane detection is enabled by platforms such as ARCore and ARKit.

\end{enumerate}

The functionality of Vuforia is based around its ability to recognize different kinds of targets. The next section explains the back-end framework that enables tracking for a variety of objects.

\section{Core Components}
\begin{figure}[!ht]
\rightline{
\begin{minipage}{\textwidth}
\begin{center}
\woopic{vuforia.png}{.32}
\vspace{-.2 in}
\caption[Vuforia core classes]{UML diagram of Vuforia core classes for Unity}\label{coreclass}
\end{center}
\end{minipage}
}
\end{figure}

An application created with Vuforia in Unity requires several core components. The Unified Modeling Language (UML) class diagram in Figure~\ref{coreclass} shows the class hierarchy of the core of Vuforia. Consider a simple AR app that scans an image target and overlays the target with a 3D model. As the app starts, a new AR session is initiated. The \texttt{VuforiaARController} class stores the information about the new AR session. At the beginning of the session, information about the camera is registered in \texttt{CameraDevice}. During each frame, the image captured by the camera feed can be obtained from \texttt{CameraDevice} as well. Each picture frame is then passed on to the \texttt{TargetFinder}, which scans the image for features and prepares it for matching using a separate process of object recognition. This process is represented by the class \texttt{ObjectRecoBehaviour}. Notice that \texttt{ObjectRecoBehaviour} is a child of \texttt{VuforiaMonoBehaviour}, which is derived from \texttt{MonoBehaviour}, the base class for all Unity scripts, which offers functions for handling the life cycles of an app. For every single object/image that is found by \texttt{TargetFinder}, the result is compared against a reference database by \texttt{ObjectRecoBehaviour}. If the result matched a preregistered target, its position and orientation are assigned to the Unity game object that represents the target. When a target is located and paired to its representation, a corresponding \texttt{ObjectTarget} is created. An \texttt{ObjectTarget} inherits the base class behavior \texttt{Trackable} for all trackable types in Vuforia. The classes for the different target types (e.g. \texttt{ImageTarget}, \texttt{MultiTarget}, etc.) are all subclasses of  \texttt{ObjectTarget}. The appropriate subclass is chosen to match the kind of target identified by \texttt{TargetFinder}.

All of these components are embedded in the Vuforia framework and are not included in the actual developing workflow. Users designing a Vuforia app with Unity can use Unity's graphical user interface to interact with these components at the graphical level.

\section{Workflow}\label{workflow}
The workflow of designing a Vuforia-enabled app in Unity follows the general workflow of the Unity game engine. This section highlights the Vuforia-specific aspects of designing an AR app in Unity.
\subsection{Installation and Activation}
The Vuforia Engine is available through two channels in Unity, as a package downloadable from the Package Manager (for Unity 2019.2 or later) and as a separate component that can be installed using the Unity Download Assistant (for Unity versions before 2019.2). Activation of the Vuforia Engine can be done by enabling Vuforia Augmented Reality Supported in Player Settings. Activation is successful when Vuforia appears in the GameObject menu (Figure~\ref{menu}).

\begin{figure}[!ht]
\rightline{
\begin{minipage}{\textwidth}
\begin{center}
\woopic{vuforiamenu.png}{.35}
\vspace{-.2 in}
\caption{Vuforia submenu}\label{menu}
\end{center}
\end{minipage}
}
\end{figure}

\subsection{Creating Target Databases}
To be able to use custom targets in Unity, developers have to create the target database(s) and import them as an asset package. This process can be done on the Developer Portal on the Vuforia Engine website. The Target Manager function has buttons for creating new databases. There are two options for storing a database, on the cloud or offline. Since the Cloud Recognition feature is for enterprise use only, this section only explains the offline mode, which is called Device on the portal. The Add Target button is for uploading a new target. Among the different target types, there are five that can be customized: image, cuboid multi target, cylinder, object, and model. Model target databases are created using the Model Target Generator. The Target Manager deals with images, multi targets, cylinders, and objects. After the file has been uploaded, the target's properties should be changed to match its actual dimensions. For all image targets, the Manger gives each target a rating of 0 to 5. A 5-star rating means that the image is highly augmentable. These are images that are rich in detail, have good contrast, and include no repetitive patterns. Figure~\ref{ratings} include two examples of a 5-star and 0-star image targets. Figure~\ref{5star} has great contrast between bright and dark areas, while Figure~\ref{0star} is too repetitive and has uneven feature distribution. The Vuforia Target Manager has a feature visualizer for image targets that are useful for understanding how well feature detection works for different images. See Figure~\ref{feature_ratings} for the features detected (marked yellow) in the two images in Figure~\ref{ratings}. The 0-star image target has much sparser features than the 5-star rating one, making it less recognizable. As a result, optimizing the trackability of image targets is important.

\begin{figure}[!ht]\centering
\subfigure[5-star rating]
{\woopic{5star}{.3}\label{5star}}
\qquad
\subfigure[0-star rating]
{\woopic{0star}{.3}\label{0star}}
\vspace{-.2 in}
\caption{Image target ratings}\label{ratings}
\end{figure}

\begin{figure}[!ht]\centering
\subfigure[5-star rating]
{\woopic{feature_5star.png}{.57}\label{feature_5star}}
\qquad
\subfigure[0-star rating]
{\woopic{feature_0star.png}{.57}\label{feature_0star}}
\vspace{-.2 in}
\caption{Vuforia feature detection}\label{feature_ratings}
\end{figure}

Because of the nature of maps, this project faces several challenges with trackability. Because maps have many repetitive details (lots of similar symbols) and not much contrast (no extreme brightness difference), it is harder to optimize maps. There are several ways to prepare images for better target detection. The actual process of optimizing a 2005 map of Ho Chi Minh City used in this project is described in Figure~\ref{optimize}. The original map in Figure~\ref{preop} has very few distinguishable interest points and poor contrast. In version~\ref{postop1}, the contrast of the map was increased by adding a black border and lightening the pink and green areas. This action increases the local contrast between darker details such as roads, and the brighter background. To achieve the 4-star rating in Figure~\ref{postop2}, the foreground/background contrast was further enhanced and repetitive features were removed. However, the optimization process has a major effect on the integrity of the 4-star image because it eliminates some important information on the map. Due to the excessive alteration to the 4-star image, it makes to choose the 2-star image, which still yields a robust performance for iOS devices. 

\begin{figure}[!ht]\centering
\subfigure[Before optimizing (0-star)]
{\woopic{map_2005}{.4}\label{preop}}
\qquad
\subfigure[After adding border and increasing local contrast (2-star)]
{\woopic{improve1}{.09}\label{postop1}}
\subfigure[After increasing foreground/background contrast and eliminating repetitiveness (4-star)]
{\woopic{improve2.png}{.3}\label{postop2}}
\vspace{-.2 in}
\caption{Optimizing image target for detections}\label{optimize}
\end{figure}

In addition to images, users can also upload 3D object scans to create object targets. The Vuforia Object Scanner is an Android application that makes object targets from $360^o$ scans of an object. The Object Scanner generates the data necessary to define an object target. Preparing the scan environment is important for obtaining a good model. The lighting in the environment should be diffuse and sufficiently bright, without any directional light sources. This should prevent specular highlights, which lead to false interest points. A cluttered background also flags false points. The object space of the model is defined relative to the coordinates provided by a target image (Figure~\ref{scanner_target}). The object to be scanned must be placed in the grey box in the top left corner of the target image. The box is delimited by the axes defining the object space. Any part of the object outside of the space delimited by the grids is culled. The pose of the object is calculated relative to the pose of the target image. The principles for detecting an object are generally similar to the ones used in feature point detection. A decent number of interest points is important, as well as good coverage. When scanning an object, the app provides the visualization of the coverage of the interest points detected. After the scanner has established a general shape for an object, a mesh that covers the object appears around it. Regions that have the sufficient number of points detected is colored green. Regions unscanned or have sparse interest point distributions are not colored. Found interest points are represented by the green dots. To improve the model, scanning should be repeated twice with the target image replaced by a dark and a light background.

\begin{figure}[!ht]
\rightline{
\begin{minipage}{\textwidth}
\begin{center}
\woopic{obj_scanner_target}{.3}
\vspace{-.2 in}
\caption{Object Scanner target image. From: Vuforia}\label{scanner_target}
\end{center}
\end{minipage}
}
\end{figure}

\begin{figure}[!ht]
\rightline{
\begin{minipage}{\textwidth}
\begin{center}
\woopic{lipstick_scan}{.15}
\vspace{-.2 in}
\caption{A scan in progress. From: author}\label{scan}
\end{center}
\end{minipage}
}
\end{figure}

The resulting file generated by the Object Scanner is an Object Data (*.OD) file. OD files can be uploaded using the online Target Manager to create object target databases. An object target is treated the same way as an image target in Unity. In the Unity hierarchy, children of the object target \textbf{GameObject} are its augmented contents.

After all the targets have been added, the database is ready for download. The Unity Editor option allows the package to be imported and extracted like any other asset package in Unity.

\subsection{Creating a Scene}
All features of Vuforia are accessible as a \textbf{GameObject}. A \textbf{GameObject} in Unity is any object in a scene, from lights to cameras to characters (cite Unity GameObjects). A default scene in Unity includes a default camera \textbf{MainCamera}. However, as the position and orientation of the camera in an AR app depends on the actual pose of the device's camera, the \textbf{MainCamera} has to be replaced by an \textbf{ARCamera}, found in the Vuforia Engine under the GameObject menu. The \textbf{ARCamera} is configured to simulate the pose of the actual device's camera. To make the app detect a target, the appropriate target GameObject needs to added to the scene. For instance, the Image GameObject creates an Image Target for tracking. The imported target database is selected in the Inspector window under the Image Target Behaviour component. The properties of the \textbf{ImageTarget} object have been already preconfigured. Unless users want to add more properties, the target is now ready for tracking. To place other virtual contents in relation to the position of the target,  the relevant \textbf{GameObjects} must be added as children of the target. This hierarchy ensures that the children objects' pose are always calculated relative to the detected target. This AR app is now deployable with one scene that tracks a single target and overlays it with the chosen virtual contents.

\subsection{Creating Augmented Contents}
As discussed, the augmented contents can be any 3D \textbf{GameObject} in Unity. This project is concerned with two main types of \textbf{GameObject}s, quads and 3D models. Quads are used to create augmented maps, textured with the actual map image. The 3D models used are COLLADA (*.dae) files, downloaded from Sketchfab and imported as assets into Unity. One of the models was created using photogrammetry, a technique that produces 3D models of objects and scenes from raw photographs. The software used to create this model is Meshroom, an open-source 3D reconstruction software.

\section{Digital Storytelling}
In addition to the technical aspect of creating an AR experience, the narrative is another important component for producing immersive content. Digital storytelling describes a form of narrative that uses technology and digital media in its production.\footnote{https://learning.oreilly.com/library/view/digital-storytelling-3rd/9780415836944/xhtml/11\_Ch1.xhtml} An AR application is a digital story which utilizers techniques in computer vision and graphics to create an experience. As a result, the AR narrative has to follow the same principles as traditional storytelling, which entails having a clear goal, an established structure and format, with consistent and logical behaviors and events.\footnote{https://learning.oreilly.com/library/view/digital-storytelling-3rd/9780415836944/xhtml/16\_Ch5.xhtml} When adding in the interactive components of storytelling, elements such as interface, navigation, point of view, gameplay and use of time and space. The process of creating an AR experience, whether done in Unity with Vuforia or on any other platforms, must take these considerations into account when designing the narrative.

A historical exhibit is also a form of storytelling. An AR-enabled historical exhibit performs digital storytelling through a screen medium. The goal of the AR experience in this project is to demonstrate how the multi-layer memory of Ho Chi Minh City is embedded in objects and spaces such as maps and buildings. To achieve this goal, the experience is structured around physical and virtual exhibit items. Both categories are interactive. Physical items are wall and table-top maps and 3D-printed models of buildings. These objects, specifically the 3D-printed models, allow users all the usual interactions; they can be picked up, rotated, and moved around. The AR virtual contents also have interactions that users can perform through the mobile device's screen, like dragging and dropping for moving them or pinching for zooming. An AR narrative progresses by prompting user interactions with visual and audio cues, both onscreen and offscreen. The audio cues in this project are provided by an audio playback during the experience, which advances the narrative through traditional (oral) storytelling and provides guidance for AR interactions. 

An important consideration when designing a digital story is deciding whether an immersive treatment is going to enhance it.\footnote{https://speakerdeck.com/rapoulson/vr-workshop-2019?slide=48} Physical immersion should only be employed if it makes the narrative more effective. For this project specifically, what augmented reality brings that is otherwise impossible is perspective agency. Part of the argument of this historical research is that the recollections provoked by cartography and architecture do not allow much freedom on the side of the citizens of Ho Chi Minh City. The exhibit was produced with the hope that it elicits a different kind of recollection that is critical and context-sensitive. Augmented reality supports that freedom, providing users with the agency to choose their perspectives. Through engagement with and manipulation of space and virtual objects, the experience encourages users to think critically about who is manipulating these sites in reality, how memory is engendered, and what technology has assisted this process of memory formation.
%!TEX root = ../username.tex
\chapter[The Urban Blueprints]{The Urban Blueprints: Cartography and the Creation of the City}\label{cartography}
\setlength{\epigraphwidth}{4.5 in}
\begin{minipage}{\textwidth}
\epigraph{\vi Mùa xuân trên thành phố Hồ Chí Minh quang vinh!

Ôi đẹp biết bao biết mấy tự hào.

Sài Gòn ơi cả nước vẫy chào.

Cờ sao đang tung bay cao qua hết rồi những năm thương đau.

Xa ba mươi năm nay đã gặp nhau vui sao nước mắt lại trào.

\vspace{.1 in}
Spring on the glorious Ho Chi Minh City!

How beautiful and proud.

Saigon, the whole country waves its salute.

The flag flies high; all the painful years have passed.

Thirty years apart now we have met again, in tears of happiness.}{\vi \textit{Xuân Hồng}\footnotemark}
\end{minipage} 
\vi Following the end of the American war, in 1976,\footnotetext{Mây Trắng, \textit{Mùa Xuân Trên Thành Phố Hồ Chí Minh}, accessed March 10, 2020, \url{https://nhac.vn/bai-hat/mua-xuan-tren-thanh-pho-ho-chi-minh-may-trang-sodRBWw.}} the National Assembly of Vietnam approved the change of the city's name from Saigon to Ho Chi Minh City, after the first Prime Minister and leader of the Democratic Republic of Vietnam.\footnote{Nghia M. Vo, \textit{Saigon: A History} (Jefferson, N.C: McFarland, 2011), 8.}  The change of name to the famed communist leader did not spark much controversy in Vietnam. Its usage, although not as prevalent as "Saigon," is not necessarily subversive.\footnote{It was mostly among the international community where there was an outrage against the new nationalist name. Erik Harms, “Beauty as Control in the New Saigon: Eviction, New Urban Zones, and Atomized Dissent in a Southeast Asian City,” \textit{American Ethnologist} 39, no. 4 (2012): 748.} The use of “Saigon” is common in daily conversations and is often interchangeable with Ho Chi Minh City, especially in cultural publications.  Songwriter Xuân Hồng, in his famous song "Spring on Ho Chi Minh City" dedicated to the liberation of Saigon in 1975, referred to both names with great pride and adoration. The name change signals a turning point in the city's memory and marks the onset of a new urban discourse of rapid development. However, as the attachment to the former name Saigon suggests, even amid the current pace of urbanization, leaders and residents of the city still turn to the past in their framing of the city's present and future. The nuances in other sites of memory are less obvious than in the city's name. They are embedded in seemingly innocuous sites like maps and land surveys, embraced by different actors to explain past creations and future visions for the city.

A remarkable feature of writings on Ho Chi Minh City is the emphasis on the short length of its history and the monumental accomplishment in this span of time.\footnote{\vi Minh Hương, \textit{Nhớ Sài Gòn} (Ho Chi Minh: Nhà xuất bản Miền Nam, 1994).} The memory of the city is rooted in a sense of exceptionalism that has allegedly enabled its enormous economical advances.\footnote{During the American war period, Saigon's population doubled from 1.3 million to 2.8 million people. Vietnam's GDP doubled from 1995 to 2006, for most which was accounted by Ho Chi Minh City. The city’s population grew from 2009 to 2019 at a rate of 1.77\% for urban areas and 4.47\% for suburban area. Economic growth was at 8.32\% for 2019. Vo, \textit{Saigon}, 16–19; Philippe Peycamm, “Saigon, From the Origins to 1859,” in Saigon, Ba Thế Kỷ Phát Triển Và Xây Dựng [Three Centuries of Urban Development], ed. Quang Ninh Lê and Stéphane Dovert, 4th ed. (Ha Noi: Nhà xuất bản Hồng Đức, 2015), 11–16; Thanh Giang, “TP Hồ Chí Minh: Tăng Dân Số Cơ Học Quá Nhanh,” \textit{Đại Đoàn Kết}, October 12, 2019, \url{http://daidoanket.vn/do-thi/tp-ho-chi-minh-tang-dan-so-co-hoc-qua-nhanh-tintuc449624}; “Ho Chi Minh City Economic Growth in 2019 Estimated at 8.32\%,” \textit{Nhân Dân Online}, December 1, 2019, \url{https://en.nhandan.org.vn/politics/item/8177902-ho-chi-minh-city-economic-growth-in-2019-estimated-at-8-32.html}; “Population Estimates for Ho Chi Minh, Viet Nam, 1950-2015,” Mongabay, accessed March 12, 2020, \url{https://books.mongabay.com/population_estimates/full/Ho_Chi_Minh-Viet_Nam.html}; “Market Report – Vietnam – Economics,” BMI, accessed March 12, 2020, \url{https://bmiglobaled.com/Market-Reports/Vietnam/economic-strength}.} Vietnamese writers sing praises of the peculiar geography, which they claim shapes the identity of its people as a resourceful, hard-working, and open-minded people. However, according to this periodization and remembrance, Saigon seems to have materialized out of nowhere in the 17th century, despite its former (and current) connections with the Kingdom of Funan, Champa and its Khmer roots.\footnote{The official periodization of Ho Chi MInh City's history is from the 1600s to the present, excluding its Cham and Khmer past.} This selective amnesia is theorized in the discourse of memory as an active process that denies the space for exchanges of remembering, or in other words, it is deleted from collective memory.\footnote{Alexandre Dessingué and J. M. Winter, eds., \textit{Beyond Memory: Silence and the Aesthetics of Remembrance}, Routledge Approaches to History 13 (New York: Routledge, 2015), 4; Paul Ricœur, \textit{Memory, History, Forgetting} (Chicago: University of Chicago Press, 2004), 448–52.} The process of ideologizing memory is the constant redefinition of the meaning and boundaries of the city, both in terms of time and space, and requires the use of special sites of memory, or \textit{lieux de mémoire}.\footnote{Pierre Nora, \textit{Rethinking France: Les Lieux de Mémoire}, trans. Mary Trouille, vol. 1 (Chicago: University of Chicago Press, 2001).} This section surveys the use of cartography by the state as a site of memory, complicit in the fashioning of the geographical and social features of Ho Chi Minh City.

The following chapter discusses four different maps of Ho Chi Minh City to explore the relationship between governance and mapmaking and how these dynamic forces manifest through modifications in the real and imagined landscapes of the city. The four examples studied include (1) an 1815 map by \vi Nguyễn dynasty official Trần Văn Học, (2) an 1895 cadastral map by the French administration of Saigon, (3) an American/South Vietnamese 1961 map, and (4) a 2005 map by the current Ministry of Natural Resources and Environment. The exhibit accompanying this analysis is a mobile augmented reality experience featuring these maps in digital forms, presenting the narrative in its time and space dimensions. \en

\section{Cartography in Pre-1800s Vietnam}\label{sec:historiography}
In the 17th century, influenced by Cartesian logic, European practices of cartography resembled that of mathematics and other sciences, which emphasize exactness, objectivity, and infallibility. Influenced by Kant's critical philosophy and Kantian space, the shift in the philosophical concept of space from strictly geometrical to social from the 18th century onward necessitates the reconsideration of representations of space not as an objective science but also as a social project.\footnote{Henri Lefebvre, \textit{The Production of Space}, trans. Donald Nicholson-Smith (Malden, MA: Blackwell, 1991), 1–2.}  Following the groundbreaking work by Henri Lefebvre on the production of space, other researchers of cartography such as J. B. Harley and Denis Wood have critiqued the relationship between power and mapping.\footnote{Denis Wood and John Fels, \textit{The Power of Maps} (New York: Guilford Press, 1992); J. B. Harley, “Maps, Knowledge and Power,” in \textit{The Iconography of Landscape: Essays on the Symbolic Representation, Design, and Use of Past Environments}, ed. Denis Cosgrove and Stephen Daniels, Cambridge Studies in Historical Geography 9 (Cambridge: Cambridge University Press, 1988), 277–312.}  In \textit{Les Lieux de mémoire}, Pierre Nora has also considered the role of geographic representations in carving out the boundaries of the state and its national borders.\footnote{Nora, \textit{Rethinking France}, 1:105–32.} Geography and visual representations of geography are powerful sites of memory for the state. Using examples from royal itineraries from the Renaissance, Nora argues that the visual memory of borders created by French kings’ tours turned preexisting borders recorded in books into a material reality.\footnote{Nora, 1:113.} According to this theory, mapping is as much a presentation of physical spaces as a representation of the political powers carving out these spaces. The social aspect of cartography is thus an important discussion in the history of Vietnam’s mapmaking, which is closely entangled with the changing social and political scene.

The history of cartography in Vietnam is a relatively untapped subject, both domestically and internationally, limited by the availability and accessibility of sources.\footnote{John K. Whitmore, “Cartography in Vietnam,” in \textit{Cartography in the Traditional East and Southeast Asian Societies}, eds. J. B. Harley and David Woodward, The History of Cartography, v. 2, bk. 2 (Chicago: University of Chicago Press, 1994), 478–79.} The historiography of the creation and usage of maps in Vietnam is thin. Vietnamese textbooks on cartography mention the existence of pre-1000 CE plans for building dams and citadels, but no records remain of these documents.\footnote{\vi Lê Văn Thơ, Phan Đình Binh, and Nguyễn Quý Ly, \textit{Giáo Trình Bản Đồ Học} (Ha Noi: Nhà Xuất Bản Nông Nghiệp, 2017), 5; Đại học Tài nguyên và Môi trường Hà Nội, \textit{Bản Đồ Học} (Ha Noi: Đại học Tài nguyên và Môi trường Hà Nội, 2010), 15–16, \url{http://lib.hunre.edu.vn/Ban-do-hoc--5158-47-47-tailieu}.}  Another explanation for this gap is the general unpopularity of cartography in the area due to historic preferences for other modes of cosmological representations of space.\footnote{Whitmore, “Cartography in Vietnam,” 479–80.} Southeast Asia as a whole and other Asian countries such as Japan observed the same void with premodern-maps because they utilized different forms of representations.\footnote{Joseph E. Schwartzberg, “Southeast Asian Geographical Maps,” in \textit{Cartography in the Traditional East and Southeast Asian Societies}, ed. J. B. Harley and David Woodward, The History of Cartography, v. 2, bk. 2 (Chicago: University of Chicago Press, 1994), 741; Mary Elizabeth Berry, “Maps Are Strange,” in \textit{Japan in Print: Information and Nation in the Early Modern Period} (Berkerley: University of California Press, 2006), 54–103..}

The first major phase in Vietnam’s cartographic history is from the 15th to the late 17th century, starting with the landmark creation of the \vi \textit{Hồng Đức bản đồ} (Maps of the Hồng Đức Period [1471-97]) under the Lê dynasty. Historian of cartography John Whitmore traces the legacy of this collection and the Lê dynasty’s cartographic traditions in his overview of the cartographic history of Vietnam. The \textit{Hồng-đức Bản-đồ} is a collection of maps of the different provinces of Đại Việt (Great Viet) commissioned by King Lê Thánh Tông in 1467 and finished in 1490.\footnote{Ngô Sĩ Liên, \textit{Đại Việt Sử Ký Toàn Thư} [Complete Annals of Đại Việt], vol. 3 (Ha Noi: Nhà Xuất Bản Khoa Học Xã Hội, 1972).} According to Whitmore, this document was the first effort by any Vietnamese courts to perform countrywide mapping.\footnote{Whitmore, “Cartography in Vietnam,” 479.} Despite the absence of existing records of the original map, Whitmore’s analysis draws evidence from later attempts at reproduction by the Mạc dynasty and the Trịnh-Nguyễn families during the 16th and 17th centuries to map Vietnam’s territories. 

Lê dynasty's records showed the influence of the Ming map system. The earliest records of Vietnamese maps date back only to the 15th century, during which time the Lê kings adopted the Ming administration model of civil service examinations and literati-officials.\footnote{Whitmore, 481.} Whitmore argues that the expansion of the bureaucratic model both created the need and provided the data for mapmaking. As the court deployed state officials to remote provinces, information began to flow back to the capital and more knowledge were accessible from previously unknown or unreachable areas. A feature of early East Asian cartography is the different modes of representation apart from scale mapping. Scale and mathematical considerations became important with advances in measured mapping under the Ming.\footnote{Cordell D. K. Yee, “Reinterpreting Traditional Chinese Geographical Maps,” in \textit{Cartography in the Traditional East and Southeast Asian Societies}, 57.} Models such as the \textit{Da Ming yitong zhi} (Comprehensive gazetteer of the Great Ming) introduced "comprehensive maps" to supplement geographic information for the gazetteer entries. Maps made during this period by the Lê, such as the \textit{tổng quát} map, illustrated heavy Ming influence. The \textit{tổng quát} map in Figure~\ref{macmap} is a possible reproduction of the maps in the Hồng Đức collection. The three-ridge representation of mountains is a distinct symbol in the Chinese map system.\footnote{Whitmore, 483}

\en
\begin{figure}[!ht]
\rightline{
\begin{minipage}{\textwidth}
\begin{center}
\woopic{macmap.png}{.4}
\vspace{-.2 in}
\caption[\vi \textit{Tổng quát} map]{\vi A \textit{tổng quát} map of Vietnam, possibly made in the 16th century (Photograph in "Cartography in Vietnam." By Whitmore, 482.)}\label{macmap}
\end{center}
\end{minipage}
}
\end{figure}

\vi Vietnamese maps under the Lê were mostly for administrative and military purposes.\footnote{Whitmore, 496.} The few maps of the south were likely plans for military excursions into the lowlands of the Cham and the Khmers. Thus, itineraries, or route maps, played an important part in strategic planning.\footnote{Whitmore, 495.} Maps helped the state with taxation and with exerting control over distant (and foreign) lands. Research on cartography of the same time period in other East Asian regions also illustrates similar connections between maps and politics. Selection of maps and symbols reflect the political significance of the places they depicted. Power, specifically the emperor's power, is a major theme; in the Gazetteer of Jiankang Prefecture, Bangbo Hu concludes that many maps emphasized the prominence of the royal dwelling palace in comparison to other sites depicted.\footnote{Bangbo Hu, “Maps and Political Power: A Cultural Interpretation of the Maps in The Gazetteer of Jiankang Prefecture,” \textit{Cartographic Perspectives}, no. 34 (September 1, 1999): 18.}  The map of the capital city Thăng Long in the Hồng Đức collection shows a similar trend.\footnote{“Bản đồ Thăng Long theo Hồng Đức Địa Dư (1490) – Plan de Thang-long,” \textit{36hn} (blog), January 1, 2015, \url{https://36hn.wordpress.com/2015/01/01/ban-do-thang-long-theo-hong-duc-dia-du-1490-plan-de-thang-long/}.} Parallels between Vietnam and other East Asian countries in terms of cartography exemplify the connection between maps and politics in the region.

The following sections explore the themes of power, water, and identity through cartography. For each theme, the four maps are analyzed and contextualized to extrapolate the patterns of mapmaking and map reading in Ho Chi Minh City. This thematic approach is done at the expense of chronology, but it allows the analysis to delve deeper into patterns across different time periods.

\section{The Mapmakers of Saigon}\label{sec:mapmakers}
One of the questions surrounding cartography is the issues of readership and authorship, or in other words, who needs maps and who gets to produce them.\footnote{Harley, “Maps, Knowledge and Power,” 278.} In the case of Ho Chi Minh City, the change in governance and political legitimacy necessitated map production. Cartography is a form of power and knowledge, and this specific form of scientific knowledge is only accessible by those in power. The cartographic history of the city demonstrates this entanglement between politics, governance, and mapmaking. Every ruling power of Ho Chi Minh City since the 19th century, including the Nguyễn dynasty, the French colonial administration, the American-backed South Vietnam government, and the current Socialist Republic of Vietnam, has extensively employed cartography and its symbolic and concrete power to exert influence or control directly over the city.

\en
\begin{figure}[!ht]
\rightline{
\begin{minipage}{\textwidth}
\begin{center}
\woopic{map_1815}{.24}
\vspace{-.4 in}
\caption[\vi Trần Văn Học’s 1815 map of Gia Định Province]{\vi Trần Văn Học’s 1815 map of Gia Định Province (Map by Trần Văn Học. In CEFURDS Map Collection)\footnotemark}\label{map_1815}
\end{center}
\end{minipage}
}
\end{figure}

Since the formal integration of Saigon into the rest of Vietnam in the 17th century, the \vi Nguyễn dynasty were the first prominent mapmakers of Saigon. The history of Vietnam’s cartography reflects the surge in cartographic activities in the $18^{th}$ and $19^{th}$ centuries, which corresponded with the southward expansion, increased contact with foreigners, and French colonization.\footnote{Whitmore, “Cartography in Vietnam,” 479.} During this period, cartography in Vietnam came under Qing and Western influence.\footnote{Whitmore, 497.} The early 19th century saw Vietnam divided under three rulers: the Nguyễn court in the center, Nguyễn Văn Thành was the governor in the north and Lê Văn Duyệt in the south.\footnote{Whitmore, 499.} Maps from this era reflected these political divisions, with the court orienting towards a Chinese model while the north continued with the Lê cartographic traditions and the south was exposed to Western influence.

In 1815, a Nguyễn dynasty official, Trần Văn Học, drew the first Vietnamese map of Gia Định Province (Saigon) (Figure~\ref{map_1815}).\footnote{Trần Nam Tiến, \textit{Sài Gòn-TP.HCM Những Sự Kiện Đầu Tiên Và Lớn Nhất} (Ho Chi Minh: Nhà Xuất Bản Trẻ, 2006), 285; Trần Văn Giàu, \textit{Địa Chí Văn Hóa Thành Phố Hồ Chí Minh Tập 1 - Lịch Sử} (Ho Chi Minh: Nhà xuất bản Thành phố Hồ Chí Minh, 1987), 190.} \footnotetext{Trần Văn Học, \textit{Plan de Gia-Định et Des Environs, Dressé Par Trần-Văn-Học, Le 4e Jour de La 12e Lune de La 14e Année de Gia-Long} (Bulletin de la Société des Etudes Indochinoises, 1815), \url{http://virtual-saigon.net/Maps/Collection?ID=1134}} Trần Văn Học was a prominent architecture for King Gia Long in the early 19th century. Fluent in modern Vietnamese and Latin, he went on various diplomatic missions to France and India.\footnote{\vi Thụy Khuê, \textit{Vua Gia Long \& Người Pháp: Khảo Sát về Ảnh Hưởng Của Người Pháp Trong Giai Đoạn Triều Nguyễn} (Ha Noi: Nhà xuất bản Hồng Đức, 2017), 267.} Trần Văn Học possessed an extensive knowledge of Western technology and translated many European scientific texts into Vietnamese. He was involved in the design of the Bát Quái citadel and the naming of streets in the inner quarter. In 1815, Trần Văn Học drew a complete map of Gia Định, presumably under the commission of the Nguyễn court. His technique was inspired by Western cartography, complete with Chinese labels of important areas. The making of this map coincided with the Nguyễn’s attempt to consolidate power in the south, especially in the context of the rise to power of Gia Định’s governor Lê Văn Duyệt, who was rivalling the imperial influence of the court. The practice of cartography in Saigon by the state took precedent from this period and persisted into the next decades of colonialism and civil war, all the way to the present.

\vi
The change in governance from the imperial Nguyễn to colonial French marked a shift in the stakeholders of the cartography industry in the south. Production of Saigon’s maps continued, undertaken by the newcomers in town, who were awash with imperial ambitions and desire to achieve them through geographical conquest. French colonial rule over Saigon began in 1961 with a series of plans for development.\footnote{François Tainturier, “Architecture and Urban Planning during the French Administration in Saigon,” in \textit{Saigon}, ed. Lê Quang Ninh and Stéphane Dovert, 77–81.} In 1962, the French military engineer in charge of the urban development of Saigon, Colonel Coffyn, produced a plan for setting the delimitations of the city of Saigon.\footnote{Tôn Nữ Quỳnh Trân, “Sài Gòn qua các bản đồ,” in \textit{Ấn tượng Sài Gòn - Thành phố Hồ Chí Minh} (Ho Chi Minh: Nhà xuất bản Trẻ, 2015), 14.} Cartographic works included a cadastral map by the French cartographer A. Chauvet for the Service du Cadastre et de la Topographie [Department of Cadastre and Topography] in Saigon (Figure~\ref{map_1898}). The abundance of colonial maps exposes the link between cartography, urban planning and power. Colonialism requires mapping in order to dominate.\footnote{James R. Akerman, ed., \textit{The Imperial Map: Cartography and the Mastery of Empire}, The Kenneth Nebenzahl, Jr., Lectures in the History of Cartography (Chicago: University of Chicago Press, 2009), 3.} The relationship between cartography and empires is implied in access to the geographical information and technological know-how of mapmaking, which comes with power.

\en
\begin{figure}[!ht]
\rightline{
\begin{minipage}{\textwidth}
\begin{center}
\woopic{map_1898}{.18}
\vspace{-.4 in}
\caption[1898 French cadastral map of Saigon]{An 1898 cadastral map of Saigon commissioned by the Service du Cadastre et de la Topographie in Saigon (Map by M. Bertaux and A. Chauvet.)\footnotemark}\label{map_1898}
\end{center}
\end{minipage}
}
\vspace{-.2 in}
\end{figure}

The theme of power mapping carried over to the Republic of Vietnam era.\footnotetext{M. Bertaux and A. Chauvet, \textit{Cochinchine Française: Plan Cadastral de La Ville de Saïgon}, 1:4000 (Service du cadastre et de la topographie, 1898), \url{https://gallica.bnf.fr/ark:/12148/btv1b530297676}.} Under the South Vietnamese government and the American authority, production and circulation of maps continued with renewed fervor. The new regime was eager to redefine Saigon as a befitting capital and established a new department for cartography in 1955, the National Geographic Service of Vietnam.\footnote{National Geographic Service of Vietnam, \textit {Nha Dia Du Quoc Gio [i.e. Gia] (National Geographic Service of Vietnam): Ten Years of Operations 1955-1965}. (Ho Chi Minh: NGS, 1965), 1.} The American Army Map Service (AMS) also partook in the mapping of Vietnam. According to its mission, the AMS was responsible for the production and compilation of maps and related geographic information in the service of the U.S. Armed Forces.\footnote{Corps of Engineers, U.S. Army, \textit{The Army Map Service: Its Mission, History and Organization} (Washington, D.C., 1960), 2.} The AMS boasted an extensive collection of military maps of Vietnam, created in close cooperation with the National Geographic Service of Vietnam. In 1958, the National Geographic Service made a map of Saigon, which was republished the AMS in 1961 (Figure~\ref{map_1961}). The AMS edition was revised for strategic purposes. These two institutions monopolized cartographic production in Saigon from the end of the First Indochina War until the fall of South Saigon.
\en

\begin{figure}[!ht]\centering
\subfigure
{\woopic{map_1961_1}{.169}}
\qquad
\hspace{-.44 in}
\subfigure
{\woopic{map_1961_2}{.169}}
\vspace{-.3 in}
\caption[U.S. Army Map Service's 1961 map of Saigon]{The U.S. Army Map Service 1961 edition of a map of Saigon, first published in 1958 (Map by the National Geographic Service of South Vietnam and the U.S. Army Map Service. In the Perry-Castañeda Library Map Collection.\footnotemark)}\label{map_1961}
\end{figure}

\begin{figure}[!ht]
\rightline{
\begin{minipage}{\textwidth}
\begin{center}
\woopic{map_2005}{.85}
\vspace{-.35 in}
\caption[2005 map of Saigon]{A 2005 map of Saigon by the Ministry of Natural Resources and Environment.\footnotemark}\label{map_2005}
\end{center}
\end{minipage}
}
\end{figure}

\vi
Nothing attests to the connection between governance and the need for cartography like the postwar nationalization of cartography.\footnotetext{National Geographic Service of Vietnam and U.S. Army Map Service, \textit{Saigon}, 1:10000 (U.S. Army Map Service, 1961), \url{http://legacy.lib.utexas.edu/maps/world_cities/txu-pclmaps-saigon_sheet1-1961.jpg}.} After 1975, the new socialist state took over the task of imposing new definitions and boundaries on the area through mapping.\footnotetext{Bộ Tài nguyên và Môi trường, \textit{Thành Phố Hồ Chí Minh. C-48-34-A-d}, 1:25000 (Ha Noi: Nhà xuất bản Bản đồ, 2005), \url{http://virtual-saigon.net/Maps/Collection?ID=1141}.}  Government institutions continue to be responsible for overseeing the creation of cadastral and land use maps for urban planning and land development purposes.\footnote{Annette Miae Kim, \textit{Sidewalk City: Remapping Public Space in Ho Chi Minh City} (Chicago: The University of Chicago Press, 2015), 58.} The current regulating body of cartographic activities in Vietnam is the Department of Survey, Mapping and Geographic Information (DOSM), a unit under the Ministry of Natural Resources and Environment (MONRE). The map in Figure~\ref{map_2005} was made by MONRE in 2005. The Vietnamese government puts special emphasis on cartography as a tool for national security and land development.\footnote{Bộ Tài nguyên và Môi trường, “Quy Định Chức Năng, Nhiệm vụ, Quyền Hạn và Cơ Cấu Tổ Chức Của Cục Đo Đạc, Bản Đồ và Thông Tin Địa Lý Việt Nam,” May 16, 2017, \url{http://dosm.gov.vn/SitePages/GioiThieu.aspx?item=568}.} Throughout its history, the mapmaking industry of Ho Chi Minh City remains an exclusively national business to facilitate the interests of whichever administration was in power at the time.

For all four maps, the symbolic and physical boundaries shifted from one period to another. The city of Saigon and now Ho Chi Minh City have been demarcated by tangible and imagined lines of waterways and borders. As the regimes produced blueprints, they also manufactured memoryscapes to facilitate remembrance in the city.

\section{The Conquest of Water}
Rulers make maps to develop and control the land, but the land, and in Ho Chi Minh City's case, its water, dictates direction of development. Water has always been a part and parcel of life in Ho Chi Minh City. The water system is intricately connected to the lifestyle of the city’s inhabitants, both as a resource and a challenge. Ho Chi Minh City was originally a marshland. The historic center of the city, Bến Nghé, lies on the west bank of the Saigon river. The closest port to the city center is about 45 miles inland from the East Sea.\footnote{Bảo tàng Thành phố Hồ Chí Minh, “Sài Gòn - Thành Phố Hồ Chí Minh: Thương Cảng, Thương Mại - Dịch Vụ,” Bảo tàng Thành phố Hồ Chí Minh, accessed November 28, 2019, \url{http://www.hcmc-museum.edu.vn/en-us/store/1123-sai-gon-thanh-pho-ho-chi-minhbrthuong-cang-thuong-mai-dich-vubr.aspx}.} This awkward position hampered Saigon’s potential to compete with more accessible ports further north and south with specialized products such as Sóc Trăng (red salt), Cà Mau (fish), and Hội An (spices).\footnote{Vo, \textit{Saigon}, 7.} The inhospitable tropical weather and marshy landscape of Saigon, with its crocodiles and tigers, limited the area’s growth under the Cham before the 11th century and later the Khmer from the 11th century to the 17th century.\footnote{Vo, 1; Sơn Nam, \textit{Đất Gia Định - Bến Nghé Xưa \& Người Sài Gòn} (Ho Chi Minh: Nhà xuất bản Trẻ, 2016), 47–60.} However, when the Vietnamese and the Chinese moved into the area and brought with them wet rice agriculture in the 16th century, Saigon became a major hub for trading, granting access to the riches of the Mekong Delta.\footnote{Vo, \textit{Saigon}, 1–7.} The marshes previously swamped with wild animals now provided access from the main river to the Chinese quarter. The declining silt conditions at central Vietnamese ports such as \vi Hội An also helped to elevate Saigon in the South China maritime trade.\footnote{Ben Kiernan, \textit{\vi Việt Nam: A History from the Earliest Times to the Present} (New York: Oxford University Press, 2017), 252.}  These developments show that the economic progression of Saigon was as dependent on the local riverine system as it was restricted by the same water lines. To harness this part of land is to control the water system that governs the life of everything on it.

As the first Vietnamese government in the south, the \vi Nguyễn dynasty put great emphasis on strategic understanding of the water network. This priority is noticeable in the 1815 map, in which \vi Trần Văn Học, charged by the rulers to map Saigon, 
 \en
\begin{wrapfigure}{l}{0in}
\woopic{river_bend}{.46}
\vspace{-.2 in}
\caption{The bend of the Saigon River depicted by \vi Trần Văn Học (Image cropped from map by Trần Văn Học.) }
\label{river_bend}
\end{wrapfigure}
 \vi paid extreme attention to depicting the water system, especially in terms of the bend of the Saigon River. He improved the measurement methods used in previous French maps, exemplified by the measurement line around the river bend (Figure~\ref{river_bend}). The proportional accuracy of this map was adjudged to surpass the quality of previous French maps of Gia Định.\footnote{Thụy Khuê, \textit{Vua Gia Long \& Người Pháp}, 267.} The map’s details demonstrate how important it was for the new rulers of Saigon to establish an understanding of the layout of the land, and in this case its water, to consolidate their power over the city.

The marshy landscape of Saigon was an obstacle for the \vi Nguyễn rulers, but it was also an opportunity, which the state exploited to build the city’s defense. Figure~\ref{citadel} shows the strategic position of the city center, bounded on the north and 
\en
\begin{wrapfigure}{r}{2.6in}
\vspace{-.1 in}
\woopic{citadel}{1.0}
\caption[The Bát Quái citadel]{The \vi Bát Quái citadel (Image cropped from map by Trần Văn Học.)}
\label{citadel}
\vspace{-.3 in}
\end{wrapfigure}
south sides by two large arroyos, \vi Bến Nghé and Chợ Lớn, and on the east side by the Saigon River. The \vi Nguyễn governors in the city built their administrative and military structures around the main waterways. In 1789, Lord \vi Nguyễn Ánh (later became King Gia Long) commissioned the construction of the Bát Quái citadel (Figure~\ref{citadel}) on the bank of the Saigon River.\footnote{Vo, \textit{Saigon}, 37.} To fight off the Siamese, in 1772, Nguyễn Cửu Đàm, a general under the Nguyễn, built the Ruột Ngựa canal and Lũy Bán Bích (Bán Bích Rampart), connecting the two main arroyos, to close off the remaining open East side with a perimeter of water and rampart.\footnote{Huỳnh Ngọc Trảng, \textit{Sài Gòn - Gia Định xưa: tư liệu \& hình ảnh} (Ho Chi Minh: Nhà xuất bản Thành phố Hồ Chí Minh, 1997), 12–13.} Controlling the water and utilizing it as a defense became a mainstay for political regimes that governed Saigon since the 18th century.

Water was the lifeline of communications in the city under the Nguyễn Dynasty. The waterways were depicted in extreme detail, compared to previous maps. Trần Văn Học provided careful annotations of rivers, canals, and bridges, in addition to residential areas. An important Vietnamese writer on southern Vietnam, Sơn Nam, notices how the term \textit{đất giồng} (alluvial banks along rivers and creeks) in the Southern Vietnamese dialect reflects the significance of water in the area.\footnote{Sơn Nam, \textit{Đất Gia Định - Bến Nghé Xưa \& Người Sài Gòn}, 47.} The map clearly identifies \textit{đất giồng} concentrations along the main riverways with small rectangular symbols. In addition to the water system and its neighborhoods, other annotations on the map include markets and pagodas. This detail shows how the importance of water at the time was comparable to these other communal spaces. Their embankments served as both commercial and spiritual centers for Saigon inhabitants.

When the French imperial power invaded the land in the second half of the 18th century, their conquest of Saigon also depended heavily on water. French strategy of “gunboat diplomacy” relied on water mobility for success.\footnote{David Biggs, \textit{Quagmire: Nation-Building and Nature in the Mekong Delta} (Seattle: University of Washington Press, 2010), 23–26.} French troops launched their first attack on Saigon in 1959. Their gunboats followed the Saigon river straight to the entrance to the citadel, whose east gate was only 500m from the nearest river port.\footnote{Chung Hai, “Nếu còn thành cũ, Gia Định không dễ thất thủ ngày 17-2-1859,” \textit{Tuổi Trẻ Online}, February 17, 2016, \url{https://tuoitre.vn/news-1052677.htm}.} Despite serving their military interests, water also created major problems for the French imperialists. After capturing Saigon, colonial forces encountered a marshy landscape of sparsely populated plains and a convoluted network of creeks and rivers.\footnote{Vo, \textit{Saigon}, 75.} Their imperial conquest did not end with the 1961 military victory. The feat only opened up more obstacles posed by the land and its people, which continued to manifest throughout French colonization of Saigon. French maps of the area during this period expose their attempts to control these tensions between geography and power.

The conquest of water is evident in the French cadastral map of 1898 (Figure~\ref{map_1898}). One of its curious features is the convenient absence of the complicated water system and marshy landscape of Saigon. The only waterways depicted are the Saigon River and short parts of the arroyos \vi Bến Nghé and Thị Nghè. The wild, swampy and flooded plains did not fit the French vision of a city. Therefore, they came up with development plans to make it habitable and conforming to Western standards.\footnote{Tainturier, “Architecture and Urban Planning during the French Administration in Saigon,” 79–80.}
\en
\begin{wrapfigure}{l}{0in}
\woopic{boulevard_charner}{.6}
\caption[Boulevard Charner]{Boulevard Charner (marked red) (Image cropped from map by M. Bertaux and A. Chauvet.)}
\label{charner}
\vspace{-.2 in}
\end{wrapfigure}
\vi Colonial alterations to the water landscape ranged from marsh cleanup to bridge construction and canal projects to facilitate the extraction of goods. In 1875, French Admiral Victor-Auguste Duperré conceived a major plan for building a new inland water network to connect Saigon to the Mekong Delta.\footnote{Biggs, \textit{Quagmire}, 32.} This mega plan included the expensive project of the Chợ Gạo canal, creating a direct route by water from Saigon to the nearest delta Port, Mỹ Tho. Other projects included the filling of canal for land transport.\footnote{Tainturier, “Architecture and Urban Planning during the French Administration in Saigon,” 78–81.} The construction of Boulevard Charner (Figure~\ref{charner}) involved the clean-up and filling of the Chợ Vải Canal, which used to be a floating market for Indian textile.\footnote{Sơn Hòa, “Những Kênh Rạch Xưa Thành Đại Lộ Đẹp Nhất Sài Gòn,” \textit{VnExpress}, April 10, 2016, \url{https://vnexpress.net/thoi-su/nhung-kenh-rach-xua-thanh-dai-lo-dep-nhat-sai-gon-3380037.html}.} Urban planning projects from this period completely reconfigured the geographical landscape of Saigon, especially on the water frontier.

After the end of the First Indochina War in 1954, under President Ngô Đình Diệm, the Republic of Vietnam took control of the city and embarked on a series of modernizing programs building upon the existing French groundwork using American aid funds.\footnote{Biggs, \textit{Quagmire}, 154–55.} Renovation projects solidified colonial alterations to the water environment by building bridges, dams, and other transportation and irrigation systems. The 1961 map of Saigon (Figure~\ref{map_1961}) shows the newly introduced ferries for transportation across the Saigon River. The South Vietnamese and American mapmakers continued to exhibit a persisting cartographic pattern from the French period, which is the conspicuous absence of marshes. Urban infrastructure and plantation replace the old flooded plains. Natural frontiers such as rivers and canals, however, still serves as a barrier to urban expansion. In the map, the east bank of the Saigon River is mostly covered by rice fields and canebrakes. The same can be said of the north and south banks of the Thị Nghè and Bến Nghé arroyos.

The battles of the American war were fought not only on land but also on the waterfront. Knowledge of water conditions became decisive for all sides. Cartographic techniques to depict water networks became more precise with the introduction of aerial photography.\footnote{Biggs, 200.} The South Vietnamese map uses the Universal Transverse Mercator coordinate system and aerial photographs by the French Department of Geography. Enabled by advanced spatial and aerial technology, mapping remains one of strongest weapons against Vietnamese guerrilla resistance. While Americans and their allies relied on staying above water for survival, Vietnamese revolutionaries depended on water for undercover. Competing technological and spatial knowledge of the delta landscape spawned different tactical orientations for each side.

The theme of water mapping carries over to the postwar period, with maps of Ho Chi Minh City focusing on urban development along the water lines. The 2005 map of the city (Figure~\ref{map_2005}) outlines the new development projects of urbanization. \en
\begin{wrapfigure}{r}{3.6in}
\woopic{saigon_bridges}{.35}
\vspace{-.2 in}
\caption[Saigon River bridge complex]{Saigon River bridge complex (circled red) (\vi Image cropped from map by Bộ Tài Nguyên và Môi Trường.)}
\label{saigon_bridges}
\vspace{-.2 in}
\end{wrapfigure}
\vi Examples include the new Saigon River bridge complex (Figure~\ref{saigon_bridges}), the first permanent structures to connect the east and west banks. The recent completion of the \vi Thủ Thiêm bridge and tunnel complex has fully integrated the swampy eastern suburbs of Ho Chi Minh City into the center. The detailed riverine classifications reveal how water is still central to the city’s livelihood. The major difference in this map is how the water system has morphed into becoming a part of the urban network of roads, bridges, railroads, ferries, etc. Meanwhile, manmade augmentations are naturalized as part of the terrain and other natural configurations.

As much as these maps tell the story of water in the city, they also obfuscate the pitfalls of geographical modifications. Topography maps blur the lines between natural waterways and artificial structures, forged through violent use of technology to change the environment.\footnote{Biggs, 71–73.} These maps fail to show the disastrous shortsightedness and a total disregard for their environmental sustainability in these masterplans. Development designs hide the permanent damages by nation-building projects on the natural and social landscapes. Specifically, public sanitation became a major problem for the French administration as issues involving freshwater accessibility and waste disposal were overlooked, indicative of the superficial and hypocritical nature of modernization.\footnote{Tainturier, “Architecture and Urban Planning during the French Administration in Saigon,” 81.} Some of the canals dredged by the colonial administrations during this period have long silted up.\footnote{Biggs, \textit{Quagmire}, 71.} From 2005 to 2012, the Thị Nghè arroyo (now Nhiêu Lộc canal) underwent a major renovation which involved digging up the riverbed to enable trash removal and reorganization of the sewage disposal system.\footnote{Văn Hiến, “Bài 3: Cải Tạo Kênh Nhiêu Lộc – Thị Nghè: ‘Công Trình Thế Kỷ’ Của TP. Hồ Chí Minh,” \textit{Báo Mới}, January 31, 2018, \url{https://baomoi.com/bai-3-cai-tao-kenh-nhieu-loc-thi-nghe-cong-trinh-the-ky-cua-tp-ho-chi-minh/c/24818795.epi}.} Maps erase from memory deteriorating environmental conditions and natural disasters resulting from transformations to the water landscapes.

\section{Imagining the City}
Compared to the Merriam-Wester definition of a city as “an inhabited place of great size, population, or importance,” pre-eighteenth century Ho Chi Minh was by no means a city.\footnote{“City,” in \textit{Merriam-Webster} (Merriam-Webster), accessed November 28, 2019, \url{https://www.merriam-webster.com/dictionary/city}.} Originally, the region was mostly marshlands and barely habitable. Compared to other towns and cities in the south, the Saigon basin did not have any economic edge over the Mekong Delta. The city only sprang into existence under the Nguyễn dynasty and continued to rise in importance to the present, now with a population of 9 million.\footnote{Thông tấn xã Việt Nam, “Dân Số TPHCM Gần 9 Triệu Người, Đông Nhất Cả Nước,” \textit{Báo Sài Gòn Đầu Tư Tài Chính}, October 12, 2019, \url{https://saigondautu.com.vn/content/NjcxMDQ=.html}.} Today, memory of Ho Chi Minh City mostly focuses on its short but rapid development and urbanization in the last three centuries, without considering the historical context of its previous geographical and economic conditions and ethnolinguistic diversity. The reimagined history of the city needs to compelling for Saigon citizens and the international community to legitimize and consolidate its creation. The history of Ho Chi Minh City must be consistently reaffirmed to legitimize and consolidate the myths of its origin and development. Consideration of the role of political and social factors in the urban construction process is vital for understanding the creation of this narrative. The planning of a city is not only practical but also symbolic. The city is therefore a product of imagination, conceived in the form of development plans, brought into existence with construction projects, and preserved as part of the land and its memory through cartography. Maps communicate envisioned boundaries that define new spaces and their meanings.

The historical context of Vietnamese migration is important for understanding the emergence of the Saigon and Chợ Lớn conurbation. During the 16th and 17th century, social and economic developments in Vietnam elevated Saigon’s status in the eyes of Vietnamese ruling powers. Evidence of this rise in ranks can be gleaned from the 1815 map (Figure~\ref{map_1815}). Saigon formally became part of “Phủ Gia Định” (Gia Định Province) under King Minh Mạng, but Vietnamese rulers had long decided its perimeter and continued to draw new boundaries in the South to create six delta provinces.\footnote{Huỳnh Ngọc Trảng, \textit{Sài Gòn - Gia Định xưa}, 37.} How the map depicts these borders reveals the value of gaining knowledge of newly acquired lands for the Nguyễn court, who later used this territorial repertoire to ratify its administrative divisions of the Mekong Delta.

\en

\begin{figure}[!ht]
\rightline{
\begin{minipage}{\textwidth}
\begin{center}
\woopic{comm_lines}{.3}
\vspace{-.2 in}
\caption[The communication lines of the \vi Nguyễn Dynasty's Saigon]{The communication lines of the \vi Nguyễn Dynasty's Saigon. The Bến Nghé Arroyo is marked red and the Royal Road blue (Image cropped from map by Trần Văn Học.)}\label{comm_lines.png}
\end{center}
\end{minipage}
}
\end{figure}

\vi
The Nguyễn lords had already begun development of the Gia Định Province long before Trần Văn Học created the map in 1815. The design shows two main centers, Bến Nghé and Chợ Lớn. Chợ Lớn was the historic Chinese settlement, dominated by Ming and Qing merchants. When Vietnamese ruling families arrived in the area, they settled down on the left bank of the Saigon River, separate from the Chinese end.\footnote{Huỳnh Ngọc Trảng, 6.} In the map, the Bát Quái citadel (Figure~\ref{citadel}) stood in the center of Bến Nghé, bounded on the north and south sides by arroyos Thị Nghè and Bến Nghé and the Saigon river in the east.\footnote{Vo, \textit{Saigon}, 8-9.} Despite the separation of administration from commerce, lines of communication connected the two ends of the city on both land and water. The map shows the Bến Nghé Arroyo joining the Chinese quarter to the administrative area. The two centers of Gia Định were also connected by Đường Cái Quan, the Royal Road that ran through the length of Việt Nam at the time, from the Northern border with China all the way to the southernmost province of Hà Tiên.\footnote{Trung Sơn, “Những Con Đường Thiên Lý Đầu Tiên Của Vùng Đất Sài Gòn,” \textit{VnExpress}, September 5, 2017, \url{https://vnexpress.net/thoi-su/nhung-con-duong-thien-ly-dau-tien-cua-vung-dat-sai-gon-3636181.html}.} In fact, communication was decisive as the Nguyễn felt the need to tighten their reins over “wayward Southern barbarians,” who were allegedly fraternizing with Catholic missionaries and abandoning their core Confucian values.\footnote{Vo, \textit{Saigon}, 54.} Urban development from this period was to fulfil the vision of a subservient imperial city resistant against the encroachment of Western and Chinese influence.

These visions of a traditional city were soon supplanted by the French conception of a European city. The colonial government took over the groundwork of city building from the imperial court in the decades following their conquest. The abundance of development plans for Saigon during the colonial era tells the story of French colonial aspirations for building the city. Urban planning was especially important because of Saigon’s special geography of convoluted water networks and uninhabitable muddy terrain. Their grand vision for a “Paris of the East” met with enormous challenges. The early development maps of Saigon were not just a symbol of power; they also exposed the colonizers’ struggle to harness the land. In their attempt to control the geography of Saigon, French urban architects employed extensive use of cartography. Cadastral maps rose in prominence as the city was subdivided into different sections and plots.{Tainturier, “Architecture and Urban Planning during the French Administration in Saigon,” 75.} The 1898 French map shows land ownership types, demarcating four main sections of the city of Saigon: administrative, commercial, industrial, and residential. The administrative quarter of the city was located on the left bank of the Saigon River, further upstream from the commercial port. Military facilities mostly occupied the northwestern bank, guarding the entrance to the city center and the administrative section. Most of the commercial and residential districts lie south of the Bến Nghé – Chợ Lớn border.

French development projects created the border between the Saigon city and Chợ Lớn. The geographical homogeneity of Saigon is a fabrication by governing regimes, and so are its imagined borders. While Trần Văn Học’s 1815 map depicts Gia Dinh as one province, the 1898 French map clearly marks the borders of the city as separate from the market town Chợ Lớn. The colonial map marks the border splitting Saigon from Chợ Lớn by a thick red line; the Chinese side reads Arrond de Cholon (District of Chợ Lớn). The city of Saigon was really only created by French urban planning, with borders divided along geographical as well as cultural lines. In a mapping project on Ho Chi Minh City, Annette Kim proposes that the notion of Ho Chi Minh City as a single, continuous space is only a recent reimagination, and that its history is really the tale of two cities.\footnote{Kim, 28-37.} French planning created a stark difference in landscape between the two parts of town, which also resulted in the social stratification within the city.

French visions for an ideal \textit{métropole} came at the exclusion of ethnic Chinese and other indigenous Cham and Khmer populations, but cultural segregation is not the only elusive aspect in cartography. Urban mapping captures the transformation of Saigon from marshlands into a city decked by colonial architecture and paved boulevards, while hiding issues with waste disposal and clean water accessibility.\footnote{Kim, 38–39; Tainturier, “Architecture and Urban Planning during the French Administration in Saigon,” 81.} Old water lines were now replaced by roads, tramways, and new canals.\footnote{Kim, \textit{Sidewalk City}, 32–39.} Grandiose monuments distracted urban planners from the problem of environmental sustainability. The fabricated colonial city was doomed to failure from the outset, and their development maps were just a series of disguise for the shortsightedness of this vision.

Similar to the French government, for the South Vietnamese state, urbanization was not only a side project; it provided justification for their authority and legitimacy. In the French case, civilization was the pretext for colonialism. This rhetoric is echoed in the Second Indochina war period. The Saigon government and its American counterparts needed to prove their supremacy over North Vietnam, and nation building was one such display of power.\footnote{Christopher Fisher, “Nation Building and the Vietnam War,” \textit{Pacific Historical Review} 74, no. 3 (2005): 441–56.} Modernization theory is the premise of America’s and South Vietnam’s nation-building programs.\footnote{Fisher, 442.} In the first decade under the South Vietnamese state (1954 – 1964), Saigon experienced a major influx of migrants, especially from North Vietnam.\footnote{Hy V. Luong, ed., \textit{Postwar Vietnam: Dynamics of a Transforming Society, Asian Voices} (Singapore: Lanham, Md: Institute of Southeast Asian Studies; Rowman \& Littlefield, 2003), 34.} The fledgling Ngô Đình Diệm regime acquired a new set of residents, and urbanization projects were well underway to cater to this growing population. Contextualizing maps from this period can shed light on how the war was also fought on the ideological front through urban development.

Evidence of this ideological struggle lies in the infrastructure illustrated in the American Army Map Service’s 1961 map (Figure~\ref{map_1961}). The historic center of Bến Nghé is densely drawn, with careful annotations of administrative and public institutions. These depictions were to showcase South Vietnam’s administrative organization and its academic and economic advances. The sophisticated infrastructure described in the map served to validate the authority of the Ngô Đình Diệm government as the only legitimate Vietnamese government in the south. The map makes no mention of any American presence, even though the funding behind most of these institutions was from American aids. This intentional omission is also consistent with the choice of language in the map. Despite having been revised by the American Army Map Service (AMS), the only English-language text is the map’s name and edition, along with captions about the AMS and the National Geographic Service of Vietnam. South Vietnamese cartography masks the real agents behind the façade of modernization and urbanization to display a modern southern capital that misleadingly appears to be uniquely Vietnamese.

While previous French, American, and Republic of Vietnam maps depict Saigon as an exception, an island protected by water, the 2005 map by the Ministry of Natural Resources and Environment (MONRE) (Figure~\ref{map_2005}) is all about connectedness. Ho Chi Minh City has expanded tremendously in the past century, with new infrastructure to accommodate this growth. The map features red lines of national highways stretching to the marshiest part of town, connecting the city center to its outskirts and to other provinces. These interconnected networks solidify claims viewing the city as a historically contingent area. The map only shows a selected number of geographical features, including rivers, forests, marshes, fields, etc, while disregarding other parts. Interestingly, the map features a large part of the northern patches of forests, cutting out most of the South Saigon area of Chợ Lớn. This observation is in keeping with previous cartographic patterns, but in this case the omission is most likely due to the MONRE’s intention to show the green suburbs, despite the largely deforested center. The production of these maps creates and formalizes the interconnectedness of the city. 

The MONRE map continues to strengthen Ho Chi Minh City's identity built on rapid development and modernization. Black cubes on pink background litter the city center, marking structures that are above three stories. Recent urban renewal policies focus on verticalizing the city landscape with skyscrapers and redefining street spaces to enable this socio-spatial restructuring.\footnote{Marie Gibert, “Moderniser La Ville, Réaménager La Rue à Ho Chi Minh Ville,” EchoGéo, no. 12 (May 31, 2010).} In the years following 2005, new maps would go on to show the new high-rise centers of the city, including master-planned upscale neighborhoods such as Thủ Thiêm, Phú Mỹ Hưng, and Landmark. These new urban development projects become new grounds for imagining the city and experimenting with ideas about urban life for both the state and private agents.\footnote{Erik Harms, \textit{Luxury and Rubble: Civility and Dispossession in the New Saigon}, Asia: Local Studies/Global Themes 32 (Oakland, California: University of California Press, 2016), 4.} Land-use rights provide the space for exercising political freedom that is otherwise restricted, hence the existence of maps as a tool to propagandize and legitimize the politically charged use of urban space.

Structural and social transformations through changes in governance constantly reshape the polysemous image of Ho Chi Minh City. From an imperial city to a European metropole, from a democratic capital to the “blossoming lotus” at the heart of the late socialist Vietnamese economy, new conceptions supplant old ones to create metamorphic narratives that are both dynamic and pervasive. Cartography is at the interface of the city’s physical development and the mental configuration facilitating this process. The intersection of power, politics, and science creates avenues for maps to transcend their material borders and bleed into the realm of urban consciousness.

\section{Ho Chi Minh City in Three Degrees of Historical Freedom}

In computer graphics, six degrees of freedom (6DOF) signify the ability to translate and rotate an object along and about three axes in three-dimensional space. A 6DOF object’s movement can be quantified by positional changes caused by these six transformations: forward/backward (x-axis), left/right (y-axis), up/down (z-axis), roll (x-axis), pitch (y-axis), yaw (z-axis). 6DOF is ultimately an index of the freedom of movement an object possesses. This section applies the concept of degrees of freedom to describe the cartographic history of Ho Chi Minh City since the 18th century by creating an AR map of Ho Chi Minh City using the historical maps discussed previously. The map provides a three degree of freedom perspective of Ho Chi Minh City by allowing the audience to view the city’s transformations along the political, geographical, and social lines. The AR app creates 2D virtual overlays of historic maps on top of a 2D real-word contemporary map of Ho Chi Minh City hanging on a vertical surface and provides an audio narrative of the historical context and significance of cartography in the past three centuries. This AR experience is created using the image target feature of the Vuforia Software Development Kit in Unity.

The Unity scene that presents this cartographic narrative is made up of a single image target, created using the 2005 map of Ho Chi Minh City (Figure~\ref{map_2005}). The three remaining maps are overlaid on the target. These maps are aligned so that the scale remains consistent across all four maps. Possible interactions with the overlaid maps include dragging and dropping. Each map can be moved to view the layers underneath. Dragging the map to the vicinity of its original position will automatically cause it to snap back into place. Figure~\ref{map_app} captures the interface of the app in use, with three active maps. The app also includes an audio guide to navigate the experience and provide the narrative.
\en 

\begin{figure}[!ht]
\rightline{
\begin{minipage}{\textwidth}
\begin{center}
\woopic{map_app}{.16}
\vspace{-.2 in}
\caption{A screenshot of the map AR experience (Screen capture by Thuy Dinh.)}\label{map_app}
\end{center}
\end{minipage}
}
\end{figure}

The motivation behind the production of this interactive map is also the driving question behind this cartographic analysis, to  attempt to understand the workings of urban memory and its mnemonic devices. To study the maps of Ho Chi Minh City is to trace the sites where remembrance is enforced and memory reinforced, imagined on paper and enacted by means of technology,  wealth, and violence. Memory is untrustworthy, every recollection subject to distortion, whether intentional or not.\footnote{J. M. Winter, \textit{Remembering War: The Great War between Memory and History in the Twentieth Century} (New Haven: Yale University Press, 2006), 4.} Memory is also powerful; collective memory provides shared identity, which enables concepts and ideologies as compelling as nationalism and regionalism.\footnote{Benedict Anderson, \textit{Imagined Communities: Reflections on the Origin and Spread of Nationalism}, Revised edition (London: Verso, 2006).} Memory creates cities. Yet because of its unreliability, memory construction is often both elusive and illusory. Every person has their own personal memory, which to them can often appear infallible. This conviction, however, could lead to the false conclusion that remembrance is a personal experience that exists without agency, while history, on the contrary, is "an objective story which exists outside of the people whose life it describes."\footnote{Winter 11.} The abilities and pitfalls of memory are what motivate this historical analysis as well as its AR component.

 %!TEX root = ../username.tex
\chapter[Architectural Monumentality]{Architectural Monumentality: Memory, Identity, and Power in Saigon Symbolic Architecture}\label{symbol}
% to do: rethink first sentence
If maps represent the mental configurations of a landscape, then buildings are their physical enactment in a process where the imagined turns concrete. Maps provide the high-level network of interrelations between power and memory, in which each building is a node with the same characteristics as the parent network. Both cartography and architecture are visual modes of remembrance, but architecture, as a site of memory, can be more direct and explicit with the transition from two- to three-dimensional space. Within the built environment, the institutionalization of memory is done on several levels, with the building’s function, form, structure, material, and facade. The monumental architecture of Saigon elicits the same questions as the city’s maps, regarding authorship, audience, narrative, and agency. These categories enable recollections of the past in an entanglement of power, memory, and identity. This chapter explores the manifestation of the interconnectedness of these three themes through an examination of the symbolic and functional significance of four monuments in Ho Chi Minh City, with each monument corresponding to a major era in the city’s history.
\vi

The interplay between memory and architecture is an important avenue for many memory studies, as monuments produce important sites of memory (cite Pierre Nora, Hue Tam Ho Tai). Eric Sandweiss’s argument about the role of museums and cities draws on Lewis Mumford’s vision of the city as a “storehouse of memory,” which remains durable in times of change, a site of both “endurance and transformation,” aided by institutions such as the city history museum (Sandweiss 26). Shelley Hornstein posits that architecture is not only defined by design, form, or structure, but as “spatialized visualizations and experiences” in the mental and emotional space (cite Hornstein). Similar framing by Mark Crinson theorizes the city as a collection of objects (or buildings) whose collective image “enables the citizen to identify with its past and present as a political, cultural and social entity” (xiv Crinson). Specifically, in terms of Vietnam, architecture is connected to notions of urbanization and nation-building. For the period when Saigon was the headquarter of French Indochina, urban development was under the influence of colonial ideologies about civilization and modernization (cite Wright and Phantasmatic). These frameworks continue into the post-colonial era, when land-use policies enforce specific framing of the urban landscape (cite erik harms rubble). Studies on architecture and memory are rich but often limited in periodization and specificity. The following analysis spans the length of Ho Chi Minh City’s history, focusing on four major architectural symbols, the Thiên Hậu temple, the Notre-Dame Cathedral, the Independence Palace, and the Bitexco Financial Tower, to illustrate the transformations of the built environment in accordance to shifting political and social processes in the city.

\section{\vi Thiên Hậu Temple}
\vi
Thiên Hậu Temple, officially Tuệ Thành Assembly Hall, was first built by the Cantonese community of Ho Chi Minh City circa 1760. Located in the heart of Chợ Lớn (the Chinese quarter), the temple (Figure~\ref{thienhau_past}) was part of a market complex with multiple other temples, assembly halls, shops, and houses. Today, Thiên Hậu Temple remains in District 5, the unofficial Chinatown of Saigon. The main alter is dedicated to the Empress of Heaven Thiên Hậu (Mazu), a protector of seafarers and ocean goers. The former assembly hall was and still is central to the spiritual and cultural practices of ethnic Chinese in Chợ Lớn. In both a symbolic and functional capacity, the site is a fascinating slice of religiosity in Ho Chi Minh City, where sacred spaces take on non-religious roles as the city transitions into a neo-capitalist period and material culture proves as central to life as spirituality. Believers still visit the temple to seek blessings, but younger generations have taken to it as a scene for having images taken for their social media.\footnote{https://thanhnien.vn/du-lich/nguoi-sai-gon-ru-nhau-di-chua-ba-cho-lon-dau-nam-moi-canh-ty-2020-1173072.html} Tourists visit from over the world, both Eastern and Western, with all levels of knowledge and belief in the sanctity of Mazu the Empress of Heaven, to see its impressive collection of religious paraphernalia and unique Chinese temple architecture. The position of Thiên Hậu Temple in the mental and physical landscape of the city maps the dynamics of its symbolism and functionality in terms of memory and identity.
 \en

\begin{figure}[!ht]
\begin{center}
\woopic{temple_1800}{0.6}
\vspace{-.2 in}
\caption[\vi Thiên Hậu Temple in the 1860s \en]{\vi Thiên Hậu Temple in the 1860s. Taken by Emile Gsell [c. 1865-1875]. From: Giáo hội Phật giáo Việt Nam \en}\label{thienhau_past}
\end{center}
\vspace{-.2 in}
\end{figure}
\vi
The natural and economic characteristics of Chợ Lớn and the larger Saigon area explain the influence of the sea goddess Mazu on the local community. Starting from the 1680s, Ming loyalists arrived in Đà Nẵng seeking refuge. Sent by the Nguyễn sovereigns to southern territories, they cultivated  land in exchange for protection (300 nam 33). Water was integral to the life of these settlers and remained important as they carved out their livelihoods. The migrants, mostly from merchant backgrounds, travelled to the southern coasts by sea and soon controlled commercial activities in the region. They established trade posts and took over import and export, mostly by monopolizing the waterway trade routes for transporting rice produce from the Mekong basin (300 years 35). Heavily involved in water-related activities, merchants looked to the Lady to protect their voyages along the South China Sea trade routes, laden with rice and other produce from the hinterlands. In the liminal space of the temple, travelers said their prayers before a trip and made offerings after one to thank the deity for their safe travels.

The shared culture of Mazu worshipping became the binding factor for communities of settlers in this foreign land. The Guangdong community who built the temple also looked to the Lady for blessings regarding health and fortune. Important events in a person’s life such as birth, marriage, new commercial ventures, illness also brought supplicants to Mazu for guidance and favor (Cherry 64). Donations went to the upkeep of the temple and community building. As an assembly hall, Thiên Hậu temple spearheaded social projects that provided education and support for members of the congregation (cite Thien Hau book). The spiritual center transcended its religious function and became a social hotspot, where seafarers and locals connected in a sphere of shared backgrounds, business ventures, on top of cultures and beliefs. This practice was replicated all over the coasts of the South China Sea. Chinese merchants built Mazu temples all over the port cities in China, Vietnam and other Southeast Asian countries, creating a spiritual network that was established on the routes existing trade network.

\vi The role of Thiên Hậu Temple in the Chinese community was not only important for travelers but for locals as well. In the $19^{th}$ century, life in Chợ Lớn was divided along geolinguistic lines, according to the \textit{bang} (congregation) system. Chinese settlers formed different congregations according to their hometown in China and their dialect. Every congregation had it own market and place of worship, along with other collective properties such as hospitals, schools, cemeteries, etc. They each managed a separate assembly hall, and on top of its religious functions, also performed administrative duties such as immigration and emigration registration and tax collection. (cite Doling 29) The Cantonese population in Chợ Lớn built the Tuệ Thành Assembly Hall (also known as Thiên Hậu Temple) circa 1790, at the height of Qing migration. (cite Doling 19) Near Tuệ Thành (Guangzhou) Assembly Hall, Ôn Lăng Assembly Hall (or Quan Âm Pagoda) was built by the Fujianese settlers. These centers regulated life outside of their thresholds through the coordination of ceremonies and rituals. Communal spaces served as a point of contact for new migrants, a gathering place for members of the same ethnic Chinese group, and as the coordinator of community-wide projects and events. These responsibilities could be educational, charitable, or ceremonial. One such evidence of these communal functions can still be found at Thiên Hậu Temple in the form of an antique fire extinguisher from the 1890s (cite). As part of a market complex, the temple resided at the intersection of all the activities that Chinese in Chợ Lớn participated in on a daily basis. Spirituality was tied to the fabric of the community.

The administrative aspect of the temple made this system desirable for the new rulers in town. Since the Chinese had great influence in Saigon due to their economic stronghold, they had great leverage for negotiating with the political powers. The semi-autonomous congregation system lasted from the Nguyễn regime through  French colonization, encouraged by an influx of Chinese migrants looking for new economic opportunities and escaping the political crisis in China in the 19th century. Under French colonization, Chợ Lớn enjoyed considerable autonomy. Colonial authorities communicated with representatives from the Chinese congregations, very often the same people who looked after the assembly halls. As an effort to appease local unrest, the French issued decrees to preserve scriptural materials and decorative objects in major temples such as the Guandong’s Thiên Hậu Temple. As a result, religious spaces were some of the places where ethnic traditions and particularities were best preserved. \en

\begin{figure}[!ht]
\begin{center}
\woopic{temple}{0.1}
\vspace{-.2 in}
\caption[\vi Thiên Hậu Temple in 2008 \en]{\vi Thiên Hậu Temple in 2008. Taken by Christopher [2008]. From: Flickr \en}\label{thienhau_present}
\end{center}
\vspace{-.2 in}
\end{figure}

\vi Today, Thiên Hậu Temple (pictured in Figure~\ref{thienhau_present}) takes on completely different meanings. Chợ Lớn remains the dominant Chinatown, but the neighborhood is no longer predominantly Chinese. The process of assimilation as well as ethnic Vietnamese migration to Ho Chi Minh City have reduced the percentage of ethnic Chinese in the city to around the 10\% mark.  During the American War of Resistance, the Saigon government embarked on the socio-cultural and economic restructuring of Saigon, which included the revocation of permits held by Chinese owners in several trade fields, forcing them to adopt  Vietnamese citizenship or change profession (300 years 205). President Ngô Đình Diệm and his successors turned away from the semi-autonomous congregation system used by the French. In their effort to assimilate and limit the influence of (non-American) foreigners, the government nationalized previously independent institutions such as schools, pagodas, cemeteries, etc.  Foreigners, including Chinese, were no longer allowed to trade meat and fish or engage in retailing consumer products, but probably the most devastating blow was the ban on transportation of people or freight by land and water. These laws blocked the Chinese community from relying on commerce as their main livelihood and drove many from Chợ Lớn into the Bến Nghé area of Saigon. The economy of Saigon was no longer controlled by the Chinese, or the French for that matter. A new body of Vietnamese nationals took over, with the support of American companies (300 years 204).

Chợ Lớn fell into dilapidation as  social stratification grew with the changes in demographics and economics. Following the fall of Saigon in 1975 and the subsequent 10-year bao cấp (subsidy) period, life of the Chinese in Saigon became even more difficult as the socialist state sought to nationalize private properties and imposed strict regulations on the exchange of goods.  Many left Vietnam during this time. It was only after the reforms in 1986 that the Chinese started to participate again in Ho Chi Minh City’s economy. These economic and political transformations led to the social restructuring among the ethnic Chinese in Chợ Lớn. These policies effectively restricted the influence of non-Vietnamese cultures and changed the role of sacred spaces like Thiên Hậu Temple in the life of Chinese and non-Chinese in Ho Chi Minh City.
% to do: footnote for name

The functions of Thiên Hậu Temple have also adapted to changes in the social fabric and economic makeup of the new urban landscape. One of the most notable changes was in the temple's name in Vietnamese, which uses the word \textit{chùa} (pagoda). The original term used by the Guangdong was \textit{miếu} (temple). Both indicate a place of worship; while \textit{chùa} is commonly affiliated with Buddhism, a \textit{miếu} is often dedicated to local deities. The blurred distinction between Buddhism and đạo mẫu (Religion of the Mother Goddesses), which includes Mazu, signifies the shifting importance of religion in the late socialist society. Visitors to the temple can be Buddhist, Christian, and often non-Chinese. The Lady’s sanctity is no longer specific to water-related prayers. Today’s worshippers are often indiscriminate about which deity to make an offering to on New Year’s Day. In this noninstitutionalized religious landscape, beliefs are shaped on an individual level by influences such as family, acquaintances, and popular publications. (Taylor 3) This phenomenon is not particular to Thiên Hậu Temple or Mazuism. Spiritual plurality gives Vietnamese temples a new look, often accompanied by new rituals and ceremonies, and eventually new meanings. In the postwar aftermath, these spaces ceased to perform the old communal functions, becoming less of a social and economic founding pillar and taking on a more strictly religious and spiritual capacity.

The rhetoric used by the temple leadership also evolves following the changes in political regime. The temple's special publication celebrating the beginning of the 21st century started with a line rehashing the 25th Anniversary of Unification (end of the American War of Resistance), followed by an expression of gratitude to the Party and the state.  The tone conforms with the national rhetoric of secularization, usually dressed in anti-superstitious language. In the post-Renovation era, the Communist Party has struggled to create a paradigm for religious freedom that still fits their agenda of urban redevelopment, specifically in the case of spiritual spaces.  An attempt to incorporate Thiên Hậu Temple into this framework was done in 1993 when the Ministry of Culture and Information recognized the temple as a national heritage site. The question of reconciling spirituality with modernity and secularization continues to plague the Vietnamese government as they maintain an ambiguous response to religion. The attitude adopted by the state is indicative of the postwar secularized identity that the Communist Party embraced. Consequently, in terms of Ho Chi Minh city, the temple's image enforces an identity built on ethnic diversity, rather than religious multiplicity. 

Despite the state's uncertainty towards the city's religious landscape, Saigon tourist agencies and the city leadership have also attempted to project the image of cultural diversity through their advertisement of the temple. The temple is among some of the most visited attractions on one-day city tours, framed as providing a slice of life in Chinatown, notwithstanding how flimsy the connection is today.  From serving as a symbol in the Guangdong Chinese community, the temple has been rebranded and resold as a representation of the entire Chinese population in Saigon. Ethno-linguistic distinctions have become obscured and redressed in one form or another,  masked as equality and accentuated to showcase only a selective and modified façade of diversity.

Sacred spaces like Thiên Hậu temple are important for cultivating cultural memory. For past and present temple-goers, the existence of these spaces provide a spiritual sphere with rites and objects that enable recollections of their origins and traditions. For the different Saigon regimes, the extent of state influence on the temple is a form of control over the Chinese community and their identity. Today, both public and private institutions in the service industry have attached new meanings to the temple such that the act of worshipping at the temple becomes a mechanism of identification for citizens, in the effort to create an identity based on cultural diversity. The same process also take place with other symbols in this architectural environment.

\section{Notre-Dame Cathedral of Saigon}

On the website of the Roman Catholic Archdiocese of Ho Chi Minh City, under the series “Trùng tu nhà thờ Đức Bà" ("Notre-Dame Restoration”), one finds the opening article “Nhà thờ Đức Bà Sài Gòn, sức hút của một công trình" ("The Notre-Dame Cathedral Basilica of Saigon, the Attraction of a Construction”). The piece is a compilation of the perspectives of several architects on the values of the cathedral as an architectural symbol of Ho Chi Minh City. Architect Nguyễn Thu Phong writes:

\begin{quotation}
I wonder why it influences the emotional life of the city’s inhabitants so much… It’s like a hyphen between an urban life and spirituality… Around the cathedral, life carries on as usual every day, parents picking up their kids after school, couples shooting their wedding photos from countless angles. On holidays, there seems to be a magnet drawing Saigon citizens to the cathedral. Nearby, the Diamond high-rise symbolizes the city’s commercial life. On another side is the Cultural House of Youth. A bit further lies the administrative and political center. On its left is the municipal post office. The cathedral fits into a cultural and communal complex. Maybe that’s why it is such an attraction in the life and mind of the people of the city.
\end{quotation}

The seamless blend of spirituality and urbanism is a common fascination when it comes to this building. Situated in the middle of a large intersection within walking distance of all major administrative and recreational sites in District 1, the proverbial downtown of Saigon, the Notre-Dame Cathedral has been well integrated into the city’s symbolic landscape as a beloved emblem. Nevertheless, even in the same article, the ambivalence with the cathedral’s history and its religious implication is palpable. This uncertainty is reflected in the words of another architect:

\begin{quotation}
A construction that represents the dream of moving forward… [a] church accompanied by a post office, government headquarters, a theater, a square… The complex projects the image of an urban center and the power of the authorities in the eyes of colonial subjects… An architectural culture is the result of the interactions within the society through the length of history. In the end, if there was no cathedral, there might have been another famous piece of architecture, but there won’t have been a hundred years of French colonization.
\end{quotation}

Such sentiment is not uncommon even though the concern is sometimes lost in total reverence of the cathedral’s architectural brilliance. In a late-socialist society, the church’s colonial ties can make it difficult to justify such adoration. The sentiment can be passed off as love of the city or pure admiration for the architecture. Oftentimes, it is mixed with a hint of nostalgia, or “colonial blues” as some critics point out, for the lost “Pearl of the Orient.” Scholars often consider the connection of colonial nostalgia and its architectural legacy. Historians of memory have long considered architecture as a reflection and agent of memory. Eleni Bastéa theorizes the relationship between memory and the built environment through design, literature, and practice.  In the cathedral, one sees the process of construction and reconstruction of memory, from its erection, to the changing iconography, and the ongoing restoration.\footnote{add 2020} The meaning of the cathedral shifts over time as agents of memory work to enact structural and symbolic modifications. A French badge of power, of religious superiority, a token of modernity, and now a proof of history, these are all the polysemous identities of an edifice representative of an equally fluid city. To understand how these complex levels of symbolism have been developed in the 140-year history of the Notre-Dame Cathedral of Saigon, it is important to grasp the context of its creation and development over time. \en

\begin{figure}[!ht]
\begin{center}
\woopic{cathedral_1890}{0.3}
\vspace{-.2 in}
\caption[Notre-Dame Cathedral in 1890]{Notre-Dame Cathedral in 1890. From: Maison Asie Pacifique \en}\label{thienhau_present}
\end{center}
\vspace{-.2 in}
\end{figure}

The Notre-Dame Cathedral Basilica of Saigon started construction in 1877 and was consecrated in 1880, 22 years after the French conquest of Saigon in 1858 (Cherry 16). By this time, the city was well established as the colonial capital of Cochinchina. To meet the religious needs of the colonial class in town and the demands of missionary work, the original Église Sainte-Marie-Immaculée was completed on the site of an old temple in 1863 but soon fell into dilapidation. A replacement, bid by numerous French contractors and won by Jules Bonard, was immediately underway (Notes and News).  Construction concluded in 1879, resulting in a Romanesque structure 93m in length and 36m in width, held by bricks imported from Marseille, in the traditional cruciform shape complete with a transept and a nave, in other words, a piece of architectural brilliance by all Western standards (Cherry 16).

As the colonial headquarter, Saigon boasted a complex of military and administrative institutions, now with the Cathedral as its religious center. Visitors to Saigon did not fail to draw the connection between the “Paris of the Far East” with its original source of inspiration. An American war correspondent, Jasper Whiting, wrote about the city's broad and immaculate boulevards and the various miniatures of Champs Élysées, Bois de Boulogne, Avenue de l’Opéra. He sang praises of the “twin-spired cathedral, the Notre-Dame of the city” (Edwards 91). The cathedral’s bell towers were the first thing that greeted travelers from the bank of the Saigon Rive after a 74km trip from the coast (Demay 125-9). Everything about the city was reminiscent of France, and adorning the grandiose of the Western-styled edifices was the tallest, grandest structure of all, the Saigon Cathedral. From the point of view of the colonial subjects, the show of power through the imposing monument was an effective move. \en

\begin{figure}[!ht]
\begin{center}
\woopic{statue_pigneau}{0.5}
\vspace{-.2 in}
\caption[Pigneau du \vi Béhaine Statue \en]{Pigneau du \vi Béhaine Statue. From: Flickr \en}\label{statue_pigneau}
\end{center}
\vspace{-.2 in}
\end{figure}

\vi By the start of the $20^{th}$ century, Saigon already had all the marks of an urban city, but it did not exude the exotic vibe of a Far Eastern metropolis. The botanical gardens, the zoos, palaces, hotels, and now a church, these are emblems of the urban identity imposed on the colony by the French settlers. It was too modern, too French, and lacking the special Asian character. The colonial government began to address this imbalance by highlighting Vietnamese-French collaboration, an example of which is the statue of Monsignor Pigneau du Béhaine and the Crown Prince Nguyễn Phúc Cảnh (Figure~\ref{statue_pigneau}). Monsignor Pigneau was the Apostolic Vicar of Cochinchina. He helped Nguyễn Ánh defeat the other factions to ascend the throne in the 18th century. Commonly known as Bá Đa Lộc in Vietnam, Pigneau was a French legend known for his assistance with the Nguyễn and guidance of Prince Cảnh (Cherry 17). The emphasis on Pigneau's role in Prince Cảnh's life reiterated the importance of French presence in Cochinchina and Vietnam, where “the French name should be synonymous with progress, civilisation and true freedom”. The depiction of the French priest with a member of the royal family signified the colonial powers' intentions for Saigon, to portray the statue, and the city by extension, as an example of cooperation between the two peoples.
%to do: rethink paragraph structure
%functioned as a gateway for incorporating Vietnamese into this colonial society on the religious front, but also served as

As much as Notre-Dame was a marker of Christianity’s spread in Saigon and Vietnam, the cathedral has always been distinguished for its symbolic merits rather than its religious capacity. Even during the French occupation period, the cathedral more often fell into the tourist attraction category.  Given the limited potential for tourism of Saigon, such a monument was immediately earmarked for prospective visitors, and still is. The allure of architectural or historical substance of any form makes the destination an instant tourist hit in this otherwise dull neo-European city. Similar observations are made by Panivong Norindr, who presents the theory that Indochina is a “phantasm”, a fiction cultivated during the French colonial hegemony to indulge the French exotic fantasies and the nostalgia for grandeur (Norindr 2). On this romantic canvas where borders were remapped, arts appropriated, and environments built, colonial configurations like the Notre-Dame were generated en masse to construct an imaginary space that both showcased indigenous cultures while appealing to European aesthetics and the colonial ethos of civilization.

The concern with a lack of heritage also motivated later Vietnamese regimes to preserve certain colonial legacies (like the cathedral) while abolishing others. The Vietnamese Communist revolutionary force Việt Minh tore down the Pigneau statue in the August Revolution of 1945. On the old pedestal, a statue of the Virgin Mary was erected in 1959 by the Archdiocese of Saigon.  These transformations reveal the fraught nature of heritage conservation. Governments base the decision over what to preserve on its ideological implications than the actual cultural values. When it comes to architectural masterpieces whose cultural values may outweigh any ideological justification for demolition, depoliticization is often the solution. One such way is through romanticization to the extent where these sites are stripped bare of any historical context, as in the case of the Notre-Dame.

Today, the cathedral is publicly recognized by the municipal government and Saigon citizens as a symbol of the city, an image among several objects and spaces around which urban identity is formed (Figure~\ref{cathedral_present}). However, the kind of memory it elicits does not focus colonial exploitation, as is the case with the official narrative on the colonial period (cite Dat Do). Instead, its symbolic status is solely as an architectural classic, a sign of European civilization, amidst a wave of nostalgia for a former Saigon, even in its colonial years. To accompany this trend of aestheticizing the past, old colonial establishments have been “architecturally re-enhanced and its aura of ‘colonial distinction’ reinvented” to cater to nostalgia seekers (Ravi 476).  The Notre-Dame is undergoing its own restoration project, using materials imported from France and Germany and the construction was overseen by French and American contractors. Through its 140-year history, the meanings assigned to the Notre-Dame might have changed, but its symbolic function, as a colonial token of power in the French Indochina capital and as a cultural and historical heritage today, continues to exist, morphing from one form to another to accommodate the city’s transformations. \en

\begin{figure}[!ht]
\begin{center}
\woopic{cathedral_today}{0.3}
\vspace{-.2 in}
\caption[Present-day Notre-Dame Cathedral]{Present-day Notre-Dame-Cathedral. From: Pinterest \en}\label{cathedral_present}
\end{center}
\vspace{-.2 in}
\end{figure}

\section{Independence Palace}
\vi

On the theme of metamorphosis, no other site in Ho Chi Minh City has undergone modifications to its physical and figurative facade as the Independence Palace. The changes to this monument are both tangible and symbolic, as forces of history perform alterations on its structure, functions, and significance throughout the years. The palace, originally named Norodom after the current Cambodian King, was inaugurated in 1875. It became known as the Palais du Gouvernement-général (Governor’s Palace)  since serving as the headquarter of the Governor-General of Cochinchia (Doling 153). The construction was an expensive and extended affair. Nevertheless, the building’s functional usefulness never lived up to the exorbitant price tag. Rarely touched by the Governors of Cochinchina, who paled in importance to a new Governor-General in Ha Noi, the place fell into disrepair, its capacity downgraded to mostly ceremonial (Doling 154). In 1954, on the withdrawal of French military following the Geneva Accords, the palace was handed over to the new South Vietnamese administration headed by President Ngô Đình Diệm (Doling 155). Thie French edifice’s time as the new presidential palace did not last very long. In 1962, disaffected members of the Saigon air force staged a coup on the Diệm government and wrecked the old structure in the process (Doling 155). At this point, Diệm commissioned a new presidential complex to be built on the site of the demolished palace, under the supervision of the up and coming French-trained Vietnamese architect Nguyễn Viết Thụ. The newly built Independence Palace became the home of succeeding South Vietnamese presidents from 1966 until the fall of Saigon in 1975. After unification, the site became a historic monument and a museum; its main hall was renamed to Reunification and serves as a venue for ceremonial and entertainment functions.

Throughout the palace's history, there is a strong link between architecture and power. The Indochina regime was well-versed in employing colonial ideology to establish architectural hegemony over Saigon (cite Norindr Phantasmatic). With the Norodom Palace, even though its practicality fell short of expectations, the production of the monument set the trend for embracing symbolism in this architectural space. The colonial ethos of westernization and civilization came across in the neoclassical edifice, later supplanted by a modernist-oriental hybrid design under the South Vietnamese era (Doling 47-51). The American War period was dominated by a new generation of Vietnamese architects eager to combine Vietnamese architectural precedents with their European training (Truong, Vu 4). Nguyễn Viết Thụ’s palace design incorporated several Chinese characters in its exterior design (Figure~\ref{palace_chinese}). The symbolic power of architecture was well articulated by the words of a French publication on the opening of the Palais d’Exposition in France, a “museum of the colonies”: the Norodom/Independence Palace was a presentation of a “great modern state with the body of its political organization, the exact representation of its economic power, and the complete tableau of its social, intellectual, and artistic activity.” (Beauplan in Phantasmatic 25) If the Palais d’Exposition was designed to showcase the exoticism of French colonies, the Independence Palace was a reversed manifesto (Phantasmatic 24). The Saigon monument was a public display of French/South Vietnamese political and aesthetic visions, an exertion of power, and a claim of dominance of the entire city. \en

\begin{figure}[!ht]
\begin{center}
\woopic{chinese}{0.6}
\vspace{-.2 in}
\caption{\vi The Chinese characters in Ngô Viết Thụ's design. From top to bottom, left to right, the characters are mouth, loyalty, three, king, ruler, and rise. From: Independence Palace website \en}\label{palace_chinese}
\end{center}
\vspace{-.2 in}
\end{figure}

\vi The Independence Palace is a site of transition, its identity influx to accommodate whichever power was seeking to exploit its symbolism. To extend the idea that the palace is an expression of power, one can argue that the literal act of occupying the palace is synonymous with gaining control of Saigon. Compared to the other symbols of Saigon in the scope of this research, the Independence Palace the most connected to the city’s history. Almost every important turn of events since the $19^{th}$ was mirrored by the monument. During Japan’s brief imperial stint in Vietnam (1941 – 1945), their military heads resided in the palace, negotiating their shared control with the Vichy representatives over meals in its dining room (Dommen 78). At the height of Japanese imperialism in Asia, they even held French officials prisoners in the palace. If one makes the link between the governing regime and the palace occupiers, one would find that the same body of residents would also be the highest authority in Saigon, South Vietnam and sometimes Vietnam. From the French settlers to Japanese imperialists, from American militarists to South Vietnamese nationalists and the present-day socialists, staking claim over the palace becomes the constant for all external forces seeking a foothold in the city.

The most striking example of the symbolic merit of this architectural complex is the events of April 30, 1975, when a tank of the North Vietnamese Army bulldozed through one of the palace’s secondary gates and Lieutenant Bùi Quang Thận replaced the South Vietnamese flag on the roof with the National Liberation Front’s. Recital of this turn of events is invariably featured in history textbooks down to the details of the tank’s model and the lieutenant’s name. The photo of the tank crashing the gate (Figure~\ref{tank}) captures the most recognizable moment of this historic day in the Vietnamese national discourse. April 30  became a national holiday in Vietnam, Ngày giải phóng miền Nam, Thống nhất Đất nước (Day of liberating the South for national reunification, or Liberation Day for short (cite Vietnamese textbook). The use of this symbolic takeover to mark the end of the war demonstrates the current authorities’ continued desire to employ the palace as a site of memory, a place where narratives are produced and reshaped, imagined and rethought, in a struggle to define the City. \en

\begin{figure}[!ht]
\begin{center}
\woopic{ky-uc}{0.25}
\vspace{-.2 in}
\caption{\vi The tank crashing the gate of the palace. A tank of the same model is still on display in the palace's grounds.}\label{tank}
\end{center}
\vspace{-.2 in}
\end{figure}

\vi Commemoration is the unifying theme for postwar framing of the American War, and the Independence Palace is no exception (Cite Hue Tam Ho Tai, Viet Thanh Nguyen). Amidst other monuments such as museums, battlefields, cemeteries, the Independence Palace becomes a commemorative class of its own. Historian Jennifer Dickey points out the distinction between the palace and other war monuments in the city, suggesting that the former presidential residence provides a more upbeat take on the war than its counterparts in Saigon, such as the War Remnants Museum or the Củ Chi Tunnels Complex, which are more focused on the strategic aspect and the heavy casualties (Dickey 153). Today, the palace functions as a museum, with most parts open to visitors interested in the former Vietnamese “White House.” The whole presentation offers little context into the lives of the palace’s former residents, opting instead to centralize the official war narrative of a national struggle for unification against an imperial power and its puppet regime. A more subtle interpretation can be gleaned from the contrast between the excessive opulence on display as opposed to the scarcities in the North Vietnamese National Liberation Front. The Independence Palace, as well as a and other tourist sites in Ho Chi Minh City, is a tool for nation building, with which the state, as a producer and curator of spaces, legitimizes their authorities and establish a shared identity among its citizens, reinforced by these physical markers of nationalism, patriotism, and belongingness.

The Independence Palace exemplifies polysemous nature of Ho Chi Minh City’s architectural symbols. April 30 may be Liberation Day in Vietnam, but for much of the Vietnamese in the diaspora, it is remembered as the Fall of Saigon or Black April.  For them, the palace has a different connotation. It is a relic of a bygone era, witness to what many see is the ultimate betrayal by the U.S. that led to Saigon’s collapse (cite New Perspectives). Mitchell Owens writes about the palace in a New York Times article with the flippant description “East meets West, in a funky monument to wartime folly.” That nostalgia for the height of the Republic of Vietnam and its elusive governing families is tangible throughout the article and its wistful title, “Madame Nhu almost slept here.” 

\section{Bitexco Financial Tower}
At the height of 262m and a mere 5-minute walk from the bank of the Saigon river, the Bitexco Financial Tower looms over the historic business district of Ho Chi Minh (Figure~\ref{bitexco}). Like various other constructions in the city, the tower's location is water-oriented, but its relationship with water is less a dependency than a strategic arrangement, an icon towering above the swamp that makes up this city. Inaugurated in 2010, Bitexco was the tallest structure in Vietnam until the Gangnam Landmark Tower opened one year later in Ha Noi, but in Ho Chi Minh City, its height reord remained unsurpassed for eight years. Even though this title has been overtaken by another superstructure, Landmark 81 (461m), Bitexco’s status has endured as the symbol of modernity in the “new” Saigon. The plan for Bitexco to define the urban landscape of Saigon was already in motion during its construction. Before the tower’s completion in 2010, the owner and development group Bitexco and the press had been touting its iconic stature as the new beacon of Saigon’s innovation and design. According to the Venezuelan American architect in charge Carlos Zapata, Bitexco was an “iconic embodiment of the energy and aspirations of the Vietnamese people.”  As far as the history of landmarks in Ho Chi Minh is concerned, this building is yet another site where narratives are defined and memories constructed, a playground for contesting ideas, ideals, and powers.
\en

\begin{figure}[!ht]
\begin{center}
\woopic{bitexco}{0.25}
\vspace{-.2 in}
\caption{\vi Bitexco from a distance. From: reatimes.vn}\label{bitexco}
\end{center}
\vspace{-.2 in}
\end{figure}

\vi From the development stages, the developers of Bitexco had shown ambitions for building an era-defining structure through their design concepts. The form of the tower was to embrace the traditional and philosophical symbol of a lotus bud, significant for its importance in Buddhism. The lotus flower represents enlightenment and purification (Tran Van Khai 1). The acclaimed national flower of Vietnam features in various folk poems:
\begin{verse}
\begin{center}
\hspace{2em} In swamps the lotus shines, \\
Green leaves, white flowers, fine stamens.\\
\hspace{2em} Blooms, leaves, and stamens gold,\\
Near mud without the moldy stink.\footnote{Luxury and Rubble}
\end{center}
\end{verse}

Said to grow in mud and bloom when reaching light, the lotus flower purifies where it grows and blossoms even in stagnant water. Its symbolism becomes a metaphor for the vitality and strength of the Vietnamese people and nation. Embracing the loaded image of the iconic lotus bud, the Bitexco owners sought to present the tower as a budding symbol rising above the swampy landscape of Saigon (both in terms of its geographical and sprawling demographic conditions). This use of cultural material in modern architecture appeals to a shared notion about the Vietnamese national identity. The Bitexco tower as a site of memory, while contributing to the national narrative of self-determination, also gains its recognizability by enabling citizens to “identify with its past and present as a political, cultural and social entity.” (Mark Crinson xiv) 

Bitexco is an example of the modern/traditional dichotomy prevalent in Vietnamese discourse of urban development. In this symbiotic relationship, the lotus bud structure was realized by high-class technologies. Its futuristic architectural style rejects the modernist trend that has dominated Saigon since the end of the First Indochina War, marked by monuments such as the Independence Palace (Tran Van Khai 1). The transition from old architectural patterns to new ones projects the inherent shift in the city’s identity, characterized by changing notions of beauty and revised urban codes. These new high-rises are the image of the integrated, modern, global city, defined by architectural ingenuity and stylistic aesthetics. A gravity-defying structure like a skyscraper in itself is a testament to advancement the sciences. Its convex sides and rounded corners are a step further from the convention, an embodiment of innovation and uniqueness. The presentation of the city in this sky-piercing depiction aims to showcase Saigon’s strides in modernization, urbanization, and rapid economic integration by means of symbolism, sign, form, scale, and materiality (Tran Van Khai 7). As an example of forward thinking and excellence in designed, the landmark was built to “surprise, to astonish, and to alter perspectives,” while still laden with ideals about culture, tradition, and history in this futuristic and unorthodox architectural rendering. 

Like the geographic and political processes in Ho Chi Minh City’s history, the transformation into the contemporary notion of modernity and urbanization exemplified by these buildings is far from straightforward. Nature-defying structures like Bitexco require tremendous alterations to the environment. Working with the soft soil conditions of the Saigon river alluvial plain, engineers had to sink the piles in the base of the building 75m underground in addition to creating a concrete mat foundation to ensure the skyscraper could withstand the force of nature (and water) that so often plagues development plans in Saigon. Natural modifications were not the only violent processes to enable the monument’s existence; controlling the human factor is also an important variable in building the perfect city. The emergence of new urban zones in Ho Chi Minh City is synonymous with mass displacement of people, in a process that is oftentimes harsh and involuntary. The current discourse of urbanization is redefining the notion of a modern city to be synonymous with “grand buildings framing breathable and beautiful open spaces” (Haims 737 beauty). Urban programs of spatial cleansing forcefully clear the city of its slums to make way for an emerging urban upper middle class. The politics of control through urban building codes manifest in the additions of high-rises like Bitexco and recently Landmark 81. Land use becomes contentious as new projects reach lower-class residential and communal spaces, creating social stratification along the city’s spatial lines.

The state was not the only players in the changing façade of Ho Chi Minh City with their zoning and urban projects. With the rise of private real estate projects comes the participation of private corporations in reimagining the city. Agents such as Bitexco, or more recently Vingroup (owner of the current highest building in Saigon), see these parcels of land as “a surface area on which to manage economies of scale.” The kind of memory engendered by these architectural structures and complexes legitimizes development plans for other peri-urban neighborhoods. The memory industry becomes lucrative in the face of diminishing unused urban land and climbing real estate prices. How enterprises justify the displacement of people and razing of current neighborhoods to make place for new middle-class urban zones is built upon ideologies about urbanization, development, and progress that continue to be reinforced with renderings such as the Bitexco Financial Tower. But urban planners also take another measure to ensure the validity of their constructions for citizens of Ho Chi Minh City by hiring French, Japanese, and American contractors in key design phases (Harms notion of beauty 743). In this sense, postwar late-socialist Vietnam has not moved away from previous Western (colonial) configurations and ideals about how a good environment looks and functions, enforced through urban codes about hygiene, order, and beauty (cite Gwendolyn Wright 1, Harms 737). Key players outside of the state have carved their space in the urban planning business of Saigon, and by extension, the memory created by these spatial transformations.

Outside of its symbolic capacity, the Bitexco Financial Tower serves all the usual functions associated with mix-use high-rises: 370,000m2 of office space, a 10,000m2 retail podium, completed with a sky deck and helipad cantilevers from the 52nd floor (cite Tran Van Khai 6). The Grade A office suites are reserved for elite finance organizations and multinational corps (cite tran Van Khai 8). The shopping center features retailers from high-end brands, as well as dining and entertainment services.  The experience provided by the shopping area of Bitexco is indicative of the upward trend in shopping malls in Vietnam. There are currently more than 40 shopping centers in Ho Chi Minh City, most of which feature some combination of a food court, a movie theater, and middle to high-end stores.  Even though the clientele of these shops is usually upper middle class, the restaurants, cafes, and movie theater can attract large crowds on week nights and weekends. Bitexco is best known for its cinema and rooftop bars. These spots, often underground or dozens of meters above ground, provide the new communal spaces for Saigon, whose public sphere is characterized by its growing consumerist tendencies.

The desire for urban landscape to manufacture an identity is made explicit in Bitexco’s case. Following the ground-breaking of the Bitexco Financial Tower, the chairman of the Board of Directors of Bitexco Group, Vũ Quang Hội, explained their visions for the construction: “Before, when mentioning Vietnam, people often thought of war and poverty… [Now] the image of the Bitexco Financial Tower will appear on all the postcards and souvenirs, so that when friends from all over the world come to Vietnam, they will bring home the image of a nation of innovation and development”. Architecture is not simply a representation of the present but a reconfiguration of the past, where new buildings replaced old monuments in a restructuring of both the physical and mental landscape. For a private enterprise like Bitexco to become the "icon builder and creator of future values," as the Vietnam National Real Estate Association press calls it, it is clear that Saigon's memory, as embodied by these structures, is not just rooted in the past, or the present, but also in an idealistic orientation for the future that concerns notions of beauty that favor a capitalist and consumerist trajectory of urban development.\footnote{http://reatimes.vn/bitexco-nguoi-di-xay-bieu-tuong-va-kien-tao-gia-tri-tuong-lai-21299.html}

\section{Creating Memory Targets}
\en

%!TEX root = ../username.tex
\chapter[The AR Model for Historical Memory Studies]{The AR Model for Historical Memory Studies: Results \& Discussion}\label{results}
The previous two chapters suggest two AR experiences for visualizing memory, particularly focusing on its dynamic nature. The different layers of the research is demonstrated through a separate component in augmented reality. Historical transformation comes across through spatial juxtaposition. The primary source-centered narrative is built on object recognition, while perspective is realized by the interactability of the exhibit items. The success of these individual elements depends on the effectiveness of each technological component and the AR system as a whole.

Overall, feature detection works well for object and image recognition. Despite some limitations with repetitive and low-contrast images, tracking is robust and efficient for most images. Once a pose has been established, tracking persists even when only a fraction of the target is in frame. The \textbf{Extended Mode} provides the option for maintaining the established pose even when the target is out of frame. 3D objects prove to be more a more challenging target for detection than 2D images. The Vuforia Object Scanner only registers larger objects with a large number of features. Objects with great difference in contrast also perform better. 3D printed models can prove to be a challenge. On the other hand, the augmentability of image targets is indicated on the Vuforia Target Manager when targets are created and Object Scanner allows instantaneous testing of targets after the scan, so that users do not have to wait until the testing phase in Unity to determine the usability of a target.

Notwithstanding the difficulties with tracking, once target detection and pose calculation are complete, the tracking behavior is relatively stable for the duration of run time. Augmented contents are highly responsive to changes in targets' poses. As mentioned previously, as long as part of the target is still in frame, the pose is maintained. For image targets, this means that augmented content remains active even when a flat target is extremely tilted. Such behavior facilitates great freedom of movement for the audience; the potential for unimpeded interactions is not hampered by technical restrictions. For object targets, the range of possible interactions is even more expansive, since flat image targets are only tracked and augmentable on one plane (the plane of the image) while object targets are augmentable from almost all angles. The only untracked plane is its bottom, which is hidden from the scan. Another limitation with Vuforia is that it does not support the simultaneous detection and augmentation of multiple object targets in the same frame.

While technical strengths and issues of the system are immediately observable, its effectiveness in terms of conveying the narrative depend on many factors. The simple AR experience cannot cover the depths of the historical research without overwhelming the audience. The tradeoff between simplicity and complexity is a balance that public history efforts constantly struggle with. The most compelling advantage that augmented reality provides is active engagement with primary sources. Handheld cross-platform AR experiences are portable and social; users are encouraged to engage in discussions or collaborate. However, since the experience is on a personal device, each user can explore the exhibit and the app at their own pace in a self-customized session. Even if the historical narrative has to be simplified, augmented reality provides affordances that are unique to digital storytelling that broaden the boundaries of traditional methods of engagement.

As the function of memory studies in history is to uncover this dynamic between remembering and its agents, the purpose of this application in particular is to visually extract what has become muddled by the human mind. By superimposing layers of historic map, the experience hopes to present the sites of memory in the history of Ho Chi Minh City in a way that shifts the agency to the audience. In the previous discussion of power and mapping, it is the ownership of knowledge and wealth that enables the creation of maps and architecture. This app provides users with the means to redress this power balance, to deconstruct the processes and practices imposed on them by cartographers and architects. In this way, the app is also a site of memory, a mnemonic device seeking to partake in the shaping of memory, but rather than dictating the perspective, it offers the freedom to decide the angles from which the past should be viewed. This activity encourages historical awareness, particularly towards forces of memory. Space is an important trigger of memory. The sources of spatial memory can be direct or indirect (cite Allen Human Spatial Memory 252). Knowledge of an environment can come from direct interactions via sensorimotor activities such as walking and observing, or from an indirect source like maps. This AR experience captures the essence of both direct and indirect sources, by designing affordances that cater to both modes of acquisition.

Much as augmented reality can empower the visualization of memory studies, as far as memory is concerned, such an application is subject to the same dangers that it seeks to deconstruct. Technology, knowledge, and power are intertwined. The use of the AR technology to create a new kind of knowledge is a kind of power on its own. It is important to consider the ethics of such a practice and its principles. The choice of which maps and buildings to use and their presentation are influenced by the need for a clear, thematic, and augmentable narrative. As a result, contextualization is crucial for understanding as well as for transparency.

%!TEX root = ../username.tex
\chapter{Conclusion}\label{conclusion}
\vi
From the perspective of memory, the Saigon urban space is a locus where multiple identities and narratives about the city are created and interact. The materials of this kind of memory are spaces such as maps and buildings that articulate the mindset of their creators, steeped in biases and intentions. In this quest to construct a city, planners and builders are also constructing a specific way to use and ultimately remember these spaces. Maps represent ideas that, once built in reality, become ideologies. Architecture exemplifies the nature of this transition from imagined to concrete. Through the creation of these structures, agents of memory like the municipal leadership, private enterprises or citizens themselves are generating a mechanism for recollecting the past that is built into the city’s own landscape. Together, the city’s sites of memory collectively produce an urban memory, which does not just indicate how the city is remembered but considers the city as “a physical landscape and collection of objects and practices that enable recollections of the past and that embody the past through traces of the city’s sequential building and rebuilding” (Crinson xii). This definition of urban memory by Mark Crinson also applies to other facets of the city such as artworks, novels, and events.

The connection between memory and power is what makes memory such an attractive avenue for various groups with a stake in the city. Spatial markers like skyscrapers or even boulevards and bridges provoke recollections of the past. Collective recollection by a city’s population enables them to identify with its history and present, providing the grounds for developing a collective identity with shared pride in the mutual past. Going back to the example of Ho Chi Minh City, for its governing powers, constructing a collective identity was instrumental in this fragmented society that was never completely Vietnamese, Chinese, or French. Using the phrase by Benedict Anderson, the fostering of an “imagined community” through instigators like maps and buildings was what enabled these agents of memory to mobilize people and resources for their purposes, whether it was colonial exploitation, civil war, or industrial production. 

As with any kind of memory processes, the making of Ho Chi Minh City and its memory is subject to coercion, loss and distortion. On the one hand, the physical process of building the landscape often involves violent removal of existing designs, whether those are marshes or communities. It can also lead to forced changes in function for the spaces in question. The Bến Nghé area under French colonization is an example of the functions of a colonial administrative headquarter was imposed on an urban area. Cartography and architecture also institute symbolic meanings through their manipulation of memory. The psychological nature of memory recollection implies loss and distortion. In the place of accurate historical representations are prompts for recollections that exploit the fickleness of remembering to institute decontextualized or synthesized memory.  The result is a kind of romantic and glorified nostalgia that should describe brutal histories but is stripped of any context, a memory without the pain. Disruptive and delusive techniques like compulsion, suppression and censorship allow for the production of specific narratives born out of political and commercial expediency.

All the same, to view memory as an expediently constructed product does not mean rejecting the city’s legacy; rather, it calls for a reevaluation of the definition of legacy. Ho Chi Minh City is not a place without history. It is the implication of this complicated history that creates the need for memory, which is a revised and streamlined version of the past. This project is a critique of the production and curation of these alternate narratives. At a time where the past is everywhere and nowhere and the business of memory booms with the commercialization of memorabilia, the context becomes lost among sentimentality and misguided notions of aesthetic. The demand for memory drives selective preservation. What is branded and resold as legacy is only an uncomplicated part of history, while the other messier aspects of colonialism, civil war, forced cultural assimilation, and environmental crises get obfuscated. Going back to Ho Chi Minh City’s legacy, this urban center has always been diverse throughout its history, a product of thriving trade activities and an intersection of various cultural influences, but its sites of memory, the maps and the architecture, are not products of its diverse and vital culture. These efforts are not only intended to be represent the city’s culture but to communicate the authority of the cartographers and architects behind them. Consumers of memory should be aware of these levels of subtext the next time they visit the Notre-Dame Cathedral and walk across the street to look at old Saigon maps in the City Post Office. And when the next skyscraper claims the title of city symbol, one should pause and ask what it symbolizes after all.

With these considerations, what augmented reality provides is a new dimension for critically engaging with memory. Technology helped create sites of memory such as maps and buildings, and technology can help deconstruct them. The motivation for using augmented reality arises from the complexities of historical subjects in general, and memory specifically. The AR model in this project is centered on spatial interactability of primary sources to exploit the spatial dimension of urban memory. Object recognition, which hinges on feature detection algorithms, is the backbone of the object-centered experience. The results demonstrate that AR can reliably be used for a variety of source materials. There are many potentials for augmented reality when it comes to non-traditional methods for studying and visualizing history.

On the other hand, the expansion of augmented reality in fields such as heritage preservation, tourism, education, and museum also creates the need for a theoretical framework for applying this technology to history. The same complexities with memory can also be applied to the use of augmented reality. The use of technology of any kind in public history is implicated in the entanglement of knowledge, power, and memory. Who owns technology, who gets to see with augmented reality, and what it depicts, these are all valid concerns. While it is important that scholarship evolves to embrace the new tools technology has to offer, the methodology and ethics of such practices must be taken into consideration to ensure that creators and users of these experiences are critical and cognizant in their usage. In addition, historiographical dialogues on the topic of science should not be limited to humanities disciplines only. In the true spirit of digital humanities, technology developers should also engage in conversations about the history and implications of their creations.

%%%%%%%%%%%%%%%%%%%%%%%%%%%%%%%%%%%%%%%%%%%%%%%%%%%%%%%
%
%  This section starts the back matter. The back matter includes appendices, indicies, and the
%  bibliography
%
%%%%%%%%%%%%%%%%%%%%%%%%%%%%%%%%%%%%%%%%%%%%%%%%%%%%%%%

\backmatter

%%!TEX root = ../username.tex
%%%%%%%%%%%%%%%%%%%%%%%%%%%%%%%%%%%%%%%%%%%%%%%%%%%%%%%%%%%%%%%%
% Contents: Math typesetting with LaTeX
% $Id: math.tex,v 1.3 2005/05/21 02:03:43 jonb Exp $
%%%%%%%%%%%%%%%%%%%%%%%%%%%%%%%%%%%%%%%%%%%%%%%%%%%%%%%%%%%%%%%%%

\chapter{Typesetting Mathematical Formulae}\label{math}
\begin{intro}
  This appendix is taken from \citet{ophs03} under the GNU open source documentation license. This appendix addresses the main strength
  of \TeX{}: mathematical typesetting. But be warned, this appendix
  only scratches the surface. While the things explained here are
  sufficient for many people, don't despair if you can't find a
  solution to your mathematical typesetting needs here. It is highly likely
  that your problem is addressed in \AmS-\LaTeX{}%
  \footnote{\texttt{CTAN:/tex-archive/macros/latex/packages/amslatex}}
  or some other package.
\end{intro}
  
\section{General}

\LaTeX{} has a special mode for typesetting mathematics\index{mathematics}.
Mathematical text within a paragraph is entered between \verb|\(|\index{\(@\verb+\(+}
and \verb|\)|\index{\)@\verb+\)+}, %$
between \texttt{\$} and \texttt{\$} or between
\verb|\begin{|{math}\verb|}| and \verb|\end{math}|.\index{formulae}

\begin{singlespace}
\begin{example}
Add $a$ squared and $b$ squared 
to get $c$ squared. Or, using 
a more mathematical approach:
$c^{2}=a^{2}+b^{2}$
\end{example}
\end{singlespace}

\begin{singlespace}
\begin{example}
\TeX{} is pronounced as 
$\tau\epsilon$.\\[6pt]
100~m$^{3}$ of water\\[6pt]
This comes from my $\heartsuit$
\end{example}
\end{singlespace}

It is preferable to \emph{display} larger mathematical equations or formulae,
rather than to typeset them on separate lines. This means you enclose them
in \verb|\[| \index{\[@\verb+\[+} and \verb|\]| \index{\]@\verb+\]+} or between
\verb|\begin{|displaymath\index{displaymath}\verb|}| and
  \verb|\end{displaymath}|.  This produces formulae which are not
numbered. If you want \LaTeX{} to number them, you can use the
equation\index{equation} environment.


\begin{singlespace}
\begin{example}
Add $a$ squared and $b$ squared 
to get $c$ squared. Or, using 
a more mathematical approach:
\begin{displaymath}
c^{2}=a^{2}+b^{2}
\end{displaymath}
And just one more line.
\end{example}
\end{singlespace}

You can reference an equation with \ic{label} and \ic{ref}

\begin{singlespace}
\begin{example}
\begin{equation} \label{eq:eps}
\epsilon > 0
\end{equation}
From (\ref{eq:eps}), we gather 
\ldots
\end{example}
\end{singlespace}

Note that expressions will be typeset in a different style if displayed:

\begin{singlespace}
\begin{example}
$\lim_{n \to \infty} 
\sum_{k=1}^n \frac{1}{k^2} 
= \frac{\pi^2}{6}$
\end{example}
\end{singlespace}
\begin{singlespace}
\begin{example}
\begin{displaymath}
\lim_{n \to \infty} 
\sum_{k=1}^n \frac{1}{k^2} 
= \frac{\pi^2}{6}
\end{displaymath}
\end{example}
\end{singlespace}

There are differences between \emph{math mode} and \emph{text mode}. For
example in \emph{math mode}: 

\begin{enumerate}

\item Most spaces and linebreaks do not have any significance, as all spaces
either are derived logically from the mathematical expressions or
have to be specified using special commands such as \verb|\,| \index{''\,@\verb+\,+}, \ic{quad}, or \ic{qquad}.
 
\item Empty lines are not allowed. Only one paragraph per formula.

\item Each letter is considered to be the name of a variable and will be
typeset as such. If you want to typeset normal text within a formula
(normal upright font and normal spacing) then you have to enter the
text using the \verb|\textrm{...}| commands.
\end{enumerate}


\begin{singlespace}
\begin{example}
\begin{equation}
\forall x \in \mathbf{R}:
\qquad x^{2} \geq 0
\end{equation}
\end{example}
\end{singlespace}

\begin{singlespace}
\begin{example}
\begin{equation}
x^{2} \geq 0\qquad
\textrm{for all }x\in\mathbf{R}
\end{equation}
\end{example}
\end{singlespace}

Mathematicians can be very fussy about which symbols are used:
it would be conventional here to use `blackboard bold\index{blackboard bold}',
bold symbols\index{bold symbols} which is obtained using \ic{mathbb} from the
package \ip{amsfonts} or \ip{amssymb}.

\ifx\mathbb\undefined\else
The last example becomes
\begin{singlespace}
\begin{example}
\begin{displaymath}
x^{2} \geq 0\qquad
\textrm{for all }x\in\mathbb{R}
\end{displaymath}
\end{example}
\end{singlespace}
\fi

\section{Grouping in Math Mode}

Most math mode commands act only on the next character. So if you
want a command to affect several characters, you have to group them
together using curly braces: \verb|{...}|.

\begin{singlespace}
\begin{example}
\begin{equation}
a^x+y \neq a^{x+y}
\end{equation}
\end{example}
\end{singlespace}
 
\section{Building Blocks of a Mathematical Formula}

In this section, the most important commands used in mathematical
typesetting will be described. Take a look at \citet{kd03} for a detailed list of commands for typesetting
mathematical symbols.

\textbf{Lowercase Greek letters\index{Greek letters}} are entered as \verb|\alpha|,
 \verb|\beta|, \verb|\gamma|, \ldots, uppercase letters
are entered as \verb|\Gamma|, \verb|\Delta|, \ldots\footnote{There is no
  uppercase Alpha defined in \LaTeXe{} because it looks the same as a
  normal roman A. Once the new math coding is done, things will
  change.} 

\begin{singlespace}
\begin{example}
$\lambda,\xi,\pi,\mu,\Phi,\Omega$
\end{example}
\end{singlespace}
 
\textbf{Exponents and Subscripts} can be specified using\index{exponent}\index{subscript}
the \verb|^|\index{^@\verb+^+} and the \verb|_|\index{_@\verb+_+} character.

\begin{singlespace}
\begin{example}
$a_{1}$ \qquad $x^{2}$ \qquad
$e^{-\alpha t}$ \qquad
$a^{3}_{ij}$\\
$e^{x^2} \neq {e^x}^2$
\end{example}
\end{singlespace}

The \textbf{square root\index{square root}} is entered as \ic{sqrt}, the
$n^\mathrm{th}$ root is generated with \verb|\sqrt[|$n$\verb|]|. The size of
the root sign is determined automatically by \LaTeX. If just the sign
is needed, use \ic{surd}.

\begin{singlespace}
\begin{example}
$\sqrt{x}$ \qquad 
$\sqrt{ x^{2}+\sqrt{y} }$ 
\qquad $\sqrt[3]{2}$\\[3pt]
$\surd[x^2 + y^2]$
\end{example}
\end{singlespace}

The commands \ic{overline} and \ic{underline} create
\textbf{horizontal lines} directly over or under an expression.
\index{horizontal!line}

\begin{singlespace}
\begin{example}
$\overline{m+n}$
\end{example}
\end{singlespace}

The commands \ic{overbrace} and \ic{underbrace} create
long \textbf{horizontal braces} over or under an expression.
\index{horizontal!brace}

\begin{singlespace}
\begin{example}
$\underbrace{ a+b+\cdots+z }_{26}$
\end{example}
\end{singlespace}

\index{mathematical!accents} To add mathematical accents such as small
arrows or {tilde} signs to variables, you can use the commands
given in \citet{kd03}.  Wide hats and
tildes covering several characters are generated with \ic{widetilde}
and \ic{widehat}.  The \verb|'|\index{'@\verb+'+} symbol gives a
prime\index{prime}.
% a dash is --

\begin{singlespace}
\begin{example}
\begin{displaymath}
y=x^{2}\qquad y'=2x\qquad y''=2
\end{displaymath}
\end{example}
\end{singlespace}

\textbf{Vectors}\index{vectors} often are specified by adding a small
arrow symbol\index{arrow symbols} on top of a variable. This is done with the
\ic{vec} command. The two commands \ic{overrightarrow} and
\ic{overleftarrow} are useful to denote the vector from $A$ to $B$.

\begin{singlespace}
\begin{example}
\begin{displaymath}
\vec a\quad\overrightarrow{AB}
\end{displaymath}
\end{example}
\end{singlespace}

Names of log-like functions are often typeset in an upright
font and not in italic like variables. Therefore \LaTeX{} supplies the
following commands to typeset the most important function names:
\index{mathematical!functions}

\begin{singlespace}
\begin{verbatim}
\arccos   \cos    \csc   \exp   \ker     \limsup  \min   \sinh
\arcsin   \cosh   \deg   \gcd   \lg      \ln      \Pr    \sup
\arctan   \cot    \det   \hom   \lim     \log     \sec   \tan
\arg      \coth   \dim   \inf   \liminf  \max     \sin   \tanh
\end{verbatim}
\end{singlespace}

\begin{singlespace}
\begin{example}
\[\lim_{x \rightarrow 0}
\frac{\sin x}{x}=1\]
\end{example}
\end{singlespace}

For the modulo function\index{modulo function}, there are two commands: \ic{bmod} for the
binary operator ``$a \bmod b$'' and \ic{pmod}
for expressions
such as ``$x\equiv a \pmod{b}$.''

A built-up \textbf{fraction\index{fraction}} is typeset with the
\ic{frac}\verb|{...}{...}| command.
Often the slashed form $1/2$ is preferable, because it looks better
for small amounts of `fraction material.'

\begin{singlespace}
\begin{example}
$1\frac{1}{2}$~hours
\begin{displaymath}
\frac{ x^{2} }{ k+1 }\qquad
x^{ \frac{2}{k+1} }\qquad
x^{ 1/2 }
\end{displaymath}
\end{example}
\end{singlespace}

To typeset binomial coefficients or similar structures, you can use
either the command \linebreak \ic{binom}\{\emph{num}\}\{\emph{denom}\} or \ic{genfrac}\{\emph{ldelim}\}\{\emph{rdelim}\}\{\emph{thickness}\}\{\emph{style}\}\{\emph{num}\}\{\emph{denom}\}. The second command can be used to produce customized fraction like output and more information can be found in \citet{mgbcr04}.

\begin{singlespace}
\begin{example}
\begin{displaymath}
\binom{n}{k}\qquad 
\genfrac{}{}{0pt}{}{x}{y+2}
\end{displaymath}
\end{example}
\end{singlespace}
 
\medskip

The \textbf{integral operator\index{integral operator}} is generated with \ic{int}, the
\textbf{sum operator\index{sum operator}} with \ic{sum}. The upper and lower limits
are specified with~\verb|^|\index{^@\verb+^+} and~\verb|_|\index{_@\verb+_+} like subscripts and superscripts.

\begin{singlespace}
\begin{example}
\begin{displaymath}
\sum_{i=1}^{n} \qquad
\int_{0}^{\frac{\pi}{2}} \qquad
\end{displaymath}
\end{example}
\end{singlespace}

For \textbf{braces\index{braces}} and other delimiters\index{delimiters}, there exist all
types of symbols in \TeX{} (e.g.~$[\;\langle\;\|\;\updownarrow$).
Round and square braces can be entered with the corresponding keys,
curly braces with \verb|\{|, all other delimiters are generated with
special commands (e.g.~\verb|\updownarrow|). For a list of all
delimiters available, check \citet{kd03}.

\begin{singlespace}
\begin{example}
\begin{displaymath}
{a,b,c}\neq\{a,b,c\}
\end{displaymath}
\end{example}
\end{singlespace}

If you put the command \ic{left} in front of an opening delimiter or
\ic{right} in front of a closing delimiter, \TeX{} will automatically
determine the correct size of the delimiter. Note that you must close
every \ic{left} with a corresponding \ic{right}, and that the size is
determined correctly only if both are typeset on the same line. If you
don't want anything on the right, use the invisible `\verb|\right .|\index{commands!right@\verb+right .+}'!

\begin{singlespace}
\begin{example}
\begin{displaymath}
1 + \left( \frac{1}{ 1-x^{2} }
    \right) ^3
\end{displaymath}
\end{example}
\end{singlespace}

In some cases it is necessary to specify the correct size of a
mathematical delimiter\index{mathematical!delimiter} by hand,
which can be done using the commands \ic{big}, \ic{Big}, \ic{bigg} and
\ic{Bigg} as prefixes to most delimiter commands.\footnote{These
  commands do not work as expected if a size changing command has been
  used, or the \texttt{11pt} or \texttt{12pt} option has been
  specified.  Use the exscale\index{exscale} or amsmath\index{amsmath} packages to
  correct this behaviour.}

\begin{singlespace}
\begin{example}
$\Big( (x+1) (x-1) \Big) ^{2}$\\
$\big(\Big(\bigg(\Bigg($\quad
$\big\}\Big\}\bigg\}\Bigg\}$\quad
$\big\|\Big\|\bigg\|\Bigg\|$
\end{example}
\end{singlespace}

To enter \textbf{three dots\index{three dots}} into a formula, you can use several
commands. \ic{ldots} typesets the dots on the baseline, \ic{cdots}
sets them centered. Besides that, there are the commands \ic{vdots} for
vertical and \ic{ddots} for diagonal dots\index{diagonal dots}.\index{vertical
  dots}\index{horizontal!dots} You can find another example in section~\ref{sec:vert}.

\begin{singlespace}
\begin{example}
\begin{displaymath}
x_{1},\ldots,x_{n} \qquad
x_{1}+\cdots+x_{n}
\end{displaymath}
\end{example}
\end{singlespace}
 
\section{Math Spacing}

\index{math spacing} If the spaces within formulae chosen by \TeX{}
are not satisfactory, they can be adjusted by inserting special
spacing commands. There are some commands for small spaces: \verb|\,| \index{\,@\verb+\,+} for
$\frac{3}{18}\:\textrm{quad}$ (\demowidth{0.166em}), \verb|\:| \index{\:@\verb+\:+} for $\frac{4}{18}\:
\textrm{quad}$ (\demowidth{0.222em}) and \verb|\;| \index{\;@\verb+\;+} for $\frac{5}{18}\:
\textrm{quad}$ (\demowidth{0.277em}).  The escaped space character
\verb*.\ . generates a medium sized space and \ic{quad}
(\demowidth{1em}) and \ic{qquad} (\demowidth{2em}) produce large
spaces. The size of a quad corresponds to the width of the
character `M' of the current font.  The \verb|\!|\index{"\"!@\texttt{\bs"!}} command produces a
negative space of $-\frac{3}{18}\:\textrm{quad}$ (\demowidth{0.166em}).

\begin{singlespace}
\begin{example}
\newcommand{\rd}{\mathrm{d}}
\begin{displaymath}
\int\!\!\!\int_{D} g(x,y)
  \, \rd x\, \rd y 
\end{displaymath}
instead of 
\begin{displaymath}
\int\int_{D} g(x,y)\rd x \rd y
\end{displaymath}
\end{example}
\end{singlespace}
Note that `d' in the differential is conventionally set in roman.

\AmS-\LaTeX{} provides another way for fine tuning
the spacing between multiple integral signs,
namely the \ic{iint}, \ic{iiint}, \ic{iiiint}, and \ic{idotsint} commands.
With the \ip{amsmath} package loaded, the above example can be
typeset this way:

\begin{singlespace}
\begin{example}
\newcommand{\rd}{\mathrm{d}}
\begin{displaymath}
\iint_{D} \, \rd x \, \rd y
\end{displaymath}
\end{example}
\end{singlespace}

See the electronic document testmath.tex (distributed with
\AmS-\LaTeX) or Chapter 8 of ``The LaTeX Companion''\footnote{
available at \texttt{CTAN:/tex-archive/info/ch8.*}.} for further details.

\section{Vertically Aligned Material}
\label{sec:vert}

To typeset \textbf{arrays}, use the \texttt{array}\index{array} environment. It works
somewhat similar to the \texttt{tabular} environment. The \verb|\\| command is
used to break the lines.

\begin{singlespace}
\begin{example}
\begin{displaymath}
\mathbf{X} =
\left( \begin{array}{ccc}
x_{11} & x_{12} & \ldots \\
x_{21} & x_{22} & \ldots \\
\vdots & \vdots & \ddots
\end{array} \right)
\end{displaymath}
\end{example}
\end{singlespace}

The \texttt{array}\index{array} environment can also be used to typeset expressions which have one
big delimiter by using a ``\verb|.|'' as an invisible right\index{commands!right@\verb+right .+} 
delimiter:

\begin{singlespace}
\begin{example}
\begin{displaymath}
y = \left\{ \begin{array}{ll}
 a & \textrm{if $d>c$}\\
 b+x & \textrm{in the morning}\\
 l & \textrm{all day long}
  \end{array} \right.
\end{displaymath}
\end{example}
\end{singlespace}


For formulae running over several lines or for equation systems\index{equation systems},
you can use the environments \texttt{eqnarray}\index{eqnarray}, and \verb|eqnarray*|
instead of \texttt{equation}. In \texttt{eqnarray} each line gets an
equation number. The \verb|eqnarray*| does not number anything.

The \texttt{eqnarray} and the \verb|eqnarray*| environments work like
a 3-column table of the form \verb|{rcl}|, where the middle column can
be used for the equal sign or the not-equal sign. Or any other sign
you see fit. The \verb|\\| command breaks the lines.

\begin{singlespace}
\begin{example}
\begin{eqnarray}
f(x) & = & \cos x     \\
f'(x) & = & -\sin x   \\
\int_{0}^{x} f(y)dy &
 = & \sin x
\end{eqnarray}
\end{example}
\end{singlespace}

\noindent Notice that the space on either side of the 
the equal signs is rather large. It can be reduced by setting
\verb|\setlength\arraycolsep{2pt}|, as in the next example.

\index{long equations} \textbf{Long equations} will not be
automatically divided into neat bits.  The author has to specify
where to break them and how much to indent. The following two methods
are the most common ones used to achieve this.

\begin{singlespace}
\begin{example}
{\setlength\arraycolsep{2pt}
\begin{eqnarray}\notag
\sin x & = & x -\frac{x^{3}}{3!}
     +\frac{x^{5}}{5!}-{}
                   \\\notag
 & & {}-\frac{x^{7}}{7!}+{}\cdots
\end{eqnarray}}
\end{example}
\end{singlespace}
\pagebreak[1]

\begin{singlespace}
\begin{example}
\begin{eqnarray}\notag
\lefteqn{ \cos x = 1
     -\frac{x^{2}}{2!} +{} }
                   \\\notag
 & & {}+\frac{x^{4}}{4!}
     -\frac{x^{6}}{6!}+{}\cdots
\end{eqnarray}
\end{example}
\end{singlespace}

\enlargethispage{\baselineskip}

\noindent The \ic{notag} command causes \LaTeX{} to not generate a number for
this equation.

It can be difficult to get vertically aligned equations to look right
with these methods; the package amsmath\index{amsmath} provides a more
powerful set of alternatives.

\section{Math Font Size}

\index{math font size} In math mode, \TeX{} selects the font size
according to the context. Superscripts, for example, get typeset in a
smaller font. If you want to typeset part of an equation in roman,
don't use the \ic{textrm} command, because the font size switching
mechanism will not work, as \verb|\textrm| temporarily escapes to text
mode. Use \verb|\mathrm| instead to keep the size switching mechanism
active. But pay attention, \ic{mathrm} will only work well on short
items. Spaces are still not active and accented characters do not
work.\footnote{The \AmS-\LaTeX{} package makes the textrm command
  work with size changing.}

\begin{singlespace}
\begin{example}
\begin{equation}
2^{\textrm{nd}} \quad 
2^{\mathrm{nd}}
\end{equation}
\end{example}
\end{singlespace}

Nevertheless, sometimes you need to tell \LaTeX{} the correct font
size. In math mode, the font size is set with the four commands:
\begin{center}
{displaystyle}~($\displaystyle 123$),
{textstyle}~($\textstyle 123$), 
{scriptstyle}~($\scriptstyle 123$) and
{scriptscriptstyle}~($\scriptscriptstyle 123$).
\end{center}

Changing styles also affects the way limits are displayed.

\begin{singlespace}
\begin{example}
\begin{displaymath}
\mathop{\mathrm{corr}}(X,Y)= 
 \frac{\displaystyle 
   \sum_{i=1}^n(x_i-\overline x)
   (y_i-\overline y)} 
  {\displaystyle\biggl[
 \sum_{i=1}^n(x_i-\overline x)^2
\sum_{i=1}^n(y_i-\overline y)^2
\biggr]^{1/2}}
\end{displaymath}    
\end{example}
\end{singlespace}
% This is not a math accent, and no maths book would be set this way.
% mathop gets the spacing right.

\noindent This is one of those examples in which we need larger
brackets than the standard \verb|\left[  \right]| provides.


\section{Theorems, Laws, \ldots}

When writing mathematical documents, you probably need a way to
typeset ``Lemmas'', ``Definitions'', ``Axioms'' and similar
structures. \LaTeX{} supports this with the command
\begin{command}
{newtheorem}\verb|{|\emph{name}\verb|}[|\emph{counter}\verb|]{|%
         \emph{text}\verb|}[|\emph{section}\verb|]|
\end{command}
The \emph{name} argument, is a short keyword used to identify the
``theorem''. With the \emph{text} argument, you define the actual name
of the ``theorem'' which will be printed in the final document.

The arguments in square brackets are optional. They are both used to
specify the numbering used on the ``theorem''. With the \emph{counter}
argument you can specify the \emph{name} of a previously declared
``theorem''. The new ``theorem'' will then be numbered in the same
sequence.  The \emph{section} argument allows you to specify the
sectional unit within which you want your ``theorem'' to be numbered.

After executing the {newtheorem} command in the preamble of your
document, you can use the following command within the document.

\begin{code}
\verb|\begin{|\emph{name}\verb|}[|\emph{text}\verb|]|\\
This is my interesting theorem\\
\verb|\end{|\emph{name}\verb|}|     
\end{code}

This should be enough theory. The following examples will hopefully
remove the final remains of doubt and make it clear that the
\verb|\newtheorem| environment is way too complex to understand.

\begin{singlespace}
\begin{example}
% definitions for the document
% preamble
\newtheorem{law}{Law}
\newtheorem{jury}[law]{Jury}
%in the document
\begin{law} \label{law:box}
Don't hide in the witness box
\end{law}
\begin{jury}[The Twelve]
It could be you! So beware and
see law~\ref{law:box}\end{jury}
\begin{law}No, No, No\end{law}
\end{example}
\end{singlespace}

The ``Jury'' theorem uses the same counter as the ``Law''
theorem. Therefore it gets a number which is in sequence with
the other ``Laws''. The argument in square brackets is used to specify 
a title or something similar for the theorem.

\begin{singlespace}
\begin{example}
\flushleft
\newtheorem{mur}{Murphy}[section]
\begin{mur}
If there are two or more 
ways to do something, and 
one of those ways can result 
in a catastrophe, then 
someone will do it.\end{mur}
\end{example}
\end{singlespace}

The ``Murphy'' theorem gets a number which is linked to the number of
the current section. You could also use another unit, for example chapter or
subsection.

\section{Bold symbols}
\index{bold symbols}

It is quite difficult to get bold symbols in \LaTeX{}; this is 
probably intentional as amateur typesetters tend to overuse them.
The font change command \verb|\mathbf| gives bold letters, but these are
roman (upright) whereas mathematical symbols are normally italic.
There is a \ic{boldmath} command, but \emph{this can only be
used outside mathematics mode}. It works for symbols too.

\begin{singlespace}
\begin{example}
\begin{displaymath}
\mu, M \qquad \mathbf{M} \qquad
\mbox{\boldmath $\mu, M$}
\end{displaymath}
\end{example}
\end{singlespace}

\noindent
Notice that the comma is bold too, which may not be what is required.

The package \ip{amsbsy} (included by \ip{amsmath}) makes this much
easier as it includes a \ic{boldsymbol} command.

\ifx\boldsymbol\undefined\else
\begin{singlespace}
\begin{example}
\begin{displaymath}
\mu, M \qquad
\boldsymbol{\mu}, \boldsymbol{M}
\end{displaymath}
\end{example}
\end{singlespace}
\fi

\section{List of Mathematical Symbols}  \label{symbols}
 
In the following tables, you find all the symbols normally accessible
from \emph{math mode}.  

%
% Conditional Text in case the AMS Fonts are installed
%
\ifx\noAMS\relax To use the symbols listed in
Tables~\ref{AMSD}--\ref{AMSNBR},\footnote{These tables were derived
  from \texttt{symbols.tex} by David~Carlisle and subsequently changed
extensively as suggested by Josef~Tkadlec.} the package
\ip{amssymb} must be loaded in the preamble of the document and the
AMS math fonts must be installed, on the system. If the AMS package and
fonts are not installed, on your system, have a look at\\ 
\texttt{CTAN:/tex-archive/macros/latex/required/amslatex}\fi
 
\begin{table}[!ht]
\caption{Math Mode Accents.}  \label{mathacc}
\begin{symbols}{*4{cl}}
\W{\hat}{a}     & \W{\check}{a} & \W{\tilde}{a} & \W{\acute}{a} \\
\W{\grave}{a} & \W{\dot}{a} & \W{\ddot}{a}    & \W{\breve}{a} \\
\W{\bar}{a} &\W{\vec}{a} &\W{\widehat}{A}&\W{\widetilde}{A}\\  
\end{symbols}
\end{table}
 
\begin{table}[!ht]
\caption{Lowercase Greek Letters.}
\begin{symbols}{*4{cl}}
 \X{\alpha}     & \X{\theta}     & \X{o}          & \X{\upsilon}  \\
 \X{\beta}      & \X{\vartheta}  & \X{\pi}        & \X{\phi}      \\
 \X{\gamma}     & \X{\iota}      & \X{\varpi}     & \X{\varphi}   \\
 \X{\delta}     & \X{\kappa}     & \X{\rho}       & \X{\chi}      \\
 \X{\epsilon}   & \X{\lambda}    & \X{\varrho}    & \X{\psi}      \\
 \X{\varepsilon}& \X{\mu}        & \X{\sigma}     & \X{\omega}    \\
 \X{\zeta}      & \X{\nu}        & \X{\varsigma}  & &             \\
 \X{\eta}       & \X{\xi}        & \X{\tau} 
\end{symbols}
\end{table}

\begin{table}[!ht]
\caption{Uppercase Greek Letters.}
\begin{symbols}{*4{cl}}
 \X{\Gamma}     & \X{\Lambda}    & \X{\Sigma}     & \X{\Psi}      \\
 \X{\Delta}     & \X{\Xi}        & \X{\Upsilon}   & \X{\Omega}    \\
 \X{\Theta}     & \X{\Pi}        & \X{\Phi} 
\end{symbols}
\end{table}
\clearpage 

\begin{table}[!tbp]
\caption{Binary Relations.}
\bigskip
You can produce corresponding negations by adding a \verb|\not| command
as prefix to the following symbols.
\begin{symbols}{*3{cl}}
 \X{<}           & \X{>}           & \X{=}          \\
 \X{\leq}or \verb|\le|   & \X{\geq}or \verb|\ge|   & \X{\equiv}     \\
 \X{\ll}         & \X{\gg}         & \X{\doteq}     \\
 \X{\prec}       & \X{\succ}       & \X{\sim}       \\
 \X{\preceq}     & \X{\succeq}     & \X{\simeq}     \\
 \X{\subset}     & \X{\supset}     & \X{\approx}    \\
 \X{\subseteq}   & \X{\supseteq}   & \X{\cong}      \\
 \X{\sqsubset}$^a$ & \X{\sqsupset}$^a$ & \X{\Join}$^a$    \\
 \X{\sqsubseteq} & \X{\sqsupseteq} & \X{\bowtie}    \\
 \X{\in}         & \X{\ni}, \verb|\owns|  & \X{\propto}    \\
 \X{\vdash}      & \X{\dashv}      & \X{\models}    \\
 \X{\mid}        & \X{\parallel}   & \X{\perp}      \\
 \X{\smile}      & \X{\frown}      & \X{\asymp}     \\
 \X{:}           & \X{\notin}      & \X{\neq}or \verb|\ne|
\end{symbols}
\centerline{\footnotesize $^a$Use the \texttt{latexsym} package to access this symbol}
\end{table}

\begin{table}[!tbp]
\caption{Binary Operators.}
\begin{symbols}{*3{cl}}
 \X{+}              & \X{-}              & &                 \\
 \X{\pm}            & \X{\mp}            & \X{\triangleleft} \\
 \X{\cdot}          & \X{\div}           & \X{\triangleright}\\
 \X{\times}         & \X{\setminus}      & \X{\star}         \\
 \X{\cup}           & \X{\cap}           & \X{\ast}          \\
 \X{\sqcup}         & \X{\sqcap}         & \X{\circ}         \\
 \X{\vee}, \verb|\lor|     & \X{\wedge}, \verb|\land|  & \X{\bullet}       \\
 \X{\oplus}         & \X{\ominus}        & \X{\diamond}      \\
 \X{\odot}          & \X{\oslash}        & \X{\uplus}        \\
 \X{\otimes}        & \X{\bigcirc}       & \X{\amalg}        \\
 \X{\bigtriangleup} &\X{\bigtriangledown}& \X{\dagger}       \\
 \X{\lhd}$^a$         & \X{\rhd}$^a$         & \X{\ddagger}      \\
 \X{\unlhd}$^a$       & \X{\unrhd}$^a$       & \X{\wr}
\end{symbols}
 
\end{table}

\begin{table}[!tbp]
\caption{BIG Operators.}
\begin{symbols}{*4{cl}}
 \X{\sum}      & \X{\bigcup}   & \X{\bigvee}   & \X{\bigoplus}\\
 \X{\prod}     & \X{\bigcap}   & \X{\bigwedge} &\X{\bigotimes}\\
 \X{\coprod}   & \X{\bigsqcup} & &             & \X{\bigodot} \\
 \X{\int}      & \X{\oint}     & &             & \X{\biguplus}
\end{symbols}
 
\end{table}


\begin{table}[!tbp]
\caption{Arrows.}
\begin{symbols}{*3{cl}}
 \X{\leftarrow}or \verb|\gets|& \X{\longleftarrow}     & \X{\uparrow}          \\
 \X{\rightarrow}or \verb|\to|& \X{\longrightarrow}    & \X{\downarrow}        \\
 \X{\leftrightarrow}    & \X{\longleftrightarrow}& \X{\updownarrow}      \\
 \X{\Leftarrow}         & \X{\Longleftarrow}     & \X{\Uparrow}          \\
 \X{\Rightarrow}        & \X{\Longrightarrow}    & \X{\Downarrow}        \\
 \X{\Leftrightarrow}    & \X{\Longleftrightarrow}& \X{\Updownarrow}      \\
 \X{\mapsto}            & \X{\longmapsto}        & \X{\nearrow}          \\
 \X{\hookleftarrow}     & \X{\hookrightarrow}    & \X{\searrow}          \\
 \X{\leftharpoonup}     & \X{\rightharpoonup}    & \X{\swarrow}          \\
 \X{\leftharpoondown}   & \X{\rightharpoondown}  & \X{\nwarrow}          \\
 \X{\rightleftharpoons} & \X{\iff}(bigger spaces)& \X{\leadsto}$^a$

\end{symbols}
\centerline{\footnotesize $^a$Use the \texttt{latexsym} package to access this symbol}
\end{table}

\begin{table}[!tbp]
\caption{Delimiters.}\label{tab:delimiters}
\begin{symbols}{*4{cl}}
 \X{(}            & \X{)}            & \X{\uparrow} & \X{\Uparrow}    \\
 \X{[}or \verb|\lbrack|   & \X{]}or \verb|\rbrack|  & \X{\downarrow}   & \X{\Downarrow}  \\
 \X{\{}or \verb|\lbrace|  & \X{\}}or \verb|\rbrace|  & \X{\updownarrow} & \X{\Updownarrow}\\
 \X{\langle}      & \X{\rangle}  & \X{|}or \verb|\vert| &\X{\|}or \verb|\Vert|\\
 \X{\lfloor}      & \X{\rfloor}      & \X{\lceil}       & \X{\rceil}      \\
 \X{/}            & \X{\backslash}   & &. (dual. empty)
\end{symbols}
\end{table}

\begin{table}[!tbp]
\caption{Large Delimiters.}
\begin{symbols}{*4{cl}}
 \Y{\lgroup}      & \Y{\rgroup}      & \Y{\lmoustache}  & \Y{\rmoustache} \\
 \Y{\arrowvert}   & \Y{\Arrowvert}   & \Y{\bracevert} 
\end{symbols}
\end{table}


\begin{table}[!tbp]
\caption{Miscellaneous Symbols.}
\begin{symbols}{*4{cl}}
 \X{\dots}       & \X{\cdots}      & \X{\vdots}      & \X{\ddots}     \\
 \X{\hbar}       & \X{\imath}      & \X{\jmath}      & \X{\ell}       \\
 \X{\Re}         & \X{\Im}         & \X{\aleph}      & \X{\wp}        \\
 \X{\forall}     & \X{\exists}     & \X{\mho}$^a$      & \X{\partial}   \\
 \X{'}           & \X{\prime}      & \X{\emptyset}   & \X{\infty}     \\
 \X{\nabla}      & \X{\triangle}   & \X{\Box}$^a$     & \X{\Diamond}$^a$ \\
 \X{\bot}        & \X{\top}        & \X{\angle}      & \X{\surd}      \\
\X{\diamondsuit} & \X{\heartsuit}  & \X{\clubsuit}   & \X{\spadesuit} \\
 \X{\neg}or \verb|\lnot| & \X{\flat}       & \X{\natural}    & \X{\sharp}

\end{symbols}
\centerline{\footnotesize $^a$Use the \texttt{latexsym} package to access this symbol}
\end{table}

\begin{table}[!tbp]
\caption{Non-Mathematical Symbols.}
\bigskip
These symbols can also be used in text mode.
\begin{symbols}{*3{cl}}
\SC{\dag} & \SC{\S} & \SC{\copyright}  \\
\SC{\ddag} & \SC{\P} & \SC{\pounds}  \\
\end{symbols}
\end{table}

%
%
% If the AMS Stuff is not available, we drop out right here :-)
%
\noAMS

\begin{table}[!tbp]
\caption{AMS Delimiters.}\label{AMSD}
\bigskip
\begin{symbols}{*4{cl}}
\X{\ulcorner}&\X{\urcorner}&\X{\llcorner}&\X{\lrcorner}
\end{symbols}
\end{table}

\begin{table}[!tbp]
\caption{AMS Greek and Hebrew.}
\begin{symbols}{*5{cl}}
\X{\digamma}     &\X{\varkappa} & \X{\beth}& \X{\daleth}     &\X{\gimel}
\end{symbols}
\end{table}

\begin{table}[!tbp]
\caption{AMS Binary Relations.}
\begin{symbols}{*3{cl}}
 \X{\lessdot}           & \X{\gtrdot}            & \X{\doteqdot}or \verb|\Doteq| \\
 \X{\leqslant}          & \X{\geqslant}          & \X{\risingdotseq}     \\
 \X{\eqslantless}       & \X{\eqslantgtr}        & \X{\fallingdotseq}    \\
 \X{\leqq}              & \X{\geqq}              & \X{\eqcirc}           \\
 \X{\lll}or \verb|\llless|      & \X{\ggg}or \verb|\gggtr| & \X{\circeq}  \\
 \X{\lesssim}           & \X{\gtrsim}            & \X{\triangleq}        \\
 \X{\lessapprox}        & \X{\gtrapprox}         & \X{\bumpeq}           \\
 \X{\lessgtr}           & \X{\gtrless}           & \X{\Bumpeq}           \\
 \X{\lesseqgtr}         & \X{\gtreqless}         & \X{\thicksim}         \\
 \X{\lesseqqgtr}        & \X{\gtreqqless}        & \X{\thickapprox}      \\
 \X{\preccurlyeq}       & \X{\succcurlyeq}       & \X{\approxeq}
 \end{symbols}
 \end{table}
 
 \begin{table}[!tbp]
\caption{AMS Binary Relations Continued.}
\begin{symbols}{*3{cl}}
 \X{\curlyeqprec}       & \X{\curlyeqsucc}       & \X{\backsim}          \\
 \X{\precsim}           & \X{\succsim}           & \X{\backsimeq}        \\
 \X{\precapprox}        & \X{\succapprox}        & \X{\vDash}            \\
 \X{\subseteqq}         & \X{\supseteqq}         & \X{\Vdash}            \\
 \X{\Subset}            & \X{\Supset}            & \X{\Vvdash}           \\
 \X{\sqsubset}          & \X{\sqsupset}          & \X{\backepsilon}      \\
 \X{\therefore}         & \X{\because}           & \X{\varpropto}        \\
 \X{\shortmid}          & \X{\shortparallel}     & \X{\between}          \\
 \X{\smallsmile}        & \X{\smallfrown}        & \X{\pitchfork}        \\
 \X{\vartriangleleft}   & \X{\vartriangleright}  & \X{\blacktriangleleft}\\
 \X{\trianglelefteq}    & \X{\trianglerighteq}   &\X{\blacktriangleright}
\end{symbols}
\end{table}

\begin{table}[!tbp]
\caption{AMS Arrows.}
\begin{symbols}{*3{cl}}
 \X{\dashleftarrow}      & \X{\dashrightarrow}     & \X{\multimap}          \\
 \X{\leftleftarrows}     & \X{\rightrightarrows}   & \X{\upuparrows}        \\
 \X{\leftrightarrows}    & \X{\rightleftarrows}    & \X{\downdownarrows}    \\
 \X{\Lleftarrow}         & \X{\Rrightarrow}        & \X{\upharpoonleft}     \\
 \X{\twoheadleftarrow}   & \X{\twoheadrightarrow}  & \X{\upharpoonright}    \\
 \X{\leftarrowtail}      & \X{\rightarrowtail}     & \X{\downharpoonleft}   \\
 \X{\leftrightharpoons}  & \X{\rightleftharpoons}  & \X{\downharpoonright}  \\
 \X{\Lsh}                & \X{\Rsh}                & \X{\rightsquigarrow}   \\
 \X{\looparrowleft}      & \X{\looparrowright}     &\X{\leftrightsquigarrow}\\
 \X{\curvearrowleft}     & \X{\curvearrowright}    & &                      \\
 \X{\circlearrowleft}    & \X{\circlearrowright}   & &
\end{symbols}
\end{table}

\begin{table}[!tbp]
\caption{AMS Negated Binary Relations and Arrows.}\label{AMSNBR}
\begin{symbols}{*3{cl}}
 \X{\nless}           & \X{\ngtr}            & \X{\varsubsetneqq}  \\
 \X{\lneq}            & \X{\gneq}            & \X{\varsupsetneqq}  \\[-0.5ex]
 \X{\nleq}            & \X{\ngeq}            & \X{\nsubseteqq}     \\
 \X{\nleqslant}       & \X{\ngeqslant}       & \X{\nsupseteqq}     \\[-0.5ex]
 \X{\lneqq}           & \X{\gneqq}           & \X{\nmid}           \\
 \X{\lvertneqq}       & \X{\gvertneqq}       & \X{\nparallel}      \\[-0.5ex]
 \X{\nleqq}           & \X{\ngeqq}           & \X{\nshortmid}      \\
 \X{\lnsim}           & \X{\gnsim}           & \X{\nshortparallel} \\[-0.5ex]
 \X{\lnapprox}        & \X{\gnapprox}        & \X{\nsim}           \\
 \X{\nprec}           & \X{\nsucc}           & \X{\ncong}          \\[-0.5ex]
 \X{\npreceq}         & \X{\nsucceq}         & \X{\nvdash}         \\
 \X{\precneqq}        & \X{\succneqq}        & \X{\nvDash}         \\[-0.5ex]
 \X{\precnsim}        & \X{\succnsim}        & \X{\nVdash}         \\
 \X{\precnapprox}     & \X{\succnapprox}     & \X{\nVDash}         \\[-0.5ex]
 \X{\subsetneq}       & \X{\supsetneq}       & \X{\ntriangleleft}  \\
 \X{\varsubsetneq}    & \X{\varsupsetneq}    & \X{\ntriangleright} \\[-0.5ex]
 \X{\nsubseteq}       & \X{\nsupseteq}       & \X{\ntrianglelefteq}\\
 \X{\subsetneqq}      & \X{\supsetneqq}      &\X{\ntrianglerighteq}\\[-0.5ex]
 \X{\nleftarrow}      & \X{\nrightarrow}     & \X{\nleftrightarrow}\\
 \X{\nLeftarrow}      & \X{\nRightarrow}     & \X{\nLeftrightarrow}
\end{symbols}
\end{table}

\begin{table}[!tbp]
\caption{AMS Binary Operators.}
\begin{symbols}{*3{cl}}
 \X{\dotplus}        & \X{\centerdot}      & \X{\intercal}      \\
 \X{\ltimes}         & \X{\rtimes}         & \X{\divideontimes} \\
 \X{\Cup}or \verb|\doublecup|& \X{\Cap}or \verb|\doublecap|& \X{\smallsetminus} \\
 \X{\veebar}         & \X{\barwedge}       & \X{\doublebarwedge}\\
 \X{\boxplus}        & \X{\boxminus}       & \X{\circleddash}   \\
 \X{\boxtimes}       & \X{\boxdot}         & \X{\circledcirc}   \\
 \X{\leftthreetimes} & \X{\rightthreetimes}& \X{\circledast}    \\
 \X{\curlyvee}       & \X{\curlywedge}  
\end{symbols}
\end{table}

\begin{table}[!tbp]
\caption{AMS Miscellaneous.}
\begin{symbols}{*3{cl}}
 \X{\hbar}             & \X{\hslash}           & \X{\Bbbk}            \\
 \X{\square}           & \X{\blacksquare}      & \X{\circledS}        \\
 \X{\vartriangle}      & \X{\blacktriangle}    & \X{\complement}      \\
 \X{\triangledown}     &\X{\blacktriangledown} & \X{\Game}            \\
 \X{\lozenge}          & \X{\blacklozenge}     & \X{\bigstar}         \\
 \X{\angle}            & \X{\measuredangle}    & \X{\sphericalangle}  \\
 \X{\diagup}           & \X{\diagdown}         & \X{\backprime}       \\
 \X{\nexists}          & \X{\Finv}             & \X{\varnothing}      \\
 \X{\eth}              & \X{\mho}       
\end{symbols}
\end{table}



\begin{table}[!tbp]
\caption{Math Alphabets.}
\begin{symbols}{@{}*3l@{}}
Example& Command &Required package\\
\hline
\rule{0pt}{1.05em}$\mathrm{ABCdef}$
        & \verb|\mathrm{ABCdef}|
        &       \\
$\mathit{ABCdef}$
        & \verb|\mathit{ABCdef}|
        &       \\
$\mathnormal{ABCdef}$
        & \verb|\mathnormal{ABCdef}|
        &       \\
$\mathcal{ABC}$
        & \verb|\mathcal{ABC}|
        &       \\
\ifx\MathRSFS\undefined\else
$\MathRSFS{ABC}$
        &\verb|\mathcal{ABC}|
        &\pai{mathrsfs}\\
\fi
\ifx\EuScript\undefined\else
$\EuScript{ABC}$
        & \verb|\mathcal{ABC}|
        &\ip{eucal} with option: \index{mathcal}  \quad or\\
        & \verb|\mathscr{ABC}|  
        &\ip{eucal}  with option: mathscr\index{mathscr}\\
$\mathfrak{ABCdef}$
        & \verb|\mathfrak{ABCdef}|
        &\ip{eufrak}                \\
\fi
$\mathbb{ABC}$
        & \verb|\mathbb{ABC}|
        &\ip{amsfonts} or \ip{amssymb}        \\
\end{symbols}
\end{table}

%%% Local Variables: 
%%% mode: latex
%%% TeX-master: "lshort2e"
%%% End: 



%%!TEX root = ../username.tex
\chapter{Examples of Java Code}
Here are some examples of Java source using the \texttt{listings} package. I have entered the following before any code examples to format the code as shown.

\begin{singlespace}
\begin{verbatim}
\lstset{language=java}
\lstset{backgroundcolor=\color{white},rulecolor=\color{black}}
\lstset{linewidth=.95\textwidth,breaklines=true}
\lstset{commentstyle=\textit,stringstyle=\upshape,showspaces=false}
\lstset{frame = trbl, frameround=tttt}
\lstset{numbers=left,numberstyle=\tiny,basicstyle=\small}
\lstset{commentstyle=\normalfont\itshape,breakautoindent=true}
\lstset{abovecaptionskip=1.2\baselineskip,xleftmargin=30pt}
\lstset{framesep=6pt}
\end{verbatim}
\end{singlespace}

I have included the code by entering
\begin{singlespace}
\begin{verbatim}
\begin{singlespace}
\lstinputlisting[caption=Clock Code,label=clock]{source/Clock.java}
\end{singlespace}
\end{verbatim}
\end{singlespace}

\lstset{language=java}
\lstset{backgroundcolor=\color{white},rulecolor=\color{black}}
\lstset{linewidth=.95\textwidth,breaklines=true}
\lstset{commentstyle=\textit,stringstyle=\upshape,showspaces=false}
\lstset{frame = trbl, frameround=tttt}
\lstset{numbers=left,numberstyle=\tiny,basicstyle=\small}
\lstset{commentstyle=\normalfont\itshape,breakautoindent=true}
\lstset{abovecaptionskip=1.2\baselineskip,xleftmargin=30pt}
\lstset{framesep=6pt}

\begin{singlespace}
\lstinputlisting[caption=Clock Code, label=clock]{source/Clock.java}
\end{singlespace}
\newpage

\begin{singlespace}
\lstinputlisting[caption=Consumer, label=consumer]{source/Consumer.java}
\end{singlespace}

\begin{singlespace}
\lstinputlisting[caption=EvilEmpire Code, label=evil]{source/EvilEmpire.java}
\end{singlespace}

%%!TEX root = ../username.tex
\chapter{C++ Examples}
This appendix demonstrates the \texttt{listings} packages ability to format C++ code.

\lstset{language =[ANSI]C++}
\lstset{backgroundcolor=\color{white},rulecolor=\color{black}}
\lstset{linewidth=.95\textwidth,breaklines=true}
\lstset{commentstyle=\textit,stringstyle=\upshape,showspaces=false}
\lstset{frame = trbl, frameround=tttt}
\lstset{numbers=left,numberstyle=\tiny,basicstyle=\small}
\lstset{commentstyle=\normalfont\itshape,breakautoindent=true}
\lstset{abovecaptionskip=1.2\baselineskip,xleftmargin=30pt}
\lstset{framesep=6pt}


\begin{singlespace}
\lstinputlisting[caption=Motion Class, label=motion]{source/Motion.cpp}
\end{singlespace}

\begin{singlespace}
\lstinputlisting[caption=Plotter Class, label=plot]{source/Plotter.cpp}
\end{singlespace}

\begin{singlespace}
\lstinputlisting[caption=Simulation Class, label=sim]{source/Simulation.cpp}
\end{singlespace}

\begin{singlespace}
\lstinputlisting[caption=Simulation Class, label=sim2]{source/Simulation.cpp}
\end{singlespace}

\begin{singlespace}
\lstinputlisting[caption=Simulation Class, label=sim3]{source/Simulation.cpp}
\end{singlespace}

\begin{singlespace}
\lstinputlisting[caption=Simulation Class, label=sim4]{source/Simulation.cpp}
\end{singlespace}

\begin{singlespace}
\lstinputlisting[caption=Simulation Class, label=sim5]{source/Simulation.cpp}
\end{singlespace}

\begin{singlespace}
\lstinputlisting[caption=Simulation Class, label=sim6]{source/Simulation.cpp}
\end{singlespace}

\begin{singlespace}
\lstinputlisting[caption=Simulation Class, label=sim7]{source/Simulation.cpp}
\end{singlespace}
%!TEX root = ../username.tex
\chapter*{Timeline}\label{timeline}

\scalebox{1}{
\begin{tabular}{r |@{\tl} l}
\vi

Pre-1600s & Funan, Chenla, Champa and highland principalities\\
1920s & Vietnamese southward expansion\\
1680s & Arrival of Ming loyalists, receiving lands from the Nguyễn lords\\
1750s & Arrival of Qing settlers\\
1771 & Tây Sơn Rebellion \& massacre of thousands of Chinese in the South.\\
1802 & Start of the Nguyễn dynasty. Nguyễn Phúc Ánh became King Gia Long.\\
1812 & Lê Văn Duyệt became the military general of Gia Định.\\
1833 & Lê Văn Khôi's rebellion\\
1859 & French conquest of Gia Định\\
1862 & Fall of Gia Định. The Nguyễn signed three southern provinces to the French.\\
1940 & Japanese involvement in Vietnam\\
1945 & Ho Chi Minh's declaration of Independence. The First Indochina War broke out.\\
1954 & French defeat at Điện Biên Phủ\\
1955 & Founding of the American-backed Republic of Vietnam under Ngô Đình Diệm\\
1975 & Liberation Day/Fall of Saigon\\
1986 & Đổi Mới (renovation reforms)\\

\end{tabular}
}

%%%%%%%%%%%%%%%%%%%%%%%%%%%%%%%%%%%%%%%%%%%%%%%%%%%%%%%
%
%  This section would be used if you are not using BibTeX. Look at Kopka and Daly for how to
%  format a bibliography manually as well as how to use BibTeX.
%
%%%%%%%%%%%%%%%%%%%%%%%%%%%%%%%%%%%%%%%%%%%%%%%%%%%%%%%

%\begin{thebibliography}{99}
%\bibitem{}
%\bibitem{}
%\end{thebibliography}

%%%%%%%%%%%%%%%%%%%%%%%%%%%%%%%%%%%%%%%%%%%%%%%%%%%%%%%
%
%  We used BibTeX to generate a Bibliography. I would recommend this method. However, it is
%  not required.
%
%%%%%%%%%%%%%%%%%%%%%%%%%%%%%%%%%%%%%%%%%%%%%%%%%%%%%%%
\vi
\renewcommand\bibname{References} % changes the name of the Bibliography

\nocite{*} % This command forces all the bibliography references to be printed -- not just 
              % those that were explicitly cited in the text.  If you comment this out, the bibliography
              % will only include cited references.
\ifthenelse{\boolean{woosterchicago}}{
\bibliographystyle{woosterchicago}}{\ifthenelse{\boolean{achemso}}{
\bibliographystyle{achemso}}{\bibliographystyle{plainnat}}}
% if you have used the woosterchicago class option then your references and citations will be in Chicago format. If you have used the achemso class option then your references and citations will be in the American Chemical Society format. If you do not specify a citation format then the default Wooster format will be used.
\bibliography{references} % load our Bibliography file
\en

%%%%%%%%%%%%%%%%%%%%%%%%%%%%%%%%%%%%%%%%%%%%%%%%%%%%%%%
%
%                                                                Index
%
%  Uncomment the lines below to include an index. To get an index you must put 
%  \index{index text} after any words that you want to appear in the index.
%  Subentries are entered as \index{index text!subentry text}. You must also run the
%  makeindex program to generate the index files that LaTeX uses. The PCs are set to run
%  makeindex automatically.
%
%%%%%%%%%%%%%%%%%%%%%%%%%%%%%%%%%%%%%%%%%%%%%%%%%%%%%%%

\ifthenelse{\boolean{index}}{
\cleardoublepage
\phantomsection
\addcontentsline{toc}{chapter}{Index}
\printindex}{}

\clearpage\thispagestyle{empty}\null\clearpage
\end{document}