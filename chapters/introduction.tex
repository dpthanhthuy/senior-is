%!TEX root = ../username.tex
\chapter{Introduction}\label{intro}
\vi
\section{Saigon in \textit{Miền Ký Ức}}
\vi In 2019, Saigon Bitexco Financial Tower held the exhibition “Memory Museum: Once There Was Saigon,” showcasing miniature replicas of old Saigon scenes (Figure~\ref{sgxua}).\footnote{“‘Bảo Tàng Ký Ức’ Lần Đầu Tiên Của Người Sài Gòn,” \textit{Dân Trí}, accessed February 26, 2020, \url{https://dantri.com.vn/van-hoa/bao-tang-ky-uc-lan-dau-tien-cua-nguoi-sai-gon-20190425180415510.htm}.} The exhibition was the brainchild of the youth art group SG-Xưa (Old-SG), whose business model profits from the creation and sales of handcrafted models replicating old symbols of Ho Chi Minh City. The commodification of memory in Ho Chi Minh City is not a new phenomenon. In recent years, there has been a growing obsession with antiques and romanticized representations of the past in the forms of cafés, restaurants, bookstores, etc.\footnote{Má Lúm, “Trào Lưu Quán Cà Phê Bao Cấp Khắp Ba Miền,” \textit{VnExpress}, April 10, 2017, \textit{\url{https://dulich.vnexpress.net/photo/am-thuc/trao-luu-quan-ca-phe-bao-cap-khap-ba-mien-3568152.html}}; Tâm An, “Uống cà phê, nhớ những tháng ngày xưa cũ ở Đà Nẵng,” \textit{Dân Trí}, November 3, 2017, \url{https://dantri.com.vn/doi-song/uong-ca-phe-nho-nhung-thang-ngay-xua-cu-o-da-nang-20171103065604954.htm}.} Such is the spread of this trend that writers have taken to call Ho Chi Minh City a place in "miền ký ức." "Miền ký ức" is a term often associated with remembering Saigon's past.\footnote{Lam Phong, “Sài Gòn của một miền ký ức,” \textit{Tuổi Trẻ Online}, January 19, 2014, \url{https://tuoitre.vn/news-590460.htm}} Saigon in "miền ký ức" is a realm (\textit{miền}) of memory (\textit{ký ức}) where life was simpler and societal morals were not corroded. This realm of memory elicits nostalgic recollections of and longing for values and embedded in sites that are now thought of as lost or at risk from the current rate of urbanization.\footnote{Ngô Minh Hùng, “Ký Ức Sài Gòn - Chợ Lớn Xưa,” \textit{Tạp Chí Kiến Trúc}, 2017.} The nostalgia for a former Saigon is often connected to notions of an ideal living environment, whether that is a Western metropolis or a close-knit Asian society. Much as they are romantic and idealistic, these projections often turn out contradictory, misleading, unfounded, and superficial. What they do indicate, nevertheless, is the contested processes that result in the formation of such memories. 
\en

\begin{figure}[!ht]
\begin{center}
\woopic{sgxua}{0.25}
\vspace{-.2 in}
\caption[\vi Sg-Xưa's Exhibit]{\vi A mini model of Bến Thành Market in SG-Xưa's exhibit (Photograph by SG Xưa. Facebook, February 21, 2020.\footnotemark)}\label{sgxua}
\end{center}
\vspace{-.4 in}
\end{figure}

\vi A city is concrete, but memory is an abstract concept.\footnotetext{SG Xưa, "Chợ Bến Thành,"  \textit{ Facebook}, February 21, 2020. \url{https://www.facebook.com/mohinhsaigonxua/photos/pcb.566548354205522/566539340873090/?type=3&theater}} How is the way a city is remembered related to the bricks and mortar used to construct it? In the case of Ho Chi Minh City, the parallels between these two concepts are even more striking, considering its relatively short history.\footnote{This 300-year periodization is only concerned with its recent history as a Vietnamese urban center. The history of this region started much earlier before the arrival of the Vietnamese.} Mythical tales of the city's miraculous development and fighting spirit through the wars have consolidated a notion of Ho Chi Minh City that exists outside of its pre-1600s history of wilderness. From the myths of its origin, it is obvious that these histories must have also been crafted alongside the physical work of building this urban center. In light of the tumultuous events and drastic transformations that transpired in the last three centuries, it is remarkable that these narratives could prove so pervasive in solidifying the position of the city in memory.

Ho Chi Minh City's history is one of foreign expansion and control. The Vietnamese were not natives in this land. Systemic Vietnamese settlement in South Vietnam only started with the waves of conquests into the South by the Vietnamese ruling dynasties since the 16th century. Before Vietnamese presence in the area, Baigaur/Prey Nokor (former names of Ho Chi Minh City) was part of Indianized principalities like Funan, Chenla, the Kingdom of Champa and other proto-Cambodian polities.\footnote{Philippe Peycamm, “Saigon, From the Origins to 1859,” in \textit{Saigon, Ba Thế Kỷ Phát Triển Và Xây Dựng [Three Centuries of Urban Development]}, ed. Quang Ninh Lê and Stéphane Dovert, 4th ed. (Ha Noi: Nhà xuất bản Hồng Đức, 2015), 28–29.} Saigon fell into the hands of French colonizers after their attack in 1859 and remained under French control until their defeat in 1954, except for a brief stint of Japanese occupation in World War Two.\footnote{Hữu Ngọc, Ca Van Thinh, and Ta Xuan Linh, “Old Saigon,” in \textit{From Saigon to Ho Chi Minh City: A Path of 300 Years}, ed. Nguyễn Khắc Viện (Ha Noi: Thế Giới Publishers, 1998), 144–45.} Following the division of Vietnam after the Geneva Conference in 1954, Ngô Đình Diệm established the South Vietnamese government with the support of the United States.\footnote{Nghia M. Vo, \textit{Saigon: A History} (Jefferson, N.C: McFarland, 2011), 126.} Communist forces retook Saigon and gained control of the entire country in April 30, 1975, known as Liberation Day in Vietnamese national discourse or the Fall of Saigon in Western media. The immediate aftermath of the war was known as the \textit{bao cấp} (subsidy) period, during which the government aggressively nationalized private enterprises and properties as well as monopolized the distribution of goods and resources.\footnote{Hy V. Luong, \textit{Postwar Vietnam: Dynamics of a Transforming Society}, Asian Voices (Singapore: Rowman \& Littlefield, 2003), 2–8.} These restrictions had devastating effects on the postwar economy and were finally lifted in 1986 by a series of reforms called Đổi Mới (Renovation).\footnote{Vo, \textit{Saigon}, 198–210.} Today, Ho Chi Minh City (renamed from Saigon in 1976) is the industrial center of the hybrid semi-capitalist economy of Vietnam. It also has the highest population in the country with almost ten million people.\footnote{Thông tấn xã Việt Nam, “Dân Số TPHCM Gần 9 Triệu Người, Đông Nhất Cả Nước,” \textit{Báo Sài Gòn Đầu Tư Tài Chính}, October 12, 2019, \url{https://saigondautu.com.vn/content/NjcxMDQ=.html}.}  

Considering past turbulence, it is unlikely that any simple narrative can uncover the true complexity of its history. The city of Ho Chi Minh painted in the current light of progress and modernity is an elaborately constructed memory.  Despite having all the designs of a modern metropolis, the area known as Ho Chi Minh City today was mostly devoid of human habitation for the larger part of history, an environment inhospitable to all except for mosquitos and crocodiles. The bifurcation between past and present also includes the change in demographics from mostly Cham, Khmer, and Chinese to predominantly Vietnamese. Power transitions from colonial to democratic to socialist created conflicting memories, forcing these ruling regimes all had to grapple with the tensions and contradictions. The rapid transformations of Ho Chi Minh City resulted a multilayered mesh of identities, warped to account for the changes to the city's environmental and social makeup. But these changes were far from smooth. After one century of colonization, three major wars, four regime changes, all initiated by foreigners to this land, it seems inevitable that the city’s history has become a contested ground for defining and legitimizing these sometimes violent alterations. As a result, memory has become such a central concept to the building and governing of the city. With this in mind, how are strands of memory embedded in the physical components of urban construction, by whom, and for what reason? This research seeks to unravel the connections between memory, power, and identity, centered around the theme of environment, including the natural, built, and symbolic. The authorities of Ho Chi Minh City have harnessed their power to shape these environments of the city to create memory, which solidifies their legitimacy and justifies future augmentations to the city.

\section{Memory, Vietnam, and Ho Chi Minh City}
The framework for understanding memory, according to scholars like Maurice Halbwachs and Pierre Nora, is based on the premise that memory is fallible, malleable, and as a result, subject to distortion.\footnote{For more on memory, see Maurice Halbwachs, \textit{The Collective Memory}, 1st ed (New York: Harper \& Row, 1980); Pierre Nora, \textit{Rethinking France: Les Lieux de Mémoire}, trans. Mary Trouille, vol. 1 (Chicago: University of Chicago Press, 2001); Paul Ricœur, \textit{Memory, History, Forgetting} (Chicago: University of Chicago Press, 2004); Alexandre Dessingué and J. M. Winter, eds., \textit{Beyond Memory: Silence and the Aesthetics of Remembrance}, Routledge Approaches to History 13 (New York: Routledge, 2015); Daniel L. Schacter, \textit{The Seven Sins of Memory: How the Mind Forgets and Remembers} (Boston: Houghton Mifflin, 2001); Jeffrey K. Olick, \textit{The Politics of Regret: On Collective Memory and Historical Responsibility} (New York: Routledge, 2007).} Psychologists like Daniel Schacter suggest that every time the human brain recollects an event from the past, it alters the memory to account for the feelings, beliefs and knowledge acquired after the experience.\footnote{Schacter, \textit{The Seven Sins of Memory}.} Because of this impressionable nature, memory changes over time, making the process of influencing recollections, or the process of memory, a historical process with context, agency, and consequences. Memory could then be social and collective.\footnote{Halbwachs, \textit{The Collective Memory}, 50–87.} It can be constructed through the use of \textit{lieux de memoir}, the term Nora uses when referring to the sites of embedded images or narratives designed to evoke a particular reconstruction of the past.\footnote{Nora, \textit{Rethinking France}.} Sites of memory can be physical or immaterial or both. A physical site of memory is meant to trigger a mental response; public sites of memory are tools for constructing national narratives.\footnote{J. M. Winter, \textit{Sites of Memory, Sites of Mourning: The Great War in European Cultural History}, Canto Classics edition, Canto Classics (Cambridge: Cambridge University Press, 2014).} Within this broader framework, there are other micro-processes concerned with specific lenses through which memories are framed, such as through commemoration, nostalgia, or suppression.\footnote{Henry Rousso, \textit{The Vichy Syndrome: History and Memory in France since 1944} (Cambridge, Mass: Harvard University Press, 1991).} 

The existing body of scholarship on remembrance in Vietnam builds upon the same framework for studying memory, which is mostly from a Eurocentric perspective, with the vast majority of works focusing on European or American history. Some of the most important memory scholarship on Vietnam is by Hue-Tam Ho Tai, whose research is concerned with the commemorative mode of reimagining the past in the public sphere through mediums such as memoirs, paintings, tourism, cinematography, or in spaces such as war monuments, cemeteries, shrines, and museums.\footnote{For more on historical memory studies on Vietnam, see Hue-Tam Ho Tai and John Bodnar, \textit{Country of Memory: Remaking the Past in Late Socialist Vietnam} (Berkeley: University of California Press, 2001); Long T. Bui, “The Debts of Memory: Historical Amnesia and Refugee Knowledge in The Reeducation of Cherry Truong,” \textit{Journal of Asian American Studies} 18, no. 1 (February 25, 2015): 73–97; David G. Marr, “History and Memory in Vietnam Today: The Journal ‘Xưa \& Nay,’” \textit{Journal of Southeast Asian Studies} 31, no. 1 (2000): 1–25; Nathalie Huynh Chau Nguyen, \textit{Memory Is Another Country: Women of the Vietnamese Diaspora} (Santa Barbara: Praeger, 2009); Christina Schwenkel, “Recombinant History: Transnational Practices of Memory and Knowledge Production in Contemporary Vietnam,” \textit{Cultural Anthropology} 21, no. 1 (2006): 3–30.} The American War of Resistance, internationally known as the Vietnam War, is a significant moment in Vietnam’s collective and personal memory. New modes of remembrance emerge as a result of the war’s disruptive nature, allowing for reconciliation and rebuilding, whether in celebration or in grief.\footnote{Nathalie Huynh Chau Nguyen, \textit{South Vietnamese Soldiers: Memories of the Vietnam War and After} (Santa Barbara: Praege, 2016); Karen Turner-Gottschang and Thanh Hao Phan, \textit{Even the Women Must Fight: Memories of War from North Vietnam} (New York: Wiley, 1998).} Authors like Viet Thanh Nguyen and Scott Laderman write about narratives that have materialized from the struggle to make sense of the war, mostly from a top-down approach through examining the national historical discourse.\footnote{Viet Thanh Nguyen, \textit{Nothing Ever Dies: Vietnam and the Memory of War} (Cambridge: Harvard University Press, 2016); Scott Laderman, \textit{Tours of Vietnam: War, Travel Guides, and Memory}, American Encounters/Global Interactions (Durham: Duke University Press, 2009).} Research on remembrance at the community level is limited to Vietnamese in the diaspora due to state suppression, but recently, especially after Renovation (1986), alternative modes to commemoration have become more common as Vietnamese learn to navigate various layers of bureaucracy and censorship to inject revolutionary beliefs into the collective, state-sanctioned public memory.\footnote{Rivka Syd Eisner, “Performing Prospective Memory,” \textit{Cultural Studies} 25, no. 6 (November 2011): 892–916.}

On the topic of Ho Chi Minh City specifically, one sees a curious case with Vietnamese-language texts where remembrance is vibrant, but its growth is at the expense of critical memory works. The most popular genre for historical writings about Ho Chi Minh City is \textit{tản văn}. These are expressive and descriptive vignettes on a particular subject and are usually without a central argument.\footnote{Hoài Nam, “Tản Văn, Từ Một Cái Nhìn Lướt,” \textit{Báo Công an Nhân Dân Điện Tử}, January 30, 2015, \url{http://antgct.cand.com.vn/Nhan-dam/Tan-van-tu-mot-cai-nhin-luot-340089/}.} The most accomplished writer of this genre on South Vietnam is Sơn Nam. His writings are compilations of accounts about the city’s history centered around themes such as nature, bureaucrats, and land reclamation.\footnote{Sơn Nam, \textit{Đất Gia Định - Bến Nghé Xưa \& Người Sài Gòn} (Ho Chi Minh: Nhà xuất bản Trẻ, 2016).} The growing fascination with Ho Chi Minh City's past has also encouraged the rise in popularity of works such as \textit{Sài Gòn Những Biểu Tượng (Saigon’s Icons)} and \textit{Vọng Sài Gòn (Saigon Echoes [of memory])}.\footnote{Du Tử Lê et al., \textit{Sài Gòn Những Biểu Tượng} (Ho Chi Minh: Nhà Xuất Bản Văn Hóa - Văn Nghệ, 2018); Trác Thúy Miêu, \textit{Vọng Sài Gòn} (Ha Noi: Nhà Xuất Bản Hội Nhà Văn, 2019).} These books have similar themes that evoke nostalgia and pride in the city, which borders on exceptionalism.\footnote{Vo, Saigon, 54; Nguyễn Khắc Viện and Hữu Ngọc, eds., \textit{From Saigon to Ho Chi Minh City: A Path of 300 Years} (Ha Noi: Thế Giới Publishers, 1998), 11–15.} There are very few instances where these memories of Ho Chi Minh City are questioned, and often they are done in conjunction with criticism of French colonialism.\footnote{Đoàn Khắc Tình, “Cái Lý Của Nghệ Thuật Kiến Trúc Thuộc Địa,” \textit{Tạp Chí Kiến Trúc}, August 11, 2014, \url{https://www.tapchikientruc.com.vn/chuyen-muc/ly-luan-phe-binh-kien-truc/cai-ly-cua-nghe-thuat-kien-truc-thuoc-dia.html}.} Critical memory studies that challenge the agency of the narratives around Ho Chi Minh City's creation and development outside the French colonization period are almost non-existent, in spite of the burgeoning business of memorializing some idealized, romanticized, and depoliticized versions of the past.\footnote{For more on Ho Chi Minh City, see Nguyễn Viết Ngoạn, \textit{Di Sản Sài Gòn [Saigon Heritage]} (Ha Noi: Nhà xuất bản Thời đại, 2014); Nguyễn Thanh Lợi, \textit{Sài Gòn Đất và Người} (Ho Chi Minh: Nhà xuất bản Tổng hợp Thành phố Hồ Chí Minh, 2015); Huỳnh Ngọc Trảng, \textit{Sài Gòn - Gia Định xưa: tư liệu \& hình ảnh} (Ho Chi Minh: Nhà xuất bản Thành phố Hồ Chí Minh, 1997).}

The interest in memory in an urban setting often stems from the fear of heritage loss in a fast-paced environment.\footnote{For more on urban memory, see Barbara E. Thornbury and Evelyn Schulz, eds., \textit{Tokyo: Memory, Imagination, and the City} (Lanham, Maryland: Lexington Books, 2018); Steve Nash and Austin Wiliams, “The Historic City: False Urban Memory Syndrome,” in \textit{The Lure of the City: From Slums to Suburbs}, ed. Austin Williams and Alastair Donald (London: Pluto Press, 2011), 98–116; Teresa Stoppani, \textit{Unorthodox Ways to Think the City: Representations, Constructions, Dynamics} (New York: Routledge, 2019).} The past is a myth, and historic sites are “little more than the carcasses of former functions,” as Mark Crinson aptly describes.\footnote{Mark Crinson, ed., \textit{Urban Memory: History and Amnesia in the Modern City} (London: Routledge, 2005), xi.} Jordan Sand’s analysis on Tokyo vernacular reveals that such concern is especially true in places where there is little to preserve. The building stock of Tokyo was repeatedly destroyed and rebuilt, and in the place of this disappearing heritage, urban planners and residents have started to recreate spaces and objects to commemorate and preserve the past.\footnote{Jordan Sand, \textit{Tokyo Vernacular: Common Spaces, Local Histories, Found Objects} (Berkeley: University of California Press, 2013), 1–5.} The materials of urban memory can be personal, in the form of literature, artwork, or public, like architecture and city plans. These tools of remembrance are political, and as Christine Boyer argues, engage with concepts such as “the public sphere” or “collective memory” in an attempt to master and dominate urban spaces and the experiences they accommodate.\footnote{M. Christine Boyer, \textit{The City of Collective Memory: Its Historical Imagery and Architectural Entertainments} (Cambridge, Mass: MIT Press, 1994), 3.} The memory phenomenon in Ho Chi Minh City is manifested in voracious consumption of historic imagery and ideals engendered in the materials of urban memory. Studying their connection to the city’s history is integral to understanding citizens’ experiences and responses to efforts by the state and other players in the urban memory environment.

This project considers the question of memory and city building from an ideological angle, using physical sites of memory including maps and buildings. Both of these sources are connected to the manipulation of space to produce meaning. To examine how cartography and architecture have been employed by different governments or enterprises to create memory, this study investigates their use of symbology, design, structure, function, and technology to convey the builders’ intentions for the city and how it should be remembered. In addition to critiquing this process of memory formation through the spatial mediums of maps and buildings, this research employs a digital spatial medium to visualize this process, in what could be described as an attempt at critical augmented reality.

\section{An Augmented Urban History}
Augmented reality (AR) describes a technology that allows for computer-generated (or virtual) contents to be superimposed or projected on the environment around us using special interfaces such as digital glasses, phone screens, or projectors. AR relies on computer vision techniques to generate a map of our surroundings, called the physical (or real) environment, and align virtual contents to this map.\footnote{H. Durrant-Whyte and T. Bailey, “Simultaneous Localization and Mapping: Part I,” \textit{IEEE Robotics Automation Magazine} 13, no. 2 (June 2006): 99–110.} It has applications in numerous fields, including healthcare, education, entertainment, and history. Augmented reality has been used in the research phase for textual and image analysis as well as for sharing findings through AR-enabled public history projects.\footnote{Kevin B. Kee and Timothy Compeau, eds., \textit{Seeing the Past with Computers: Experiments with Augmented Reality and Computer Vision for History}, Digital Humanities (Ann Arbor: University of Michigan Press, 2019).} Technologies like augmented reality and virtual reality are increasingly commonplace at sites of public memory such as museums and historic monuments. It has grown to become a site of memory of its own, with the same characteristics as augmented reality's traditional counterpart, concerning issues with agency, audience, and agenda.

The desire to control the surrounding environment and to “augment” it with interactive and maneuverable data is what motivates the various iterations of augmented reality in history. In 1968, Ivan Sutherland introduced the first augmented reality system, but the term “augmented reality” did not officially come about until 1992.\footnote{Clemens Arth et al., “The History of Mobile Augmented Reality,” \textit{ArXiv}, no. 1505.01319 (November 10, 2015): 3, \url{http://arxiv.org/abs/1505.01319}.} As the field of computer vision develops, scientists like Paul Milgram and Fumio Kishino have come up with different ways to describe the emerging technologies, such as the Reality-Virtuality Continuum (explained in Chapter 2).\footnote{Arth et al., 5.} Since then, AR contents have gone from simple wireframe drawings to complicated three-dimensional models and clunky hardware is gradually replaced by more portable and efficient processors and displays.  Alongside hardware, AR software also benefits from the advances in detection and tracking algorithms. In 1991, Durrant-Whyte et al. made groundbreaking progress with the simultaneous localization and mapping problem, which is used extensively in augmented and virtual reality.\footnote{Durrant-Whyte and Bailey, “Simultaneous Localization and Mapping.”} The increased portability and compactness of mobile devices have allowed for AR to be deployed more effectively and with advanced features empowered by motion measuring components in the devices’ hardware. These developments make AR accessible to many disciplines and industries without the technical overhead.

The parallels regarding space usage between the sites of memory of interest in this project (maps and buildings) and augmented reality can provide an illuminating case study on how effective these new computer vision techniques could help historians better understand processes of history and convey their understanding to the general public, especially the communities whose lives these histories directly affect. In this example of Ho Chi Minh City, the introduction of an augmented reality component is an attempt to present a top-down perspective (government and ideology) to a grassroots audience. The critical part of this project is delivered through a close examination of the back-end implementation of augmented reality as well as a critique of the ethics and implications of this technology, particularly in relation to memory and power.

Through close reading of maps and architectural structures of Ho Chi Minh City, this Independent Study argues that the ruling regimes of the city as well as private enterprises and local communities have fashioned these sites of memory into devices for mass-mediating the façade and function of the city’s spaces. The mobilization of cartography and monumental architecture to create memory is a form of control made permanent by accumulation of wealth and power. These spaces are the manifestation of ideologies such as modernity and urbanism, conceptualized on plans and blueprints and realized with bricks and mortar. The augmented reality exhibit accompanying this project provides an opportunity for users to deconstruct these narratives by giving them a vantage point to view these historical processes, enabled by computer vision algorithms and techniques. Such an attempt also serves to demonstrate how perspectives are created with the assistance of power (AR technology) and that science is also a form of power capable of manufacturing memory. Cartography and architecture are two examples of how science has participated in both the physical and mental construction of Ho Chi Minh City.

This independent study is made up of two main sections: the first part discusses the technical aspect of augmented reality and the process of designing an AR historical exhibit on the Vuforia platform, while the second part includes historical analysis of maps and sites from four major periods in Vietnam’s history. Although this organization creates a dichotomy between technology and history, each chapter considers both aspects within the context of each other. Chapter 2 offers a high-level overview of the technical underpinnings of augmented reality as well as an analysis of feature detection algorithms, while considering the implication of these methods from a humanities perspective. Chapter 3 moves on to the workflow of designing with Vuforia and the elements of digital storytelling for historical narratives. The contents of the exhibit are drawn from the analyses in chapters 4 and 5, which discuss cartography and architecture respectively and their relationship with memory. Each chapter concludes with a description of the augmented reality experience associated with each site of memory. Exhibit items include an augmented map and several 3D models of the buildings considered in chapter 5. All software AR components are included in a mobile app available for both Android and Apple mobile devices.

\en
