%!TEX root = ../username.tex
\chapter{Introduction}\label{intro}
\vi
\section{Saigon in Miền Ký Ức}
\vi In 2019, Bitexco Financial Tower held the exhibit “Memory Museum: Once there was Saigon” showcasing miniature replicas of old Saigon scenes (Figure~\ref{sgxua}). The exhibit was the brainchild of the youth art group SG-Xưa (Old-SG), who makes a business of selling handcrafted models of old symbols of the city. The commodification of memory in Ho Chi Minh City is not a new phenomenon. In recent years, there has been a growing obsession with antiques and romanticized representations of the past in the forms of cafes, restaurants, bookstores, etc. (cite subsidy). Miền kí ức is a term often associated with Saigon in the past. Saigon in miền kí ức is a region (miền) of memory (ký ức) where life was simpler, society still had all its moral merits intact (cite).  This region of memory elicits nostalgic recollections of and longing for values and sites that are now thought lost or to be at risk from the current rate of urbanization.  The nostalgia for a former Saigon is often connected to notions about an ideal living environment, whether that is a Western metropolis or a close-knit society. Much as they are romantic and idealistic, these projections often turn out contradictory, misleading, unfounded, and superficial. What they do indicate, nevertheless, is the contested processes that resulted in the formation of such memories.
\en

\begin{figure}[!ht]
\rightline{
\begin{minipage}{\textwidth}
\begin{center}
\woopic{sgxua}{0.25}
\vspace{-.2 in}
\caption[\vi Sg-Xưa's Exhibit]{\vi A mini model of Bến Thành Market in SG-Xưa's exhibit. Source: SG-Xưa Facebook}\label{sgxua}
\end{center}
\end{minipage}
}
\end{figure}

\vi A city is concrete, but memory is such an abstract concept. How is the way a city is remembered connected to the bricks and mortar used to construct it? In the case of Ho Chi Minh City, the parallels between these two concepts are even more striking, considering its relatively short history.  From the narratives that surround the creation of the city, it is obvious that these histories must have also been crafted alongside the physical work of building this urban center. In light of the tumultuous events and drastic transformations that transpired in the last three centuries, it is remarkable that these narratives could prove so pervasive in solidifying the position of the city in memory.

Saigon, the city, is an elaborately constructed memory.  Despite having all the designs of a modern metropolis, the area known as Ho Chi Minh City today was mostly devoid of human habitation for the larger part of history, an environment inhospitable to all except for mosquitos and crocodiles. The bifurcation between past and present also includes the change in demographics from mostly Cham, Khmer, and Chinese to predominantly Vietnamese. Governmental transitions from colonial to democratic to socialist create conflicting memories that eventually these ruling regimes all had to grapple with. A result of the rapid transformations to Ho Chi Minh City is the multilayered mesh of identities that were woven to account for the changes to its environmental and social makeup. But these changes were far from smooth and plain sailing. After one century of colonization, three major wars, four regime changes, all initiated by foreigners to this land (the Vietnamese were not natives in Saigon), it is inevitable that the city’s history becomes a contested ground for defining and legitimizing these, sometimes violent, alterations. This is where memory becomes such a central concept to the building and governing of the city. If the creation of the city is done on both a physical and mental level, what are some of the signs of this process? In other words, how are strands of memory embedded in the physical components of urban construction, by whom, and for what reason? This research seeks to unravel the connections between memory, power, and identity, centered around the theme of environment, including the natural, built, symbolic one.

\section{Saigon in History}
Systemic Vietnamese settlements in South Vietnam only started with the waves of conquests into the South by the Vietnamese ruling dynasties since the 16th century. Before Vietnamese presence in the area, Baigaur/Prey Nokor (former names of Saigon) was part of Indianized principalities like Funan, Chenla, the Kingdom of Champa and other proto-Cambodia polities. Saigon fell into the hands of French colonizers after their attack in 1859 and remained under their control until their defeat in 1954, except for a brief stint of Japanese occupation in World War Two (Nguyen 300, Ngoc 144-5). Following the division of Vietnam after the Geneva Conference in 1954, Ngo Dinh Diem established the South Vietnamese government with the support of the United States (Vo 126). Communist forces retook Saigon and gained control of the entire Vietnam in April 30, 1975, known as Liberation Day in Vietnamese national discourse or the Fall of Saigon in Western media. The immediate aftermath of the war was known as the bao cấp (subsidy) period, during which the government cracked down on the nationalization private enterprises and properties as well as monopolized the distribution of goods and resources (cite). These restrictions had devastating effects on the postwar economy and were finally lifted in 1986 by a series of reforms called Đổi mới (Renovation) (Vo 198-210). Today, Ho Chi Minh City (renamed from Saigon in 1976) is the industrial center of the hybrid semi-capitalist economy of Vietnam. It is also the most populous city, with a population of almost ten million people (cite). 
%to do: add transition

\section{Memory, Vietnam, and Ho Chi Minh City}
The framework for understanding memory, according to scholars like Maurice Halbwachs and Pierre Nora, is based on the premise that memory is fallible, malleable, and as a result, subject to distortion.  Psychologists like Daniel Schacter suggest that every time the human brain recollects an event from the past, it alters the memory to account for the feelings, beliefs and knowledge acquired after the experience.  Because of this impressionable nature, memory changes over time, making the process of influencing recollections, or the process of memory, a historical process with context, agency, and consequences. Memory could then be social and collective.  It can be constructed through the use of lieux de memoir, the term Nora uses when referring to the sites of embedded images or narratives designed to evoke a particular reconstruction of the past.  Sites of memory can be physical or immaterial or both. A physical site of memory is meant to trigger a mental response; public sites of memory are the tool for constructing national narratives.  Within this broader framework, there are other micro-processes concerned with specific lenses through which memories are framed such as through commemoration, nostalgia, or suppression. 

The existing body of scholarship on remembrance in Vietnam builds upon the same framework for studying memory, which is mostly from a Eurocentric perspective, with the vast majority of works focusing on European or American history. Some of the most important memory scholarship on Vietnam is by Hue-Tam Ho Tai, whose research is concerned with the commemorative mode of reimagining the past in the public sphere through mediums such as memoirs, paintings, tourism, cinematography, or in spaces such war monuments, cemeteries, shrines, and museums.  The American War of Resistance, internationally known as the Vietnam War, is a significant moment in Vietnam’s collective and personal memory. New modes of remembrance emerge as a result of the war’s disruptive nature that allows for reconciliation and rebuild, whether in celebration or in grief. Authors like Viet Thanh Nguyen and Scott Laderman write about the narratives that have materialized from the struggle to make sense of the war, mostly from a top-down approach through examining the national historical discourse.  Research on remembrance at the community level is limited to Vietnamese in the diaspora due to state suppression, but recently, especially after Renovation, alternative modes to commemoration have become more common as Vietnamese learn to navigate various layers of bureaucracy and censorship to inject revolutionary beliefs into the collective, state-sanctioned public memory. 

On the topic of Ho Chi Minh City specifically, one sees a curious case with Vietnamese-language texts where remembrance is vibrant, but its growth is at the expense of critical memory works. The most popular genre for historical writings about Ho Chi Minh City is tản văn. These are expressive and descriptive vignettes on a particular subject and are usually without a central argument.  The most accomplished writer of this genre on South Vietnam is Sơn Nam. His writings are compilations of accounts about the city’s history centered around themes such as ……..  The growing fascination with Saigon’s past has also encouraged the rise in popularity of works such as Sài Gòn Những Biểu Tượng (Saigon’s Icons) and Vọng Sài Gòn (Saigon Echoes [of memory]). These books are unified by a heavy sense of nostalgia and pride in the city, which is sometimes borderline exceptionalism.  There are very few instances where these memories of Saigon are questioned, and often they are in done in conjunction with criticism of French colonialism.  Critical memory studies that challenge the agency of the narratives around Saigon’s creation and development outside the French colonization period are almost non-existent, in spite of the burgeoning business of memorializing some idealized, romanticized, and depoliticized version of the past. 

Despite the abundance of textual sources on literary nostalgia, this project considers the question of memory and city building from an ideological angle, using physical sites of memory like maps and buildings. Both of these sources are connected to the manipulation of space to produce meaning. To examine how cartography and architecture have been employed by different governments or enterprises to create memory, this study investigates their use of symbology, design, structure, function, and technology to convey the builders’ intentions for the city and how it should be remembered. In addition to critiquing this process of memory formation through spatial mediums like maps and buildings, this research employs another technological medium to visualize this process, in what could be described an attempt at critical augmented reality.

\section{An Augmented Reality Urban History}
Augmented reality (AR) describes a technology that allows for computer-generated (or virtual) contents to be superimposed or projected on the environment around us using special interfaces such as digital glasses, phone screens, or projectors. AR relies on computer vision techniques to generate a map of our surroundings, called the physical (or real) environment, and align virtual contents to this map (cite). It has applications in numerous fields, including healthcare, education, entertainment, and history. Augmented reality has been used in the research phase for textual and image analysis as well as for sharing their findings through AR-enabled public history projects.  Technologies like augmented reality and virtual reality are increasingly commonplace at sites of public memory such as museums and historic monuments. It has grown to become a site of memory of its own, with the same characteristics as its predecessors, concerning issues with agency, audience, and agenda.

The parallels regarding space usage between the sites of memory of interest in this project (maps and buildings) and augmented reality can provide an illuminating case study on how effective these new computer vision techniques could help historians better understand processes of history and convey their understanding to the general public, especially the communities whose lives these histories directly affect. In this example of Ho Chi Minh City, the introduction of an augmented reality component is an attempt to present a top-down perspective (government and ideology) to a grassroots audience. The critical part of critical augmented reality is delivered through a close examination of the back-end implementation of augmented reality as well as a critique of the ethics and implications of this technology, particularly in relations with memory and power.

Through close reading of the maps and architectural structures of Saigon, I argue that the ruling regimes of the city as well as private enterprises and local communities have fashioned these sites of memory into devices for mass-mediating the façade and function of the city’s spaces. The mobilization of cartography and monumental architecture to create memory is a form of control made permanent by an accumulation of wealth and power. These spaces are the manifestation of ideologies such as modernity and urbanism, conceptualized on plans and blueprints and realized with bricks and mortar. The augmented reality exhibit accompanying this project provides an opportunity for users to deconstruct these narratives by giving them a vantage point to view these historical processes, enabled by computer vision algorithms and techniques. Such an attempt also serves to demonstrate how perspectives are created with the assistance of power (AR technology) and that science is also a form of power capable of manufacturing memory. Cartography and architecture are two examples of how science has participated in both the physical and mental construction of Ho Chi Minh City.

This independent study is made up of two main sections: the first part discusses the technical aspect of augmented reality and the process of designing an AR historical exhibit on the Vuforia platform, while the second part includes historical analysis of maps and sites from four major periods in Vietnam’s history. Although this organization creates a dichotomy between technology and history, each chapter, whether the focus is on the former or the latter, considers both aspects within the context of one another. Chapter 1 offers a high-level overview of the technical underpinnings of augmented reality as well as an analysis of feature detection algorithms, while considering the implication of these methods from a humanities perspective. Chapter 2 moves on to the workflow of designing with Vuforia and the elements of digital storytelling for historical narratives. The contents of the exhibit are drawn from the analyses in chapter 3 and 4, which discusses cartography and architecture respectively and their relationship with memory. Each chapter concludes with a description of the augmented reality experience associated with each site of memory. Exhibit items include an augmented map and several 3D models of the buildings considered in the architecture chapter. All software AR components are included in a mobile app available for both Android and Apple mobile devices.

\en
