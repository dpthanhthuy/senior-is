%!TEX root = ../username.tex
\chapter[The AR Model for Historical Memory Studies]{The AR Model for Historical Memory Studies: Results \& Discussion}\label{results}
Chapters 4 and 5 present two AR experiences for visualizing memory, particularly focusing on its dynamic nature. The three layers of the research are demonstrated by different concepts in augmented reality. First, historical transformation comes across through spatial juxtaposition. Second, the artifact-centered narrative is enabled by object recognition. Third, perspective is realized by the interactability of the exhibit items. The success of these individual elements depends on the effectiveness of each technological component and the AR system as a whole.

Overall, feature detection works well for object and image recognition. Despite some limitations with repetitive and low-contrast images, tracking is robust and efficient for most images. The augmentability of image targets is indicated on the Vuforia Target Manager when targets are created. Once a pose has been established, tracking persists even when only a fraction of the target is in frame. \textbf{Extended Mode} in the Vuforia configurations in Unity provides the option for maintaining the established pose even when the target is out of frame. 3D objects prove to be a more challenging target for detection than 2D images. The Vuforia Object Scanner only registers larger objects with a substantial number of features. Objects with great difference in contrast also perform better. 3D printed models are especially hard to stably detect and track because they only have one color and very little contrast. Despite these drawbacks, the Object Scanner allows instantaneous testing of targets after the scan, so that users do not have to wait until the testing phase in Unity to determine the usability of a target.

Notwithstanding the difficulties with tracking, once target detection and pose calculation are complete, the tracking behavior is relatively stable for the duration of the app's run time. Augmented contents are highly responsive to changes in targets' poses. As mentioned previously, as long as part of the target is still in frame, the pose is maintained. For image targets, this means that augmented content remains active even when a flat target is extremely tilted. Such behavior facilitates great freedom of movement for the audience; the potential for unimpeded interactions is not hampered by technical restrictions. For object targets, the range of possible interactions is even more expansive, since flat image targets are only tracked and augmentable on one plane (the plane of the image), while object targets are augmentable from almost all angles. The only untracked plane is its bottom, which is hidden from the scan. Another limitation with Vuforia is that it does not support the simultaneous detection and augmentation of multiple object targets in the same frame.

While technical strengths and issues of the system are immediately observable, its effectiveness in terms of conveying the narrative depends on many factors. The simple AR experience cannot cover the depths of the historical research without overwhelming the audience. The tradeoff between simplicity and complexity is a balance that public historians constantly struggle with. The most compelling advantage that augmented reality provides is active engagement with primary sources. Handheld cross-platform AR experiences are portable and social; users are encouraged to engage in discussions or collaborate. Meanwhile, since the experience is on a personal device, each user can explore the exhibit and the app at their own pace in a self-customized session. Even if the historical narrative has to be simplified, augmented reality provides affordances that are unique to digital storytelling that broaden the boundaries of traditional methods of engagement.

As the function of memory studies in history is to uncover this dynamic between memory and its actors, the purpose of this application in particular is to visually extract what has become muddled by the human mind and reclaim agency over the city's history. By superimposing layers of historic maps, the experience hopes to present the sites of memory in the history of Ho Chi Minh City in a way that shifts the agency to the audience. In the previous discussion of power and mapping, it is the ownership of knowledge and wealth that enables the creation of maps and architecture. This app provides users with the means to redress this power imbalance and to deconstruct the processes and practices imposed on them by cartographers and architects. In this way, the app is also a site of memory, a mnemonic device seeking to partake in the shaping of memory, but rather than dictating the perspective, it offers users the freedom to decide the angles from which the past should be viewed. This activity encourages historical awareness, particularly of forces of memory. Space is an important trigger of memory. The sources of spatial memory can be direct or indirect (cite Allen Human Spatial Memory 252). Knowledge of an environment can come from direct interactions via sensorimotor activities such as walking and observing, or from indirect sources like maps. This AR experience captures the essence of both direct and indirect sources by designing affordances that cater to both modes of acquisition.

Much as augmented reality can empower the visualization of memory studies, as far as memory is concerned, such an application is subject to the same dangers that it seeks to deconstruct. Technology, knowledge, and power are intertwined. The use of AR technologies to create a new kind of knowledge is a kind of power on its own. It is important to consider the ethics of such a practice and its principles. The choice of which maps and buildings to use and their presentation are influenced by the need for a clear, thematic, and augmentable narrative. As a result, contextualization is crucial for understanding as well as for transparency.
