%!TEX root = ../username.tex
\chapter[The Urban Blueprints]{The Urban Blueprints: Cartography and the Creation of the City}\label{cartography}
\setlength{\epigraphwidth}{4.5 in}
\begin{minipage}{\textwidth}
\epigraph{\vi Mùa xuân trên thành phố Hồ Chí Minh quang vinh!

Ôi đẹp biết bao biết mấy tự hào.

Sài Gòn ơi cả nước vẫy chào.

Cờ sao đang tung bay cao qua hết rồi những năm thương đau.

Xa ba mươi năm nay đã gặp nhau vui sao nước mắt lại trào.

\vspace{.1 in}
Spring on the glorious Ho Chi Minh City!

How beautiful and proud.

Saigon, the whole country waves its salute.

The flag flies high; all the painful years have passed.

Thirty years apart now we have met again, in tears of happiness.}{\vi \textit{Xuân Hồng}\footnotemark}
\end{minipage} 
\vi Following the end of the American war, in 1976,\footnotetext{Mây Trắng, \textit{Mùa Xuân Trên Thành Phố Hồ Chí Minh}, accessed March 10, 2020, \url{https://nhac.vn/bai-hat/mua-xuan-tren-thanh-pho-ho-chi-minh-may-trang-sodRBWw.}} the National Assembly of Vietnam approved the change of the city's name from Saigon to Ho Chi Minh City, after the first Prime Minister and leader of the Democratic Republic of Vietnam.\footnote{Nghia M. Vo, \textit{Saigon: A History} (Jefferson, N.C: McFarland, 2011), 8.}  The change of name to the famed communist leader did not spark much controversy in Vietnam. Its usage, although not as prevalent as "Saigon," is not necessarily subversive.\footnote{It was mostly among the international community where there was an outrage against the new nationalist name. Erik Harms, “Beauty as Control in the New Saigon: Eviction, New Urban Zones, and Atomized Dissent in a Southeast Asian City,” \textit{American Ethnologist} 39, no. 4 (2012): 748.} The use of “Saigon” is common in daily conversations and is often interchangeable with Ho Chi Minh City, especially in cultural publications.  Songwriter Xuân Hồng, in his famous song "Spring on Ho Chi Minh City" dedicated to the liberation of Saigon in 1975, referred to both names with great pride and adoration. The name change signals a turning point in the city's memory and marks the onset of a new urban discourse of rapid development. However, as the attachment to the former name Saigon suggests, even amid the current pace of urbanization, leaders and residents of the city still turn to the past in their framing of the city's present and future. The nuances in other sites of memory are less obvious than in the city's name. They are embedded in seemingly innocuous sites like maps and land surveys, embraced by different actors to explain past creations and future visions for the city.

A remarkable feature of writings on Ho Chi Minh City is the emphasis on the short length of its history and the monumental accomplishment in this span of time.\footnote{\vi Minh Hương, \textit{Nhớ Sài Gòn} (Ho Chi Minh: Nhà xuất bản Miền Nam, 1994).} The memory of the city is rooted in a sense of exceptionalism that has allegedly enabled its enormous economical advances.\footnote{During the American war period, Saigon's population doubled from 1.3 million to 2.8 million people. Vietnam's GDP doubled from 1995 to 2006, for most which was accounted by Ho Chi Minh City. The city’s population grew from 2009 to 2019 at a rate of 1.77\% for urban areas and 4.47\% for suburban area. Economic growth was at 8.32\% for 2019. Vo, \textit{Saigon}, 16–19; Philippe Peycamm, “Saigon, From the Origins to 1859,” in Saigon, Ba Thế Kỷ Phát Triển Và Xây Dựng [Three Centuries of Urban Development], ed. Quang Ninh Lê and Stéphane Dovert, 4th ed. (Ha Noi: Nhà xuất bản Hồng Đức, 2015), 11–16; Thanh Giang, “TP Hồ Chí Minh: Tăng Dân Số Cơ Học Quá Nhanh,” \textit{Đại Đoàn Kết}, October 12, 2019, \url{http://daidoanket.vn/do-thi/tp-ho-chi-minh-tang-dan-so-co-hoc-qua-nhanh-tintuc449624}; “Ho Chi Minh City Economic Growth in 2019 Estimated at 8.32\%,” \textit{Nhân Dân Online}, December 1, 2019, \url{https://en.nhandan.org.vn/politics/item/8177902-ho-chi-minh-city-economic-growth-in-2019-estimated-at-8-32.html}; “Population Estimates for Ho Chi Minh, Viet Nam, 1950-2015,” Mongabay, accessed March 12, 2020, \url{https://books.mongabay.com/population_estimates/full/Ho_Chi_Minh-Viet_Nam.html}; “Market Report – Vietnam – Economics,” BMI, accessed March 12, 2020, \url{https://bmiglobaled.com/Market-Reports/Vietnam/economic-strength}.} Vietnamese writers sing praises of the peculiar geography, which they claim shapes the identity of its people as a resourceful, hard-working, and open-minded people. However, according to this periodization and remembrance, Saigon seems to have materialized out of nowhere in the 17th century, despite its former (and current) connections with the Kingdom of Funan, Champa and its Khmer roots.\footnote{The official periodization of Ho Chi MInh City's history is from the 1600s to the present, excluding its Cham and Khmer past.} This selective amnesia is theorized in the discourse of memory as an active process that denies the space for exchanges of remembering, or in other words, it is deleted from collective memory.\footnote{Alexandre Dessingué and J. M. Winter, eds., \textit{Beyond Memory: Silence and the Aesthetics of Remembrance}, Routledge Approaches to History 13 (New York: Routledge, 2015), 4; Paul Ricœur, \textit{Memory, History, Forgetting} (Chicago: University of Chicago Press, 2004), 448–52.} The process of ideologizing memory is the constant redefinition of the meaning and boundaries of the city, both in terms of time and space, and requires the use of special sites of memory, or \textit{lieux de mémoire}.\footnote{Pierre Nora, \textit{Rethinking France: Les Lieux de Mémoire}, trans. Mary Trouille, vol. 1 (Chicago: University of Chicago Press, 2001).} This section surveys the use of cartography by the state as a site of memory, complicit in the fashioning of the geographical and social features of Ho Chi Minh City.

The following chapter discusses four different maps of Ho Chi Minh City to explore the relationship between governance and mapmaking and how these dynamic forces manifest through modifications in the real and imagined landscapes of the city. The four examples studied include (1) an 1815 map by \vi Nguyễn dynasty official Trần Văn Học, (2) an 1895 cadastral map by the French administration of Saigon, (3) an American/South Vietnamese 1961 map, and (4) a 2005 map by the current Ministry of Natural Resources and Environment. The exhibit accompanying this analysis is a mobile augmented reality experience featuring these maps in digital forms, presenting the narrative in its time and space dimensions. \en

\section{Cartography in Pre-1800s Vietnam}\label{sec:historiography}
In the 17th century, influenced by Cartesian logic, European practices of cartography resembled that of mathematics and other sciences, which emphasize exactness, objectivity, and infallibility. Influenced by Kant's critical philosophy and Kantian space, the shift in the philosophical concept of space from strictly geometrical to social from the 18th century onward necessitates the reconsideration of representations of space not as an objective science but also as a social project.\footnote{Henri Lefebvre, \textit{The Production of Space}, trans. Donald Nicholson-Smith (Malden, MA: Blackwell, 1991), 1–2.}  Following the groundbreaking work by Henri Lefebvre on the production of space, other researchers of cartography such as J. B. Harley and Denis Wood have critiqued the relationship between power and mapping.\footnote{Denis Wood and John Fels, \textit{The Power of Maps} (New York: Guilford Press, 1992); J. B. Harley, “Maps, Knowledge and Power,” in \textit{The Iconography of Landscape: Essays on the Symbolic Representation, Design, and Use of Past Environments}, ed. Denis Cosgrove and Stephen Daniels, Cambridge Studies in Historical Geography 9 (Cambridge: Cambridge University Press, 1988), 277–312.}  In \textit{Les Lieux de mémoire}, Pierre Nora has also considered the role of geographic representations in carving out the boundaries of the state and its national borders.\footnote{Nora, \textit{Rethinking France}, 1:105–32.} Geography and visual representations of geography are powerful sites of memory for the state. Using examples from royal itineraries from the Renaissance, Nora argues that the visual memory of borders created by French kings’ tours turned preexisting borders recorded in books into a material reality.\footnote{Nora, 1:113.} According to this theory, mapping is as much a presentation of physical spaces as a representation of the political powers carving out these spaces. The social aspect of cartography is thus an important discussion in the history of Vietnam’s mapmaking, which is closely entangled with the changing social and political scene.

The history of cartography in Vietnam is a relatively untapped subject, both domestically and internationally, limited by the availability and accessibility of sources.\footnote{John K. Whitmore, “Cartography in Vietnam,” in \textit{Cartography in the Traditional East and Southeast Asian Societies}, eds. J. B. Harley and David Woodward, The History of Cartography, v. 2, bk. 2 (Chicago: University of Chicago Press, 1994), 478–79.} The historiography of the creation and usage of maps in Vietnam is thin. Vietnamese textbooks on cartography mention the existence of pre-1000 CE plans for building dams and citadels, but no records remain of these documents.\footnote{\vi Lê Văn Thơ, Phan Đình Binh, and Nguyễn Quý Ly, \textit{Giáo Trình Bản Đồ Học} (Ha Noi: Nhà Xuất Bản Nông Nghiệp, 2017), 5; Đại học Tài nguyên và Môi trường Hà Nội, \textit{Bản Đồ Học} (Ha Noi: Đại học Tài nguyên và Môi trường Hà Nội, 2010), 15–16, \url{http://lib.hunre.edu.vn/Ban-do-hoc--5158-47-47-tailieu}.}  Another explanation for this gap is the general unpopularity of cartography in the area due to historic preferences for other modes of cosmological representations of space.\footnote{Whitmore, “Cartography in Vietnam,” 479–80.} Southeast Asia as a whole and other Asian countries such as Japan observed the same void with premodern-maps because they utilized different forms of representations.\footnote{Joseph E. Schwartzberg, “Southeast Asian Geographical Maps,” in \textit{Cartography in the Traditional East and Southeast Asian Societies}, ed. J. B. Harley and David Woodward, The History of Cartography, v. 2, bk. 2 (Chicago: University of Chicago Press, 1994), 741; Mary Elizabeth Berry, “Maps Are Strange,” in \textit{Japan in Print: Information and Nation in the Early Modern Period} (Berkerley: University of California Press, 2006), 54–103..}

The first major phase in Vietnam’s cartographic history is from the 15th to the late 17th century, starting with the landmark creation of the \vi \textit{Hồng Đức bản đồ} (Maps of the Hồng Đức Period [1471-97]) under the Lê dynasty. Historian of cartography John Whitmore traces the legacy of this collection and the Lê dynasty’s cartographic traditions in his overview of the cartographic history of Vietnam. The \textit{Hồng-đức Bản-đồ} is a collection of maps of the different provinces of Đại Việt (Great Viet) commissioned by King Lê Thánh Tông in 1467 and finished in 1490.\footnote{Ngô Sĩ Liên, \textit{Đại Việt Sử Ký Toàn Thư} [Complete Annals of Đại Việt], vol. 3 (Ha Noi: Nhà Xuất Bản Khoa Học Xã Hội, 1972).} According to Whitmore, this document was the first effort by any Vietnamese courts to perform countrywide mapping.\footnote{Whitmore, “Cartography in Vietnam,” 479.} Despite the absence of existing records of the original map, Whitmore’s analysis draws evidence from later attempts at reproduction by the Mạc dynasty and the Trịnh-Nguyễn families during the 16th and 17th centuries to map Vietnam’s territories. 

Lê dynasty's records showed the influence of the Ming map system. The earliest records of Vietnamese maps date back only to the 15th century, during which time the Lê kings adopted the Ming administration model of civil service examinations and literati-officials.\footnote{Whitmore, 481.} Whitmore argues that the expansion of the bureaucratic model both created the need and provided the data for mapmaking. As the court deployed state officials to remote provinces, information began to flow back to the capital and more knowledge were accessible from previously unknown or unreachable areas. A feature of early East Asian cartography is the different modes of representation apart from scale mapping. Scale and mathematical considerations became important with advances in measured mapping under the Ming.\footnote{Cordell D. K. Yee, “Reinterpreting Traditional Chinese Geographical Maps,” in \textit{Cartography in the Traditional East and Southeast Asian Societies}, 57.} Models such as the \textit{Da Ming yitong zhi} (Comprehensive gazetteer of the Great Ming) introduced "comprehensive maps" to supplement geographic information for the gazetteer entries. Maps made during this period by the Lê, such as the \textit{tổng quát} map, illustrated heavy Ming influence. The \textit{tổng quát} map in Figure~\ref{macmap} is a possible reproduction of the maps in the Hồng Đức collection. The three-ridge representation of mountains is a distinct symbol in the Chinese map system.\footnote{Whitmore, 483}

\en
\begin{figure}[!ht]
\rightline{
\begin{minipage}{\textwidth}
\begin{center}
\woopic{macmap.png}{.4}
\vspace{-.2 in}
\caption[\vi \textit{Tổng quát} map]{\vi A \textit{tổng quát} map of Vietnam, possibly made in the 16th century (Photograph in "Cartography in Vietnam." By Whitmore, 482.)}\label{macmap}
\end{center}
\end{minipage}
}
\end{figure}

\vi Vietnamese maps under the Lê were mostly for administrative and military purposes.\footnote{Whitmore, 496.} The few maps of the south were likely plans for military excursions into the lowlands of the Cham and the Khmers. Thus, itineraries, or route maps, played an important part in strategic planning.\footnote{Whitmore, 495.} Maps helped the state with taxation and with exerting control over distant (and foreign) lands. Research on cartography of the same time period in other East Asian regions also illustrates similar connections between maps and politics. Selection of maps and symbols reflect the political significance of the places they depicted. Power, specifically the emperor's power, is a major theme; in the Gazetteer of Jiankang Prefecture, Bangbo Hu concludes that many maps emphasized the prominence of the royal dwelling palace in comparison to other sites depicted.\footnote{Bangbo Hu, “Maps and Political Power: A Cultural Interpretation of the Maps in The Gazetteer of Jiankang Prefecture,” \textit{Cartographic Perspectives}, no. 34 (September 1, 1999): 18.}  The map of the capital city Thăng Long in the Hồng Đức collection shows a similar trend.\footnote{“Bản đồ Thăng Long theo Hồng Đức Địa Dư (1490) – Plan de Thang-long,” \textit{36hn} (blog), January 1, 2015, \url{https://36hn.wordpress.com/2015/01/01/ban-do-thang-long-theo-hong-duc-dia-du-1490-plan-de-thang-long/}.} Parallels between Vietnam and other East Asian countries in terms of cartography exemplify the connection between maps and politics in the region.

The following sections explore the themes of power, water, and identity through cartography. For each theme, the four maps are analyzed and contextualized to extrapolate the patterns of mapmaking and map reading in Ho Chi Minh City. This thematic approach is done at the expense of chronology, but it allows the analysis to delve deeper into patterns across different time periods.

\section{The Mapmakers of Saigon}\label{sec:mapmakers}
One of the questions surrounding cartography is the issues of readership and authorship, or in other words, who needs maps and who gets to produce them.\footnote{Harley, “Maps, Knowledge and Power,” 278.} In the case of Ho Chi Minh City, the change in governance and political legitimacy necessitated map production. Cartography is a form of power and knowledge, and this specific form of scientific knowledge is only accessible by those in power. The cartographic history of the city demonstrates this entanglement between politics, governance, and mapmaking. Every ruling power of Ho Chi Minh City since the 19th century, including the Nguyễn dynasty, the French colonial administration, the American-backed South Vietnam government, and the current Socialist Republic of Vietnam, has extensively employed cartography and its symbolic and concrete power to exert influence or control directly over the city.

\en
\begin{figure}[!ht]
\rightline{
\begin{minipage}{\textwidth}
\begin{center}
\woopic{map_1815}{.24}
\vspace{-.4 in}
\caption[\vi Trần Văn Học’s 1815 map of Gia Định Province]{\vi Trần Văn Học’s 1815 map of Gia Định Province (Map by Trần Văn Học. In CEFURDS Map Collection)\footnotemark}\label{map_1815}
\end{center}
\end{minipage}
}
\end{figure}

Since the formal integration of Saigon into the rest of Vietnam in the 17th century, the \vi Nguyễn dynasty were the first prominent mapmakers of Saigon. The history of Vietnam’s cartography reflects the surge in cartographic activities in the $18^{th}$ and $19^{th}$ centuries, which corresponded with the southward expansion, increased contact with foreigners, and French colonization.\footnote{Whitmore, “Cartography in Vietnam,” 479.} During this period, cartography in Vietnam came under Qing and Western influence.\footnote{Whitmore, 497.} The early 19th century saw Vietnam divided under three rulers: the Nguyễn court in the center, Nguyễn Văn Thành was the governor in the north and Lê Văn Duyệt in the south.\footnote{Whitmore, 499.} Maps from this era reflected these political divisions, with the court orienting towards a Chinese model while the north continued with the Lê cartographic traditions and the south was exposed to Western influence.

In 1815, a Nguyễn dynasty official, Trần Văn Học, drew the first Vietnamese map of Gia Định Province (Saigon) (Figure~\ref{map_1815}).\footnote{Trần Nam Tiến, \textit{Sài Gòn-TP.HCM Những Sự Kiện Đầu Tiên Và Lớn Nhất} (Ho Chi Minh: Nhà Xuất Bản Trẻ, 2006), 285; Trần Văn Giàu, \textit{Địa Chí Văn Hóa Thành Phố Hồ Chí Minh Tập 1 - Lịch Sử} (Ho Chi Minh: Nhà xuất bản Thành phố Hồ Chí Minh, 1987), 190.} \footnotetext{Trần Văn Học, \textit{Plan de Gia-Định et Des Environs, Dressé Par Trần-Văn-Học, Le 4e Jour de La 12e Lune de La 14e Année de Gia-Long} (Bulletin de la Société des Etudes Indochinoises, 1815), \url{http://virtual-saigon.net/Maps/Collection?ID=1134}} Trần Văn Học was a prominent architecture for King Gia Long in the early 19th century. Fluent in modern Vietnamese and Latin, he went on various diplomatic missions to France and India.\footnote{\vi Thụy Khuê, \textit{Vua Gia Long \& Người Pháp: Khảo Sát về Ảnh Hưởng Của Người Pháp Trong Giai Đoạn Triều Nguyễn} (Ha Noi: Nhà xuất bản Hồng Đức, 2017), 267.} Trần Văn Học possessed an extensive knowledge of Western technology and translated many European scientific texts into Vietnamese. He was involved in the design of the Bát Quái citadel and the naming of streets in the inner quarter. In 1815, Trần Văn Học drew a complete map of Gia Định, presumably under the commission of the Nguyễn court. His technique was inspired by Western cartography, complete with Chinese labels of important areas. The making of this map coincided with the Nguyễn’s attempt to consolidate power in the south, especially in the context of the rise to power of Gia Định’s governor Lê Văn Duyệt, who was rivalling the imperial influence of the court. The practice of cartography in Saigon by the state took precedent from this period and persisted into the next decades of colonialism and civil war, all the way to the present.

\vi
The change in governance from the imperial Nguyễn to colonial French marked a shift in the stakeholders of the cartography industry in the south. Production of Saigon’s maps continued, undertaken by the newcomers in town, who were awash with imperial ambitions and desire to achieve them through geographical conquest. French colonial rule over Saigon began in 1961 with a series of plans for development.\footnote{François Tainturier, “Architecture and Urban Planning during the French Administration in Saigon,” in \textit{Saigon}, ed. Lê Quang Ninh and Stéphane Dovert, 77–81.} In 1962, the French military engineer in charge of the urban development of Saigon, Colonel Coffyn, produced a plan for setting the delimitations of the city of Saigon.\footnote{Tôn Nữ Quỳnh Trân, “Sài Gòn qua các bản đồ,” in \textit{Ấn tượng Sài Gòn - Thành phố Hồ Chí Minh} (Ho Chi Minh: Nhà xuất bản Trẻ, 2015), 14.} Cartographic works included a cadastral map by the French cartographer A. Chauvet for the Service du Cadastre et de la Topographie [Department of Cadastre and Topography] in Saigon (Figure~\ref{map_1898}). The abundance of colonial maps exposes the link between cartography, urban planning and power. Colonialism requires mapping in order to dominate.\footnote{James R. Akerman, ed., \textit{The Imperial Map: Cartography and the Mastery of Empire}, The Kenneth Nebenzahl, Jr., Lectures in the History of Cartography (Chicago: University of Chicago Press, 2009), 3.} The relationship between cartography and empires is implied in access to the geographical information and technological know-how of mapmaking, which comes with power.

\en
\begin{figure}[!ht]
\rightline{
\begin{minipage}{\textwidth}
\begin{center}
\woopic{map_1898}{.18}
\vspace{-.4 in}
\caption[1898 French cadastral map of Saigon]{An 1898 cadastral map of Saigon commissioned by the Service du Cadastre et de la Topographie in Saigon (Map by M. Bertaux and A. Chauvet.)\footnotemark}\label{map_1898}
\end{center}
\end{minipage}
}
\vspace{-.2 in}
\end{figure}

The theme of power mapping carried over to the Republic of Vietnam era.\footnotetext{M. Bertaux and A. Chauvet, \textit{Cochinchine Française: Plan Cadastral de La Ville de Saïgon}, 1:4000 (Service du cadastre et de la topographie, 1898), \url{https://gallica.bnf.fr/ark:/12148/btv1b530297676}.} Under the South Vietnamese government and the American authority, production and circulation of maps continued with renewed fervor. The new regime was eager to redefine Saigon as a befitting capital and established a new department for cartography in 1955, the National Geographic Service of Vietnam.\footnote{National Geographic Service of Vietnam, \textit {Nha Dia Du Quoc Gio [i.e. Gia] (National Geographic Service of Vietnam): Ten Years of Operations 1955-1965}. (Ho Chi Minh: NGS, 1965), 1.} The American Army Map Service (AMS) also partook in the mapping of Vietnam. According to its mission, the AMS was responsible for the production and compilation of maps and related geographic information in the service of the U.S. Armed Forces.\footnote{Corps of Engineers, U.S. Army, \textit{The Army Map Service: Its Mission, History and Organization} (Washington, D.C., 1960), 2.} The AMS boasted an extensive collection of military maps of Vietnam, created in close cooperation with the National Geographic Service of Vietnam. In 1958, the National Geographic Service made a map of Saigon, which was republished the AMS in 1961 (Figure~\ref{map_1961}). The AMS edition was revised for strategic purposes. These two institutions monopolized cartographic production in Saigon from the end of the First Indochina War until the fall of South Saigon.
\en

\begin{figure}[!ht]\centering
\subfigure
{\woopic{map_1961_1}{.169}}
\qquad
\hspace{-.44 in}
\subfigure
{\woopic{map_1961_2}{.169}}
\vspace{-.3 in}
\caption[U.S. Army Map Service's 1961 map of Saigon]{The U.S. Army Map Service 1961 edition of a map of Saigon, first published in 1958 (Map by the National Geographic Service of South Vietnam and the U.S. Army Map Service. In the Perry-Castañeda Library Map Collection.\footnotemark)}\label{map_1961}
\end{figure}

\begin{figure}[!ht]
\rightline{
\begin{minipage}{\textwidth}
\begin{center}
\woopic{map_2005}{.85}
\vspace{-.35 in}
\caption[2005 map of Saigon]{A 2005 map of Saigon by the Ministry of Natural Resources and Environment.\footnotemark}\label{map_2005}
\end{center}
\end{minipage}
}
\end{figure}

\vi
Nothing attests to the connection between governance and the need for cartography like the postwar nationalization of cartography.\footnotetext{National Geographic Service of Vietnam and U.S. Army Map Service, \textit{Saigon}, 1:10000 (U.S. Army Map Service, 1961), \url{http://legacy.lib.utexas.edu/maps/world_cities/txu-pclmaps-saigon_sheet1-1961.jpg}.} After 1975, the new socialist state took over the task of imposing new definitions and boundaries on the area through mapping.\footnotetext{Bộ Tài nguyên và Môi trường, \textit{Thành Phố Hồ Chí Minh. C-48-34-A-d}, 1:25000 (Ha Noi: Nhà xuất bản Bản đồ, 2005), \url{http://virtual-saigon.net/Maps/Collection?ID=1141}.}  Government institutions continue to be responsible for overseeing the creation of cadastral and land use maps for urban planning and land development purposes.\footnote{Annette Miae Kim, \textit{Sidewalk City: Remapping Public Space in Ho Chi Minh City} (Chicago: The University of Chicago Press, 2015), 58.} The current regulating body of cartographic activities in Vietnam is the Department of Survey, Mapping and Geographic Information (DOSM), a unit under the Ministry of Natural Resources and Environment (MONRE). The map in Figure~\ref{map_2005} was made by MONRE in 2005. The Vietnamese government puts special emphasis on cartography as a tool for national security and land development.\footnote{Bộ Tài nguyên và Môi trường, “Quy Định Chức Năng, Nhiệm vụ, Quyền Hạn và Cơ Cấu Tổ Chức Của Cục Đo Đạc, Bản Đồ và Thông Tin Địa Lý Việt Nam,” May 16, 2017, \url{http://dosm.gov.vn/SitePages/GioiThieu.aspx?item=568}.} Throughout its history, the mapmaking industry of Ho Chi Minh City remains an exclusively national business to facilitate the interests of whichever administration was in power at the time.

For all four maps, the symbolic and physical boundaries shifted from one period to another. The city of Saigon and now Ho Chi Minh City have been demarcated by tangible and imagined lines of waterways and borders. As the regimes produced blueprints, they also manufactured memoryscapes to facilitate remembrance in the city.

\section{The Conquest of Water}
Rulers make maps to develop and control the land, but the land, and in Ho Chi Minh City's case, its water, dictates direction of development. Water has always been a part and parcel of life in Ho Chi Minh City. The water system is intricately connected to the lifestyle of the city’s inhabitants, both as a resource and a challenge. Ho Chi Minh City was originally a marshland. The historic center of the city, Bến Nghé, lies on the west bank of the Saigon river. The closest port to the city center is about 45 miles inland from the East Sea.\footnote{Bảo tàng Thành phố Hồ Chí Minh, “Sài Gòn - Thành Phố Hồ Chí Minh: Thương Cảng, Thương Mại - Dịch Vụ,” Bảo tàng Thành phố Hồ Chí Minh, accessed November 28, 2019, \url{http://www.hcmc-museum.edu.vn/en-us/store/1123-sai-gon-thanh-pho-ho-chi-minhbrthuong-cang-thuong-mai-dich-vubr.aspx}.} This awkward position hampered Saigon’s potential to compete with more accessible ports further north and south with specialized products such as Sóc Trăng (red salt), Cà Mau (fish), and Hội An (spices).\footnote{Vo, \textit{Saigon}, 7.} The inhospitable tropical weather and marshy landscape of Saigon, with its crocodiles and tigers, limited the area’s growth under the Cham before the 11th century and later the Khmer from the 11th century to the 17th century.\footnote{Vo, 1; Sơn Nam, \textit{Đất Gia Định - Bến Nghé Xưa \& Người Sài Gòn} (Ho Chi Minh: Nhà xuất bản Trẻ, 2016), 47–60.} However, when the Vietnamese and the Chinese moved into the area and brought with them wet rice agriculture in the 16th century, Saigon became a major hub for trading, granting access to the riches of the Mekong Delta.\footnote{Vo, \textit{Saigon}, 1–7.} The marshes previously swamped with wild animals now provided access from the main river to the Chinese quarter. The declining silt conditions at central Vietnamese ports such as \vi Hội An also helped to elevate Saigon in the South China maritime trade.\footnote{Ben Kiernan, \textit{\vi Việt Nam: A History from the Earliest Times to the Present} (New York: Oxford University Press, 2017), 252.}  These developments show that the economic progression of Saigon was as dependent on the local riverine system as it was restricted by the same water lines. To harness this part of land is to control the water system that governs the life of everything on it.

As the first Vietnamese government in the south, the \vi Nguyễn dynasty put great emphasis on strategic understanding of the water network. This priority is noticeable in the 1815 map, in which \vi Trần Văn Học, charged by the rulers to map Saigon, 
 \en
\begin{wrapfigure}{l}{0in}
\woopic{river_bend}{.46}
\vspace{-.2 in}
\caption{The bend of the Saigon River depicted by \vi Trần Văn Học (Image cropped from map by Trần Văn Học.) }
\label{river_bend}
\end{wrapfigure}
 \vi paid extreme attention to depicting the water system, especially in terms of the bend of the Saigon River. He improved the measurement methods used in previous French maps, exemplified by the measurement line around the river bend (Figure~\ref{river_bend}). The proportional accuracy of this map was adjudged to surpass the quality of previous French maps of Gia Định.\footnote{Thụy Khuê, \textit{Vua Gia Long \& Người Pháp}, 267.} The map’s details demonstrate how important it was for the new rulers of Saigon to establish an understanding of the layout of the land, and in this case its water, to consolidate their power over the city.

The marshy landscape of Saigon was an obstacle for the \vi Nguyễn rulers, but it was also an opportunity, which the state exploited to build the city’s defense. Figure~\ref{citadel} shows the strategic position of the city center, bounded on the north and 
\en
\begin{wrapfigure}{r}{2.6in}
\vspace{-.1 in}
\woopic{citadel}{1.0}
\caption[The Bát Quái citadel]{The \vi Bát Quái citadel (Image cropped from map by Trần Văn Học.)}
\label{citadel}
\vspace{-.3 in}
\end{wrapfigure}
south sides by two large arroyos, \vi Bến Nghé and Chợ Lớn, and on the east side by the Saigon River. The \vi Nguyễn governors in the city built their administrative and military structures around the main waterways. In 1789, Lord \vi Nguyễn Ánh (later became King Gia Long) commissioned the construction of the Bát Quái citadel (Figure~\ref{citadel}) on the bank of the Saigon River.\footnote{Vo, \textit{Saigon}, 37.} To fight off the Siamese, in 1772, Nguyễn Cửu Đàm, a general under the Nguyễn, built the Ruột Ngựa canal and Lũy Bán Bích (Bán Bích Rampart), connecting the two main arroyos, to close off the remaining open East side with a perimeter of water and rampart.\footnote{Huỳnh Ngọc Trảng, \textit{Sài Gòn - Gia Định xưa: tư liệu \& hình ảnh} (Ho Chi Minh: Nhà xuất bản Thành phố Hồ Chí Minh, 1997), 12–13.} Controlling the water and utilizing it as a defense became a mainstay for political regimes that governed Saigon since the 18th century.

Water was the lifeline of communications in the city under the Nguyễn Dynasty. The waterways were depicted in extreme detail, compared to previous maps. Trần Văn Học provided careful annotations of rivers, canals, and bridges, in addition to residential areas. An important Vietnamese writer on southern Vietnam, Sơn Nam, notices how the term \textit{đất giồng} (alluvial banks along rivers and creeks) in the Southern Vietnamese dialect reflects the significance of water in the area.\footnote{Sơn Nam, \textit{Đất Gia Định - Bến Nghé Xưa \& Người Sài Gòn}, 47.} The map clearly identifies \textit{đất giồng} concentrations along the main riverways with small rectangular symbols. In addition to the water system and its neighborhoods, other annotations on the map include markets and pagodas. This detail shows how the importance of water at the time was comparable to these other communal spaces. Their embankments served as both commercial and spiritual centers for Saigon inhabitants.

When the French imperial power invaded the land in the second half of the 18th century, their conquest of Saigon also depended heavily on water. French strategy of “gunboat diplomacy” relied on water mobility for success.\footnote{David Biggs, \textit{Quagmire: Nation-Building and Nature in the Mekong Delta} (Seattle: University of Washington Press, 2010), 23–26.} French troops launched their first attack on Saigon in 1959. Their gunboats followed the Saigon river straight to the entrance to the citadel, whose east gate was only 500m from the nearest river port.\footnote{Chung Hai, “Nếu còn thành cũ, Gia Định không dễ thất thủ ngày 17-2-1859,” \textit{Tuổi Trẻ Online}, February 17, 2016, \url{https://tuoitre.vn/news-1052677.htm}.} Despite serving their military interests, water also created major problems for the French imperialists. After capturing Saigon, colonial forces encountered a marshy landscape of sparsely populated plains and a convoluted network of creeks and rivers.\footnote{Vo, \textit{Saigon}, 75.} Their imperial conquest did not end with the 1961 military victory. The feat only opened up more obstacles posed by the land and its people, which continued to manifest throughout French colonization of Saigon. French maps of the area during this period expose their attempts to control these tensions between geography and power.

The conquest of water is evident in the French cadastral map of 1898 (Figure~\ref{map_1898}). One of its curious features is the convenient absence of the complicated water system and marshy landscape of Saigon. The only waterways depicted are the Saigon River and short parts of the arroyos \vi Bến Nghé and Thị Nghè. The wild, swampy and flooded plains did not fit the French vision of a city. Therefore, they came up with development plans to make it habitable and conforming to Western standards.\footnote{Tainturier, “Architecture and Urban Planning during the French Administration in Saigon,” 79–80.}
\en
\begin{wrapfigure}{l}{0in}
\woopic{boulevard_charner}{.6}
\caption[Boulevard Charner]{Boulevard Charner (marked red) (Image cropped from map by M. Bertaux and A. Chauvet.)}
\label{charner}
\vspace{-.2 in}
\end{wrapfigure}
\vi Colonial alterations to the water landscape ranged from marsh cleanup to bridge construction and canal projects to facilitate the extraction of goods. In 1875, French Admiral Victor-Auguste Duperré conceived a major plan for building a new inland water network to connect Saigon to the Mekong Delta.\footnote{Biggs, \textit{Quagmire}, 32.} This mega plan included the expensive project of the Chợ Gạo canal, creating a direct route by water from Saigon to the nearest delta Port, Mỹ Tho. Other projects included the filling of canal for land transport.\footnote{Tainturier, “Architecture and Urban Planning during the French Administration in Saigon,” 78–81.} The construction of Boulevard Charner (Figure~\ref{charner}) involved the clean-up and filling of the Chợ Vải Canal, which used to be a floating market for Indian textile.\footnote{Sơn Hòa, “Những Kênh Rạch Xưa Thành Đại Lộ Đẹp Nhất Sài Gòn,” \textit{VnExpress}, April 10, 2016, \url{https://vnexpress.net/thoi-su/nhung-kenh-rach-xua-thanh-dai-lo-dep-nhat-sai-gon-3380037.html}.} Urban planning projects from this period completely reconfigured the geographical landscape of Saigon, especially on the water frontier.

After the end of the First Indochina War in 1954, under President Ngô Đình Diệm, the Republic of Vietnam took control of the city and embarked on a series of modernizing programs building upon the existing French groundwork using American aid funds.\footnote{Biggs, \textit{Quagmire}, 154–55.} Renovation projects solidified colonial alterations to the water environment by building bridges, dams, and other transportation and irrigation systems. The 1961 map of Saigon (Figure~\ref{map_1961}) shows the newly introduced ferries for transportation across the Saigon River. The South Vietnamese and American mapmakers continued to exhibit a persisting cartographic pattern from the French period, which is the conspicuous absence of marshes. Urban infrastructure and plantation replace the old flooded plains. Natural frontiers such as rivers and canals, however, still serves as a barrier to urban expansion. In the map, the east bank of the Saigon River is mostly covered by rice fields and canebrakes. The same can be said of the north and south banks of the Thị Nghè and Bến Nghé arroyos.

The battles of the American war were fought not only on land but also on the waterfront. Knowledge of water conditions became decisive for all sides. Cartographic techniques to depict water networks became more precise with the introduction of aerial photography.\footnote{Biggs, 200.} The South Vietnamese map uses the Universal Transverse Mercator coordinate system and aerial photographs by the French Department of Geography. Enabled by advanced spatial and aerial technology, mapping remains one of strongest weapons against Vietnamese guerrilla resistance. While Americans and their allies relied on staying above water for survival, Vietnamese revolutionaries depended on water for undercover. Competing technological and spatial knowledge of the delta landscape spawned different tactical orientations for each side.

The theme of water mapping carries over to the postwar period, with maps of Ho Chi Minh City focusing on urban development along the water lines. The 2005 map of the city (Figure~\ref{map_2005}) outlines the new development projects of urbanization. \en
\begin{wrapfigure}{r}{3.6in}
\woopic{saigon_bridges}{.35}
\vspace{-.2 in}
\caption[Saigon River bridge complex]{Saigon River bridge complex (circled red) (\vi Image cropped from map by Bộ Tài Nguyên và Môi Trường.)}
\label{saigon_bridges}
\vspace{-.2 in}
\end{wrapfigure}
\vi Examples include the new Saigon River bridge complex (Figure~\ref{saigon_bridges}), the first permanent structures to connect the east and west banks. The recent completion of the \vi Thủ Thiêm bridge and tunnel complex has fully integrated the swampy eastern suburbs of Ho Chi Minh City into the center. The detailed riverine classifications reveal how water is still central to the city’s livelihood. The major difference in this map is how the water system has morphed into becoming a part of the urban network of roads, bridges, railroads, ferries, etc. Meanwhile, manmade augmentations are naturalized as part of the terrain and other natural configurations.

As much as these maps tell the story of water in the city, they also obfuscate the pitfalls of geographical modifications. Topography maps blur the lines between natural waterways and artificial structures, forged through violent use of technology to change the environment.\footnote{Biggs, 71–73.} These maps fail to show the disastrous shortsightedness and a total disregard for their environmental sustainability in these masterplans. Development designs hide the permanent damages by nation-building projects on the natural and social landscapes. Specifically, public sanitation became a major problem for the French administration as issues involving freshwater accessibility and waste disposal were overlooked, indicative of the superficial and hypocritical nature of modernization.\footnote{Tainturier, “Architecture and Urban Planning during the French Administration in Saigon,” 81.} Some of the canals dredged by the colonial administrations during this period have long silted up.\footnote{Biggs, \textit{Quagmire}, 71.} From 2005 to 2012, the Thị Nghè arroyo (now Nhiêu Lộc canal) underwent a major renovation which involved digging up the riverbed to enable trash removal and reorganization of the sewage disposal system.\footnote{Văn Hiến, “Bài 3: Cải Tạo Kênh Nhiêu Lộc – Thị Nghè: ‘Công Trình Thế Kỷ’ Của TP. Hồ Chí Minh,” \textit{Báo Mới}, January 31, 2018, \url{https://baomoi.com/bai-3-cai-tao-kenh-nhieu-loc-thi-nghe-cong-trinh-the-ky-cua-tp-ho-chi-minh/c/24818795.epi}.} Maps erase from memory deteriorating environmental conditions and natural disasters resulting from transformations to the water landscapes.

\section{Imagining the City}
Compared to the Merriam-Wester definition of a city as “an inhabited place of great size, population, or importance,” pre-eighteenth century Ho Chi Minh was by no means a city.\footnote{“City,” in \textit{Merriam-Webster} (Merriam-Webster), accessed November 28, 2019, \url{https://www.merriam-webster.com/dictionary/city}.} Originally, the region was mostly marshlands and barely habitable. Compared to other towns and cities in the south, the Saigon basin did not have any economic edge over the Mekong Delta. The city only sprang into existence under the Nguyễn dynasty and continued to rise in importance to the present, now with a population of 9 million.\footnote{Thông tấn xã Việt Nam, “Dân Số TPHCM Gần 9 Triệu Người, Đông Nhất Cả Nước,” \textit{Báo Sài Gòn Đầu Tư Tài Chính}, October 12, 2019, \url{https://saigondautu.com.vn/content/NjcxMDQ=.html}.} Today, memory of Ho Chi Minh City mostly focuses on its short but rapid development and urbanization in the last three centuries, without considering the historical context of its previous geographical and economic conditions and ethnolinguistic diversity. The reimagined history of the city needs to compelling for Saigon citizens and the international community to legitimize and consolidate its creation. The history of Ho Chi Minh City must be consistently reaffirmed to legitimize and consolidate the myths of its origin and development. Consideration of the role of political and social factors in the urban construction process is vital for understanding the creation of this narrative. The planning of a city is not only practical but also symbolic. The city is therefore a product of imagination, conceived in the form of development plans, brought into existence with construction projects, and preserved as part of the land and its memory through cartography. Maps communicate envisioned boundaries that define new spaces and their meanings.

The historical context of Vietnamese migration is important for understanding the emergence of the Saigon and Chợ Lớn conurbation. During the 16th and 17th century, social and economic developments in Vietnam elevated Saigon’s status in the eyes of Vietnamese ruling powers. Evidence of this rise in ranks can be gleaned from the 1815 map (Figure~\ref{map_1815}). Saigon formally became part of “Phủ Gia Định” (Gia Định Province) under King Minh Mạng, but Vietnamese rulers had long decided its perimeter and continued to draw new boundaries in the South to create six delta provinces.\footnote{Huỳnh Ngọc Trảng, \textit{Sài Gòn - Gia Định xưa}, 37.} How the map depicts these borders reveals the value of gaining knowledge of newly acquired lands for the Nguyễn court, who later used this territorial repertoire to ratify its administrative divisions of the Mekong Delta.

\en

\begin{figure}[!ht]
\rightline{
\begin{minipage}{\textwidth}
\begin{center}
\woopic{comm_lines}{.3}
\vspace{-.2 in}
\caption[The communication lines of the \vi Nguyễn Dynasty's Saigon]{The communication lines of the \vi Nguyễn Dynasty's Saigon. The Bến Nghé Arroyo is marked red and the Royal Road blue (Image cropped from map by Trần Văn Học.)}\label{comm_lines.png}
\end{center}
\end{minipage}
}
\end{figure}

\vi
The Nguyễn lords had already begun development of the Gia Định Province long before Trần Văn Học created the map in 1815. The design shows two main centers, Bến Nghé and Chợ Lớn. Chợ Lớn was the historic Chinese settlement, dominated by Ming and Qing merchants. When Vietnamese ruling families arrived in the area, they settled down on the left bank of the Saigon River, separate from the Chinese end.\footnote{Huỳnh Ngọc Trảng, 6.} In the map, the Bát Quái citadel (Figure~\ref{citadel}) stood in the center of Bến Nghé, bounded on the north and south sides by arroyos Thị Nghè and Bến Nghé and the Saigon river in the east.\footnote{Vo, \textit{Saigon}, 8-9.} Despite the separation of administration from commerce, lines of communication connected the two ends of the city on both land and water. The map shows the Bến Nghé Arroyo joining the Chinese quarter to the administrative area. The two centers of Gia Định were also connected by Đường Cái Quan, the Royal Road that ran through the length of Việt Nam at the time, from the Northern border with China all the way to the southernmost province of Hà Tiên.\footnote{Trung Sơn, “Những Con Đường Thiên Lý Đầu Tiên Của Vùng Đất Sài Gòn,” \textit{VnExpress}, September 5, 2017, \url{https://vnexpress.net/thoi-su/nhung-con-duong-thien-ly-dau-tien-cua-vung-dat-sai-gon-3636181.html}.} In fact, communication was decisive as the Nguyễn felt the need to tighten their reins over “wayward Southern barbarians,” who were allegedly fraternizing with Catholic missionaries and abandoning their core Confucian values.\footnote{Vo, \textit{Saigon}, 54.} Urban development from this period was to fulfil the vision of a subservient imperial city resistant against the encroachment of Western and Chinese influence.

These visions of a traditional city were soon supplanted by the French conception of a European city. The colonial government took over the groundwork of city building from the imperial court in the decades following their conquest. The abundance of development plans for Saigon during the colonial era tells the story of French colonial aspirations for building the city. Urban planning was especially important because of Saigon’s special geography of convoluted water networks and uninhabitable muddy terrain. Their grand vision for a “Paris of the East” met with enormous challenges. The early development maps of Saigon were not just a symbol of power; they also exposed the colonizers’ struggle to harness the land. In their attempt to control the geography of Saigon, French urban architects employed extensive use of cartography. Cadastral maps rose in prominence as the city was subdivided into different sections and plots.{Tainturier, “Architecture and Urban Planning during the French Administration in Saigon,” 75.} The 1898 French map shows land ownership types, demarcating four main sections of the city of Saigon: administrative, commercial, industrial, and residential. The administrative quarter of the city was located on the left bank of the Saigon River, further upstream from the commercial port. Military facilities mostly occupied the northwestern bank, guarding the entrance to the city center and the administrative section. Most of the commercial and residential districts lie south of the Bến Nghé – Chợ Lớn border.

French development projects created the border between the Saigon city and Chợ Lớn. The geographical homogeneity of Saigon is a fabrication by governing regimes, and so are its imagined borders. While Trần Văn Học’s 1815 map depicts Gia Dinh as one province, the 1898 French map clearly marks the borders of the city as separate from the market town Chợ Lớn. The colonial map marks the border splitting Saigon from Chợ Lớn by a thick red line; the Chinese side reads Arrond de Cholon (District of Chợ Lớn). The city of Saigon was really only created by French urban planning, with borders divided along geographical as well as cultural lines. In a mapping project on Ho Chi Minh City, Annette Kim proposes that the notion of Ho Chi Minh City as a single, continuous space is only a recent reimagination, and that its history is really the tale of two cities.\footnote{Kim, 28-37.} French planning created a stark difference in landscape between the two parts of town, which also resulted in the social stratification within the city.

French visions for an ideal \textit{métropole} came at the exclusion of ethnic Chinese and other indigenous Cham and Khmer populations, but cultural segregation is not the only elusive aspect in cartography. Urban mapping captures the transformation of Saigon from marshlands into a city decked by colonial architecture and paved boulevards, while hiding issues with waste disposal and clean water accessibility.\footnote{Kim, 38–39; Tainturier, “Architecture and Urban Planning during the French Administration in Saigon,” 81.} Old water lines were now replaced by roads, tramways, and new canals.\footnote{Kim, \textit{Sidewalk City}, 32–39.} Grandiose monuments distracted urban planners from the problem of environmental sustainability. The fabricated colonial city was doomed to failure from the outset, and their development maps were just a series of disguise for the shortsightedness of this vision.

Similar to the French government, for the South Vietnamese state, urbanization was not only a side project; it provided justification for their authority and legitimacy. In the French case, civilization was the pretext for colonialism. This rhetoric is echoed in the Second Indochina war period. The Saigon government and its American counterparts needed to prove their supremacy over North Vietnam, and nation building was one such display of power.\footnote{Christopher Fisher, “Nation Building and the Vietnam War,” \textit{Pacific Historical Review} 74, no. 3 (2005): 441–56.} Modernization theory is the premise of America’s and South Vietnam’s nation-building programs.\footnote{Fisher, 442.} In the first decade under the South Vietnamese state (1954 – 1964), Saigon experienced a major influx of migrants, especially from North Vietnam.\footnote{Hy V. Luong, ed., \textit{Postwar Vietnam: Dynamics of a Transforming Society, Asian Voices} (Singapore: Lanham, Md: Institute of Southeast Asian Studies; Rowman \& Littlefield, 2003), 34.} The fledgling Ngô Đình Diệm regime acquired a new set of residents, and urbanization projects were well underway to cater to this growing population. Contextualizing maps from this period can shed light on how the war was also fought on the ideological front through urban development.

Evidence of this ideological struggle lies in the infrastructure illustrated in the American Army Map Service’s 1961 map (Figure~\ref{map_1961}). The historic center of Bến Nghé is densely drawn, with careful annotations of administrative and public institutions. These depictions were to showcase South Vietnam’s administrative organization and its academic and economic advances. The sophisticated infrastructure described in the map served to validate the authority of the Ngô Đình Diệm government as the only legitimate Vietnamese government in the south. The map makes no mention of any American presence, even though the funding behind most of these institutions was from American aids. This intentional omission is also consistent with the choice of language in the map. Despite having been revised by the American Army Map Service (AMS), the only English-language text is the map’s name and edition, along with captions about the AMS and the National Geographic Service of Vietnam. South Vietnamese cartography masks the real agents behind the façade of modernization and urbanization to display a modern southern capital that misleadingly appears to be uniquely Vietnamese.

While previous French, American, and Republic of Vietnam maps depict Saigon as an exception, an island protected by water, the 2005 map by the Ministry of Natural Resources and Environment (MONRE) (Figure~\ref{map_2005}) is all about connectedness. Ho Chi Minh City has expanded tremendously in the past century, with new infrastructure to accommodate this growth. The map features red lines of national highways stretching to the marshiest part of town, connecting the city center to its outskirts and to other provinces. These interconnected networks solidify claims viewing the city as a historically contingent area. The map only shows a selected number of geographical features, including rivers, forests, marshes, fields, etc, while disregarding other parts. Interestingly, the map features a large part of the northern patches of forests, cutting out most of the South Saigon area of Chợ Lớn. This observation is in keeping with previous cartographic patterns, but in this case the omission is most likely due to the MONRE’s intention to show the green suburbs, despite the largely deforested center. The production of these maps creates and formalizes the interconnectedness of the city. 

The MONRE map continues to strengthen Ho Chi Minh City's identity built on rapid development and modernization. Black cubes on pink background litter the city center, marking structures that are above three stories. Recent urban renewal policies focus on verticalizing the city landscape with skyscrapers and redefining street spaces to enable this socio-spatial restructuring.\footnote{Marie Gibert, “Moderniser La Ville, Réaménager La Rue à Ho Chi Minh Ville,” EchoGéo, no. 12 (May 31, 2010).} In the years following 2005, new maps would go on to show the new high-rise centers of the city, including master-planned upscale neighborhoods such as Thủ Thiêm, Phú Mỹ Hưng, and Landmark. These new urban development projects become new grounds for imagining the city and experimenting with ideas about urban life for both the state and private agents.\footnote{Erik Harms, \textit{Luxury and Rubble: Civility and Dispossession in the New Saigon}, Asia: Local Studies/Global Themes 32 (Oakland, California: University of California Press, 2016), 4.} Land-use rights provide the space for exercising political freedom that is otherwise restricted, hence the existence of maps as a tool to propagandize and legitimize the politically charged use of urban space.

Structural and social transformations through changes in governance constantly reshape the polysemous image of Ho Chi Minh City. From an imperial city to a European metropole, from a democratic capital to the “blossoming lotus” at the heart of the late socialist Vietnamese economy, new conceptions supplant old ones to create metamorphic narratives that are both dynamic and pervasive. Cartography is at the interface of the city’s physical development and the mental configuration facilitating this process. The intersection of power, politics, and science creates avenues for maps to transcend their material borders and bleed into the realm of urban consciousness.

\section{Ho Chi Minh City in Three Degrees of Historical Freedom}

In computer graphics, six degrees of freedom (6DOF) signify the ability to translate and rotate an object along and about three axes in three-dimensional space. A 6DOF object’s movement can be quantified by positional changes caused by these six transformations: forward/backward (x-axis), left/right (y-axis), up/down (z-axis), roll (x-axis), pitch (y-axis), yaw (z-axis). 6DOF is ultimately an index of the freedom of movement an object possesses. This section applies the concept of degrees of freedom to describe the cartographic history of Ho Chi Minh City since the 18th century by creating an AR map of Ho Chi Minh City using the historical maps discussed previously. The map provides a three degree of freedom perspective of Ho Chi Minh City by allowing the audience to view the city’s transformations along the political, geographical, and social lines. The AR app creates 2D virtual overlays of historic maps on top of a 2D real-word contemporary map of Ho Chi Minh City hanging on a vertical surface and provides an audio narrative of the historical context and significance of cartography in the past three centuries. This AR experience is created using the image target feature of the Vuforia Software Development Kit in Unity.

The Unity scene that presents this cartographic narrative is made up of a single image target, created using the 2005 map of Ho Chi Minh City (Figure~\ref{map_2005}). The three remaining maps are overlaid on the target. These maps are aligned so that the scale remains consistent across all four maps. Possible interactions with the overlaid maps include dragging and dropping. Each map can be moved to view the layers underneath. Dragging the map to the vicinity of its original position will automatically cause it to snap back into place. Figure~\ref{map_app} captures the interface of the app in use, with three active maps. The app also includes an audio guide to navigate the experience and provide the narrative.
\en 

\begin{figure}[!ht]
\rightline{
\begin{minipage}{\textwidth}
\begin{center}
\woopic{map_app}{.16}
\vspace{-.2 in}
\caption{A screenshot of the map AR experience (Screen capture by Thuy Dinh.)}\label{map_app}
\end{center}
\end{minipage}
}
\end{figure}

The motivation behind the production of this interactive map is also the driving question behind this cartographic analysis, to  attempt to understand the workings of urban memory and its mnemonic devices. To study the maps of Ho Chi Minh City is to trace the sites where remembrance is enforced and memory reinforced, imagined on paper and enacted by means of technology,  wealth, and violence. Memory is untrustworthy, every recollection subject to distortion, whether intentional or not.\footnote{J. M. Winter, \textit{Remembering War: The Great War between Memory and History in the Twentieth Century} (New Haven: Yale University Press, 2006), 4.} Memory is also powerful; collective memory provides shared identity, which enables concepts and ideologies as compelling as nationalism and regionalism.\footnote{Benedict Anderson, \textit{Imagined Communities: Reflections on the Origin and Spread of Nationalism}, Revised edition (London: Verso, 2006).} Memory creates cities. Yet because of its unreliability, memory construction is often both elusive and illusory. Every person has their own personal memory, which to them can often appear infallible. This conviction, however, could lead to the false conclusion that remembrance is a personal experience that exists without agency, while history, on the contrary, is "an objective story which exists outside of the people whose life it describes."\footnote{Winter 11.} The abilities and pitfalls of memory are what motivate this historical analysis as well as its AR component.
