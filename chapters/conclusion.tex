%!TEX root = ../username.tex
\chapter{Conclusion}\label{conclusion}
\vi
From the perspective of memory, the Saigon urban space is a locus where multiple identities and narratives about the city are created and interact. The materials of this kind of memory are spaces such as maps and buildings that articulate the mindset of their creators, steeped in biases and intentions. In this quest to construct a city, planners and builders are also constructing a specific way to use and ultimately remember these spaces. Maps represent ideas that, once built in reality, become ideologies. Architecture exemplifies the nature of this transition from imagined to concrete. Through the creation of these structures, agents of memory like the municipal leadership, private enterprises or citizens themselves are generating a mechanism for recollecting the past that is built into the city’s own landscape. Together, the city’s sites of memory collectively produce an urban memory, which does not just indicate how the city is remembered but considers the city as “a physical landscape and collection of objects and practices that enable recollections of the past and that embody the past through traces of the city’s sequential building and rebuilding” (Crinson xii). This definition of urban memory by Mark Crinson also applies to other facets of the city such as artworks, novels, and events.

The connection between memory and power is what makes memory such an attractive avenue for various groups with a stake in the city. Spatial markers like skyscrapers or even boulevards and bridges provoke recollections of the past. Collective recollection by a city’s population enables them to identify with its history and present, providing the grounds for developing a collective identity with shared pride in the mutual past. Going back to the example of Ho Chi Minh City, for its governing powers, constructing a collective identity was instrumental in this fragmented society that was never completely Vietnamese, Chinese, or French. Using the phrase by Benedict Anderson, the fostering of an “imagined community” through instigators like maps and buildings was what enabled these agents of memory to mobilize people and resources for their purposes, whether it was colonial exploitation, civil war, or industrial production. 

As with any kind of memory processes, the making of Ho Chi Minh City and its memory is subject to coercion, loss and distortion. On the one hand, the physical process of building the landscape often involves violent removal of existing designs, whether those are marshes or communities. It can also lead to forced changes in function for the spaces in question. The Bến Nghé area under French colonization is an example of the functions of a colonial administrative headquarter was imposed on an urban area. Cartography and architecture also institute symbolic meanings through their manipulation of memory. The psychological nature of memory recollection implies loss and distortion. In the place of accurate historical representations are prompts for recollections that exploit the fickleness of remembering to institute decontextualized or synthesized memory.  The result is a kind of romantic and glorified nostalgia that should describe brutal histories but is stripped of any context, a memory without pain. Disruptive and delusive techniques like compulsion, suppression and censorship allow for the production of specific narratives born out of political and commercial expediency.

All the same, to view memory as an expediently constructed product does not mean rejecting the city’s legacy; rather, it calls for a reevaluation of the definition of legacy. Ho Chi Minh City is not a place without history. It is the implication of this complicated history that creates the need for memory, which is a revised and streamlined version of the past. This project is a critique of the production and curation of these alternate narratives. At a time where the past is everywhere and nowhere and the business of memory booms with the commercialization of memorabilia, the context becomes lost among sentimentality and misguided notions of aesthetic. The demand for memory drives selective preservation. What is branded and resold as legacy is only an uncomplicated part of history, while the other messier aspects of colonialism, civil war, forced cultural assimilation, and environmental crises get obfuscated. Going back to Ho Chi Minh City’s legacy, this urban center has always been diverse throughout its history, a product of thriving trade activities and an intersection of various cultural influences, but its sites of memory, the maps and the architecture, are not products of its diverse and vital culture. These efforts are not only intended to be represent the city’s culture but to communicate the authority of the cartographers and architects behind them. Consumers of memory should be aware of these levels of subtext the next time they visit the Notre-Dame Cathedral and walk across the street to look at old Saigon maps in the City Post Office. And when the next skyscraper claims the title of city symbol, one should pause and ask what it symbolizes after all.

With these considerations, what augmented reality provides is a new dimension for critically engaging with memory. Technology helped create sites of memory such as maps and buildings, and technology can help deconstruct them. The motivation for using augmented reality arises from the complexities of historical subjects in general, and memory specifically. The AR model in this project is centered on spatial interactability of primary sources to exploit the spatial dimension of urban memory. Object recognition, which hinges on feature detection algorithms, is the backbone of the object-centered experience. The results demonstrate that AR can reliably be used for a variety of source materials. There are many potentials for augmented reality when it comes to non-traditional methods for studying and visualizing history.

On the other hand, the expansion of augmented reality in fields such as heritage preservation, tourism, education, and museum also creates the need for a theoretical framework for applying this technology to history. The same complexities with memory can also be applied to the use of augmented reality. The use of technology of any kind in public history is implicated in the entanglement of knowledge, power, and memory. Who owns technology, who gets to see with augmented reality, and what it depicts, these are all valid concerns. While it is important that scholarship evolves to embrace the new tools technology has to offer, the methodology and ethics of such practices must be taken into consideration to ensure that creators and users of these experiences are critical and cognizant in their usage. In addition, historiographical dialogues on the topic of science should not be limited to humanities disciplines only. In the true spirit of digital humanities, technology developers should also engage in conversations about the history and implications of their creations.