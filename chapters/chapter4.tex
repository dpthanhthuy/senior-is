 %!TEX root = ../username.tex
\chapter[Architectural Monumentality]{Architectural Monumentality: Memory, Identity, and Power in Saigon Symbolic Architecture}\label{symbol}
% to do: rethink first sentence
If maps represent the mental configurations of a landscape, then buildings are their physical enactment in a process where the imagined turns concrete. Maps provide the high-level network of interrelations between power and memory, in which each building is a node with the same characteristics as the parent network. Both cartography and architecture are visual modes of remembrance, but architecture, as a site of memory, can be more direct and explicit with the transition from two- to three-dimensional space. Within the built environment, the institutionalization of memory is done on several levels, with the building’s function, form, structure, material, and façade. The monumental architecture of Saigon elicits the same questions as the city’s maps, regarding authorship, audience, narrative, and agency. These categories enable recollections of the past in an entanglement of power, memory, and identity. This chapter explores the manifestation of the interconnectedness of these three themes through an examination of the symbolic and functional significance of four monuments in Ho Chi Minh City, with each monument corresponding to a major era in the city’s history.
\vi

The interplay between memory and architecture is an important avenue for many memory studies, as monuments produce important sites of memory (cite Pierre Nora, Hue Tam Ho Tai). Eric Sandweiss’s argument about the role of museums and cities draws on Lewis Mumford’s vision of the city as a “storehouse of memory,” which remains durable in times of change, a site of both “endurance and transformation,” aided by institutions such as the city history museum (Sandweiss 26). Shelley Hornstein posits that architecture is not only defined by design, form, or structure, but as “spatialized visualizations and experiences” in the mental and emotional space (cite Hornstein). Similar framing by Mark Crinson theorizes the city as a collection of objects (or buildings) whose collective image “enables the citizen to identify with its past and present as a political, cultural and social entity” (xiv Crinson). Specifically, in terms of Vietnam, architecture is connected to notions of urbanization and nation-building. For the period when Saigon was the headquarter of French Indochina, urban development was under the influence of colonial ideologies about civilization and modernization (cite Wright and Phantasmatic). These frameworks continue into the post-colonial era, when land-use policies enforce specific framing of the urban landscape (cite erik harms rubble). Studies on architecture and memory are rich but often limited in periodization and specificity. The following analysis spans the length of Ho Chi Minh City’s history, focusing on four major architectural symbols, the Thiên Hậu temple, the Notre-Dame Cathedral, the Independence Palace, and the Bitexco Financial Tower, to illustrate the transformations of the built environment in accordance to shifting political and social processes in the city.

\section{\vi Thiên Hậu Temple}
\vi
Thiên Hậu Temple, officially Tuệ Thành Assembly Hall, was first built by the Cantonese community of Ho Chi Minh City circa 1760. Located in the heart of Chợ Lớn (the Chinese quarter), the temple (Figure~\ref{thienhau_past}) was part of a market complex with multiple other temples, assembly halls, shops, and houses. Today, Thiên Hậu Temple remains in District 5, the unofficial Chinatown of Saigon. The main alter is dedicated to the Empress of Heaven Thiên Hậu (Mazu), a protector of seafarers and ocean goers. The former assembly hall was and still is central to the spiritual and cultural practices of ethnic Chinese in Chợ Lớn. In both a symbolic and functional capacity, the site is a fascinating slice of religiosity in Ho Chi Minh City, where sacred spaces take on non-religious roles as the city transitions into a neo-capitalist period and material culture proves as central to life as spirituality. Believers still visit the temple to seek blessings, but younger generations have taken to it as a scene for having images taken for their social media.\footnote{https://thanhnien.vn/du-lich/nguoi-sai-gon-ru-nhau-di-chua-ba-cho-lon-dau-nam-moi-canh-ty-2020-1173072.html} Tourists visit from over the world, both Eastern and Western, with all levels of knowledge and belief in the sanctity of Mazu the Empress of Heaven, to see its impressive collection of religious paraphernalia and unique Chinese temple architecture. The position of Thiên Hậu Temple in the mental and physical landscape of the city maps the dynamics of its symbolism and functionality in terms of memory and identity.
 \en

\begin{figure}[!ht]
\begin{center}
\woopic{temple_1800}{0.6}
\vspace{-.2 in}
\caption[\vi Thiên Hậu Temple in the 1860s \en]{\vi Thiên Hậu Temple in the 1860s. Taken by Emile Gsell [c. 1865-1875]. From: Giáo hội Phật giáo Việt Nam \en}\label{thienhau_past}
\end{center}
\vspace{-.2 in}
\end{figure}
\vi
The natural and economic characteristics of Chợ Lớn and the larger Saigon area explain the influence of the sea goddess Mazu on the local community. Starting from the 1680s, Ming loyalists arrived in Đà Nẵng seeking refuge. Sent by the Nguyễn sovereigns to southern territories, they cultivated  land in exchange for protection (300 nam 33). Water was integral to the life of these settlers and remained important as they carved out their livelihoods. The migrants, mostly from merchant backgrounds, travelled to the southern coasts by sea and soon controlled commercial activities in the region. They established trade posts and took over import and export, mostly by monopolizing the waterway trade routes for transporting rice produce from the Mekong basin (300 years 35). Heavily involved in water-related activities, merchants looked to the Lady to protect their voyages along the South China Sea trade routes, laden with rice and other produce from the hinterlands. In the liminal space of the temple, travelers said their prayers before a trip and made offerings after one to thank the deity for their safe travels.

The shared culture of Mazu worshipping became the binding factor for communities of settlers in this foreign land. The Guangdong community who built the temple also looked to the Lady for blessings regarding health and fortune. Important events in a person’s life such as birth, marriage, new commercial ventures, illness also brought supplicants to Mazu for guidance and favor (Cherry 64). Donations went to the upkeep of the temple and community building. As an assembly hall, Thiên Hậu temple spearheaded social projects that provided education and support for members of the congregation (cite Thien Hau book). The spiritual center transcended its religious function and became a social hotspot, where seafarers and locals connected in a sphere of shared backgrounds, business ventures, on top of cultures and beliefs. This practice was replicated all over the coasts of the South China Sea. Chinese merchants built Mazu temples all over the port cities in China, Vietnam and other Southeast Asian countries, creating a spiritual network that was established on the routes existing trade network.

\vi The role of Thiên Hậu Temple in the Chinese community was not only important for travelers but for locals as well. In the $19^{th}$ century, life in Chợ Lớn was divided along geolinguistic lines, according to the \textit{bang} (congregation) system. Chinese settlers formed different congregations according to their hometown in China and their dialect. Every congregation had it own market and place of worship, along with other collective properties such as hospitals, schools, cemeteries, etc. They each managed a separate assembly hall, and on top of its religious functions, also performed administrative duties such as immigration and emigration registration and tax collection. (cite Doling 29) The Cantonese population in Chợ Lớn built the Tuệ Thành Assembly Hall (also known as Thiên Hậu Temple) circa 1790, at the height of Qing migration. (cite Doling 19) Near Tuệ Thành (Guangzhou) Assembly Hall, Ôn Lăng Assembly Hall (or Quan Âm Pagoda) was built by the Fujianese settlers. These centers regulated life outside of their thresholds through the coordination of ceremonies and rituals. Communal spaces served as a point of contact for new migrants, a gathering place for members of the same ethnic Chinese group, and as the coordinator of community-wide projects and events. These responsibilities could be educational, charitable, or ceremonial. One such evidence of these communal functions can still be found at Thiên Hậu Temple in the form of an antique fire extinguisher from the 1890s (cite). As part of a market complex, the temple resided at the intersection of all the activities that Chinese in Chợ Lớn participated in on a daily basis. Spirituality was tied to the fabric of the community.

The administrative aspect of the temple made this system desirable for the new rulers in town. Since the Chinese had great influence in Saigon due to their economic stronghold, they had great leverage for negotiating with the political powers. The semi-autonomous congregation system lasted from the Nguyễn regime through  French colonization, encouraged by an influx of Chinese migrants looking for new economic opportunities and escaping the political crisis in China in the 19th century. Under French colonization, Chợ Lớn enjoyed considerable autonomy. Colonial authorities communicated with representatives from the Chinese congregations, very often the same people who looked after the assembly halls. As an effort to appease local unrest, the French issued decrees to preserve scriptural materials and decorative objects in major temples such as the Guandong’s Thiên Hậu Temple. As a result, religious spaces were some of the places where ethnic traditions and particularities were best preserved. \en

\begin{figure}[!ht]
\begin{center}
\woopic{temple}{0.1}
\vspace{-.2 in}
\caption[\vi Thiên Hậu Temple in 2008 \en]{\vi Thiên Hậu Temple in 2008. Taken by Christopher [2008]. From: Flickr \en}\label{thienhau_present}
\end{center}
\vspace{-.2 in}
\end{figure}

\vi Today, Thiên Hậu Temple (pictured in Figure~\ref{thienhau_present}) takes on completely different meanings. Chợ Lớn remains the dominant Chinatown, but the neighborhood is no longer predominantly Chinese. The process of assimilation as well as ethnic Vietnamese migration to Ho Chi Minh City have reduced the percentage of ethnic Chinese in the city to around the 10\% mark.  During the American War of Resistance, the Saigon government embarked on the socio-cultural and economic restructuring of Saigon, which included the revocation of permits held by Chinese owners in several trade fields, forcing them to adopt  Vietnamese citizenship or change profession (300 years 205). President Ngô Đình Diệm and his successors turned away from the semi-autonomous congregation system used by the French. In their effort to assimilate and limit the influence of (non-American) foreigners, the government nationalized previously independent institutions such as schools, pagodas, cemeteries, etc.  Foreigners, including Chinese, were no longer allowed to trade meat and fish or engage in retailing consumer products, but probably the most devastating blow was the ban on transportation of people or freight by land and water. These laws blocked the Chinese community from relying on commerce as their main livelihood and drove many from Chợ Lớn into the Bến Nghé area of Saigon. The economy of Saigon was no longer controlled by the Chinese, or the French for that matter. A new body of Vietnamese nationals took over, with the support of American companies (300 years 204).

Chợ Lớn fell into dilapidation as  social stratification grew with the changes in demographics and economics. Following the fall of Saigon in 1975 and the subsequent 10-year bao cấp (subsidy) period, life of the Chinese in Saigon became even more difficult as the socialist state sought to nationalize private properties and imposed strict regulations on the exchange of goods.  Many left Vietnam during this time. It was only after the reforms in 1986 that the Chinese started to participate again in Ho Chi Minh City’s economy. These economic and political transformations led to the social restructuring among the ethnic Chinese in Chợ Lớn. These policies effectively restricted the influence of non-Vietnamese cultures and changed the role of sacred spaces like Thiên Hậu Temple in the life of Chinese and non-Chinese in Ho Chi Minh City.
% to do: footnote for name

The functions of Thiên Hậu Temple have also adapted to changes in the social fabric and economic makeup of the new urban landscape. One of the most notable changes was in the temple's name in Vietnamese, which uses the word \textit{chùa} (pagoda). The original term used by the Guangdong was \textit{miếu} (temple). Both indicate a place of worship; while \textit{chùa} is commonly affiliated with Buddhism, a \textit{miếu} is often dedicated to local deities. The blurred distinction between Buddhism and đạo mẫu (Religion of the Mother Goddesses), which includes Mazu, signifies the shifting importance of religion in the late socialist society. Visitors to the temple can be Buddhist, Christian, and often non-Chinese. The Lady’s sanctity is no longer specific to water-related prayers. Today’s worshippers are often indiscriminate about which deity to make an offering to on New Year’s Day. In this noninstitutionalized religious landscape, beliefs are shaped on an individual level by influences such as family, acquaintances, and popular publications. (Taylor 3) This phenomenon is not particular to Thiên Hậu Temple or Mazuism. Spiritual plurality gives Vietnamese temples a new look, often accompanied by new rituals and ceremonies, and eventually new meanings. In the postwar aftermath, these spaces ceased to perform the old communal functions, becoming less of a social and economic founding pillar and taking on a more strictly religious and spiritual capacity.

The rhetoric used by the temple leadership also evolves following the changes in political regime. The temple's special publication celebrating the beginning of the 21st century started with a line rehashing the 25th Anniversary of Unification (end of the American War of Resistance), followed by an expression of gratitude to the Party and the state.  The tone conforms with the national rhetoric of secularization, usually dressed in anti-superstitious language. In the post-Renovation era, the Communist Party has struggled to create a paradigm for religious freedom that still fits their agenda of urban redevelopment, specifically in the case of spiritual spaces.  An attempt to incorporate Thiên Hậu Temple into this framework was done in 1993 when the Ministry of Culture and Information recognized the temple as a national heritage site. The question of reconciling spirituality with modernity and secularization continues to plague the Vietnamese government as they maintain an ambiguous response to religion. The attitude adopted by the state is indicative of the postwar secularized identity that the Communist Party embraced. Consequently, in terms of Ho Chi Minh city, the temple's image enforces an identity built on ethnic diversity, rather than religious multiplicity.\footnote{Other scholars on Vietnam such as Christina Schwenkel have pointed out that urban growth does not necessarily entail secularization in the case of socialist cities. See Religious resemblage for an example of how different actors use religious spaces in Vinh, a city in central Vietnam, to vie for control over memory.}

Despite the state's uncertainty towards the city's religious landscape, Saigon tourist agencies and the city leadership have also attempted to project the image of cultural diversity through their advertisement of the temple. The temple is among some of the most visited attractions on one-day city tours, framed as providing a slice of life in Chinatown, notwithstanding how flimsy the connection is today.  From serving as a symbol in the Guangdong Chinese community, the temple has been rebranded and resold as a representation of the entire Chinese population in Saigon. Ethno-linguistic distinctions have become obscured and redressed in one form or another,  masked as equality and accentuated to showcase only a selective and modified façade of diversity.

The city's sacred topography is important for cultivating cultural memory. For past and present temple-goers, the existence of these spaces provide a spiritual sphere with rites and objects that enable recollections of their origins and traditions. For the different Saigon regimes, the extent of state influence on the temple is a form of control over the Chinese community and their identity. Today, both public and private institutions in the service industry have attached new meanings to the temple such that the act of worshipping at the temple becomes a mechanism of identification for citizens, in the effort to create an identity based on cultural diversity. The same process also take place with other symbols in this architectural environment.

\section{Notre-Dame Cathedral of Saigon}

On the website of the Roman Catholic Archdiocese of Ho Chi Minh City, under the series “Trùng tu nhà thờ Đức Bà" ("Notre-Dame Restoration”), one finds the opening article “Nhà thờ Đức Bà Sài Gòn, sức hút của một công trình" ("The Notre-Dame Cathedral Basilica of Saigon, the Attraction of a Construction”). The piece is a compilation of the perspectives of several architects on the values of the cathedral as an architectural symbol of Ho Chi Minh City. Architect Nguyễn Thu Phong writes:

\begin{quotation}
I wonder why it influences the emotional life of the city’s inhabitants so much… It’s like a hyphen between an urban life and spirituality… Around the cathedral, life carries on as usual every day, parents picking up their kids after school, couples shooting their wedding photos from countless angles. On holidays, there seems to be a magnet drawing Saigon citizens to the cathedral. Nearby, the Diamond high-rise symbolizes the city’s commercial life. On another side is the Cultural House of Youth. A bit further lies the administrative and political center. On its left is the municipal post office. The cathedral fits into a cultural and communal complex. Maybe that’s why it is such an attraction in the life and mind of the people of the city.
\end{quotation}

The seamless blend of spirituality and urbanism is a common fascination when it comes to this building. Situated in the middle of a large intersection within walking distance of all major administrative and recreational sites in District 1, the proverbial downtown of Saigon, the Notre-Dame Cathedral has been well integrated into the city’s symbolic landscape as a beloved emblem. Nevertheless, even in the same article, the ambivalence with the cathedral’s history and its religious implication is palpable. This uncertainty is reflected in the words of another architect:

\begin{quotation}
A construction that represents the dream of moving forward… [a] church accompanied by a post office, government headquarters, a theater, a square… The complex projects the image of an urban center and the power of the authorities in the eyes of colonial subjects… An architectural culture is the result of the interactions within the society through the length of history. In the end, if there was no cathedral, there might have been another famous piece of architecture, but there won’t have been a hundred years of French colonization.
\end{quotation}

Such sentiment is not uncommon even though the concern is sometimes lost in total reverence of the cathedral’s architectural brilliance. In a late-socialist society, the church’s colonial ties can make it difficult to justify such adoration. The sentiment can be passed off as love of the city or pure admiration for the architecture. Oftentimes, it is mixed with a hint of nostalgia, or “colonial blues” as some critics point out, for the lost “Pearl of the Orient.” Scholars often consider the connection of colonial nostalgia and its architectural legacy. Historians of memory have long considered architecture as a reflection and agent of memory. Eleni Bastéa theorizes the relationship between memory and the built environment through design, literature, and practice.  In the cathedral, one sees the process of construction and reconstruction of memory, from its erection, to the changing iconography, and the ongoing restoration.\footnote{add 2020} The meaning of the cathedral shifts over time as agents of memory work to enact structural and symbolic modifications. A French badge of power, of religious superiority, a token of modernity, and now a proof of history, these are all the polysemous identities of an edifice representative of an equally fluid city. To understand how these complex levels of symbolism have been developed in the 140-year history of the Notre-Dame Cathedral of Saigon, it is important to grasp the context of its creation and development over time. \en

\begin{figure}[!ht]
\begin{center}
\woopic{cathedral_1890}{0.3}
\vspace{-.2 in}
\caption[Notre-Dame Cathedral in 1890]{Notre-Dame Cathedral in 1890. From: Maison Asie Pacifique \en}\label{thienhau_present}
\end{center}
\vspace{-.2 in}
\end{figure}

The Notre-Dame Cathedral Basilica of Saigon started construction in 1877 and was consecrated in 1880, 22 years after the French conquest of Saigon in 1858 (Cherry 16). By this time, the city was well established as the colonial capital of Cochinchina. To meet the religious needs of the colonial class in town and the demands of missionary work, the original Église Sainte-Marie-Immaculée was completed on the site of an old temple in 1863 but soon fell into dilapidation. A replacement, bid by numerous French contractors and won by Jules Bonard, was immediately underway (Notes and News).  Construction concluded in 1879, resulting in a Romanesque structure 93m in length and 36m in width, held by bricks imported from Marseille, in the traditional cruciform shape complete with a transept and a nave, in other words, a piece of architectural brilliance by all Western standards (Cherry 16).

As the colonial headquarter, Saigon boasted a complex of military and administrative institutions, now with the Cathedral as its religious center. Visitors to Saigon did not fail to draw the connection between the “Paris of the Far East” with its original source of inspiration. An American war correspondent, Jasper Whiting, wrote about the city's broad and immaculate boulevards and the various miniatures of Champs Élysées, Bois de Boulogne, Avenue de l’Opéra. He sang praises of the “twin-spired cathedral, the Notre-Dame of the city” (Edwards 91). The cathedral’s bell towers were the first thing that greeted travelers from the bank of the Saigon Rive after a 74km trip from the coast (Demay 125-9). Everything about the city was reminiscent of France, and adorning the grandiose of the Western-styled edifices was the tallest, grandest structure of all, the Saigon Cathedral. From the point of view of the colonial subjects, the show of power through the imposing monument was an effective move. \en

\begin{figure}[!ht]
\begin{center}
\woopic{statue_pigneau}{0.5}
\vspace{-.2 in}
\caption[Pigneau du \vi Béhaine Statue \en]{Pigneau du \vi Béhaine Statue. From: Flickr \en}\label{statue_pigneau}
\end{center}
\vspace{-.2 in}
\end{figure}

\vi By the start of the $20^{th}$ century, Saigon already had all the marks of an urban city, but it did not exude the exotic vibe of a Far Eastern metropolis. The botanical gardens, the zoos, palaces, hotels, and now a church, these are emblems of the urban identity imposed on the colony by the French settlers. It was too modern, too French, and lacking the special Asian character. The colonial government began to address this imbalance by highlighting Vietnamese-French collaboration, an example of which is the statue of Monsignor Pigneau du Béhaine and the Crown Prince Nguyễn Phúc Cảnh (Figure~\ref{statue_pigneau}). Monsignor Pigneau was the Apostolic Vicar of Cochinchina. He helped Nguyễn Ánh defeat the other factions to ascend the throne in the 18th century. Commonly known as Bá Đa Lộc in Vietnam, Pigneau was a French legend known for his assistance with the Nguyễn and guidance of Prince Cảnh (Cherry 17). The emphasis on Pigneau's role in Prince Cảnh's life reiterated the importance of French presence in Cochinchina and Vietnam, where “the French name should be synonymous with progress, civilisation and true freedom”. The depiction of the French priest with a member of the royal family signified the colonial powers' intentions for Saigon, to portray the statue, and the city by extension, as an example of cooperation between the two peoples.
%to do: rethink paragraph structure
%functioned as a gateway for incorporating Vietnamese into this colonial society on the religious front, but also served as

As much as Notre-Dame was a marker of Christianity’s spread in Saigon and Vietnam, the cathedral has always been distinguished for its symbolic merits rather than its religious capacity. Even during the French occupation period, the cathedral was more popular among tourists than worshippers.  Given the limited potential for tourism of Saigon, such a monument was immediately earmarked for prospective visitors, and still is. The allure of architectural or historical substance of any form made the destination an instant tourist hit in this otherwise dull neo-European city. Similar observations are made by Panivong Norindr, who presents the theory that Indochina is a “phantasm”, a fiction cultivated during the French colonial hegemony to indulge the French exotic fantasies and the nostalgia for grandeur (Norindr 2). On this romantic canvas where borders were remapped, arts appropriated, and environments built, colonial configurations like the Notre-Dame were generated en masse to construct an imaginary space that both showcased indigenous cultures while appealing to European aesthetics and the colonial ethos of civilization.

The concern with a lack of heritage also motivated later Vietnamese regimes to preserve certain colonial legacies (like the cathedral) while abolishing others. The Vietnamese Communist revolutionary force Việt Minh tore down the Pigneau statue in the August Revolution of 1945. On the old pedestal, a statue of the Virgin Mary was erected in 1959 by the Archdiocese of Saigon.  These transformations reveal the fraught nature of heritage conservation. Governments based the decision about which sites to preserve on ideological implications rather than their cultural values. When it comes to architectural masterpieces whose cultural values may outweigh any ideological justification for demolition, depoliticization was often the solution. One form of depoliticization was through romanticization to the extent where these sites are stripped bare of any historical context, as in the case of the Notre-Dame.

Today, the cathedral is publicly recognized by the municipal government and Saigon citizens as a symbol of the city, an image among several objects and spaces around which urban identity is formed (Figure~\ref{cathedral_present}). However, the kind of memory it elicits does not focus colonial exploitation, as is the case with the official narrative on the colonial period (cite Dat Do). Instead, its symbolic status is solely as an architectural classic, a sign of European civilization, amidst a wave of nostalgia for a former Saigon, even in its colonial years. To accompany this trend of aestheticizing the past, old colonial establishments have been “architecturally re-enhanced and [their] aura of ‘colonial distinction’ reinvented” to cater to nostalgia seekers (Ravi 476).  The Notre-Dame is undergoing its own restoration project, using materials imported from France and Germany and the construction continues to be overseen by French and American contractors. Throughout its 140-year history, the meanings assigned to the Notre-Dame might have changed, but its symbolic function, as a colonial token of power in the French Indochina capital and as a cultural and historical heritage site today, continues to exist, morphing from one form to another to accommodate the city’s transformations. \en

\begin{figure}[!ht]
\begin{center}
\woopic{cathedral_today}{0.3}
\vspace{-.2 in}
\caption[Present-day Notre-Dame Cathedral]{Present-day Notre-Dame-Cathedral. From: Pinterest \en}\label{cathedral_present}
\end{center}
\vspace{-.2 in}
\end{figure}

\section{Independence Palace}
\vi

On the theme of metamorphosis, no other site in Ho Chi Minh City has undergone as many modifications to its physical and figurative façade as the Independence Palace. The changes to this monument are both tangible and symbolic, as forces of history perform alterations on its structure, functions, and significance throughout the years. The palace, originally named Norodom after a Cambodian king, was inaugurated in 1875. It served as the headquarter of the Governor-General of Cochinchia and became known as the Palais du Gouvernement-général (Governor’s Palace) (Doling 153). Its construction was an expensive and extended affair. Nevertheless, the building’s usefulness never lived up to the exorbitant price tag. Rarely touched by the Governors of Cochinchina, who paled in importance to a new Governor-General in Ha Noi, the place fell into disrepair, its capacity downgraded to mostly ceremonial (Doling 154). In 1954, on the withdrawal of French military following the Geneva Accords, the palace was handed over to the new South Vietnamese administration headed by President Ngô Đình Diệm (Doling 155). The French edifice’s time as the new presidential palace did not last very long. In 1962, disaffected members of the Saigon Air Force staged a coup on the Diệm government and wrecked the old structure in the process (Doling 155). At this point, Diệm commissioned a new presidential complex to be built on the site of the demolished palace, under the supervision of the up and coming French-trained Vietnamese architect Nguyễn Viết Thụ. The newly built Independence Palace became the home of succeeding South Vietnamese presidents from 1966 until the fall of Saigon in 1975. After unification, the site became a historic monument and a museum; its main hall was renamed to Reunification Hall and serves as a venue for ceremonial and entertainment functions.

Throughout the palace's history, there is a strong link between architecture and power. The Indochina regime was well-versed in employing colonial ideology to establish architectural hegemony over Saigon (cite Norindr Phantasmatic). With the Norodom Palace, even though its practicality fell short of expectations, the production of the monument set the trend for embracing symbolism in this architectural space. The colonial ethos of westernization and civilization came across in the neoclassical edifice, later supplanted by a modernist-oriental hybrid design under the South Vietnamese era (Doling 47-51). The American War period was dominated by a new generation of Vietnamese architects eager to combine Vietnamese architectural precedents with their European training (Truong, Vu 4). Nguyễn Viết Thụ’s palace design incorporated several Chinese characters in its exterior façade (Figure~\ref{palace_chinese}). The symbolic power of architecture was well articulated by the words of a French publication on the opening of the Palais d’Exposition in France, a “museum of the colonies”: the Norodom/Independence Palace was a presentation of a “great modern state with the body of its political organization, the exact representation of its economic power, and the complete tableau of its social, intellectual, and artistic activity.” (Beauplan in Phantasmatic 25) If the Palais d’Exposition was designed to showcase the exoticism of French colonies, the Independence Palace was a reversed manifesto (Phantasmatic 24). The Saigon monument was a public display of French/South Vietnamese political and aesthetic visions, an exertion of power, and a claim of dominance over the entire city. \en
%to do: change quote on the exposition. split to two sentences

\begin{figure}[!ht]
\begin{center}
\woopic{chinese}{0.6}
\vspace{-.2 in}
\caption{\vi The Chinese characters in Ngô Viết Thụ's design. From top to bottom, left to right, the characters are mouth, loyalty, three, king, ruler, and rise. From: Independence Palace website \en}\label{palace_chinese}
\end{center}
\vspace{-.2 in}
\end{figure}

\vi The Independence Palace is a site of transition, its identity in flux to accommodate whichever power was seeking to exploit its symbolism. To extend the idea that the palace is an expression of power, one can argue that the literal act of occupying the palace has been synonymous with gaining control of Saigon. Compared to the other symbols of Saigon in the scope of this research, the Independence Palace is the most connected to the city’s history. Almost every important turn of events since the $19^{th}$ century was mirrored by the monument. During Japan’s brief imperial stint in Vietnam (1941 – 1945), their military heads resided in the palace and negotiated their shared control with the Vichy representatives over meals in its dining room (Dommen 78). At the height of Japanese imperialism in Asia, they even held French officials prisoners in the palace. The palace has always served as the residence of the highest authority in Saigon, South Vietnam and sometimes Vietnam. From the French settlers to Japanese imperialists, from American militarists to South Vietnamese nationalists and the present-day socialists, staking claim over the palace has become the constant for all external forces seeking a foothold in the city.

The most striking example of the symbolic merit of this architectural complex is the events of April 30, 1975, when a tank of the North Vietnamese Army bulldozed through one of the palace’s secondary gates and Lieutenant Bùi Quang Thận replaced the South Vietnamese flag on the roof with the National Liberation Front’s flag. This turn of events is invariably featured in history textbooks down to the details of the tank’s model. The photo of the tank crashing the gate (Figure~\ref{tank}) captures the most recognizable moment of this historic day in the Vietnamese national discourse. April 30  became a national holiday in Vietnam, Ngày giải phóng miền Nam, Thống nhất Đất nước (Day of liberating the South for national reunification, or Liberation Day for short (cite Vietnamese textbook). The use of this symbolic takeover to mark the end of the war demonstrates the current authorities’ continued desire to employ the palace as a site of memory, a place where narratives are produced and reshaped, imagined and rethought, in a struggle to define the city. \en

\begin{figure}[!ht]
\begin{center}
\woopic{ky-uc}{0.25}
\vspace{-.2 in}
\caption{\vi The tank crashing the gate of the palace. A tank of the same model is still on display in the palace's grounds.}\label{tank}
\end{center}
\vspace{-.2 in}
\end{figure}

\vi Commemoration is the unifying theme for postwar framing of the American War, and the Independence Palace is no exception (Cite Hue Tam Ho Tai, Viet Thanh Nguyen). Amidst other monuments such as museums, battlefields, and cemeteries, the Independence Palace has become a commemorative class of its own. Historian Jennifer Dickey points out the distinction between the palace and other war monuments in the city, suggesting that the former presidential residence provides a more upbeat take on the war than its counterparts in Saigon, such as the War Remnants Museum or the Củ Chi Tunnels Complex, which are more focused on the strategic aspect and the heavy casualties (Dickey 153). Today, the palace functions as a museum, with most parts open to visitors interested in the former Vietnamese “White House.” The whole presentation offers little context into the lives of the palace’s former residents, opting instead to centralize the official war narrative of a national struggle for unification against an imperial power and its puppet regime. A more subtle interpretation can be gleaned from the contrast between the excessive opulence on display as opposed to the scarcities in the North Vietnamese National Liberation Front. The Independence Palace, as well as war tourist sites in Ho Chi Minh City, is a tool for nation building, with which the state, as a producer and curator of spaces, legitimizes their authorities and establish a shared identity among its citizens, reinforced by these physical markers of nationalism, patriotism, and belongingness.

The Independence Palace exemplifies the polysemous nature of Ho Chi Minh City’s architectural symbols. April 30 may be Liberation Day in Vietnam, but for many Vietnamese in the diaspora, it is remembered as the Fall of Saigon or Black April.  For them, the palace has a different meaning. It is a relic of a bygone era, witness to what many see is the ultimate betrayal by the U.S. that led to Saigon’s collapse (cite New Perspectives). Mitchell Owens writes about the palace in a \textit{New York Times} article with the flippant description “East meets West, in a funky monument to wartime folly.” That nostalgia for the height of the Republic of Vietnam and its elusive governing families is tangible throughout the article and its wistful title, “Madame Nhu Almost Slept Here.” %to do: elaborate on identity and memory, concluding sentence

\section{Bitexco Financial Tower}
At the height of 262m and a mere 5-minute walk from the bank of the Saigon river, the Bitexco Financial Tower looms over the historic business district of Ho Chi Minh (Figure~\ref{bitexco}). Like various other constructions in the city, the tower's location is water-oriented, but its relationship with water is less a dependency than a strategic arrangement, an icon towering above the swamp that makes up this city. Inaugurated in 2010, Bitexco was the tallest structure in Vietnam until the Gangnam Landmark Tower opened one year later in Ha Noi, but in Ho Chi Minh City, its height record remained unsurpassed for eight years. Even though this title has been overtaken by another superstructure, Landmark 81 (461m), Bitexco’s status has endured as the symbol of modernity in the “new” Saigon. The plan for Bitexco to define the urban landscape of Saigon was already in motion during its construction. Before the tower’s completion in 2010, the owner and development group of Bitexco and the press had been touting its iconic stature as the new beacon of Saigon’s innovation and design. According to the Venezuelan American architect in charge, Carlos Zapata, Bitexco was an “iconic embodiment of the energy and aspirations of the Vietnamese people.”  As far as the history of landmarks in Ho Chi Minh City is concerned, this building is yet another site where narratives are defined and memories constructed, a playground for contesting ideas, ideals, and powers.
\en

\begin{figure}[!ht]
\begin{center}
\woopic{bitexco}{0.25}
\vspace{-.2 in}
\caption{\vi Bitexco from a distance. From: reatimes.vn}\label{bitexco}
\end{center}
\vspace{-.2 in}
\end{figure}

\vi From the earliest stages, the developers of Bitexco had shown ambitions for building an era-defining structure through their design concepts. The form of the tower was to embrace the traditional and philosophical symbol of a lotus bud, significant for its importance in Buddhism. The lotus flower represents enlightenment and purification (Tran Van Khai 1). The acclaimed national flower of Vietnam features in various folk poems:
\begin{verse}
\begin{center}
\hspace{2em} In swamps the lotus shines, \\
Green leaves, white flowers, fine stamens.\\
\hspace{2em} Blooms, leaves, and stamens gold,\\
Near mud without the moldy stink.\footnote{Luxury and Rubble}
\end{center}
\end{verse}

Said to grow in mud and bloom when reaching light, the lotus flower purifies where it grows and blossoms even in stagnant water. Its symbolism becomes a metaphor for the vitality and strength of the Vietnamese people and nation. Embracing the loaded image of the iconic lotus bud, the Bitexco owners sought to present the tower as a budding symbol rising above the swampy landscape of Saigon (both in terms of its geographical and sprawling demographic conditions). This use of cultural symbolism in modern architecture appeals to a shared notion about the Vietnamese national identity. The Bitexco tower as a site of memory, while contributing to the national narrative of self-determination, also gains its recognizability by enabling citizens to “identify with its past and present as a political, cultural and social entity.” (Mark Crinson xiv)  %to do: explain the national narratives of self-determination

Bitexco is an example of the modern/traditional dichotomy prevalent in Vietnamese discourse of urban development. In this symbiotic relationship, the lotus bud structure was realized by high-class technologies. Its futuristic architectural style rejects the modernist trend that has dominated Saigon since the end of the First Indochina War, marked by monuments such as the Independence Palace (Tran Van Khai 1). The transition from old architectural patterns to new ones projects the inherent shift in the city’s identity, characterized by changing notions of beauty and revised urban codes. These new high-rises are the image of the integrated, modern, global city, defined by architectural ingenuity and stylistic aesthetics. A gravity-defying structure like a skyscraper in itself is a testament to advancement in the sciences. Its convex sides and rounded corners are a step further from the convention, an embodiment of innovation and uniqueness. The presentation of the city in this sky-piercing depiction aims to showcase Saigon’s strides in modernization, urbanization, and rapid economic integration by means of symbolism, sign, form, scale, and materiality (Tran Van Khai 7). As an example of forward thinking and excellence in designed, the landmark was built to “surprise, to astonish, and to alter perspectives,” while still laden with ideals about culture, tradition, and history in this futuristic and unorthodox architectural rendering. %to do: different ways to think about identity, national, city, colonial

Like other geographic and political processes in Ho Chi Minh City’s history, the transformation into the contemporary notion of modernity and urbanization exemplified by these buildings is far from straightforward. Nature-defying structures like Bitexco require tremendous alterations to the environment. Working with the soft soil conditions of the Saigon river alluvial plain, engineers had to sink the piles in the base of the building 75m underground in addition to creating a concrete mat foundation to ensure the skyscraper could withstand the force of nature (and water) that so often plagues development plans in Saigon. Natural modifications were not the only violent processes to enable the monument’s existence; controlling the human factor is also an important variable in building the perfect city. The emergence of new urban zones in Ho Chi Minh City is synonymous with mass displacement of people, in a process that is oftentimes brutal and involuntary. The current discourse of urbanization is redefining the notion of a modern city to be synonymous with “grand buildings framing breathable and beautiful open spaces” (Haims 737 beauty). Urban programs of spatial cleansing forcefully clear the city of its slums to make way for an emerging urban upper middle class. The politics of control through urban building codes manifest in the additions of high-rises like Bitexco and recently Landmark 81. Land use becomes contentious as new projects reach lower-class residential and communal spaces, creating social stratification along the city’s spatial lines.

 The state is not the only player in the changing façade of Ho Chi Minh City with its zoning and urban projects. With the rise of private real estate projects comes the participation of private corporations in reimagining the city. Agents such as Bitexco, or more recently Vingroup (owner of the current highest building in Saigon), see these parcels of land as “a surface area on which to manage economies of scale.” The kind of memory engendered by these architectural structures and complexes legitimizes development plans for other peri-urban neighborhoods. The memory industry becomes lucrative in the face of diminishing unused urban land and climbing real estate prices. How enterprises justify the displacement of people and razing of current neighborhoods to make place for new middle-class urban zones is built upon ideologies about urbanization, development, and progress that continue to be reinforced with renderings such as the Bitexco Financial Tower. But urban planners also take another measure to ensure the validity of their constructions for citizens of Ho Chi Minh City by hiring French, Japanese, and American contractors in key design phases (Harms notion of beauty 743). In this sense, postwar late-socialist Vietnam has not moved away from previous Western (colonial) configurations and ideals about how a good environment looks and functions, enforced through urban codes about hygiene, order, and beauty (cite Gwendolyn Wright 1, Harms 737). Key players outside of the state have carved their space in the urban planning business of Saigon, and by extension, the memory created by these spatial transformations.
 %to do: change topic sentence to reflect memory

Outside of its symbolic capacity, the Bitexco Financial Tower serves all the usual functions associated with mix-use high-rises: 370,000 $m^2$ of office space, a 10,000 $m^2$ retail podium, completed with a sky deck and helipad cantilevers from the 52nd floor (cite Tran Van Khai 6). The Grade A office suites are reserved for elite finance organizations and multinational corps (cite tran Van Khai 8). The shopping center features retailers from high-end brands, as well as dining and entertainment services.  The experience provided by the shopping area of Bitexco is indicative of the upward trend in shopping malls in Vietnam. There are currently more than 40 shopping centers in Ho Chi Minh City, most of which feature some combination of a food court, a movie theater, and middle to high-end stores.  Even though the clientele of these shops is usually upper middle class, the restaurants, cafes, and movie theaters can attract large crowds on week nights and weekends. Bitexco is best known for its cinema and rooftop bars. These spots, often underground or dozens of meters above ground, provide the new communal spaces for Saigon, whose public sphere is characterized by its growing consumerist tendencies.

The desire for urban landscape to manufacture the city's identity is made explicit in Bitexco’s case. Following the ground-breaking of the Bitexco Financial Tower, the chairman of the Board of Directors of Bitexco Group, Vũ Quang Hội, explained their visions for the construction: “Before, when mentioning Vietnam, people often thought of war and poverty… [Now] the image of the Bitexco Financial Tower will appear on all the postcards and souvenirs, so that when friends from all over the world come to Vietnam, they will bring home the image of a nation of innovation and development.” Architecture is not simply a representation of the present but a reconfiguration of the past, where new buildings have replaced old monuments in a restructuring of both the physical and mental landscape. For a private enterprise like Bitexco to become the "icon builder and creator of future values," as the Vietnam National Real Estate Association Press calls it, it is clear that Saigon's memory, as embodied by these structures, is not just rooted in the past, or the present, but also in an idealistic orientation for the future that concerns notions of beauty that favor a capitalist and consumerist trajectory of urban development.\footnote{http://reatimes.vn/bitexco-nguoi-di-xay-bieu-tuong-va-kien-tao-gia-tri-tuong-lai-21299.html} %to do: change paragraph to identity

\section{Creating Memory Targets}
\en
